\newcommand{\frfb}{{\mathsf{fb}}}
%---------------------------------------
\begin{definition}\label{definition:FourBoxes}
We define the reduction $\to_\frfb$ (where $\frfb$ stands for \emph{four boxes}) as follows, for any atomic flows $\phi$ and $\psi$ with a polarity assignment $\pi$:
\[
\atomicflow{
(-8, 6)*{\afvjdm4{\boldsymbol{\epsilon}}{}};
( 0, 8)*{\afaidm{}{}{}{}{}{}};
( 8, 6)*{\afvjdm4{}{\boldsymbol{\epsilon'}}};
%-
(-5, 0)*{\affr88};
(-6, 2)*{\aflabelleft\ppl};
(-4, 2)*{\aflabelright\phi};
( 5, 0)*{\affr88};
( 4, 2)*{\aflabelleft\pmi};
( 6, 2)*{\aflabelright\psi};
%-
(-8,-6)*{\afvjum4{\boldsymbol{\iota}}{}};
( 0,-8)*{\afaium{}{}{}{}{}{}};
( 8,-6)*{\afvjum4{}{\boldsymbol{\iota'}}};
}
\quad\to_\frfb\quad
\atomicflow{
%one
(-27, 8)*{\afawdm{}{}{\boldsymbol{\epsilon_1}}{}};
(-19,16)*{\afvjdm8{\boldsymbol{\epsilon}}{}};
(-19, 8)*{\afacum{}{}{}{}{}{}};
(-24, 0)*{\affr88};
(-20, 2)*{\aflabelleft{f_1(\phi)}};
(-27,-8)*{\afawum{}{}{\boldsymbol{\iota_1}}{}};
(-19,-8)*{\afacdm{}{}{}{}{}{}};
(-19,-16)*{\afvjum8{\boldsymbol{\iota}}{}};
%two
(-15,18)*{\afawdm{}{}{\boldsymbol{\epsilon_2}}{}};
(  0,18)*{\afaidmex{}{}{}{}{}{}92};
(-12,10)*{\affr88};
( -8,12)*{\aflabelleft{f_2(\phi)}};
(-16, 1)*{\afcjrm2{10}};
( -9, 2)*{\afawum{}{}{}{}};
%three
(-16,-1)*{\afcjlm2{10}};
( -9,-2)*{\afawdm{}{}{}{}};
(-12,-10)*{\affr88};
( -8, -8)*{\aflabelleft{f_3(\phi)}};
(  0,-18)*{\afaiumex{}{}{}{}{}{}92};
(-15,-18)*{\afawum{}{}{\boldsymbol{\iota_3}}{}};
%four
(-5, 8)*{\afawdm{}{}{\boldsymbol{\epsilon_4}}{}};
( 1, 8)*{\afvjm8};
(-2, 0)*{\affr88};
( 2, 2)*{\aflabelleft{f_4(\phi)}};
( 1,-8)*{\afvjm8};
(-5,-8)*{\afawum{}{}{\boldsymbol{\iota_4}}{}};
%psi
(13, 12)*{\afvjdm{16}{}{g(\boldsymbol{\epsilon'})}};
( 9, 13)*{\afvjm2};
( 7,  8)*{\afacdm{}{}{}{}{}{}};
( 3, 12)*{\afaidnw{}{}};
(10,  0)*{\affr88};
(14,  2)*{\aflabelleft{g(\psi)}};
( 3,-14)*{\afaiunw{}{}};
( 7, -8)*{\afacum{}{}{}{}{}{}};
( 9,-13)*{\afvjm2};
(13,-12)*{\afvjum{16}{}{g(\boldsymbol{\iota'})}};
}\quad,
\]
where $\boldsymbol{\epsilon_1}=f_1(\boldsymbol\epsilon)$, $\boldsymbol{\iota_1}=f_1(\boldsymbol\iota)$, $\boldsymbol{\epsilon_2}=f_2(\boldsymbol\epsilon)$, $\boldsymbol{\iota_3}=f_3(\boldsymbol\iota)$, $\boldsymbol{\epsilon_4}=f_4(\boldsymbol\epsilon)$ and $\boldsymbol{\iota_4}=f_4(\boldsymbol\iota)$, and for every $\epsilon$ in $\boldsymbol\epsilon$ (resp., $\iota$ in $\boldsymbol\iota$) there is a path from $\epsilon$ (resp., $\iota$) to $f_1(\epsilon)$ and $f_3(\epsilon)$ (resp., $f_1(\iota)$ and $f_2(\iota)$).
\end{definition}

%---------------------------------------
\begin{theorem}\label{theorem:SoundFourBoxes}
Reduction\/ $\to_\frfb$ is sound.
\end{theorem}

%---------------------------------------
\begin{proof}
Let $\Phi$ be a derivation with flow $\phi'$, such that $\phi'\to_\frfb\psi'$. We show that there exists a derivation $\Psi$ with flow $\psi'$ and with the same premiss and conclusion as $\Phi$. In the following, we refer to the figure in Definition~\vref{definition:FourBoxes}.

We obtain $\Psi$, with atomic flow $\psi'$, from $\Phi$ by, for every atom occurrence $a^\phi$ in $\Phi$:
\begin{enumerate}
\item If $a^\phi$ is not in the premiss or conclusion of $\Phi$, substitute it with the formula
\[
 \vls([a^{f_1(\phi)}.a^{f_2(\phi)}].[a^{f_3(\phi)}.a^{f_4(\phi)}])\quad.
\]
\item If $a^\phi$ is in the premiss or conclusion of $\Phi$, substitute it with the derivation
\[
\vlinf{}{}
{
 \vls(
  [
   a^{f_1(\phi)}
  \;.\;
   \vlinf{}{}{a^{f_2(\phi)}}{\fff}
  ]
 \;.\;
  [
   a^{f_3(\phi)}
  \;.\;
   \vlinf{}{}{a^{f_4(\phi)}}{\fff}
  ]
 )  
}
{a}
\quad\mbox{or}\quad
\vlinf{}{}
{a}
{
 \vls
 (
  [a^{f_1(\phi)}.a^{f_2(\phi)}]
 \;.\;
  [
   \vlinf{}{}{\ttt}{a^{f_3(\phi)}}
  \;.\;
   \vlinf{}{}{\ttt}{a^{f_4(\phi)}}
  ]
 )
}
\quad,
\]
respectively.
\item Substitute rule instances $\vlinf{}{}{\vls[a^\phi.\bar a^\psi]}{\ttt}$ and $\vlinf{}{}{\fff}{\vls(a^\phi.\bar a^\psi)}$ with the derivations
\[
\vlderivation
{
 \vlin{=}{}{\vls[([\vlinf{}{}{a^{f_1(\phi)}}{\fff}\;.\;a^{f_2(\phi)}]\;.\;[\vlinf{}{}{a^{f_3(\phi)}}{\fff}\;.\;a^{f_4(\phi)}])\;.\;\vlinf{}{}{\bar a^\psi}{\vls[\bar a.\bar a]}]}
 {
  \vlin{\swi}{}{\vls[\vlinf{\swi}{}{\vls\vlsmallbrackets[(a^{f_2(\phi)}.a^{f_4(\phi)}).\bar a]}{\vls\vlsmallbrackets(a^{f_2(\phi)}.[a^{f_4(\phi)}.\bar a])}\;.\;\bar a]}
  {
   \vlhy{\vls(\vlinf{}{}{\vls[a^{f_2(\phi)}.\bar a]}{\ttt}\;.\;\vlinf{}{}{\vls[a^{f_4(\phi)}.\bar a]}{\ttt})}
  }
 }
}\qquad\mbox{and}
\]
\[
\vlderivation
{
 \vlin{\swi}{}{\vls[\vlinf{}{}{\fff}{\vls(a^{f_3(\phi)}.\bar a)}\;.\;\vlinf{}{}{\fff}{\vls(a^{f_4(\phi)}.\bar a)}]}
 {
  \vlin{=}{}{\vls(\vlinf{\swi}{}{\vls[a^{f_3(\phi)}.(a^{f_4(\phi)}.\bar a)]}{\vls([a^{f_3(\phi)}.a^{f_4(\phi)}].\bar a)}\;.\;\bar a)}
  {
   \vlhy{\vls(([\vlinf{}{}{\ttt}{a^{f_1(\phi)}}\;.\;\vlinf{}{}{\ttt}{a^{f_2(\phi)}}]\;.\;\vlsmallbrackets[a^{f_3(\phi)}.a^{f_4(\phi)}])\;.\;\vlinf{}{}{\vls(\bar a.\bar a)}{\bar a^\psi})}
  }
 }
}\quad,
\]
respectively.
\item Substitute rule instances $\acd$, $\acu$, $\awd$, $\awu$ which apply to $a^\phi$ with $\cod$, $\cou$, $\wed$, $\weu$, respectively, applied to $\vls\vlsmallbrackets([a^{f_1(\phi)}.a^{\phi_2}].[a^{f_3(\phi)}.a^{f_4(\phi)}])$.
\end{enumerate}
\end{proof}

\begin{lemma}\label{lemma:FourBoxesSimpleForm}
Given two atomic flows $\phi$ and $\psi$ and a polarity assignment $\pi$, such that $\phi\to_\frfb\psi$, then $\psi$ is on simple form with respect to $\pi$.
\end{lemma}

\begin{proof}
By Definition~\vref{definition:FlowNormalForms} and Definition~\vref{definition:FourBoxes}.
\end{proof}

\newcommand{\Simpl}{\mathsf{Simpl}}
\begin{definition}\label{definition:DerSimpleForm}
Given a derivation $\Phi$ with atomic flow $\phi$ and a polarity assignment $\pi$, let $\Psi$ with atomic flow $\psi$ be the derivation obtained in the proof of Theorem~\ref{theorem:SoundFourBoxes}, such that $\phi\to_\frfb\psi$, then we say that $\Psi$ is \emph{the simple form of $\Phi$ with respect to $\pi$}, denoted $\Simpl(\Phi,\pi)$.
\end{definition}

\begin{lemma}\label{lemma:FourBoxesStreamlining}
Given two atomic flows $\phi$ and $\psi$ with polarity assignment $\pi$, such that $\phi\to_\frfb\psi$ and $\phi$ is weakly streamlined with respect to $\bar\pi$ then $\psi$ is weakly streamlined with respect to $\bar\pi$.
\end{lemma}

\begin{proof}
By studying the atomic flows in Definition~\vref{definition:FourBoxes} we can observe that for every path from an interaction vertex to a cut vertex in $\psi$ there is a path from an interaction vertex to a cut vertex in $\phi$ with the same polarity assignment.
\end{proof}

\TODO{Check:}

\begin{remark}\label{remark:FourBoxesDestroySimpleForm}
Given an atomic flow $\phi$ and a polarity assignment $\pi$, such that $\phi$ is not weakly streamlined with respect to $\bar\pi$, then $\Simpl(\phi,\pi)$ is not on simple form with respect to $\bar\pi$.
\end{remark}

\TODO{Rephrase:}

\begin{lemma}\label{lemma:FourBoxesSize}
For any two atomic flows $\phi$ and $\psi$ such that $\phi\to_\frfb\psi$ the size of $\psi$ is at most four times the size of $\phi$.
\end{lemma}
