\newcommand{\frfb}{{\mathsf{fb}}}
%---------------------------------------
\begin{definition}\label{definition:FourBoxes}
We define the reduction $\to_\frfb$ (where $\frfb$ stands for \emph{four boxes}) as follows, for any atomic flows $\phi$ and $\psi$ with a polarity assignment $\pi$:
\[
\atomicflow{
(-8, 6)*{\afvjdm4{\boldsymbol{\epsilon}}{}};
( 0, 8)*{\afaidm{}{}{}{}{}{}};
( 8, 6)*{\afvjdm4{}{\boldsymbol{\epsilon'}}};
%-
(-5, 0)*{\affr88};
(-6, 2)*{\aflabelleft\ppl};
(-4, 2)*{\aflabelright\phi};
( 5, 0)*{\affr88};
( 4, 2)*{\aflabelleft\pmi};
( 6, 2)*{\aflabelright\psi};
%-
(-8,-6)*{\afvjum4{\boldsymbol{\iota}}{}};   
( 0,-8)*{\afaium{}{}{}{}{}{}};
( 8,-6)*{\afvjum4{}{\boldsymbol{\iota'}}};
}
\quad\to_\frfb\quad
\atomicflow{
%one
(-22,18)*{\afvjdm4{\boldsymbol{\epsilon}}{}};
(-22,10.25)*{\afacumexsq{}{}{}{}{}{}52};
(-21, 8)*{\afawdm{}{}{}{}};
(-24, 0)*{\affr88};
(-20, 2)*{\aflabelleft{f_1(\phi)}};
(-22,-10)*{\afacdmexsq{}{}{}{}{}{}52};
(-22,-18)*{\afvjum4{\boldsymbol{\iota}}{}};
(-21,-8)*{\afawum{}{}{}{}};
%two
(-15,18)*{\afawdm{}{}{\boldsymbol{\epsilon_2}}{}};
(  0,18)*{\afaidmex{}{}{}{}{}{}92};
(-12,10)*{\affr88};
( -8,12)*{\aflabelleft{f_2(\phi)}};
(-16, 1)*{\afcjrm2{10}};
( -9, 2)*{\afawum{}{}{}{}};
%three
(-16,-1)*{\afcjlm2{10}};
( -9,-2)*{\afawdm{}{}{}{}};
(-12,-10)*{\affr88};
( -8, -8)*{\aflabelleft{f_3(\phi)}};
(  0,-18)*{\afaiumex{}{}{}{}{}{}92};
(-15,-18)*{\afawum{}{}{\boldsymbol{\iota_3}}{}};
%four
(-5, 8)*{\afawdm{}{}{\boldsymbol{\epsilon_4}}{}};
( 1, 8)*{\afvjm8};
(-2, 0)*{\affr88};
( 2, 2)*{\aflabelleft{f_4(\phi)}};
( 1,-8)*{\afvjm8};
(-5,-8)*{\afawum{}{}{\boldsymbol{\iota_4}}{}};
%psi
(13, 12)*{\afvjdm{16}{}{\boldsymbol{\epsilon'}}};
( 9, 13)*{\afvjm2};
( 7,  8)*{\afacdm{}{}{}{}{}{}};
( 3, 12)*{\afaidnw{}{}};
(10,  0)*{\affr88};
(14,  2)*{\aflabelleft{g(\psi)}};
( 3,-14)*{\afaiunw{}{}};
( 7, -8)*{\afacum{}{}{}{}{}{}};
( 9,-13)*{\afvjm2};
(13,-12)*{\afvjum{16}{}{\boldsymbol{\iota'}}};
}\quad,
\]
where $\boldsymbol{\epsilon_2}=f_2(\boldsymbol\epsilon)$, $\boldsymbol{\iota_3}=f_3(\boldsymbol\iota)$, $\boldsymbol{\epsilon_4}=f_4(\boldsymbol\epsilon)$ and $\boldsymbol{\iota_4}=f_4(\boldsymbol\iota)$, $\boldsymbol{\epsilon'}=g(\boldsymbol{\epsilon'})$, $\boldsymbol{\iota'}=g(\boldsymbol{\iota'})$, and for every $\epsilon$ in $\boldsymbol\epsilon$ (resp., $\iota$ in $\boldsymbol\iota$) there are paths from $\epsilon$ (resp., $\iota$) to $f_1(\epsilon)$ and $f_3(\epsilon)$ (resp., $f_1(\iota)$ and $f_2(\iota)$).
\end{definition}

%---------------------------------------
\begin{theorem}\label{theorem:SoundFourBoxes}
Reduction\/ $\to_\frfb$ is sound.
\end{theorem}

%---------------------------------------
\begin{proof}
Let $\Phi$ be a derivation with flow $\phi'$, such that $\phi'\to_\frfb\psi'$. We show that there exists a derivation $\Psi$ with flow $\psi'$ and with the same premiss and conclusion as $\Phi$. In the following, we refer to the figure in Definition~\vref{definition:FourBoxes}.

Let $a_1^{\phi}$, $\dots$, $a_n^{\phi}$ be all the atoms whose occurrences map to $\phi$, and let
\[
\vlder{\Phi'}{\{\aid,\aiu\}}
{
 \vlsbr[\beta\;.\;\vlinf{}{}{\fff}{\vlsmallbrackets\vls(a_n^\phi.\bar a_n^\psi)}\;.\;\cdots\;.\;\vlinf{}{}{\fff}{\vlsmallbrackets\vls(a_1^\phi.\bar a_1^\psi)}]
}
{
 \vlsbr(\vlinf{}{}{\vlsmallbrackets\vls[a_1^\phi.\bar a_1^\psi]}{\ttt}\;.\;\cdots\;.\;\vlinf{}{}{\vlsmallbrackets\vls[a_n^\phi.\bar a_n^\psi]}{\ttt}\;.\;\alpha)
}\quad,
\]
be the $\ai$-decomposed form of $\Phi$.

Consider the substitution
\[
\sigma\equiv\{a_1^\phi/\vlsmallbrackets\vlsbr([a_1^{f_1(\phi)}.a_1^{f_2(\phi)}].[a_1^{f_3(\phi)}.a_1^{f_4(\phi)}]),\dots,a_n^\phi/\vlsmallbrackets\vlsbr([a_n^{f_1(\phi)}.a_n^{f_2(\phi)}].[a_n^{f_3(\phi)}.a_n^{f_4(\phi)}])\}\;.
\]
We can then obtain, by Definition~\vref{TODO}, the derivation $\Phi'\sigma$ with atomic flow
\[
\atomicflow
{
(0,0)="A";
"A"+(-3, 6)*{\afvjdm4{f_1(\boldsymbol\epsilon)}{}};
"A"+( 3, 6)*{\afvjm4};
"A"+( 0, 0)*{\affr88};
"A"+(-1, 2)*{\aflabelright{f_1(\phi)}};
"A"+(-3,-6)*{\afvjum4{f_1(\boldsymbol\iota)}{}};
"A"+( 3,-6)*{\afvjm4};
%---
"A"+(14, 0)="A";
"A"+(-3, 6)*{\afvjdm4{f_2(\boldsymbol\epsilon)}{}};
"A"+( 3, 6)*{\afvjm4};
"A"+( 0, 0)*{\affr88};
"A"+(-1, 2)*{\aflabelright{f_2(\phi)}};
"A"+(-3,-6)*{\afvjum4{f_2(\boldsymbol\iota)}{}};
"A"+( 3,-6)*{\afvjm4};
%---
"A"+(14, 0)="A";
"A"+(-3, 6)*{\afvjdm4{f_3(\boldsymbol\epsilon)}{}};
"A"+( 3, 6)*{\afvjm4};
"A"+( 0, 0)*{\affr88};
"A"+(-1, 2)*{\aflabelright{f_3(\phi)}};
"A"+(-3,-6)*{\afvjum4{f_3(\boldsymbol\iota)}{}};
"A"+( 3,-6)*{\afvjm4};
%---
"A"+(14, 0)="A";
"A"+(-3, 6)*{\afvjdm4{f_4(\boldsymbol\epsilon)}{}};
"A"+( 3, 6)*{\afvjm4};
"A"+( 0, 0)*{\affr88};
"A"+(-1, 2)*{\aflabelright{f_4(\phi)}};
"A"+(-3,-6)*{\afvjum4{f_4(\boldsymbol\iota)}{}};
"A"+( 3,-6)*{\afvjm4};
%---
"A"+(14, 0)="A";
"A"+(-3, 6)*{\afvjm4};
"A"+( 3, 6)*{\afvjdm4{}{g(\boldsymbol\epsilon)}};
"A"+( 0, 0)*{\affr88};
"A"+(-1, 2)*{\aflabelright{g(\phi)}};
"A"+(-3,-6)*{\afvjm4};
"A"+( 3,-6)*{\afvjum4{}{g(\boldsymbol\iota)}};
%---
}
\quad.
\]
For every $1\le i\le n$, there exist derivations
\[
\vlinf{}{}
{
 \vls(
  [
   a_i^{f_1(\phi)}
  \;.\;
   \vlinf{}{}{a_i^{f_2(\phi)}}{\fff}
  ]
 \;.\;
  [
   a_i^{f_3(\phi)}
  \;.\;
   \vlinf{}{}{a_i^{f_4(\phi)}}{\fff}
  ]
 )  
}
{a_i}
\quad\mbox{and}\quad
\vlinf{}{}
{a_i}
{
 \vls
 (
  [a_i^{f_1(\phi)}.a_i^{f_2(\phi)}]
 \;.\;
  [
   \vlinf{}{}{\ttt}{a_i^{f_3(\phi)}}
  \;.\;
   \vlinf{}{}{\ttt}{a_i^{f_4(\phi)}}
  ]
 )
}
\quad,
\]
which allow us to build
\[
\vlder{}{}{\alpha\sigma}{\alpha}
\quad\mbox{and}\quad
\vlder{}{}{\beta}{\beta\sigma}
\quad,
\]
with atomic flows
\[
\atomicflow
{
( 0, 0)*{\afacumexsq{f_1(\boldsymbol\epsilon)}{}{f_3(\boldsymbol\epsilon)}{}{}{}52};
(12,-2)*{\afawdm{}{}{f_2(\boldsymbol\epsilon)}{}};
(20,-2)*{\afawdm{}{}{f_4(\boldsymbol\epsilon)}{}};
(24, 0)*{\afvjum{12}{}{g(\boldsymbol\epsilon)}};
}
\qquad\mbox{and}\qquad
\atomicflow
{
( 0,0)*{\afacdmexsq{f_1(\boldsymbol\iota)}{}{f_2(\boldsymbol\iota)}{}{}{}52};
(12,2)*{\afawum{}{}{f_3(\boldsymbol\iota)}{}};
(20,2)*{\afawum{}{}{f_4(\boldsymbol\iota)}{}};
(24,0)*{\afvjdm{12}{}{g(\boldsymbol\iota)}};
}
\qquad,
\]
respectively.
Furthermore, for every $1\le i\le n$, there exist derivations
\[
\vlderivation
{
 \vlin{=}{}{\vls[([\vlinf{}{}{a_i^{f_1(\phi)}}{\fff}\;.\;a_i^{f_2(\phi)}]\;.\;[\vlinf{}{}{a_i^{f_3(\phi)}}{\fff}\;.\;a_i^{f_4(\phi)}])\;.\;\vlinf{}{}{\bar a_i^\psi}{\vls[\bar a_i.\bar a_i]}]}
 {
  \vlin{\swi}{}{\vls[\vlinf{\swi}{}{\vls\vlsmallbrackets[(a_i^{f_2(\phi)}.a_i^{f_4(\phi)}).\bar a_i]}{\vls\vlsmallbrackets(a_i^{f_2(\phi)}.[a_i^{f_4(\phi)}.\bar a_i])}\;.\;\bar a_i]}
  {
   \vlhy{\vls(\vlinf{}{}{\vls[a_i^{f_2(\phi)}.\bar a_i]}{\ttt}\;.\;\vlinf{}{}{\vls[a_i^{f_4(\phi)}.\bar a_i]}{\ttt})}
  }
 }
}
\]
and
\[
\vlderivation
{
 \vlin{\swi}{}{\vls[\vlinf{}{}{\fff}{\vls(a_i^{f_3(\phi)}.\bar a_i)}\;.\;\vlinf{}{}{\fff}{\vls(a_i^{f_4(\phi)}.\bar a_i)}]}
 {
  \vlin{=}{}{\vls(\vlinf{\swi}{}{\vls[a_i^{f_3(\phi)}.(a_i^{f_4(\phi)}.\bar a_i)]}{\vls([a_i^{f_3(\phi)}.a_i^{f_4(\phi)}].\bar a_i)}\;.\;\bar a_i)}
  {
   \vlhy{\vls(([\vlinf{}{}{\ttt}{a_i^{f_1(\phi)}}\;.\;\vlinf{}{}{\ttt}{a_i^{f_2(\phi)}}]\;.\;\vlsmallbrackets[a_i^{f_3(\phi)}.a_i^{f_4(\phi)}])\;.\;\vlinf{}{}{\vls(\bar a_i.\bar a_i)}{\bar a_i^\psi})}
  }
 }
}\quad,
\]
which allows us to build
\[
\vlder{}{}
{
 \vlsmallbrackets\vlsbr([a^\phi_1.\bar a^\psi_1].\cdots.[a^\phi_n.\bar a^\psi_n])\sigma
}
{
 \ttt
}
\quad\mbox{and}\quad
\vlder{}{}
{
 \fff
}
{
 \vlsmallbrackets\vlsbr([a^\phi_n.\bar a^\psi_n].\cdots.[a^\phi_1.\bar a^\psi_1])\sigma
}
\quad,
\]
with atomic flows
\[
\quad,
\]
respectively.

Combining these derivations we can build
\[
\vlderivation
{
 \vlde{}{}
 {
  \beta
 }
 {
  \vlde{\Phi'\sigma}{}
  {
   \vlsbr[\beta.(a^\phi_n.\bar a^\psi_n).\cdots.(a^\phi_1.\bar a^\psi_1)]\sigma
  }
  {
   \vlde{}{}
   {
    \vlsbr([a^\phi_1.\bar a^\phi_1].\cdots.[a^\phi_n.\bar a^\phi_n].\alpha)\sigma
   }
   {
    \vlhy
    {
     \alpha
    }
   }
  }
 }
}\quad,
\]
with the desired atomic flow.
\end{proof}

\begin{lemma}\label{lemma:FourBoxesSimpleForm}
Given two atomic flows $\phi$ and $\psi$ and a polarity assignment $\pi$, such that $\phi\to_\frfb\psi$, then $\psi$ is on simple form with respect to $\pi$.
\end{lemma}

\begin{proof}
By Definition~\vref{definition:FlowNormalForms} and Definition~\vref{definition:FourBoxes}.
\end{proof}

\newcommand{\Simpl}{\mathsf{Simpl}}
\begin{definition}\label{definition:DerSimpleForm}
Given a derivation $\Phi$ with atomic flow $\phi$ and a polarity assignment $\pi$, let $\Psi$ with atomic flow $\psi$ be the derivation obtained in the proof of Theorem~\ref{theorem:SoundFourBoxes}, such that $\phi\to_\frfb\psi$, then we say that $\Psi$ is \emph{the simple form of $\Phi$ with respect to $\pi$}, denoted $\Simpl(\Phi,\pi)$.
\end{definition}

\begin{lemma}\label{lemma:FourBoxesStreamlining}
Given two atomic flows $\phi$ and $\psi$ with polarity assignment $\pi$, such that $\phi\to_\frfb\psi$ and $\phi$ is weakly streamlined with respect to $\bar\pi$ then $\psi$ is weakly streamlined with respect to $\bar\pi$.
\end{lemma}

\begin{proof}
By studying the atomic flows in Definition~\vref{definition:FourBoxes} we can observe that for every path from an interaction vertex to a cut vertex in $\psi$ there is a path from an interaction vertex to a cut vertex in $\phi$ with the same polarity assignment.
\end{proof}

\TODO{Check:}

\begin{remark}\label{remark:FourBoxesDestroySimpleForm}
Given an atomic flow $\phi$ and a polarity assignment $\pi$, such that $\phi$ is not weakly streamlined with respect to $\bar\pi$, then $\Simpl(\phi,\pi)$ is not on simple form with respect to $\bar\pi$.
\end{remark}

\TODO{Rephrase:}

\begin{lemma}\label{lemma:FourBoxesSize}
For any two atomic flows $\phi$ and $\psi$ such that $\phi\to_\frfb\psi$ the size of $\psi$ is at most four times the size of $\phi$.
\end{lemma}
