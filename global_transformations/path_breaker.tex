\TODO{Import from AFII when the paper is final.}

\TODO{Consider changing into the style of simple flow remover above: first define a reduction on atomic flows then show it is sound. It fits better with the message of the thesis.}

\newcommand{\Break}{\mathsf{Break}}
\begin{definition}\label{DefPathBreak}
The \emph{path breaker}, $\Break$, is an operator whose arguments are a derivation $\vlder{\Phi}{}{\beta}{\alpha}$ and the atom occurrences $a^\lambda$ and $a^\mu$, such that $\alpha=\vlsmallbrackets\vls([a^\lambda.\bar a].\gamma)$, $\beta=\vls[\delta.(a^\mu.\bar a)]$, and the results are a derivation and two atom occurrences $a^\epsilon$ and $a^\iota$ such that $\Break(\Phi,a^\lambda,a^\mu)=(\Psi,a^\epsilon,a^\iota)$, where:
\newbox\DeltaTopK
\setbox\DeltaTopK=
\hbox{$
\vlderivation
{
 \vlin{=}{}{\vls[\delta.(\vlinf{}{}{\ttt}{a^{\mu'}}.\bar a)]}
 {
  \vlde{\Phi}{}{\beta}
  {
   \vlhy{\alpha}
  }
 }
}$
}
\newbox\DeltaK
\setbox\DeltaK=
\hbox{$
\vlderivation
{
 \vlin{=}{}{\vls[\delta.(a^{\epsilon'}.\vlinf{}{}{\ttt}{\bar a})]}
 {
  \vlde{\Phi}{}{\beta}
  {
   \vlin{=}{}{\alpha}
   {
    \vlhy{\vls([\vlinf{}{}{a^{\lambda'}}{\fff}.\bar a].\gamma)}
   }
  }
 }
}$
}
\newbox\DeltaBotK
\setbox\DeltaBotK=
\hbox{$
\vlderivation
{
 \vlde{\Phi}{}{\beta}
 {
  \vlin{=}{}{\alpha}
  {
   \vlhy{\vls([a^{\epsilon'}.\vlinf{}{}{\bar a}{\fff}].\gamma)}
  }
 }
}$
}
\[
\Psi\quad=\quad
\vlderivation
{
 \vlin{=}{}{\vls[\vlinf{2\cdot\cod}{}{\delta}{\vls[\delta.\delta.\delta]}\;.\;(a^\iota.\bar a)]}
 {
  \vlin{=}{}{\vlsbr[\delta\;\;\;.\;\;\;\delta\;\;\;.\;\;\;\box\DeltaBotK]}
  {
   \vlin{\swi}{}{\vlsmallbrackets\vls[(\gamma.a^{\epsilon'}).[\delta.\delta]]}
   {
    \vlin{=}{}{\vlsmallbrackets\vls(\gamma.[a^{\epsilon'}.[\delta.\delta]])}
    {
     \vlin{=}{}{\vlsbr([\delta\;\;\;\;\;.\;\;\;\;\;\box\DeltaK]\;\;\;\;\;.\;\;\;\;\;\gamma)}
     {
      \vlin{=}{}{\vls(\vlinf{\swi}{}{\vls[(\gamma.\bar a).\delta]}{\vls(\gamma.[\bar a.\delta])}\;.\;\gamma)}
      {
       \vlin{=}{}{\vlsbr(\box\DeltaTopK\;\;\;.\;\;\;\gamma\;\;\;.\;\;\;\gamma)}
       {
        \vlhy{\vls([a^\epsilon.\bar a]\;.\;\vlinf{2\cdot\cou}{}{\vls(\gamma.\gamma.\gamma)}{\gamma})}
       }
      }
     }
    }
   }
  }
 } 
}\quad,
\]
where $\epsilon$, $\lambda'$ and $\epsilon'$ correspond to $\lambda$ and $\mu'$, $\epsilon'$ and $\iota$ correspond to $\mu$ in the atomic flow of $\Phi$.
\end{definition}

\begin{proposition}\label{PropPathBreak}
If, for some flows $\phi$ and $\psi$, the atomic flow of $\vlder{\Phi}{}{\vlsmallbrackets\vls[\beta.(a^{\mu}.\bar a^{\mu'})]}{\vlsmallbrackets\vls([a^{\lambda}.\bar a^{\lambda'}].\alpha)}$ is of shape
\[
\atomicflow
{
(-8, 6)*{\afvjm{4}};
(-2, 6)*{\afvju{4}{\lambda}{}};
( 2, 6)*{\afvju{4}{}{\lambda'}};
( 8, 6)*{\afvjm{4}};
(-5, 0)*{\affr{8}{8}};
(-4, 2)*{\aflabelright\phi};
%---
( 5, 0)*{\affr{8}{8}};
( 6, 2)*{\aflabelright{\psi}};
( 8,-6)*{\afvjm{4}};
(-2,-6)*{\afvjd{4}{\mu}{}};
( 2,-6)*{\afvjd{4}{}{\mu'}};
(-8,-6)*{\afvjm{4}};
}\quad,
\]
and if\/ $\Break(\Phi,a^\lambda,a^\mu)=\left(\vlder{\Psi}{}{\beta}{\alpha},a^\epsilon,a^\iota\right)$, where $\alpha=\vlsmallbrackets\vls([a^\epsilon.\bar a^{\epsilon'}].\gamma)$ and $\beta=\vlsmallbrackets\vls[\delta.(a^\iota.\bar a^{\iota'})]$, then
\begin{itemize}
	\item there are no paths from $\epsilon$ to $\iota$ and no paths from $\epsilon'$ to $\iota'$ in the atomic flow of\/ $\Psi$; and
	\item there are no paths from an upper path $\epsilon''$ to a lower path $\iota''$ in the atomic flow of $\Psi$ unless there is a path from the corresponding upper path $\lambda''$ to the lower path $\mu''$ in the atomic flow of $\Phi$.
\end{itemize}
\end{proposition}

\begin{figure}
\[
\atomicflow
{
(-8, 6)*{\afvjm{4}};
(-2, 6)*{\afvju{4}{\lambda}{}};
( 2, 6)*{\afvju{4}{}{\lambda'}};
( 8, 6)*{\afvjm{4}};
(-5, 0)*{\affr{8}{8}};
(-4, 2)*{\aflabelright\phi};
%---
( 5, 0)*{\affr{8}{8}};
( 6, 2)*{\aflabelright{\psi}};
( 8,-6)*{\afvjm{4}};
(-2,-6)*{\afvjd{4}{\mu}{}};
( 2,-6)*{\afvjd{4}{}{\mu'}};
(-8,-6)*{\afvjm{4}};
}\qquad\qquad
\atomicflow
{
%%%%% RED %%%%%
(0,-20)="D";
(0,-10)="Dhalf";
%% contractions
"D"+"D"="A";
%left
"A"+(-14,-15.5)-"D"*{\afvjmcol{23}{Red}};
"A"+(-11,-16.5)*{\afvjmcol{3}{Red}};
%right
"A"+(11,-11.5)-"Dhalf"*{\afvjmcol{11}{Red}};
"A"+(14,-15.5)-"D"*{\afvjmcol{23}{Red}};
"A"+(11,-16.5)*{\afvjmcol{3}{Red}};
% top boxes
(0,0)="A";
"A"+(-11,-14)*{\afcjrmcol{6}{20}{Red}};
"A"+(11,-14)*{\afcjlmcol{6}{20}{Red}};
"A"+(-2, 11)*{\afvjdcol{14}{\epsilon}{}{Red}};
"A"+( 2, 11)*{\afvjdcol{14}{}{\epsilon'}{Red}};
"A"+(-2, -8)*{\afawucol{}{}{\mu}{}{}{Red}};
% join one
"A"+(2,-7)*{\afvjdcol6{}{\mu'}{Red}};
"A"+(2,-13)*{\afvjucol6{}{\lambda'}{Red}};
% middle boxes
"A"+"D"="A";
"A"+(9.5,-10)*{\afcjlmcol{3}{12}{Red}};
"A"+( 2,-8)*{\afawucol{}{}{}{\mu'}{}{Red}};
%%%%% GREEN %%%%%
%% cocontractions
(0,0)="A";
%left
"A"+(-11,16.5)*{\afvjmcol{3}{OliveGreen}};
"A"+"D"+(-14,15.5)*{\afvjmcol{23}{OliveGreen}};
"A"+"Dhalf"+(-11,11.5)*{\afvjmcol{11}{OliveGreen}};
%right
"A"+(11,16.5)*{\afvjmcol{3}{OliveGreen}};
"A"+"D"+(14,15.5)*{\afvjmcol{23}{OliveGreen}};
% middle boxes
"A"+"D"="A";
"A"+(-9.5,10)*{\afcjlmcol{3}{12}{OliveGreen}};
"A"+(-2, 8)*{\afawdcol{}{}{\lambda}{}{}{OliveGreen}};
% join two
"A"+(-2, -7)*{\afvjdcol6{\mu}{}{OliveGreen}};
"A"+(-2,-13)*{\afvjucol6{\lambda}{}{OliveGreen}};
% bottom boxes
"A"+"D"="A";
"A"+(-11,14)*{\afcjlmcol{6}{20}{OliveGreen}};
"A"+(11,14)*{\afcjrmcol{6}{20}{OliveGreen}};
"A"+(-2,-11)*{\afvjucol{14}{\iota}{}{OliveGreen}};
"A"+( 2,-11)*{\afvjucol{14}{}{\iota'}{OliveGreen}};
"A"+( 2, 8)*{\afawdcol{}{}{}{\lambda'}{}{OliveGreen}};
%%%%% BLACK %%%%%
%% cocontractions
(0,0)="A";
%left
"A"+(-8,5.5)*{\afvjm3};
"A"+(-11,11)*{\affr88};
"A"+(-11,11)*{\copy\contrup};
%right
"A"+(8,5.5)*{\afvjm3};
"A"+"Dhalf"+(11,11.5)*{\afvjm{11}};
"A"+(11,11)*{\affr88};
"A"+(11,11)*{\copy\contrup};
%% contractions
"D"+"D"="A";
%left
"A"+(-11,-11.5)-"Dhalf"*{\afvjm{11}};
"A"+(-8,-5.5)*{\afvjm3};
"A"+(-11,-11)*{\affr88};
"A"+(-11,-11)*{\copy\contrdown};
%right
"A"+(8,-5.5)*{\afvjm3};
"A"+(11,-11)*{\affr88};
"A"+(11,-11)*{\copy\contrdown};
% top boxes
(0,0)="A";
"A"+(-5,  0)*{\affr{8}{8}};
"A"+(-4,  2)*{\aflabelright{\phi}};
"A"+( 5,  0)*{\affr{8}{8}};
"A"+( 6,  2)*{\aflabelright{\psi}};
% middle boxes
"A"+"D"="A";
"A"+(9.5,10)*{\afcjrm{3}{12}};
"A"+(-9.5,-10)*{\afcjrm{3}{12}};
"A"+(-5, 0)*{\affr{8}{8}};
"A"+(-4, 2)*{\aflabelright{\phi}};
"A"+( 5, 0)*{\affr{8}{8}};
"A"+( 6, 2)*{\aflabelright{\psi}};
% bottom boxes
"A"+"D"="A";
"A"+(-5, 0)*{\affr{8}{8}};
"A"+(-4, 2)*{\aflabelright{\phi}};
"A"+( 5, 0)*{\affr{8}{8}};
"A"+( 6, 2)*{\aflabelright{\psi}};
}
\]
\caption{The atomic flows of $\Phi$ (left) and $\Psi$ (right) such that $\Break(\Phi,a^\lambda,a^\mu)=(\Psi,a^\epsilon,a^\iota)$.}
\label{FigFlowBreak}
\end{figure}

\begin{proof}
Refer to the atomics in Figure~\ref{FigFlowBreak}.
\begin{itemize}
	\item All the edges that might be in paths from $\epsilon$ or $\epsilon'$ are coloured in red and all the edges that might be in paths to $\iota$ or $\iota'$ are coloured in green. Since the red and the green paths do not overlap, there are no paths from $\epsilon$ to $\iota$ and no paths from $\epsilon'$ to $\iota'$; and
	\item If there is a path from an upper edge $\epsilon''$ to a lower edge $\mu''$ in the atomic flow of $\Psi$ the path must pass through one of the three copies of either $\phi$ or $\psi$, hence there must be a path from the corresponding upper edge $\lambda''$ to the corresponding lower edge $\mu''$ in the atomic flow of $\Phi$.
\end{itemize}
\end{proof}
