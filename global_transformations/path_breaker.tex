\newcommand{\frpb}{{\mathsf{pb}}}
%---------------------------------------
\begin{definition}\label{definition:PathBreaker}
We define the reduction $\to_\frpb$ (where $\frpb$ stands for \emph{path breaker}) as follows, for any atomic flows $\phi$ and $\psi$:
\[
\atomicflow
{
(-8, 7)*{\afvjdm{6}{\boldsymbol\epsilon}{}};
( 0, 8)*{\afaid{}{}{}{}{}{}};
( 8, 7)*{\afvjdm{6}{}{\boldsymbol{\epsilon'}}};
(-5, 0)*{\affr{8}{8}};
(-4, 2)*{\aflabelright\phi};
%---
( 5, 0)*{\affr{8}{8}};
( 6, 2)*{\aflabelright{\psi}};
( 8,-7)*{\afvjum{6}{}{\boldsymbol{\iota'}}};
( 0,-8)*{\afaiu{}{}{}{}{}{}};
(-8,-7)*{\afvjum{6}{\boldsymbol\iota}{}};
}
\quad\to_\frpb\quad
\atomicflow
{
%%%%% RED %%%%%
(0,-20)="D";
(0,-10)="Dhalf";
%% contractions
"D"+"D"="A";
%left
"A"+(-14,-15.5)-"D"*{\afvjmcol{23}{Red}};
"A"+(-11,-17)*{\afvjumcol{4}{\boldsymbol\iota}{}{Red}};
%right
"A"+(11,-11.5)-"Dhalf"*{\afvjmcol{11}{Red}};
"A"+(14,-15.5)-"D"*{\afvjmcol{23}{Red}};
"A"+(11,-17)*{\afvjumcol{4}{}{\boldsymbol{\iota'}}{Red}};
% top boxes
(0,0)="A";
"A"+(-11,-14)*{\afcjrmcol{6}{20}{Red}};
"A"+(11,-14)*{\afcjlmcol{6}{20}{Red}};
"A"+( 0,  8)*{\afaidcol{}{}{}{}{}{}{Red}{Red}};
"A"+(-2, -8)*{\afawucol{}{}{}{}{}{Red}};
% join one
"A"+(2,-10)*{\afvjcol{12}{Red}};
% middle boxes
"A"+"D"="A";
"A"+(9.5,-10)*{\afcjlmcol{3}{12}{Red}};
"A"+( 2,-8)*{\afawucol{}{}{}{}{}{Red}};
%%%%% GREEN %%%%%
%% cocontractions
(0,0)="A";
%left
"A"+(-11,17)*{\afvjdmcol{4}{\boldsymbol\epsilon}{}{OliveGreen}};
"A"+"D"+(-14,15.5)*{\afvjmcol{23}{OliveGreen}};
"A"+"Dhalf"+(-11,11.5)*{\afvjmcol{11}{OliveGreen}};
%right
"A"+(11,17)*{\afvjdmcol{4}{}{\boldsymbol{\epsilon'}}{OliveGreen}};
"A"+"D"+(14,15.5)*{\afvjmcol{23}{OliveGreen}};
% middle boxes
"A"+"D"="A";
"A"+(-9.5,10)*{\afcjlmcol{3}{12}{OliveGreen}};
"A"+(-2, 8)*{\afawdcol{}{}{}{}{}{OliveGreen}};
% join two
"A"+(-2,-10)*{\afvjcol{12}{OliveGreen}};
% bottom boxes
"A"+"D"="A";
"A"+(-11,14)*{\afcjlmcol{6}{20}{OliveGreen}};
"A"+(11,14)*{\afcjrmcol{6}{20}{OliveGreen}};
"A"+( 0,-8)*{\afaiucol{}{}{}{}{}{}{OliveGreen}{OliveGreen}};
"A"+( 2, 8)*{\afawdcol{}{}{}{}{}{OliveGreen}};
%%%%% BLACK %%%%%
%% cocontractions
(0,0)="A";
%left
"A"+(-8,5.5)*{\afvjm3};
"A"+(-11,11)*{\affr88};
"A"+(-11,11)*{\copy\contrup};
%right
"A"+(8,5.5)*{\afvjm3};
"A"+"Dhalf"+(11,11.5)*{\afvjm{11}};
"A"+(11,11)*{\affr88};
"A"+(11,11)*{\copy\contrup};
%% contractions
"D"+"D"="A";
%left
"A"+(-11,-11.5)-"Dhalf"*{\afvjm{11}};
"A"+(-8,-5.5)*{\afvjm3};
"A"+(-11,-11)*{\affr88};
"A"+(-11,-11)*{\copy\contrdown};
%right
"A"+(8,-5.5)*{\afvjm3};
"A"+(11,-11)*{\affr88};
"A"+(11,-11)*{\copy\contrdown};
% top boxes
(0,0)="A";
"A"+(-5,  0)*{\affr{8}{8}};
"A"+(-6,  2)*{\aflabelright{f_1(\phi)}};
"A"+( 5,  0)*{\affr{8}{8}};
"A"+( 4,  2)*{\aflabelright{g_1(\psi)}};
% middle boxes
"A"+"D"="A";
"A"+(9.5,10)*{\afcjrm{3}{12}};
"A"+(-9.5,-10)*{\afcjrm{3}{12}};
"A"+(-5, 0)*{\affr{8}{8}};
"A"+(-6, 2)*{\aflabelright{f_2(\phi)}};
"A"+( 5, 0)*{\affr{8}{8}};
"A"+( 4, 2)*{\aflabelright{g_2(\psi)}};
% bottom boxes
"A"+"D"="A";
"A"+(-5, 0)*{\affr{8}{8}};
"A"+(-6, 2)*{\aflabelright{f_3(\phi)}};
"A"+( 5, 0)*{\affr{8}{8}};
"A"+( 4, 2)*{\aflabelright{g_3(\psi)}};
}\quad,
\]
where the evidenced interaction and cut vertices belong to the same connected component.
\end{definition}

\begin{theorem}\label{theorem:PathBreakerSound}
Reduction $\to_\frpb$ is sound.
\end{theorem}

\TODO{Define `evidenced'.}

\begin{proof}
Let $\Phi$ be a derivation with flow $\phi'$, such that $\phi'\to_\frpb\psi'$. We show that there exists a derivation $\Psi$ with flow $\psi'$ and with the same premiss and conclusion as $\Phi$. In the following, we refer to the figure in Definition~\vref{definition:PathBreaker}.

Since the evidenced interaction and cut vertices belong to the same connected component, we assume, using Remark~\vref{remark:AlternativeAiDecomposedForm}, that the following derivation is an $\ai$-decomposed form of $\Phi$:
\[
\vlder{\Phi'}{}
{
 \vlsbr[\beta\;.\;\vlinf{}{}{\fff}{\vls(a^\phi.\bar a^\psi)}]
}
{
 \vlsbr(\vlinf{}{}{\vls[a^\phi.\bar a^\psi]}{\ttt}\;.\;\alpha)
}\quad,
\]
for some atom $a$ and formulae $\alpha$ and $\beta$.

We combine three copies of $\Phi'$ to obtain the desired derivation $\Psi$ with flow $\psi'$ and the same premiss and conclusion as $\Phi$:

\TODO{Make an example of this use of switches:}

\newbox\DeltaTopK
\setbox\DeltaTopK=
\hbox{$
\vlder{\Phi'}{}
{
 \vlsbr[\beta\;.\;(\vlinf{}{}{\ttt}{a^\phi}\;.\;\bar a^\psi)]
}
{
 \vlsbr(\vlinf{}{}{\vls[a^\phi.\bar a^\psi]}{\ttt}\;.\;\alpha)
}
$}
\newbox\DeltaK
\setbox\DeltaK=
\hbox{$
\vlder{\Phi'}{}
{
 \vlsbr[\beta\;.\;(a^\phi\;.\;\vlinf{}{}{\ttt}{\bar a^\psi})]
}
{
 \vlsbr([\vlinf{}{}{a^\phi}{\fff}\;.\;\bar a^\psi]\;.\;\alpha)
}
$}
\newbox\DeltaBotK
\setbox\DeltaBotK=
\hbox{$
\vlder{\Phi'}{}
{
 \vlsbr[\beta\;.\;\vlinf{}{}{\fff}{\vls(a^\phi.\bar a^\psi)}]
}
{
 \vlsbr([a^\phi\;.\;\vlinf{}{}{\bar a^\psi}{\fff}]\;.\;\alpha)
}
$}
\[
\Psi\quad=\quad
\vlderivation
{
 \vlin{\cod}{}{\beta}
 {
  \vlin{\swi}{}
  {
   \vls
   [
    \vlinf{\cod}{}{\beta}{\vls[\beta.\beta]}
   \;\;\;\;.\;\;\;\;
    \box\DeltaBotK
   ]
  }
  {
   \vlin{\swi}{}
   {
    \vls
    (
%     \vlinf{\swi}{}
%     {
%      \vls
      [
       \beta
      \;\;\;\;.\;\;\;\;
       \box\DeltaK
      ]
%     }
%     {
%      \vls(\vlsmallbrackets[\beta.\bar a^\psi].\alpha)
%     }
    \;\;\;\;\;.\;\;\;\;\;
     \alpha
    )   
   }
   {
    \vlin{\cod}{}
    {
     \vls
     (
      \box\DeltaTopK
     \;\;\;\;.\;\;\;\;
      \vlinf{\cou}{}{\vls(\alpha.\alpha)}{\alpha}
     )
    }
    {
     \vlhy{\alpha}
    }
   }
  }
 } 
}\qquad.
\]
\end{proof}

\TODO{Find a way to define `evidenced' so this makes sense...}

\begin{proposition}\label{proposition:PathBreak}
Given atomic flows $\phi$ and $\psi$, such that $\phi\to_\frpb\psi$, then, referring to the atomic flows in Definition~\vref{definition:PathBreaker},
\begin{itemize}
 \item there is no path from the evidenced interaction vertex to the evidenced cut vertex in $\psi$, and
 \item let $\epsilon$ (resp., $\iota$) refer to a pair of corresponding upper (resp., lower) edges in $\phi$ and in $\psi$, then there is a path from $\epsilon$ to $\iota$ in $\psi$ if and only if there is a path from $\epsilon$ to $\iota$ in $\phi$.
\end{itemize}
\end{proposition}

\TODO{Redo proof.}

\begin{proof}
\begin{itemize}
	\item In the atomic flow $\psi$, we have colored all the edges that might be in paths from the evidenced interaction vertex in red, and all the edges that might be in paths to the evidenced cut vertex in green. Since the red and the green paths never overlap, we know there is no path from the interaction to the cut vertex.
	\item By Proposition~\vref{TODO}, we know that for every path from $\epsilon$ to $\iota$ in $\phi'$ there are three paths from $\epsilon$ to $\iota$ in $\psi'$, and if there is a path from $\epsilon$ to $\iota$ in $\psi'$, it must pass through one of the three subflows isomorphic to $\phi$ (resp., $\psi$) so there is a path from $\epsilon$ to $\iota$ in $\phi$.
\end{itemize}
\end{proof}

\newcommand{\PB}{\mathsf{PB}}
\begin{definition}\label{definition:DerPathBreaker}
The \emph{Path Breaker}, $\PB$, is an operator whose arguments are an atom $a$, and a derivation of shape
\[
\Phi\;=\;
\vlder{\Phi'}{}
{
 \vlsbr
 [
  \beta
 \;.\;
  \vlinf{}{}
  {
   \fff
  }
  {
   \vls(a^\psi.\bar a)
  }
 \;.\;\cdots\;.\;
  \vlinf{}{}
  {
   \fff
  }
  {
   \vls(a^\psi.\bar a)
  }
 ]
}
{
 \vlsbr
 (
  \vlinf{}{}
  {
   \vls[a^\psi.\bar a]
  }
  {
   \ttt
  }
 \;.\;\cdots\;.\;
  \vlinf{}{}
  {
   \vls[a^\psi.\bar a]
  }
  {
   \ttt
  }
 \;.\;
  \alpha
 )
}\quad,
\]
with atomic flow
\[
\atomicflow
{
(-8, 7)*{\afvjdm{6}{\boldsymbol\epsilon}{}};
( 0, 8)*{\afaidm{}{}{}{}{}{}};
( 8, 7)*{\afvjdm{6}{}{\boldsymbol{\epsilon'}}};
(-5, 0)*{\affr{8}{8}};
(-4, 2)*{\aflabelright\phi};
%---
( 5, 0)*{\affr{8}{8}};
( 6, 2)*{\aflabelright{\psi}};
( 8,-7)*{\afvjum{6}{}{\boldsymbol{\iota'}}};
( 0,-8)*{\afaium{}{}{}{}{}{}};
(-8,-7)*{\afvjum{6}{\boldsymbol\iota}{}};
}
\quad,
\]
such that $a$ does not appear in an interaction or cut instance in $\Phi'$. Consider the derivation
\[
\Psi'\;=\;
\vlder{\Phi'}{}
{
 \vlsbr
 [
  \beta
 \;\;\;.\;\;\;
  \vlderivation
  {
   \vlin{}{}
   {
    \fff
   }
   {
    \vlde{}{\{\cod\}}
    {
     \vls(a.\bar a)
    }
    {
     \vlhy
     {
      \vls[(a.\bar a).\cdots.(a.\bar a)]
     }
    }
   }
  }
 ]
}
{
 \vlsbr
 (
  \vlderivation
  {
   \vlde{}{\{\cou\}}
   {
    \vls([a.\bar a].\cdots.[a.\bar a])
   }
   {
    \vlin{}{}
    {
     \vls[a.\bar a]
    }
    {
     \vlhy
     {
      \ttt
     }
    }
   }
  }
 \;\;\;.\;\;\;
  \alpha
 )
}\quad,
\]
with atomic flow
\[
\psi'\;=\;
\atomicflow{
(-8, 11)*{\afvjdm{14}{\boldsymbol\epsilon}{}};
(11, 11)*{\afvjdm{14}{}{\boldsymbol{\epsilon'}}};
( 2, 18)*{\afaidex{}{}{}{}{}{}{8}{4}};
%-
(-2, 10)*{\affr68};
(-2, 10)*{\copy\contrup};
(-2,  5)*{\afvjm2};
(-5,  0)*{\affr{8}{8}};
(-1,  2)*{\aflabelleft{\phi}};
(-2, -5)*{\afvjm2};
(-2,-10)*{\affr68};
(-2,-10)*{\copy\contrdown};
%-
( 6, 10)*{\affr68};
( 6, 10)*{\copy\contrup};
( 6,  5)*{\afvjm2};
( 8,  0)*{\affr{8}{8}};
(12,  2)*{\aflabelleft{\psi}};
( 6, -5)*{\afvjm2};
( 6,-10)*{\affr68};
( 6,-10)*{\copy\contrdown};
%-
( 2,-18)*{\afaiuex{}{}{}{}{}{}{8}{4}};
(-8,-11)*{\afvjum{14}{\boldsymbol\iota}{}};
(11,-11)*{\afvjum{14}{}{\boldsymbol{\iota'}}};
}\quad.
\]
We then define $\PB(\Phi,a)$ to be such that $\Psi'\to_\frpb\PB(\Phi,a)$.
\end{definition}

\begin{proposition}
Given an atom $a$, and a derivation
\[
\Phi\;=\;
\vlder{}{}
{
 \vlsbr
 [
  \beta
 \;.\;
  \vlinf{}{}
  {
   \fff
  }
  {
   \vls(a^\psi.\bar a)
  }
 \;.\;\cdots\;.\;
  \vlinf{}{}
  {
   \fff
  }
  {
   \vls(a^\psi.\bar a)
  }
 ]
}
{
 \vlsbr
 (
  \vlinf{}{}
  {
   \vls[a^\psi.\bar a]
  }
  {
   \ttt
  }
 \;.\;\cdots\;.\;
  \vlinf{}{}
  {
   \vls[a^\psi.\bar a]
  }
  {
   \ttt
  }
 \;.\;
  \alpha
 )
}\quad,
\]
with atomic flow
\[
\atomicflow
{
(-8, 7)*{\afvjdm{6}{\boldsymbol\epsilon}{}};
( 0, 8)*{\afaidm{}{}{}{}{}{}};
( 8, 7)*{\afvjdm{6}{}{\boldsymbol{\epsilon'}}};
(-5, 0)*{\affr{8}{8}};
(-4, 2)*{\aflabelright\phi};
%---
( 5, 0)*{\affr{8}{8}};
( 6, 2)*{\aflabelright{\psi}};
( 8,-7)*{\afvjum{6}{}{\boldsymbol{\iota'}}};
( 0,-8)*{\afaium{}{}{}{}{}{}};
(-8,-7)*{\afvjum{6}{\boldsymbol\iota}{}};
}
\quad,
\]
such that $a$ does not appear in an interaction or cut instance in $\Phi'$.
\begin{itemize}
\item $\PB(\Phi,a)$ is weakly streamlined with respect to $a$ and with respect to $\bar a$;
\item for any atom $b$, if $\Phi$ is weakly streamlined with respect to $b$, then $\PB(\Phi,a)$ is weakly streamlined with respect to $b$;
\item if $\Phi$ is on simple form with respect to $\pi$ then $\PB(\Phi,a)$ is on simple form with respect to $\pi$; and
\item $\PB(\Phi,a)$ depends at most linearly on the size of $\Phi$.
\end{itemize}
\end{proposition}

\begin{example}\label{example:PathBreaker}
Given a derivation $\Phi$ with atomic flow
\[
\phi\;\;=\;\;\atomicflow
{
(-13,0)*{\affr{22}{20}};
(-5,8)*{\aflabelright{\phi_1}};
%
(-20, 9)*{\afvjm{10}};
(-13, 8)*{\afaid{}{}{}{}{}{}};
( -6, 9)*{\afvjm{10}};
(-18, 0)*{\affr{8}{8}};
(-17, 2)*{\aflabelright{\phi'_1}};
( -8, 0)*{\affr{8}{8}};
( -7, 2)*{\aflabelright{\phi''_1}};
( -6,-9)*{\afvjm{10}};
(-13,-8)*{\afaiu{}{}{}{}{}{}};
(-20,-9)*{\afvjm{10}};
%------------
(13,0)*{\affr{22}{20}};
(21,8)*{\aflabelright{\phi_2}};
%
(20, 9)*{\afvjm{10}};
(13, 8)*{\afaid{}{}{}{}{}{}};
( 6, 9)*{\afvjm{10}};
( 8, 0)*{\affr{8}{8}};
( 9, 2)*{\aflabelright{\phi'_2}};
(18, 0)*{\affr{8}{8}};
(19, 2)*{\aflabelright{\phi''_2}};
( 6,-9)*{\afvjm{10}};
(13,-8)*{\afaiu{}{}{}{}{}{}};
(20,-9)*{\afvjm{10}};
}\qquad
\atomicflow
{
(0,9)*{\afvjm{10}};
(0,0)*{\affr88};
(2,2)*{\aflabelright{\psi}};
(0,-9)*{\afvjm{10}};
}\quad,
\]
where occurrences of $a_1$ are mapped to all the edges in $\phi'_1$, occurrences of $\bar a_1$ are mapped to all the edges in $\phi''_1$, occurrences of $a_2$ are mapped to all the edges in $\phi'_2$, and occurrences of $\bar a_2$ are mapped to all the edges in $\phi''_2$; the atomic flow of $\PB(\PB(\Phi,a_1),a_2)$ is
\[
\atomicflow
{
%cocontraction - top
(12,43.5)*{\afvjm{3}};
(12,38)*{\affr{34}8};
(12,38)*{\copy\contrup};
(-18,46)*{\afaidex{}{}{}{}{}{}31};
(-24,38)*{\affr{10}8};
(-24,38)*{\copy\contrup};
(-12,38)*{\affr{10}8};
(-12,38)*{\copy\contrup};
%contraction - bot
(-12,-43.5)*{\afvjm{3}};
(-12,-38)*{\affr{34}8};
(-12,-38)*{\copy\contrdown};
(18,-46)*{\afaiuex{}{}{}{}{}{}31};
(24,-38)*{\affr{10}8};
(24,-38)*{\copy\contrdown};
(12,-38)*{\affr{10}8};
(12,-38)*{\copy\contrdown};
%---------------------
(4,-20)="D";
(0,-10)="Dhalf";
%----------------
%%first
(-20,0)="B";
% cocontractions
"B"-"D"-"D"-(-12,7)="A";
"A"+(4,-4)*{\afcjrm{16}{10}};
"A"+(8,-4)*{\afcjrm{16}{10}};
"A"+(12,-4)*{\afcjrm{16}{10}};
"A"+"Dhalf"+"Dhalf"+(4,-9)*{\afvjm{40}};
"A"+"Dhalf"+(0,-9)*{\afvjm{20}};
"A"+(-12,-4)*{\afvj{10}};
"A"+(-4,-4)*{\afcjr{8}{10}};
% contractions
"B"+"D"+"D"+(-12,7)="A";
"A"+(4,4)*{\afvjm{10}};
"A"+(0,4)-"Dhalf"*{\afvjm{30}};
"A"+(-4,4)-"Dhalf"-"Dhalf"*{\afvjm{50}};
"A"-(-20,-4)*{\afcjl{24}{10}};
"A"-(-28,-4)*{\afcjl{32}{10}};
%---
% top boxes
"B"-"D"="A";
"A"+( 0,-8)*{\afawu{}{}{}{}{}};
"A"+( 0, 0)*{\affr{10}{8}};
"A"+( 2, 2)*{\aflabelright{\phi_1}};
% join one
"A"+(4,-10)*{\afvjcol{12}{Red}};
% middle boxes
"B"="A";
"A"+(-4, 8)*{\afawd{}{}{}{}{}};
"A"+( 4,-8)*{\afawu{}{}{}{}{}};
"A"+( 0, 0)*{\affr{10}{8}};
"A"+( 2, 2)*{\aflabelright{\phi_1}};
% join two
"A"+(0,-10)*{\afvjcol{12}{Red}};
% bottom boxes
"B"+"D"="A";
"A"+(0, 8)*{\afawd{}{}{}{}{}};
"A"+(0, 0)*{\affr{10}{8}};
"A"+( 2, 2)*{\aflabelright{\phi_1}};
%----------------
%%second
(0,0)="B";
% cocontractions
"B"-"D"-"D"-(-12,7)="A";
"A"+(0,-4)*{\afcjrm8{10}};
"A"+(4,-4)*{\afcjrm8{10}};
"A"+(8,-4)*{\afcjrm8{10}};
"A"+"Dhalf"+"Dhalf"+(4,-9)*{\afvjm{40}};
"A"+"Dhalf"+(0,-9)*{\afvjm{20}};
"A"+(-20,-4)*{\afcjl{16}{10}};
"A"+(-12,-4)*{\afcjl{8}{10}};
% contractions
"B"+"D"+"D"+(-12,7)="A";
"A"-(0,-4)*{\afcjrm8{10}};
"A"-(4,-4)*{\afcjrm8{10}};
"A"-(8,-4)*{\afcjrm8{10}};
"A"-"Dhalf"-"Dhalf"-(4,-9)*{\afvjm{40}};
"A"-"Dhalf"-(0,-9)*{\afvjm{20}};
"A"-(-20,-4)*{\afcjl{16}{10}};
"A"-(-12,-4)*{\afcjl{8}{10}};
%---
% top boxes
"B"-"D"="A";
"A"+( 0,-8)*{\afawu{}{}{}{}{}};
"A"+( 0, 0)*{\affr{10}{8}};
"A"+( 2, 2)*{\aflabelright{\phi_1}};
% join one
"A"+(4,-10)*{\afvjcol{12}{Red}};
% middle boxes
"B"="A";
"A"+(-4, 8)*{\afawd{}{}{}{}{}};
"A"+( 4,-8)*{\afawu{}{}{}{}{}};
"A"+( 0, 0)*{\affr{10}{8}};
"A"+( 2, 2)*{\aflabelright{\phi_1}};
% join two
"A"+(0,-10)*{\afvjcol{12}{Red}};
% bottom boxes
"B"+"D"="A";
"A"+(0, 8)*{\afawd{}{}{}{}{}};
"A"+(0, 0)*{\affr{10}{8}};
"A"+( 2, 2)*{\aflabelright{\phi_1}};
%----------------
%%third
(20,0)="B";
% cocontractions
"B"-"D"-"D"-(-12,7)="A";
"A"+"Dhalf"+"Dhalf"+(4,-4)*{\afvjm{50}};
"A"+"Dhalf"+(0,-4)*{\afvjm{30}};
"A"+(-4,-4)*{\afvjm{10}};
"A"+(-20,-4)*{\afcjl{24}{10}};
"A"+(-28,-4)*{\afcjl{32}{10}};
% contractions
"B"+"D"+"D"+(-12,7)="A";
"A"-(4,-4)*{\afcjrm{16}{10}};
"A"-(8,-4)*{\afcjrm{16}{10}};
"A"-(12,-4)*{\afcjrm{16}{10}};
"A"-"Dhalf"-"Dhalf"-(4,-9)*{\afvjm{40}};
"A"-"Dhalf"-(0,-9)*{\afvjm{20}};
"A"-(-12,-4)*{\afvj{10}};
"A"-(-4,-4)*{\afcjr{8}{10}};
%---
% top boxes
"B"-"D"="A";
"A"+( 0,-8)*{\afawu{}{}{}{}{}};
"A"+( 0, 0)*{\affr{10}{8}};
"A"+( 2, 2)*{\aflabelright{\phi_1}};
% join one
"A"+(4,-10)*{\afvjcol{12}{Red}};
% middle boxes
"B"="A";
"A"+(-4, 8)*{\afawd{}{}{}{}{}};
"A"+( 4,-8)*{\afawu{}{}{}{}{}};
"A"+( 0, 0)*{\affr{10}{8}};
"A"+( 2, 2)*{\aflabelright{\phi_1}};
% join two
"A"+(0,-10)*{\afvjcol{12}{Red}};
% bottom boxes
"B"+"D"="A";
"A"+(0, 8)*{\afawd{}{}{}{}{}};
"A"+(0, 0)*{\affr{10}{8}};
"A"+( 2, 2)*{\aflabelright{\phi_1}};
}
\]
\[
\atomicflow
{
(12,-52)="D";
% cocontractions
(18,-11)="A";
%
"A"+(6,5.5)*{\afvjm3};
"A"+(6,0)*{\affr{34}8};
"A"+(6,0)*{\copy\contrup};
"A"+(2,-30)*{\afvjm{52}};
"A"+(6,-30)*{\afvjm{52}};
"A"+(10,-34)*{\afvjm{60}};
"A"+(14,-56)*{\afvjm{104}};
"A"+(18,-56)*{\afvjm{104}};
"A"+(22,-60)*{\afvjm{112}};
% === BOX ONE ===
(0,-27)="A";
%-
"A"+( -6,24)*{\afaidex{}{}{}{}{}{}31};
%-
"A"+(-12,16)*{\affr{10}8};
"A"+(-12,16)*{\copy\contrup};
"A"+(  0,16)*{\affr{10}8};
"A"+(  0,16)*{\copy\contrup};
%-
"A"+(-16,8)*{\afvj8};
"A"+(-8,8)*{\afcjr88};
"A"+(0,8)*{\afcjrm{16}8};
%
"A"+(-8,8)*{\afcjl88};
"A"+(0,8)*{\afvj8};
"A"+(8,8)*{\afcjrm88};
%
"A"+(0,8)*{\afcjl{16}8};
"A"+(8,8)*{\afcjl88};
"A"+(16,8)*{\afvjm8};
%-
"A"+(-12,0)*{\affr{10}8};
"A"+(0,0)*{\affr{10}8};
"A"+(12,0)*{\affr{10}8};
"A"+(-10,2)*{\aflabelright{\phi_2}};
"A"+(2,2)*{\aflabelright{\phi_2}};
"A"+(14,2)*{\aflabelright{\phi_2}};
%-
"A"+(-8,-8)*{\afcjl88};
"A"+(0,-8)*{\afcjlcol{16}8{Red}};
%-
"A"+(-8,-8)*{\afcjrm88};
"A"+(0,-8)*{\afvj8};
"A"+(8,-8)*{\afcjlcol88{Red}};
%
"A"+(0,-8)*{\afcjrm{16}8};
"A"+(8,-8)*{\afcjr88};
"A"+(16,-8)*{\afvjcol8{Red}};
%
"A"+(  0,-16)*{\affr{10}8};
"A"+(  0,-16)*{\copy\contrdown};
"A"+( 12,-16)*{\affr{10}8};
"A"+( 12,-16)*{\copy\contrdown};
%---
"A"+(0,-24)*{\afawu{}{}{}{}};
"A"+(12,-26)*{\afvjcol{12}{Red}};
% === BOX TWO ===
"A"+"D"="A";
%-
"A"+(-12,24)*{\afawd{}{}{}{}};
%-
"A"+(-12,16)*{\affr{10}8};
"A"+(-12,16)*{\copy\contrup};
"A"+(  0,16)*{\affr{10}8};
"A"+(  0,16)*{\copy\contrup};
%-
"A"+(-16,8)*{\afvj8};
"A"+(-8,8)*{\afcjrcol88{Red}};
"A"+(0,8)*{\afcjrm{16}8};
%
"A"+(-8,8)*{\afcjl88};
"A"+(0,8)*{\afvjcol8{Red}};
"A"+(8,8)*{\afcjrm88};
%
"A"+(0,8)*{\afcjl{16}8};
"A"+(8,8)*{\afcjlcol88{Red}};
%-
"A"+(-12,0)*{\affr{10}8};
"A"+(0,0)*{\affr{10}8};
"A"+(12,0)*{\affr{10}8};
"A"+(-10,2)*{\aflabelright{\phi_2}};
"A"+(2,2)*{\aflabelright{\phi_2}};
"A"+(14,2)*{\aflabelright{\phi_2}};
%-
"A"+(-8,-8)*{\afcjlcol88{Red}};
"A"+(0,-8)*{\afcjl{16}8};
%-
"A"+(-8,-8)*{\afcjrm88};
"A"+(0,-8)*{\afvjcol8{Red}};
"A"+(8,-8)*{\afcjl88};
%
"A"+(0,-8)*{\afcjrm{16}8};
"A"+(8,-8)*{\afcjrcol88{Red}};
"A"+(16,-8)*{\afvj8};
%
"A"+(  0,-16)*{\affr{10}8};
"A"+(  0,-16)*{\copy\contrdown};
"A"+( 12,-16)*{\affr{10}8};
"A"+( 12,-16)*{\copy\contrdown};
%---
"A"+(12,-24)*{\afawu{}{}{}{}};
"A"+(0,-26)*{\afvjcol{12}{Red}};
% === BOX THREE ===
"A"+"D"="A";
%-
"A"+(  0,24)*{\afawd{}{}{}{}};
%-
"A"+(-12,16)*{\affr{10}8};
"A"+(-12,16)*{\copy\contrup};
"A"+(  0,16)*{\affr{10}8};
"A"+(  0,16)*{\copy\contrup};
%-
"A"+(-16,8)*{\afvjcol8{Red}};
"A"+(-8,8)*{\afcjr88};
"A"+(0,8)*{\afcjrm{16}8};
%
"A"+(-8,8)*{\afcjlcol88{Red}};
"A"+(0,8)*{\afvj8};
"A"+(8,8)*{\afcjrm88};
%
"A"+(0,8)*{\afcjlcol{16}8{Red}};
"A"+(8,8)*{\afcjl88};
%-
"A"+(-12,0)*{\affr{10}8};
"A"+(0,0)*{\affr{10}8};
"A"+(12,0)*{\affr{10}8};
"A"+(-10,2)*{\aflabelright{\phi_2}};
"A"+(2,2)*{\aflabelright{\phi_2}};
"A"+(14,2)*{\aflabelright{\phi_2}};
%-
"A"+(-16,-8)*{\afvjm8};
"A"+(-8,-8)*{\afcjl88};
"A"+(0,-8)*{\afcjl{16}8};
%-
"A"+(-8,-8)*{\afcjrm88};
"A"+(8,-8)*{\afcjl88};
"A"+(0,-8)*{\afvj8};
%
"A"+(0,-8)*{\afcjrm{16}8};
"A"+(8,-8)*{\afcjr88};
"A"+(16,-8)*{\afvj8};
%
"A"+(  0,-16)*{\affr{10}8};
"A"+(  0,-16)*{\copy\contrdown};
"A"+( 12,-16)*{\affr{10}8};
"A"+( 12,-16)*{\copy\contrdown};
%-
"A"+(  6,-24)*{\afaiuex{}{}{}{}{}{}31};
% contractions
"A"+(-20,-16)="A";
"A"+(-20,60)*{\afvjm{112}};
"A"+(-16,56)*{\afvjm{104}};
"A"+(-12,56)*{\afvjm{104}};
"A"+(-8,34)*{\afvjm{60}};
"A"+(-4,30)*{\afvjm{52}};
"A"+( 0,30)*{\afvjm{52}};
"A"+(-4,0)*{\affr{34}8};
"A"+(-4,0)*{\copy\contrdown};
"A"+(-4,-5.5)*{\afvjm3};
}
\]
\[
\atomicflow{
(0,34.5)*{\afvjm3};
(0,29)*{\affr{82}8};
(0,29)*{\copy\contrup};
%
(0,-29)*{\affr{82}8};
(0,-29)*{\copy\contrdown};
(0,-34.5)*{\afvjm3};
%-------------------
(30,0)="B";
%---------------
(0,0)-"B"="A";
"A"+(-10,14.5)*{\afvjm{21}};
"A"+(0,14.5)*{\afvjm{21}};
"A"+(10,14.5)*{\afvjm{21}};
%---
"A"+(-10,0)*{\affr88};
"A"+( -9,2)*{\aflabelright\psi};
"A"+(  0,0)*{\affr88};
"A"+(  1,2)*{\aflabelright\psi};
"A"+( 10,0)*{\affr88};
"A"+( 11,2)*{\aflabelright\psi};
%---
"A"+(-10,-14.5)*{\afvjm{21}};
"A"+(0,-14.5)*{\afvjm{21}};
"A"+(10,-14.5)*{\afvjm{21}};
%---------------
(0,0)="A";
%---
"A"+(-10,14.5)*{\afvjm{21}};
"A"+(0,14.5)*{\afvjm{21}};
"A"+(10,14.5)*{\afvjm{21}};
%---
"A"+(-10,0)*{\affr88};
"A"+( -9,2)*{\aflabelright\psi};
"A"+(  0,0)*{\affr88};
"A"+(  1,2)*{\aflabelright\psi};
"A"+( 10,0)*{\affr88};
"A"+( 11,2)*{\aflabelright\psi};
%---
"A"+(-10,-14.5)*{\afvjm{21}};
"A"+(0,-14.5)*{\afvjm{21}};
"A"+(10,-14.5)*{\afvjm{21}};
%---------------
"A"+"B"="A";
%---
"A"+(-10,14.5)*{\afvjm{21}};
"A"+(0,14.5)*{\afvjm{21}};
"A"+(10,14.5)*{\afvjm{21}};
%---
"A"+(-10,0)*{\affr88};
"A"+( -9,2)*{\aflabelright\psi};
"A"+(  0,0)*{\affr88};
"A"+(  1,2)*{\aflabelright\psi};
"A"+( 10,0)*{\affr88};
"A"+( 11,2)*{\aflabelright\psi};
%---
"A"+(-10,-14.5)*{\afvjm{21}};
"A"+(0,-14.5)*{\afvjm{21}};
"A"+(10,-14.5)*{\afvjm{21}};
}\quad.
\]
By comparing the copies of $\phi_1$ with the copies of $\phi_2$ in $\PB(\PB(\Phi,a_1),a_2)$ we can see why the procedure is non-confluent. We could trivially change our procedure to make the subflows more similar by permuting the weakening (resp., coweakening) vertices down (resp., up) in the second subflow. However, the subflows marked in red will still differ.
\end{example}