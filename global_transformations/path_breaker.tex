\newcommand{\frpb}{{\mathsf{pb}}}
%---------------------------------------
\begin{definition}\label{definition:PathBreaker}
We define the reduction $\to_\frpb$ (where $\frpb$ stands for \emph{path breaker}) as follows, for any atomic flows $\phi$ and $\psi$:
\[
\atomicflow
{
(-8, 7)*{\afvjdm{6}{\boldsymbol\epsilon}{}};
( 0, 8)*{\afaid{}{}{}{}{}{}};
( 8, 7)*{\afvjdm{6}{}{\boldsymbol{\epsilon'}}};
(-5, 0)*{\affr{8}{8}};
(-4, 2)*{\aflabelright\phi};
%---
( 5, 0)*{\affr{8}{8}};
( 6, 2)*{\aflabelright{\psi}};
( 8,-7)*{\afvjum{6}{}{\boldsymbol{\iota'}}};
( 0,-8)*{\afaiu{}{}{}{}{}{}};
(-8,-7)*{\afvjum{6}{\boldsymbol\iota}{}};
}
\quad\to_\frpb\quad
\atomicflow
{
%%%%% RED %%%%%
(0,-20)="D";
(0,-10)="Dhalf";
%% contractions
"D"+"D"="A";
%left
"A"+(-14,-15.5)-"D"*{\afvjmcol{23}{Red}};
"A"+(-11,-17)*{\afvjumcol{4}{\boldsymbol\iota}{}{Red}};
%right
"A"+(11,-11.5)-"Dhalf"*{\afvjmcol{11}{Red}};
"A"+(14,-15.5)-"D"*{\afvjmcol{23}{Red}};
"A"+(11,-17)*{\afvjumcol{4}{}{\boldsymbol{\iota'}}{Red}};
% top boxes
(0,0)="A";
"A"+(-11,-14)*{\afcjrmcol{6}{20}{Red}};
"A"+(11,-14)*{\afcjlmcol{6}{20}{Red}};
"A"+( 0,  8)*{\afaidcol{}{}{}{}{}{}{Red}{Red}};
"A"+(-2, -8)*{\afawucol{}{}{}{}{}{Red}};
% join one
"A"+(2,-10)*{\afvjcol{12}{Red}};
% middle boxes
"A"+"D"="A";
"A"+(9.5,-10)*{\afcjlmcol{3}{12}{Red}};
"A"+( 2,-8)*{\afawucol{}{}{}{}{}{Red}};
%%%%% GREEN %%%%%
%% cocontractions
(0,0)="A";
%left
"A"+(-11,17)*{\afvjdmcol{4}{\boldsymbol\epsilon}{}{OliveGreen}};
"A"+"D"+(-14,15.5)*{\afvjmcol{23}{OliveGreen}};
"A"+"Dhalf"+(-11,11.5)*{\afvjmcol{11}{OliveGreen}};
%right
"A"+(11,17)*{\afvjdmcol{4}{}{\boldsymbol{\epsilon'}}{OliveGreen}};
"A"+"D"+(14,15.5)*{\afvjmcol{23}{OliveGreen}};
% middle boxes
"A"+"D"="A";
"A"+(-9.5,10)*{\afcjlmcol{3}{12}{OliveGreen}};
"A"+(-2, 8)*{\afawdcol{}{}{}{}{}{OliveGreen}};
% join two
"A"+(-2,-10)*{\afvjcol{12}{OliveGreen}};
% bottom boxes
"A"+"D"="A";
"A"+(-11,14)*{\afcjlmcol{6}{20}{OliveGreen}};
"A"+(11,14)*{\afcjrmcol{6}{20}{OliveGreen}};
"A"+( 0,-8)*{\afaiucol{}{}{}{}{}{}{OliveGreen}{OliveGreen}};
"A"+( 2, 8)*{\afawdcol{}{}{}{}{}{OliveGreen}};
%%%%% BLACK %%%%%
%% cocontractions
(0,0)="A";
%left
"A"+(-8,5.5)*{\afvjm3};
"A"+(-11,11)*{\affr88};
"A"+(-11,11)*{\copy\contrup};
%right
"A"+(8,5.5)*{\afvjm3};
"A"+"Dhalf"+(11,11.5)*{\afvjm{11}};
"A"+(11,11)*{\affr88};
"A"+(11,11)*{\copy\contrup};
%% contractions
"D"+"D"="A";
%left
"A"+(-11,-11.5)-"Dhalf"*{\afvjm{11}};
"A"+(-8,-5.5)*{\afvjm3};
"A"+(-11,-11)*{\affr88};
"A"+(-11,-11)*{\copy\contrdown};
%right
"A"+(8,-5.5)*{\afvjm3};
"A"+(11,-11)*{\affr88};
"A"+(11,-11)*{\copy\contrdown};
% top boxes
(0,0)="A";
"A"+(-5,  0)*{\affr{8}{8}};
"A"+(-4,  2)*{\aflabelright{\phi}};
"A"+( 5,  0)*{\affr{8}{8}};
"A"+( 6,  2)*{\aflabelright{\psi}};
% middle boxes
"A"+"D"="A";
"A"+(9.5,10)*{\afcjrm{3}{12}};
"A"+(-9.5,-10)*{\afcjrm{3}{12}};
"A"+(-5, 0)*{\affr{8}{8}};
"A"+(-4, 2)*{\aflabelright{\phi}};
"A"+( 5, 0)*{\affr{8}{8}};
"A"+( 6, 2)*{\aflabelright{\psi}};
% bottom boxes
"A"+"D"="A";
"A"+(-5, 0)*{\affr{8}{8}};
"A"+(-4, 2)*{\aflabelright{\phi}};
"A"+( 5, 0)*{\affr{8}{8}};
"A"+( 6, 2)*{\aflabelright{\psi}};
}\quad,
\]
where the evidenced interaction and cut vertices belong to the same connected component.
\end{definition}

\begin{theorem}\label{theorem:PathBreakerSound}
Reduction $\to_\frpb$ is sound.
\end{theorem}

\TODO{Define `evidenced'.}

\begin{proof}
Let $\Phi$ be a derivation with flow $\phi'$, such that $\phi'\to_\frpb\psi'$. We show that there exists a derivation $\Psi$ with flow $\psi'$ and with the same premiss and conclusion as $\Phi$. In the following, we refer to the figure in Definition~\vref{definition:PathBreaker}.

Since the evidenced interaction and cut vertices belong to the same connected component, we assume, using Remark~\vref{remark:AlternativeAiDecomposedForm}, that the following derivation is an $\ai$-decomposed form of $\Phi$:
\[
\vlder{\Phi'}{}
{
 \vlsbr[\beta\;.\;\vlinf{}{}{\fff}{\vls(a^\phi.\bar a^\psi)}]
}
{
 \vlsbr(\vlinf{}{}{\vls[a^\phi.\bar a^\psi]}{\ttt}\;.\;\alpha)
}\quad,
\]
for some atom $a$ and formulae $\alpha$ and $\beta$.

We combine three copies of $\Phi'$ to obtain the desired derivation $\Psi$ with flow $\psi'$ and the same premiss and conclusion as $\Phi$:

\TODO{Make an example of this use of switches:}

\newbox\DeltaTopK
\setbox\DeltaTopK=
\hbox{$
\vlder{\Phi'}{}
{
 \vlsbr[\beta\;.\;(\vlinf{}{}{\ttt}{a^\phi}\;.\;\bar a^\psi)]
}
{
 \vlsbr(\vlinf{}{}{\vls[a^\phi.\bar a^\psi]}{\ttt}\;.\;\alpha)
}
$}
\newbox\DeltaK
\setbox\DeltaK=
\hbox{$
\vlder{\Phi'}{}
{
 \vlsbr[\beta\;.\;(a^\phi\;.\;\vlinf{}{}{\ttt}{\bar a^\psi})]
}
{
 \vlsbr([\vlinf{}{}{a^\phi}{\fff}\;.\;\bar a^\psi]\;.\;\alpha)
}
$}
\newbox\DeltaBotK
\setbox\DeltaBotK=
\hbox{$
\vlder{\Phi'}{}
{
 \vlsbr[\beta\;.\;\vlinf{}{}{\fff}{\vls(a^\phi.\bar a^\psi)}]
}
{
 \vlsbr([a^\phi\;.\;\vlinf{}{}{\bar a^\psi}{\fff}]\;.\;\alpha)
}
$}
\[
\Psi\quad=\quad
\vlderivation
{
 \vlin{\cod}{}{\beta}
 {
  \vlin{\swi}{}
  {
   \vls
   [
    \vlinf{\cod}{}{\beta}{\vls[\beta.\beta]}
   \;\;\;\;.\;\;\;\;
    \box\DeltaBotK
   ]
  }
  {
   \vlin{\swi}{}
   {
    \vls
    (
%     \vlinf{\swi}{}
%     {
%      \vls
      [
       \beta
      \;\;\;\;.\;\;\;\;
       \box\DeltaK
      ]
%     }
%     {
%      \vls(\vlsmallbrackets[\beta.\bar a^\psi].\alpha)
%     }
    \;\;\;\;\;.\;\;\;\;\;
     \alpha
    )   
   }
   {
    \vlin{\cod}{}
    {
     \vls
     (
      \box\DeltaTopK
     \;\;\;\;.\;\;\;\;
      \vlinf{\cou}{}{\vls(\alpha.\alpha)}{\alpha}
     )
    }
    {
     \vlhy{\alpha}
    }
   }
  }
 } 
}\qquad.
\]
\end{proof}

\TODO{Find a way to define `evidenced' so this makes sense...}

\begin{proposition}\label{proposition:PathBreak}
Given atomic flows $\phi$ and $\psi$, such that $\phi\to_\frpb\psi$, then, referring to the atomic flows in Definition~\vref{definition:PathBreaker},
\begin{itemize}
 \item there is no path from the evidenced interaction vertex to the evidenced cut vertex in $\psi$, and
 \item let $\epsilon$ (resp., $\iota$) refer to a pair of corresponding upper (resp., lower) edges in $\phi$ and in $\psi$, then there is a path from $\epsilon$ to $\iota$ in $\psi$ if and only if there is a path from $\epsilon$ to $\iota$ in $\phi$.
\end{itemize}
\end{proposition}

\TODO{Redo proof.}

\begin{proof}
\begin{itemize}
	\item In the atomic flow $\psi$, we have colored all the edges that might be in paths from the evidenced interaction vertex in red, and all the edges that might be in paths to the evidenced cut vertex in green. Since the red and the green paths never overlap, we know there is no path from the interaction to the cut vertex.
	\item By Proposition~\vref{TODO}, we know that for every path from $\epsilon$ to $\iota$ in $\phi'$ there are three paths from $\epsilon$ to $\iota$ in $\psi'$, and if there is a path from $\epsilon$ to $\iota$ in $\psi'$, it must pass through one of the three subflows isomorphic to $\phi$ (resp., $\psi$) so there is a path from $\epsilon$ to $\iota$ in $\phi$.
\end{itemize}
\end{proof}
