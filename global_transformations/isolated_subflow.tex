\TODO{Throughout: use `isolated subflow' and define it somewhere else.}

\TODO{Change the following definition to be inline with notation and conventions in the rest of the thesis.}

\TODO{Change the definition to show a simple sub flow instead of a simple edge.}

\TODO{Define simple subflow and make sure we use that instead of simple edge elsewhere.}

\newcommand{\frsf}{{\mathsf{sf}}}
%---------------------------------------
\begin{definition}\label{DefSimSubRem}
We define the reduction $\to_\frsf$ (where $\frsf$ stands for \emph{simple flow}) as follows, for every atomic flow $A$:
\[
\atomicflow{
(10,16)*{\afaidnw{}{}};
( 0,14)*{\afvju4{\epsilon_1}{}};
( 2,14)*{\cdots};
( 4,14)*{\afvju4{}{\epsilon_h}};
( 8,14)*{\afvju4{}\two};
( 9,10)*{\aflabelleft A};
( 4, 9)*{\affr{10}6};
(12,11)*{\afvj{10}};
( 0, 4)*{\afvjd4{\epsilon'_1}{}};
( 2, 4)*{\cdots};
( 4, 4)*{\afvjd4{}{\epsilon'_k}};
(10, 2)*{\afaiu{}\three{}\one{}{}};
(-3, 0)*{\invisiblemark};
(14, 0)*{\invisiblemark}}
\quad\to_\frsf\quad
\atomicflow{
( 0,24   )*{\afvju{16}{\hat\epsilon_1}{}};
( 2,18   )*{\cdots};
( 4,24   )*{\afvju{16}{}{\hat\epsilon_h}};
( 8,18.25)*{\aflabelright{\hat\two}};
( 9,14   )*{\aflabelleft {\hat A}};
( 4,13   )*{\affr{10}6};
( 0, 8.4 )*{\afvjd2{\hat\epsilon'_1}{}};
( 2, 8   )*{\cdots};
( 4, 8.4 )*{\afvjd2{}{\hat\epsilon'_k}};
(12,32   )*{\afvju2{\tilde\epsilon_1}{}};
(14,32   )*{\cdots};
(16,32   )*{\afvju2{}{\tilde\epsilon_h}};
(17,28   )*{\aflabelleft {\tilde A}};
(12,27   )*{\affr{10}6};
(12,16   )*{\afvjd{16}{\tilde\epsilon'_1}{}};
(14,22   )*{\cdots}; 
(16,16   )*{\afvjd{16}{}{\tilde\epsilon'_k}};
( 8,22.25)*{\aflabelleft{\tilde\three}};
( 8,34   )*{\afawd{}{}{\tilde\two}{}};
( 8,42   )*{\cdots};
( 6,38   )*{\afacuexsq{}{}{}{}{\epsilon_1}{}31};
(10,38   )*{\afacuexsq{}{}{}{}{}{\epsilon_h}31};
( 8,20   )*{\afvj8};
( 8, 6   )*{\afawu{}{}{}{\hat\three}};
( 6, 2   )*{\afacdexsq{}{}{}{}{\epsilon'_1}{}31};
(10, 2   )*{\afacdexsq{}{}{}{}{}{\epsilon'_k}31};
( 8,-2   )*{\cdots};
(-3, 0   )*{\invisiblemark};
(19, 0   )*{\invisiblemark}}
\quad,
\]
where $h,k\ge0$, edges have been renamed with $\hat{\enspace}$ and $\tilde{\enspace}$ accents, flows $\tilde A$ and $\hat A$ are both isomorphic to $A$, and edges $\hat\two$ and $\tilde\three$ are identified.
\end{definition}

%---------------------------------------
\begin{theorem}\label{ThSFSound}
Reduction\/ $\to_\frsf$ is sound.
\end{theorem}

\TODO{Redo proof to be in line with the rest of the thesis.}

\TODO{Use Formalism A notation.}

\TODO{Define super switch, super switch down and super switch up.}

%---------------------------------------
\begin{proof}
Let $\Phi$ be a derivation with flow $B$, such that $B\to_\frsf C$. We show that there exists a derivation $\Psi$ with flow $C$ and with the same premiss and conclusion as $\Phi$. In the following, we refer to the figure in Definition~\ref{DefRedS}. We assume that $\Phi$ has premiss $\xi\{\ttt\}$ and conclusion $\zeta\{\fff\}$, where the evidenced and labelled $\ttt$ and $\fff$ can be traced to the interaction and cointeraction vertices eliminated by $\to_\frsf$, respectively (this can always be done by using switches and unit equations). Intuitively, we can think of $\ttt$ and $\fff$ as mapping to special `unit edges', which can be substituted just like normal edges. So, we assume that $\Phi$ is
\[
\vlderivation                                           {
\vlde{\Phi_3}{}{\zeta\{ \fff\}                 }    {
\vlin{\aiu  }{}{\zeta'\{ \fff\}                }   {
\vlde{\Phi_2}{}{\zeta'\vlscn(\bar a^\three.a^\one)}  {
\vlin{\aid  }{}{\xi'\vlscn[\bar a^\two.a^\one]    } {
\vlde{\Phi_1}{}{\xi'\{ \ttt\}                  }{
\vlhy          {\xi\{ \ttt\}                   }}}}}}}
\quad.
\]
We obtain the two derivations $\Psi'$ and $\Psi''$ from $\Phi$ as follows:
\[
\Psi'=\;\;
\vlderivation                                                  {
\vlde{\Phi_3\{\fff\ot\bar a\}}
             {}{\zeta\{\bar a^{\tilde\three}\}          }     {
\vlin{=     }{}{\zeta'\{\bar a^{\tilde\three}\}         }    {
\vlde{\Phi_2\{a^\one\ot\ttt\}}
             {}{\zeta'\vlscn(\bar a^{\tilde\three}.\ttt)}   {
\vlin{\awd  }{}{\xi'\vlscn[\bar a^{\tilde\two}.\ttt]    }  {
\vlin{=     }{}{\xi'\vlscn[\fff.\ttt]                   } {
\vlde{\Phi_1}{}{\xi'\{\ttt\}                            }{
\vlhy          {\xi\{\ttt\}                             }}}}}}}}
\qquad\hbox{and}\qquad
\Psi''=\;\;
\vlderivation                                                                  {
\vlde{\Phi_3                  }{}{\zeta\{\fff\}                         }     {
\vlin{=                       }{}{\zeta'\{\fff\}                        }    {
\vlin{\awu                    }{}{\zeta'\vlscn(\ttt.\fff)               }   {
\vlde{\Phi_2\{a^\one\ot \fff\}}{}{\zeta'\vlscn(\bar a^{\hat\three}.\fff)}  {
\vlin{=                       }{}{\xi'\vlscn[\bar a^{\hat\two}.\fff]    } {
\vlde{\Phi_1\{\ttt\ot\bar a\}}
                               {}{\xi'\{\bar a^{\hat\two}\}             }{
\vlhy                            {\xi\{\bar a^{\hat\two}\}              }}}}}}}}
\quad.
\]
Derivation $\Psi'$ has flow $B'$ and $\Psi''$ has $B''$:
\nopagebreak[4]\bigskip\afnegspace
\[
B'=\raise2\atflowelheight\hbox{$
\atomicflow{
( 4,14)*{\afvju4{\tilde\epsilon_1}{}};
( 6,14)*{\cdots};
( 8,14)*{\afvju4{}{\tilde\epsilon_h}};
( 0, 4)*{\afvjd4{\tilde\three}{}};
( 9,10)*{\aflabelleft {\tilde A}};
( 4, 9)*{\affr{10}6};
( 4, 4)*{\afvjd4{\tilde\epsilon'_1}{}};
( 6, 4)*{\cdots};
( 8, 4)*{\afvjd4{}{\tilde\epsilon'_k}};
( 0,16)*{\afawd{}{}{\tilde\two}{}};
(-2, 4)*{\invisiblemark};
(11, 4)*{\invisiblemark}}$}
\qquad\hbox{and}\qquad
B''=\lower2\atflowelheight\hbox{$
\atomicflow{
( 0,14)*{\afvju4{\hat\epsilon_1}{}};
( 2,14)*{\cdots};
( 4,14)*{\afvju4{}{\hat\epsilon_h}};
( 8,14)*{\afvju4{}{\hat\two}};
( 9,10)*{\aflabelleft {\hat A}};
( 4, 9)*{\affr{10}6}; 
( 0, 4)*{\afvjd4{\hat\epsilon'_1}{}};
( 2, 4)*{\cdots};
( 4, 4)*{\afvjd4{}{\hat\epsilon'_k}};
( 8, 2)*{\afawu{}{}{}{\hat\three}};
(-3, 4)*{\invisiblemark}}$}
\quad.
\]
\afnegspace
We combine $\Psi'$ and $\Psi''$ to get the desired derivation $\Psi$ with flow $C$ and the same premiss and conclusion as $\Phi$:
\[
\Psi=\;\;
\vlderivation                                                              {
\vlin{\cod                      }{}{\zeta\{\fff\}                    }    {
\vlde{\vls[\Psi''.\zeta\{\fff\}]}{}{\vls[\zeta\{\fff\}.\zeta\{\fff\}]}   {
\vlin{\swi                      }{}{\vls[\xi\{\bar a\}.\zeta\{\fff\}]}  {
\vlde{\vls(\xi\{\ttt\}.\Psi')   }{}{\vls(\xi\{\ttt\}.\zeta\{\bar a\})} {
\vlin{\cou                      }{}{\vls(\xi\{\ttt\}.\xi\{\ttt\})    }{
\vlhy                              {\xi\{\ttt\}                      }}}}}}}
\quad,
\]
where $\swi$, $\cod$ and $\cou$ are `macro' rules introduced in Remarks~\ref{RemSupSwitch} and \ref{RemGenContr}.
\end{proof}

\TODO{Add Lemata similar to the ones in the section on Four Boxes.}