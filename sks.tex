\newcommand{\SKS}{\mathsf{SKS}}

\section{System SKS}

We now define the standard deductive system for classical propositional logic in deep inference $\SKS$\index{$\SKS$} \cite{Brun:03:Atomic-C:oz,Brun:06:Cut-Elim:cq,Brun:06:Locality:zh,BrunTiu:01:A-Local-:mz}. For an excellent reference to previous work on normalisation in $\SKS$, see \cite{Brun:04:Deep-Inf:rq}. Subsystems of $\SKS$ are used throughout this thesis.

The results presented in this section, with the exception of Theorem~\vref{theorem:SKSComplete}, are standard results which can be found in the literature. We include the proofs for completeness and as means for giving examples of the functorial calculus.

\newcommand{\ai   }{{\mathsf{ai}}}
\newcommand{\aw   }{{\mathsf{aw}}}
\newcommand{\ac   }{{\mathsf{ac}}}
\newcommand{\aid  }{{\ai{\downarrow}}}
\newcommand{\awd  }{{\aw{\downarrow}}}
\newcommand{\acd  }{{\ac{\downarrow}}}
\newcommand{\aiu  }{{\ai{\uparrow}}}
\newcommand{\awu  }{{\aw{\uparrow}}}
\newcommand{\acu  }{{\ac{\uparrow}}}
\newcommand{\swi  }{\mathsf{s}}
\newcommand{\med  }{\mathsf{m}}
\newcommand{\asor }{{=_\mathsf{a}{\downarrow}}}
\newcommand{\asand}{{=_\mathsf{a}{\uparrow}}}
\newcommand{\coor }{{=_{\vee\mathsf{c}}}}
\newcommand{\coand}{{=_{\wedge\mathsf{c}}}}
\newcommand{\fffd }{{{=_{\fff}}{\downarrow}}}
\newcommand{\fffu }{{{=_{\fff}}{\uparrow}}}
\newcommand{\tttd }{{{=_{\ttt}}{\downarrow}}}
\newcommand{\tttu }{{{=_{\ttt}}{\uparrow}}}
\newcommand{\tttord }{{{=_{\ttt\vee}}{\downarrow}}}
\newcommand{\fffandd }{{{=_{\fff\wedge}}{\downarrow}}}
\newcommand{\tttoru }{{{=_{\ttt\vee}}{\uparrow}}}
\newcommand{\fffandu }{{{=_{\fff\wedge}}{\uparrow}}}


\begin{definition}\label{definition:SKS}
System $\SKS$ is defined by the following \emph{structural}\index{inference rule!structural} inference rules:
\[
\vlinf{\aid}{}{\vls[a.\bar a]}{\ttt}\qquad
\vlinf{\awd}{}{a}{\fff}\qquad
\vlinf{\acd}{}{a}{\vls[a.a]}
\]
\[
\vlinf{\aiu}{}{\fff}{\vls(a.\bar a)}\qquad
\vlinf{\awu}{}{\ttt}{a}\qquad
\vlinf{\acu}{}{\vls(a.a)}{a}\quad,
\]
the \emph{logical}\index{inference rule!logical} inference rules:
\[
\vlinf{\swi}{}{\vls[(A.B).C]}{\vls(A.[B.C])}\qquad
\vlinf{\med}{}{\vls([A.C].[B.D])}{\vls[(A.B).(C.D)]}
\quad,
\]
and the \emph{invertible}\index{inference rule!invertible} (\emph{logical}) rules:
\[
\vlinf{\coor}{}{\vls[B.A]}{\vls[A.B]}\qquad
\vlinf{\coand}{}{\vls(B.A)}{\vls(A.B)}\qquad
\vlinf{\asor}{}{\vls[[A.B].C]}{\vls[A.[B.C]]}\qquad
\vlinf{\asand}{}{\vls(A.(B.C))}{\vls((A.B).C)}
\]
\[
\vlinf{\fffd}{}{\vls[A.\fff]}{A}\qquad
\vlinf{\tttd}{}{\vls(A.\ttt)}{A}\qquad
\vlinf{\fffu}{}{A}{\vls(\ttt.A)}\qquad
\vlinf{\tttu}{}{A}{\vls[\fff.A]}
\]
\[
\vlinf{\fffandd}{}{\vls(\fff.\fff)}{\fff}\qquad
\vlinf{\tttord}{}{\vls[\ttt.\ttt]}{\ttt}\qquad
\vlinf{\fffandu}{}{\ttt}{\vls[\ttt.\ttt]}\qquad
\vlinf{\tttoru}{}{\fff}{\vls(\fff.\fff)}\quad.
\]
\end{definition}

The calculus of structures and system $\SKS$ were originally defined in terms of equivalence classes of formulae, called `structures', and without the above invertible logical rules. However, we find it more convenient to use formulae instead, since it makes it simpler to `trace atom occurrences', which we will see in Section~\vref{section:ExtractingFlowsFromDerivations}. We now show that the two approaches are morally the same.

%------------------------------------------------------
\begin{definition}\label{definition:EquivalenceRelation}
We define the relation $=$ such that, given formulae $\alpha$ and $\beta$, $\alpha=\beta$ if there is a derivation $\vlder{\Phi}{\{\coor,\coand,\asor,\asand,\fffd,\tttd,\fffu,\tttu,\fffandd,\tttord,\fffandu,\tttoru\}}{\beta}{\alpha}$.
\end{definition}
%---------------

%----------------------------------------
\begin{notation}\label{notation:Equation}
If $\alpha=\beta$, we often write $\vlinf{=}{}{\beta}{\alpha}$.
\end{notation}
%-------------

%------------------------------------------
\begin{remark}\label{remark:ContextClosure}
By Notation~\vref{notation:Equation} and Lemma~\vref{lemma:DerInContext}, for any formulae $\alpha$ and $\beta$ and context $\xi\vlhole$ we have that $\alpha=\beta$ implies $\xi\{\alpha\}=\xi\{\beta\}$.
\end{remark}
%-----------

%---------------------------------------------------------
\begin{proposition}\label{proposition:EquivalenceRelation}
The relation $=$ defined in Definition~\vref{definition:EquivalenceRelation} is an equivalence relation.
\end{proposition}
%----------------

It turns out that the equivalence class induced by $=$ is the same as the structures used in \cite{Brun:04:Deep-Inf:rq}.

%----------------------------------------
\begin{remark}\label{remark:EquationSize}
If $\alpha=\beta$, then (as remarked in \cite{BrusGugl:07:On-the-P:fk}) there exists a derivation
\[
\vlder{\Phi}{\{\coor,\coand,\asor,\asand,\fffd,\tttd,\fffu,\tttu,\fffandd,\tttord,\fffandu,\tttoru\}}{\beta}{\alpha}
\]
whose size depends at most quadratically on the sum of the sizes of $\alpha$ and $\beta$.
\end{remark}
%-----------

\begin{notation}\label{notation:ImplicitEquations}
When we work in (subsystems of) $\SKS$, we often omit mentioning the invertible rules. Let $\mathcal{S}$ be a subsystem of $\SKS$, then, unless specified otherwise, when we write $\mathcal{S}$ we mean $\mathcal{S}\cup\{\coor,\coand,\asor,\asand,\fffd,\tttd,\fffu,\tttu,\fffandd,\tttord,\fffandu,\tttoru\}$. Furthermore, if $\rho\in\SKS$, and there is a derivation
\[
\vlderivation
{
 \vlin{=}{}
 {
  \beta
 }
 {
  \vlin{\rho}{}
  {
   \beta'
  }
  {
   \vlin{=}{}
   {
    \alpha'
   }
   {
    \vlhy
    {
     \alpha
    }
   }
  }
 }
}
\]
we sometimes write
\[
\vlinf{\rho}{}
{
 \beta
}
{
 \alpha
}\quad.
\]
\emph{E.g.}, instead of the derivation
\[
\vlderivation
{
 \vlin{\coor}{}
 {
  \vls
  [
   \alpha
  \;.\;
   \vlinf{\coand}{}
   {
    \vls(\beta.\gamma)
   }
   {
    \vls(\gamma.\beta)
   }
  ]
 }
 {
  \vlin{\swi}{}
  {
   \vls[(\gamma.\beta).\alpha]
  }
  {
   \vlin{\coand}{}
   {
    \vls(\gamma.[\beta.\alpha])
   }
   {
    \vlhy
    {
     \vls
     (
      \vlinf{\coor}{}
      {
       \vls[\beta.\alpha]
      }
      {
       \vls[\alpha.\beta]
      }
     \;.\;
      \gamma
     )
    }
   }
  }
 }
}\quad,
\]
we write
\[
\vlinf{\swi}{}{\vls[\alpha.(\beta.\gamma)]}{\vls([\alpha.\beta].\gamma)}\quad.
\]
See the proofs of Theorems~\vrefrange{theorem:SoundPathBreaker}{theorem:SoundMultipleIsolatedSubflowsRemoval} for more examples of implicit equations.
\end{notation}

We now give some standard results which will also serve as examples of system $\SKS$ and the functorial calculus.

\begin{lemma}\label{lemma:SuperSwitch}
Given a context $\xi\vlhole$ and a formula $\alpha$ there exist derivations $\vlder{}{\{\swi\}}{\xi\{\alpha\}}{\vls(\alpha.\xi\{\ttt\})}$ and $\vlder{}{\{\swi\}}{\vls[\xi\{\fff\}.\alpha]}{\xi\{\alpha\}}$; both of whose size depend at most quadratically on the size of $\xi\{\alpha\}$.
\end{lemma}

\Tom{atoms -> atom occurrences}

\begin{proof}
We show how to construct the first derivation, the second one can be done symmetrically. We argue by induction on the number of atom occurrences in $\xi\vlhole$. The base case, $\xi\vlhole=\vlhole$, is trivial and the inductive cases are:
\[
\vlderivation
{
 \vlin{=}{}
 {
  \xi\{\alpha\}
 }
 {
  \vlin{\swi}{}
  {
   \vlsbr[\vlder{\Psi}{\{\swi\}}{\xi'\{\alpha\}}{\vls(\alpha.\xi'\{\ttt\})}\;\;.\;\;\beta]
  }
  {
   \vlhy
   {
    \vls
    (
     \alpha
    \;.\;
     \vlinf{=}{}
     {
      \vls[\xi'\{\ttt\}.\beta]
     }
     {
      \xi\{\ttt\}
     }
    )
   }
  }
 }
}\qquad\mbox{and}\qquad
\vlderivation
{
 \vlin{=}{}{\xi\{\alpha\}}
 {
  \vlin{=}{}{\vlsbr(\vlder{\Psi'}{\{\swi\}}{\xi'\{\alpha\}}{\vls(\alpha.\xi'\{\ttt\})}\;\;.\;\;\beta)}
  {
   \vlhy{\vls(\alpha.\xi\{\ttt\})}
  }
 }
}\quad,
\]
for some $\xi'\vlhole$ and $\beta$ where $\beta$ is not a unit and $\Psi$ and $\Psi'$ exist by the inductive hypothesis.
\end{proof}

\newcommand{\supers}{\mathsf{ss}}
\newcommand{\ssu}{\supers\uparrow}
\newcommand{\ssd}{\supers\downarrow}

\begin{notation}\label{notation:SuperSwitch}
We often write $\vlinf{\ssu}{}{\xi\{\alpha\}}{\vls(\alpha.\xi\{\ttt\})}$ and $\vlinf{\ssd}{}{\vls[\xi\{\fff\}.\alpha]}{\xi\{\alpha\}}$, instead of, respectively, the derivations $\vlder{}{\{\swi\}}{\xi\{\alpha\}}{\vls(\alpha.\xi\{\ttt\})}$ and $\vlder{}{\{\swi\}}{\vls[\xi\{\fff\}.\alpha]}{\xi\{\alpha\}}$, as defined in the proof of Lemma~\vref{lemma:SuperSwitch}. Instead of the derivation
\[
\vlinf{\swi}{}
{
 \vls[\zeta\{\fff\}\;.\;\vlinf{\ssu}{}{\xi\{\alpha\}}{\vls(\alpha.\xi\{\ttt\})}]
}
{
 \vls(\vlinf{\ssd}{}{\vls[\zeta\{\fff\}.\alpha]}{\zeta\{\alpha\}}\;.\;\xi\{\ttt\})
}
\]
we write $\vlinf{\supers}{}{\vls[\zeta\{\fff\}.\xi\{\alpha\}]}{\vls(\zeta\{\alpha\}.\xi\{\ttt\})}$.
\end{notation}

\Tom{Added Lemma with proof. This was implicitly proven in Theorem~\vref{theorem:ThresholdDerivations}.}

We now show a consequenc of the previous Lemma, which will be very useful in Subsection~\vref{subsection:ThresholdFormulae}.

%------------------------------------------------
\begin{lemma}\label{lemma:SuperSwitchContraction}
Given a formula $\alpha$ and an atom $a$, there exist derivations $\vlder{}{\{\acu,\swi\}}{\alpha}{\vls(a.\alpha\{a/\ttt\})}$ and $\vlder{}{\{\acd,\swi\}}{\vls[\alpha\{a/\fff\}.a]}{\alpha}$; both of whose size depend at most quadratically on the size of $\alpha$.
\end{lemma}

\begin{proof}
We show how to construct the first derivation, the second one can be done symmetrically. The result follows by induction on the number of occurrences of $a$ in $\alpha$, and Lemma~\vref{lemma:SuperSwitch}. The base case is trivial. Let $\xi\vlhole$ be some context such that, $\alpha=\xi\{a\}$, then the inductive case is:
\[
\vlder{}{\{\swi\}}{\vls(a.(\xi\{a/\ttt\})\{a\})}{\vls(\vlinf{}{}{\vls(a.a)}{a}.(\xi\{a/\ttt\})\{\ttt\})}
\quad.
\]
\end{proof}
%----------


\begin{lemma}\label{lemma:GenericWeakening}
Given a formula $\alpha$, there exist derivations $\vlupsmash{\vlder{}{\{\awd,\swi\}}{\alpha}{\fff}}$ and $\vlupsmash{\vlder{}{\{\awu,\swi\}}{\ttt}{\alpha}}$; both of whose size depend at most quadratically on the size of $\alpha$.
\end{lemma}

\Tom{atoms -> atom occurrences}

\begin{proof}
We show how to construct the first derivation, the second one can be done symmetrically.
Let $a_1$, $\dots$, $a_n$ be the atoms appearing in $\alpha$, then there exists a derivation
\[
\vlder{}{\{\awd\}}{\alpha}{\alpha\{a_1/\fff,\dots,a_n/\fff\}}
\quad.
\]
Since $\alpha\{a_1/\fff,\dots,a_n/\fff\}$ contains no atom occurrences, there exists a derivation
\[
\vlder{}{\{\fffd,\tttd,\fffandd,\tttord\}}{\alpha\{a_1/\fff,\dots,a_n/\fff\}}{\fff}
\qquad\mbox{or}\qquad
\vlderivation
{
 \vlde{}{\{\fffd,\tttd,\fffandd,\tttord\}}
 {
  \alpha\{a_1/\fff,\dots,a_n/\fff\}
 }
 {
  \vlin{=}{}
  {
   \ttt
  }
  {
   \vlin{\swi}{}
   {
    \vls[(\fff.\ttt).\ttt]
   }
   {
    \vlin{=}{}
    {
     \vls(\fff.[\ttt.\fff])
    }
    {
     \vlhy
     {
      \fff
     }
    }
   }
  }
 }
}
\quad.
\]
\end{proof}

\begin{lemma}\label{lemma:GenericContraction}
Given a formula $\alpha$, there exist derivations $\vlupsmash{\vlder{}{\{\acd,\med\}}{\alpha}{\vls[\alpha.\alpha]}}$ and $\vlupsmash{\vlder{}{\{\acu,\med\}}{\vls(\alpha.\alpha)}{\alpha}}$; both of whose size depend at most quadratically on the size of $\alpha$.
\end{lemma}

\Tom{number of atoms in $\alpha$ -> size of $\alpha$; added two base casse}

\begin{proof}
We show how to construct the first derivation, the second one can be done symmetrically. We argue by induction on the size of $\alpha$. We have to consider the following three base cases
\[
\vlinf{\fffandu}{}{\ttt}{\vls[\ttt.\ttt]}
\qquad,\qquad
\vlinf{\tttu}{}{\fff}{\vls[\fff.\fff]}
\qquad\mbox{and}\qquad
\vlinf{}{}{a}{\vls[a.a]}
\quad,
\]
and two inductive cases:
\[
\vlinf{\med}{}
{
 \vls
 (
  \vlder{}{\{\acd,\med\}}
  {
   \alpha
  }
  {
   \vls[\alpha.\alpha]
  }
 \;\;.\;\;
  \vlder{}{\{\acd,\med\}}
  {
   \beta
  }
  {
   \vls[\beta.\beta]
  }
 )
}
{
 \vls[(\alpha.\beta).(\alpha.\beta)]
}
\qquad\hbox{and}\qquad
\vlinf{=}{}
{
 \vls
 [
  \vlder{}{\{\acd,\med\}}
  {
   \alpha
  }
  {
   \vls[\alpha.\alpha]
  }
 \;\;.\;\;
  \vlder{}{\{\acd,\med\}}
  {
   \beta
  }
  {
   \vls[\beta.\beta]
  }
 ]
}
{
 \vls[[\alpha.\beta].[\alpha.\beta]]
}
\quad.
\]
\end{proof}

\newcommand{\contr}{\mathsf{c}}
\newcommand{\cod}{{\contr{\downarrow}}}
\newcommand{\cou}{{\contr{\uparrow}}}
\newcommand{\weakn}{\mathsf{w}}
\newcommand{\wed}{{\weakn{\downarrow}}}
\newcommand{\weu}{{\weakn{\uparrow}}}

\begin{notation}\label{notation:GenericWeakeningContraction}
In the non-atomic version of system $\SKS$ the derivations shown in the proofs of Lemma~\vref{lemma:GenericWeakening} and Lemma~\vref{lemma:GenericContraction} correspond to (co)\-weak\-en\-ing and (co)contractions, respectively. For this reason we sometimes write the inference rules $\vlinf{\wed}{}{\alpha}{\fff}$, $\vlinf{\weu}{}{\ttt}{\alpha}$, $\vlinf{\cod}{}{\alpha}{\vls[\alpha.\alpha]}$ and $\vlinf{\cou}{}{\vls(\alpha.\alpha)}{\alpha}$ instead of the derivations $\vlder{}{\{\awd,\swi\}}{\alpha}{\fff}$, $\vlder{}{\{\awu,\swi\}}{\ttt}{\alpha}$, $\vlder{}{\{\acd,\med\}}{\alpha}{\vls[\alpha.\alpha]}$ and $\vlder{}{\{\acu,\med\}}{\vls(\alpha.\alpha)}{\alpha}$, respectively.
\end{notation}

To give an example of the notions defined so far, we now show a completeness proof of system $\SKS$.

\begin{theorem}\label{theorem:SKSComplete}
System $\SKS$ is complete for propositional classical logic.
\end{theorem}

\begin{proof}
Consider a tautology $\alpha$. We show by induction on the number of atoms appearing in $\alpha$ that there exists a proof of $\alpha$ in $\SKS$. For the base case, let $\alpha$ consist only of units. Then, since $\alpha$ is a tautology, we can build
\[
\vlder{}{\{\fffd,\tttd,\fffandd,\tttord\}}{\alpha}{\ttt}\quad.
\]

For the inductive case, let $\alpha$ be a tautology containing instances of the atom $a$. We consider two cases:
\begin{itemize}
\item if $\alpha$ does not contain an instance of $\bar a$, then $\alpha\{a/\fff\}$ is a tautology, so by the inductive hypothesis we can build
\[
\vlderivation
{
 \vlde{}{\{\awd\}}
 {
  \alpha
 }
 {
  \vlde{}{}
  {
   \alpha\{a/\fff\}
  }
  {
   \vlhy{\ttt}
  }
 }
}\quad;
\]
\item otherwise, both $\alpha\{a/\ttt,\bar a/\fff\}$ and $\alpha\{a/\fff,\bar a/\ttt\}$ are tautologies, so by the inductive hypothesis we can build
\[
\Phi\;=\;
\vlderivation
{
 \vlde{}{\{\ssd\}}
 {
  \vlsbr[\alpha\;\;.\;\;\vlder{}{\{\acd\}}{\bar a}{\vls[\bar a.\cdots.\bar a]}]
 }
 {
  \vlde{}{\{\aid,\awd\}}
  {
   \alpha\{a/\vlsbr[a.\bar a]\}
  }
  {
   \vlde{}{}
   {
    \alpha\{a/\ttt,\bar a/\fff\}
   }
   {
    \vlhy{\ttt}
   }
  }
 }
}\quad.
\]
Using $\Phi$ and the inductive hypothesis we can build the desired derivation:
\[
\vlderivation
{
 \vlde{}{\cod}
 {
  \alpha
 }
 {
  \vlde{}{\{\ssd\}}
  {
   \vls[\alpha.\cdots.\alpha]
  }
  {
   \vlde{}{}
   {
    \alpha\{\bar a/\vlsbr[\alpha.\bar a]\}
   }
   {
    \vlde{}{\{\awd\}}
    {
     \alpha\{\bar a/\ttt\}
    }
    {
     \vlde{}{}
     {
      \alpha\{a/\fff,\bar a/\ttt\}
     }
     {
      \vlhy{\ttt}
     }
    }
   }
  }
 }
}\quad.
\]
\end{itemize}
\end{proof}

\begin{remark}
Given any formulae $\alpha$ and $\beta$ and any context $\xi\vlhole$, then, by a construction similar to the one in the proof of Lemma~\vref{lemma:SuperSwitch}, we can build a derivation $\vlder{}{\{\swi,\coor,\coand,\asor\}}{\vls[\xi\{\alpha\}.\beta]}{\xi\{\vls[\alpha.\beta]\}}$. If we use this derivation instead of the rule $\ssd$ in the proof of Theorem~\vref{theorem:SKSComplete}, it follows that the system
\[
\{\aid,\acd,\awd,\swi,\med,\coor,\coand,\asor,\fffd,\tttd,\fffandd,\tttord\}
\]
is complete for classical logic. This justifies the naming of the invertible rules, as the tradition is in deep inference to label admissible rules with an $\uparrow$.
\end{remark}

\begin{remark}
If we allow the use of all the rules of $\SKS$, as opposed to only the down fragment, we can build a more compact proof than what we do in Theorem~\vref{theorem:SKSComplete}, by using the following as the inductive case:
\[
\vlderivation
{
 \vlin{\cod}{}
 {
  \alpha
 }
 {
  \vlin{\swi}{}
  {
   \vls
   [
    \vlder{}{\{\ssu\}}
    {
     \alpha
    }
    {
     \vlsbr
     (
      \vlder{}{\{\acu\}}
      {
       \vls(a.\cdots.a)
      }
      {
       a
      }
     \;\;.\;\;
      \vlder{}{\{\awd\}}
      {
       \alpha\{a/\ttt\}
      }
      {
       \alpha\{a/\ttt,\bar a/\fff\}
      }
     )
    }
   \;\;\;\;.\;\;\;\;
    \vlder{}{\{\ssu\}}
    {
     \alpha
    }
    {
     \vlsbr
     (
      \vlder{}{\{\acu\}}
      {
       \vls(\bar a.\cdots.\bar a)
      }
      {
       \bar a
      }
     \;\;.\;\;
      \vlder{}{\{\awd\}}
      {
       \alpha\{\bar a/\ttt\}
      }
      {
       \alpha\{a/\fff,\bar a/\ttt\}
      }
     )
    }
   ]
  }
  {
   \vlde{}{}
   {
    \vlsbr
    (
     \vlinf{\swi}{}
     {
      \vls[(a.\alpha\{a/\ttt,\bar a/\fff\}).\bar a]
     }
     {
      \vls
      (
       \vlinf{}{}{\vls[a.\bar a]}{\ttt}
      \;.\;
       \alpha\{a/\ttt,\bar a/\fff\}
      )
     }
    \;\;.\;\;
     \alpha\{a/\fff,\bar a/\ttt\}
    )
   }
   {
    \vlhy
    {
     \ttt
    }
   }
  }
 }
}\quad.
\]
\end{remark}