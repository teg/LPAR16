\chapter{Cut Elimination}

\newcommand{\Exp}{\mathsf{Exp}}

\begin{definition}\label{definition:FlowExperiment}
Given a polarity assignment $\pi$ and an atomic flow
\[
\phi\;\;=\;\;\atomicflow
{
(0,17.5)*{\invisiblemark};
%---
(-13,0)*{\affr{22}{20}};
(-5,8)*{\aflabelright{\phi_1}};
%
(-13, 8)*{\afaidm{}{}{}{}{}{}};
(-18, 0)*{\affr{8}{8}};
(-19, 2)*{\aflabelleft\ppl};
(-17, 2)*{\aflabelright{\phi'_1}};
( -8, 0)*{\affr{8}{8}};
( -9, 2)*{\aflabelleft\pmi};
( -7, 2)*{\aflabelright{\phi''_1}};
( -6,-11)*{\afvjum{14}{}{\boldsymbol\iota''_1}};
(-13,-8)*{\afaium{}{}{}{}{}{}};
(-20,-11)*{\afvjum{14}{\boldsymbol\iota'_1}{}};
%------------
(0,0)*{\cdots};
%------------
(13,0)*{\affr{22}{20}};
(21,8)*{\aflabelright{\phi_n}};
%
(13, 8)*{\afaidm{}{}{}{}{}{}};
( 8, 0)*{\affr{8}{8}};
( 7, 2)*{\aflabelleft\ppl};
( 9, 2)*{\aflabelright{\phi'_n}};
(18, 0)*{\affr{8}{8}};
(17, 2)*{\aflabelleft\pmi};
(19, 2)*{\aflabelright{\phi''_n}};
( 6,-11)*{\afvjum{14}{\boldsymbol\iota'_n}{}};
(13,-8)*{\afaium{}{}{}{}{}{}};
(20,-11)*{\afvjum{14}{}{\boldsymbol\iota''_n}};
}\qquad
\atomicflow{
( -5, 13.5)*{\afawdm{}{}{}{}{}{}};
%---
( -8, 5.5)*{\copy\contrup};
( -5, 5.5)*{\affr{18}8};
( -2, 5.5)*{\copy\contrdown};
%---
(-10,-2.5)*{\afaium{}{}{}{}{}{}};
(  0,   0)*{\afvjm3};
( 10, 2.5)*{\afaidm{}{}{}{}{}{}};
%---
(  2,-5.5)*{\copy\contrup};
(  5,-5.5)*{\affr{18}8};
(  8,-5.5)*{\copy\contrdown};
%---
(  0,-13.5)*{\afawum{}{}{}{}{}{}};
( 10,-13.5)*{\afvjm8};
%---------
( 0, 0)*{\affr{30}{32}};
(12,14)*{\aflabelright{\psi}};
}\quad,
\]
where, for $1\le i\le n$, $\phi_i$ is a connected, non-weakly-cut-free subflow, we define the \emph{experiment on $\phi$ with respect to $\pi$}, denoted $\Exp(\phi,\pi)$, to be
\[
\atomicflow
{
(0,17.5)*{\invisiblemark};
%-----
(-18, 6)*{\afvjdm4{}{\boldsymbol{\lambda'_1}}};
( -8, 8)*{\afawdm{}{}{}{}{}{}};
(-18, 0)*{\affr{8}{8}};
(-19, 2)*{\aflabelleft\ppl};
(-17, 2)*{\aflabelright{\phi'_1}};
( -8, 0)*{\affr{8}{8}};
( -9, 2)*{\aflabelleft\pmi};
( -7, 2)*{\aflabelright{\phi''_1}};
( -6,-11)*{\afvjum{14}{}{\boldsymbol\iota''_1}};
(-13,-8)*{\afaium{}{}{}{}{}{}};
(-20,-11)*{\afvjum{14}{\boldsymbol\iota'_1}{}};
%------------
(0,0)*{\cdots};
%------------
( 8, 6)*{\afvjdm4{}{\boldsymbol{\lambda'_n}}};
(18, 8)*{\afawdm{}{}{}{}{}{}};
( 8, 0)*{\affr{8}{8}};
( 7, 2)*{\aflabelleft\ppl};
( 9, 2)*{\aflabelright{\phi'_n}};
(18, 0)*{\affr{8}{8}};
(17, 2)*{\aflabelleft\pmi};
(19, 2)*{\aflabelright{\phi''_n}};
( 6,-11)*{\afvjum{14}{\boldsymbol\iota'_n}{}};
(13,-8)*{\afaium{}{}{}{}{}{}};
(20,-11)*{\afvjum{14}{}{\boldsymbol\iota''_n}};
}\qquad
\atomicflow{
( -5, 13.5)*{\afawdm{}{}{}{}{}{}};
%---
( -8, 5.5)*{\copy\contrup};
( -5, 5.5)*{\affr{18}8};
( -2, 5.5)*{\copy\contrdown};
%---
(-10,-2.5)*{\afaium{}{}{}{}{}{}};
(  0,   0)*{\afvjm3};
( 10, 2.5)*{\afaidm{}{}{}{}{}{}};
%---
(  2,-5.5)*{\copy\contrup};
(  5,-5.5)*{\affr{18}8};
(  8,-5.5)*{\copy\contrdown};
%---
(  0,-13.5)*{\afawum{}{}{}{}{}{}};
( 10,-13.5)*{\afvjm8};
%---------
( 0, 0)*{\affr{30}{32}};
(12,14)*{\aflabelright{\psi}};
}\quad.
\]
For every $1\le i\le n$, the \emph{new upper edges of} $\phi_i$ are $\boldsymbol{\lambda'_i}$.
\end{definition}


\begin{definition}\label{definition:DerExperiment}
Given a proof $\vlproof{\Pi}{}{\alpha}$ with associated atomic flow $\phi$ and a polarity assignment, $\pi$, to $\phi$, such that $\phi_1$, $\dots$, $\phi_n$ are all the non-weakly-cut-free, connected subflows of $\phi$, an \emph{experiment on\/ $\Pi$ with respect to $\pi$} is defined to be a derivation
\[
\vlder{}{}{\alpha}{\vlsmallbrackets\vls((a^{\lambda'_{1,1}}_1.\cdots.a^{\lambda'_{1,k_1}}_1).\cdots.(a^{\lambda'_{n,1}}_n.\cdots.a^{\lambda'_{n,k_n}}_n))}
\]
with associated atomic flow $\Exp(\phi,\pi)$, such that, for every $1\le i\le n$, $\lambda'_{i,1}$, $\dots$, $\lambda'_{i,k_i}$ are the new upper edges of $\phi_i$.
\end{definition}

\TODO{Update definition of `new upper edges of $\phi_i$' to also mention $\Exp$}

\TODO{We are glossing over some associativity/commutativity...}

\begin{theorem}\label{theorem:DerExperiment}
Given a proof\/ $\vlproof{\Pi}{}{\alpha}$ with associated atomic flow $\phi$, and a polarity assignment, $\pi$, to $\phi$, an experiment on $\Pi$ with respect to $\pi$ can be constructed.
\end{theorem}

\begin{proof}
Let $\phi_1$, $\dots$, $\phi_n$ be all the non-weakly-cut-free, connected subflows of $\phi$ and, for every $1\le i\le n$, let $\lambda'_{i,1}$, $\dots$, $\lambda'_{i,k_i}$ be the new upper edges of $\phi_i$ in $\Exp(\phi,\pi)$.

For every $1\le i\le n$ and every $1\le j\le k_i$, substitute $\vlinf{}{}{\vls[a_i^{\lambda'_{i,j}}.\bar a_i]}{\ttt}$ with $\vlsbr[a_i^{\lambda'_{i,j}}.\vlinf{}{}{\bar a_i}{\fff}]$ in the $\ai$-decomposed form of $\Pi$ to obtain a derivation with atomic flow $\Exp(\phi,\pi)$:
\[
\vlder{}{}{\alpha}
{
 \vlsbr
 (
  (
   [a_1^{\lambda'_{1,1}}.\vlinf{}{}{\bar a_1}{\fff}]
  \;.\;\cdots\;.\;
   [a_1^{\lambda'_{1,k_1}}.\vlinf{}{}{\bar a_1}{\fff}]
  )
 \;.\;\cdots\;.\;
  (
   [a_n^{\lambda'_{n,1}}.\vlinf{}{}{\bar a_n}{\fff}]
  \;.\;\cdots\;.\;
   [a_n^{\lambda'_{n,k_n}}.\vlinf{}{}{\bar a_n}{\fff}]
  )
 )
}\quad.
\]
\end{proof}

\begin{definition}\label{definition:DerTheExperiment}
Given a proof $\Pi$ and a polarity assignment to its atomic flow $\pi$, the experiment on $\Phi$ with respect to $\pi$ obtained as described in the proof of Theorem~\vref{theorem:DerExperiment} is called \emph{the experiment on\/ $\Pi$ with respect to $\pi$}, denoted $\Exp(\Pi,\pi)$.
\end{definition}

\TODO{In what sense is this unique?}

\TODO{Integrate this section with the rest of the thesis. In particular do it in terms of weak cut-elimination, which will make it much more elegant.}

% TODO: big remark or section on philosophy
% TODO: remark on symmetry/confluence

\newcommand{\Assignments}{\mathcal A}
\newcommand{\Sym}{\mathsf{Sym}}

\begin{definition}\label{DefSymmetricProof}
Given distinct and pairwise non-dual atoms, $a_1,\dots,a_n$, define
\begin{itemize}
\item the set $\Assignments_k=\{\{b_1,\dots,b_k\}|b_i\in\{a_i,\bar a_i\}\}$ for $1\leq k\leq n$ and
\item a \emph{symmetric proof of }$\bigvee_{\{b_1,\dots,b_k\}\in\Assignments_k}\vlsbr(b_1.\cdots.b_n)$, denoted $\Sym(a_1,\dots,a_k)$, by induction on $k$:

The base case is
\[
\Sym(a_1)=\vlinf{}{}{\vls[a_1.\bar a_1]}{\ttt}\quad,
\]
and the inductive case is
\[
\newbox\DerCap
\setbox\DerCap=
\hbox{$
\vlderivation
{
 \vlde{}{\{\acu,\med\}}{\bigwedge_{i=1}^{2^{k-1}}[b_k.\bar b_k]}
 {
  \vlin{}{}{\vls[b_k.\bar b_k]}
  {
   \vlhy{\ttt}
  }
 }
}$
}
\newbox\DerCap
\setbox\DerCap=
\hbox{$
\vlderivation
{
 \vlde{}{\{\acu,\med\}}{\bigwedge_{i=1}^{2^{k-1}}\vls[b_k.\bar b_k]}
 {
  \vlin{}{}{\vls[b_k.\bar b_k]}
  {
   \vlhy{\ttt}
  }
 }
}$
}
\Sym(a_1,\dots,a_k)\quad=\quad
\vlderivation
{
 \vlin{=}{}{\bigvee_{\{b_1,\dots,b_k\}\in \Assignments_k}\vlsbr(b_1.\cdots.b_k)}
 {
  \vlde{}{\{\swi\}}{\bigvee_{\{b_1,\dots,b_{k-1}\}\in \Assignments_{n-1}}\vlsbr[(b_1.\cdots.b_{k-1}.b_n).(b_1.\cdots.b_{k-1}.\bar b_k)]}
  {
  \vlpr{\Sym(a_1,\dots,a_{k-1})}{\{\aid,\acu,\swi,\med\}}{\vlsbr(\box\DerCap.\bigvee_{\{b_1,\dots,b_{k-1}\}\in \Assignments_{k-1}}(\vlinf{}{}{\vls(b_1.b_1)}{b_1}.\cdots.\vlinf{}{}{\vls(b_{k-1}.b_{k-1})}{b_{k-1}}))}
  }
 }
}\quad,
\]
for $1 < k \leq n$.
\end{itemize}
\end{definition}

\begin{proposition}
The atomic flow associated with $\Sym(a_1,\dots,a_n)$ is
\[
\atomicflow
{
(-13, 8)*{\afaid{}{}{}{}{}{}};
(-18, 0)*{\affr{8}{8}};
(-18, 0)*{\copy\contrup};
(-16, 2)*{a_1};
( -8, 0)*{\affr{8}{8}};
( -8, 0)*{\copy\contrup};
( -6, 2)*{\bar a_1};
(-18,-6)*{\afvjm{4}};
( -8,-6)*{\afvjm{4}};
%------------
(0,0)*{\cdots};
%------------
(13, 8)*{\afaid{}{}{}{}{}{}};
( 8, 0)*{\affr{8}{8}};
( 8, 0)*{\copy\contrup};
(10, 2)*{a_n};
(18, 0)*{\affr{8}{8}};
(18, 0)*{\copy\contrup};
(20, 2)*{\bar a_n};
( 8,-6)*{\afvjm{4}};
(18,-6)*{\afvjm{4}};
}\quad.
\]
\end{proposition}

\TODO{This is not weakly-cut-free... There are paths from cuts to axioms...}

\begin{definition}
Given a proof $\vlproof{\Pi}{}{\alpha}$ with atomic flow $\phi$, the set $P$ of all the polarity assignments to $\phi$, and, for every $\pi$ in $P$ and every non-weakly-cut-free subflow $\phi_i$ of $\phi$, let $\lambda'_{i,1}$, $\dots$, $\lambda'_{i,k_i}$ be the new upper edges of $\phi_i$, then \emph{a symmetric weakly-cut-free proof obtained from $\Phi$} is:
\[
\vlderivation
{
 \vlin{(|P|-1)\cdot\cod}{}{\alpha}
 {
  \vlpr{\Sym(a_1,\dots,a_n)}{\{\aid,\acu,\swi,\med\}}
  {
  \bigvee_{\pi\in P}
  \left(
   \vlder{\Exp(\Pi,\pi)}{}{\alpha}
    {
     \vls(
      \vlinf{}{}{\vls(b^{\lambda'_{1,1}}_1.\cdots.b^{\lambda'_{1,k_1}}_1)}{b_1}
     \;.\;\cdots\;.\;
      \vlinf{}{}{\vls(b^{\lambda'_{n,1}}_n.\cdots.b^{\lambda'_{n,k_n}}_n)}{b_n}
     )
    }
  \right)}
 }
}\quad.
\]
\end{definition}


\begin{proposition}\label{ProUniqueCutFreeFlow}
Given a proof $\Phi$ with atomic flow
\[
A=
\atomicflow
{
(-18, 0)*{\affr{8}{8}};
(-16, 2)*{a_1};
(-21,-6)*{\afvjm4};
%
(-13,8)*{\afaidm{}{}{}{}{}{}};
(-13,-8)*{\afaium{}{}{}{}{}{}};
%
( -8, 0)*{\affr{8}{8}};
( -6, 2)*{\bar a_1};
( -5,-6)*{\afvjm4};
%------------
(0,0)*{\cdots};
%------------
( 8, 0)*{\affr{8}{8}};
(10, 2)*{a_n};
( 5,-6)*{\afvjm4};
%
(13,8)*{\afaidm{}{}{}{}{}{}};
(13,-8)*{\afaium{}{}{}{}{}{}};
%
(18, 0)*{\affr{8}{8}};
(20, 2)*{\bar a_n};
(21,-6)*{\afvjm4};
}\quad,
\]
the atomic flow associated with a symmetric cut-free proof obtained from $\Phi$ is
\[
\atomicflow
{
(-27,18)*{\afaid{}{}{}{}{}{}};
%----
(27,18)*{\afaid{}{}{}{}{}{}};
%--------
(-38,10)*{\affr{20}8};
(-38,10)*{\copy\contrup};
(-16,10)*{\affr{20}8};
(-16,10)*{\copy\contrup};
%----
(16,10)*{\affr{20}8};
(16,10)*{\copy\contrup};
(38,10)*{\affr{20}8};
(38,10)*{\copy\contrup};
%--------
(-44,5)*{\afvjm2};
(-32,5)*{\afvjm2};
(-22,5)*{\afvjm2};
(-10,5)*{\afvjm2};
%----
(10,5)*{\afvjm2};
(22,5)*{\afvjm2};
(33,5)*{\afvjm2};
(44,5)*{\afvjm2};
%--------
(-44,0)*{\affr88};
(-42,2)*{a_1};
(-38,0)*{\vldots};
(-32,0)*{\affr88};
(-30,2)*{a_1};
(-22,0)*{\affr88};
(-20,2)*{\bar a_1};
(-16,0)*{\vldots};
(-10,0)*{\affr88};
(-8,2)*{\bar a_1};
%----
(0,0)*{\vldots};
%----
(10,0)*{\affr88};
(12,2)*{a_n};
(16,0)*{\vldots};
(22,0)*{\affr88};
(24,2)*{a_n};
(32,0)*{\affr88};
(34,2)*{\bar a_n};
(38,0)*{\vldots};
(44,0)*{\affr88};
(46,2)*{\bar a_n};
%--------
(-50,-2)*{\afawdm{}{}{}{}};
(-44,-5)*{\afvjm2};
(-32,-5)*{\afvjm2};
(-22,-5)*{\afvjm2};
(-10,-5)*{\afvjm2};
(-4,-2)*{\afawdm{}{}{}{}};
%----
(4,-2)*{\afawdm{}{}{}{}};
(10,-5)*{\afvjm2};
(22,-5)*{\afvjm2};
(32,-5)*{\afvjm2};
(44,-5)*{\afvjm2};
(50,-2)*{\afawdm{}{}{}{}};
%--------
(-40,-10)*{\affr{24}8};
(-40,-10)*{\copy\contrdown};
(-14,-10)*{\affr{24}8};
(-14,-10)*{\copy\contrdown};
%----
(14,-10)*{\affr{24}8};
(14,-10)*{\copy\contrdown};
(40,-10)*{\affr{24}8};
(40,-10)*{\copy\contrdown};
%--------
(-40,-16)*{\afvjm4};
(-14,-16)*{\afvjm4};
%----
(14,-16)*{\afvjm4};
(40,-16)*{\afvjm4};
}\quad,
\]
where the subflow of the former atomic flow labelled $a_i$ (resp., $\bar a_i$) is isomorphic to the subflow of the latter atomic flow labelled $a_i$ (resp., $\bar a_i$) for every $1\leq i\leq n$ and the subflows labelled with contraction vertices are unique modulo associativity of contraction by Remark~\ref{RemUniquGenContr}.
\end{proposition}
