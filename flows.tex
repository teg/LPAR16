\chapter{Definitions}

%---------------------------------------
\begin{definition}
We call the following six diagrams (\emph{atomic-flow}) \emph{labels}:
\[
\begin{array}{@{}c@{}c@{}c@{}}
      \vcenter{\afaid{}{}{}{}{}{}}&
\qquad\vcenter{\afawd{}{}{}{}}&
\qquad\vcenter{\afacd{}{}{}{}{}{}}\\
\noalign{\smallskip}
      \mbox{$\aid$ or \emph{interaction}}&
\qquad\mbox{$\awd$ or \emph{weakening}}&
\qquad\mbox{$\acd$ or \emph{contraction}}\\
\noalign{\bigskip}
      \vcenter{\afaiu{}{}{}{}{}{}}&
\qquad\vcenter{\afawu{}{}{}{}}&
\qquad\vcenter{\afacu{}{}{}{}{}{}}\\
\noalign{\smallskip}
      \mbox{$\aiu$ or \emph{cut}}&
\qquad\mbox{$\awu$ or \emph{coweakening}}&
\qquad\mbox{$\acu$ or \emph{cocontraction}}\\
\end{array}\quad.
\]
\end{definition}

\TODO{Make sure the symbols we use later are all defined here. In particular we need $\iota$ to denote edges, maybe also more.}

\newcommand{\ppl  }{{\mathchoice{\scriptstyle+}
                                {\scriptstyle+}
                                {\scriptstyle+}
                                {\scriptscriptstyle+}}}
\newcommand{\pmi  }{{\mathchoice{\scriptstyle-}
                                {\scriptstyle-}
                                {\scriptstyle-}
                                {\scriptscriptstyle-}}}
\newcommand{\card}[1]{\mathord\vert #1\mathord\vert}
\newcommand{\up}{{\mathit up}}
\newcommand{\lo}{{\mathit lo}}
%---------------------------------------
\begin{definition}
An (\emph{atomic}) \emph{flow} is a tuple $(V,E,\eta,\up,\lo)$, denoted $\phi$ or $\psi$, such that:
\begin{enumerate}
%-------------------
\item $V$ is a finite set of \emph{vertices}, denoted by $\nu$;
%-------------------
\item $E$ is a finite set of \emph{edges}, denoted by $\epsilon$;
%-------------------
\item $\eta\colon V\to\{\aid,\aiu,\awd,\awu,\acd,\acu\}$ maps vertices to their \emph{labels};
%-------------------
\item $\up\colon E\to V\cup\{\top\}$ and $\lo\colon E\to V\cup\{\bot\}$ are, respectively, the \emph{upper} and \emph{lower} maps, and $\top$ and $\bot$ are special vertices not belonging to $V$; we define, for every $\nu\in V\cup\{\top,\bot\}$, the set $L_\nu=\{\,\epsilon\mid\up(\epsilon)=\nu\,\}$ of \emph{lower edges of $\nu$}, the set $U_\nu=\{\,\epsilon\mid\lo(\epsilon)=\nu\,\}$ of \emph{upper edges of $\nu$}, and the set $E_\nu=L_\nu\cup U_\nu$ of \emph{edges of $\nu$};
%-------------------
\item if $\card S$ denotes the cardinality of set $S$, we have that
\begin{align*}
\mbox{if $\eta(\nu)=\aid$ then $\card{L_\nu}=2$ and $\card{U_\nu}=0$,}&\\
\mbox{if $\eta(\nu)=\aiu$ then $\card{L_\nu}=0$ and $\card{U_\nu}=2$,}&\\
\mbox{if $\eta(\nu)=\awd$ then $\card{L_\nu}=1$ and $\card{U_\nu}=0$,}&\\
\mbox{if $\eta(\nu)=\awu$ then $\card{L_\nu}=0$ and $\card{U_\nu}=1$,}&\\
\mbox{if $\eta(\nu)=\acd$ then $\card{L_\nu}=1$ and $\card{U_\nu}=2$,}&\\
\mbox{if $\eta(\nu)=\acu$ then $\card{L_\nu}=2$ and $\card{U_\nu}=1$;}&
\end{align*}
%-------------------
\item\label{ItAcycl} there is no sequence $\epsilon_1,\dots,\epsilon_h$ of edges of $V$ such that $\up(\epsilon_i)=\lo(\epsilon_{i+1\pmod h})$, for $1\le i\le h$;
%-------------------
\item\label{ItPol} there is a \emph{polarity assignment} $\pi\colon E\to\{\pmi,\ppl\}$ such that, for every $\nu\in V$,
\begin{enumerate}
%---------
\item if $\eta(\nu)\in\{\acd,\acu\}$ then $\pi(E_\nu)=\{\pmi\}$ or $\pi(E_\nu)=\{\ppl\}$;
%---------
\item if $\eta(\nu)\in\{\aid,\aiu\}$ then $\pi(E_\nu)=\{\pmi,\ppl\}$.
\end{enumerate}
\end{enumerate}
Given an atomic flow $\phi$, we say that the sets $L_\top=\{\epsilon_1,\dots,\epsilon_h\}$ and $U_\bot=\{\epsilon'_1,\dots,\epsilon'_k\}$ contain, respectively, the \emph{upper} and \emph{lower edges of $\phi$}.
\end{definition}

\TODO{Explain the use of colors and small numerals.}

\newcommand{\one  }{{\mathchoice{\scriptstyle\mathbf1}
                                {\scriptstyle\mathbf1}
                                {\scriptstyle\mathbf1}
                                {\scriptscriptstyle\mathbf1}}}
\newcommand{\two  }{{\mathchoice{\scriptstyle\mathbf2}
                                {\scriptstyle\mathbf2}
                                {\scriptstyle\mathbf2}
                                {\scriptscriptstyle\mathbf2}}}
\newcommand{\three}{{\mathchoice{\scriptstyle\mathbf3}
                                {\scriptstyle\mathbf3}
                                {\scriptstyle\mathbf3}
                                {\scriptscriptstyle\mathbf3}}}
\newcommand{\four }{{\mathchoice{\scriptstyle\mathbf4}
                                {\scriptstyle\mathbf4}
                                {\scriptstyle\mathbf4}
                                {\scriptscriptstyle\mathbf4}}}
\newcommand{\five }{{\mathchoice{\scriptstyle\mathbf5}
                                {\scriptstyle\mathbf5}
                                {\scriptstyle\mathbf5}
                                {\scriptscriptstyle\mathbf5}}}
\newcommand{\six  }{{\mathchoice{\scriptstyle\mathbf6}
                                {\scriptstyle\mathbf6}
                                {\scriptstyle\mathbf6}
                                {\scriptscriptstyle\mathbf6}}}
\newcommand{\seven}{{\mathchoice{\scriptstyle\mathbf7}
                                {\scriptstyle\mathbf7}
                                {\scriptstyle\mathbf7}
                                {\scriptscriptstyle\mathbf7}}}
\newcommand{\eight}{{\mathchoice{\scriptstyle\mathbf8}
                                {\scriptstyle\mathbf8}
                                {\scriptstyle\mathbf8}
                                {\scriptscriptstyle\mathbf8}}}
\newcommand{\nine }{{\mathchoice{\scriptstyle\mathbf9}
                                {\scriptstyle\mathbf9}
                                {\scriptstyle\mathbf9}
                                {\scriptscriptstyle\mathbf9}}}

\TODO{Make sure we define atomic flows modulo associativity of contraction and graph isomorphism.}

\TODO{Define \emph{paths}, \emph{$\aid$-paths} and `\emph{super paths}'.}

\TODO{Define connected component.}

\TODO{Define subflow}.