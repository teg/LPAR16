\chapter{Global Reductions}\label{chapter:GlobalReductions}

\TODO{Consider internal/external instead of global/local}

In this and the next chapter we see the second use of atomic flows: Controlling normalisation of derivations. Conventional wisdom teaches us that normalisation is a delicate property, and that a careful design of inference rules is necessary in order to obtain it. Atomic flows were designed to describe normal forms, by removing a lot of information about the inference rules, it is therefore surprising that they contain enough information to design normalisation procedures.

There are two kinds of flow reductions: global and local ones. In global reductions, the entire flow is rewritten: normally, two or more slightly altered copies of a flow are connected together. In local reductions, a bounded subflow in a flow is substituted by another subflow that fits in the context.

This chapter is dedicated to the most challenging part of normalisation: obtaining weakly streamlined derivations through global reductions. The process is non-confluent, and at first glance it increases the size of derivations exponentially. However, a second surprise was the fact that we are able to design procedures for weakly streamlining which only grow derivations quasipolynomially.

We will define several `atomic flow reductions' which can be combined in different ways in order to obtain normalisation. Since we aim to produce derivations on normal forms, and not only their atomic flows, we find it convenient to define \emph{operators} on derivations in terms of the flow reductions. It is important to note that we could have performed all the procedures purely in terms of atomic flows. The final results about derivations would follow from the `soundness' of the flow reductions. We chose to be a bit more explicit and provide the derivations directly.

\TODO{Change $f$ and $g$ into something else, as not coincide whith function names used afterwards.}

%-------------------------------------------------
\begin{definition}\label{definition:FlowReduction}
An (\emph{atomic-flow}) \emph{reduction $R$}\index{reduction} is a binary relation on the set of atomic flows, such that $\phi\mathrel{R}\psi$ if
\begin{enumerate}
\item
there is a one-to-one map, $f$, from the upper edges of $\phi$ to the upper edges of $\psi$;
\item
there is a one-to-one map, $g$, from the lower edges of $\phi$ to the lower edges of $\psi$; and
\item\label{definition:FlowReduction:item:Polarity}
for every polarity assignment $\pi$ for $\phi$, there is a polarity assignment $\pi'$ for $\psi$ such that $\pi'(f(\epsilon))=\pi(\epsilon)$ and $\pi'(g(\iota))=\pi(\iota)$, for any upper edge $\epsilon$ and any lower edge $\iota$ of $\phi'$.
\end{enumerate}
We call $\phi$ a \emph{redex} and $\psi$ a \emph{contractum} of $\mathrel R$.
\end{definition}
%---------------

%-----------------------------------------------
\begin{remark}\label{remark:LabelBijectionEdges}
Given a reduction $R$ and two atomic flows $\phi$ and $\psi$, such that $\phi\mathrel{R}\psi$, we indicate the bijections $f$ and $g$ by labeling the upper (resp., lower) edge $f(\epsilon)$ (resp., $g(\epsilon)$) of $\psi$ by $\epsilon$, for every upper (resp., lower) edge $\epsilon$ of $\phi$.
\end{remark}
%-----------

\TODO{``lifting''}

%--------------------------------------------------
\begin{definition}\label{definition:SoundRedcution}
A reduction $\mathrel{R}$ is \emph{sound}\index{reduction!sound} if, for every $\phi$ and $\psi$, such that $\phi\mathrel{R}\psi$, and for every derivation $\Phi$ with atomic flow $\phi$, there is a derivation $\Psi$ with atomic flow $\psi$ such that $\Phi$ and $\Psi$ have the same premiss and conclusion; in this case we write $\Phi\mathrel{R}\Psi$.
\end{definition}
%---------------

%-----------------
\begin{convention}
We provide constructive soundness proofs for every reduction in this thesis, so from now on, for any reduction $\mathrel{R}$, when we write $\Phi\mathrel{R}\Psi$, we mean that $\Psi$ is the derivation obtained form $\Phi$ in the soundness proof of $\mathrel{R}$.
\end{convention}
%---------------

%------------------------------------------------------------
\begin{proposition}\label{proposition:DerivationSubstitution}
Given a derivation\/ $\vlder\Phi\SKS{\beta}\alpha$, let its associated flow have shape
\[
\atomicflow{
(-5, 6)*{\afvjdm4{}{}};
( 5, 6)*{\afvjdm4{}{}};
(-1, 2)*{\aflabelleft\phi};
( 9, 2)*{\aflabelleft\psi};
(-5, 0)*{\affr88};
( 5, 0)*{\affr88};
(-5,-6)*{\afvjum4{}{}};
( 5,-6)*{\afvjum4{}{}}
}
\quad,
\]
such that $\phi$ is a connected component whose edges are each associated with atom $a$; then, for any formula $\gamma$, there exists a derivation
\[
\vlder\Psi\SKS{\beta \{a^\phi/\gamma\}}
              {\alpha\{a^\phi/\gamma\}}
\]
whose associated flow is
\[
\atomicflow{
(-17, 6)*{\afvjdm4{}{}};
( -5, 6)*{\afvjdm4{}{}};
(  5, 6)*{\afvjdm4{}{}};
(-13, 2)*{\aflabelleft{f_1(\phi)}};
( -1, 2)*{\aflabelleft{f_n(\phi)}};
(  9, 2)*{\aflabelleft\psi};
(-17, 0)*{\affr88};
(-11, 0)*{\cdots};
( -5, 0)*{\affr88};
(  5, 0)*{\affr88};
(-17,-6)*{\afvjum4{}{}};
( -5,-6)*{\afvjum4{}{}};
(  5,-6)*{\afvjum4{}{}};
}
\quad,
\]
where $n$ is the number of atom occurrences in $\gamma$; moreover, the size of\/ $\Psi$ depends linearly on the size of\/ $\Phi$ and quadratically on the size of $\gamma$.
\end{proposition}
%----------------

%-------------------------------------------------------------------------------
\begin{proof}
We can proceed by structural induction on $\Phi$. For every formula in $\Phi$ we substitute $a^\phi$ with $\gamma$. Since all the edges in $\phi$ are mapped to from $a$ (and not $\bar a$), we know that all the vertices in $\phi$ are mapped to from instances of $\acd$, $\acu$, $\awd$ and $\awu$. We substitute every instance of $\acd$, $\acu$, $\awd$ and $\awu$ where $a^\phi$ appears, by $\cod$, $\cou$, $\wed$, $\weu$, respectively, with $\gamma$ in the place of $a^\phi$. The result then follows by Lemma~\vref{lemma:GenericWeakening} and Lemma~\vref{lemma:GenericContraction}.
\end{proof}
%----------

%---------------
\begin{notation}
The derivation $\Psi$ obtained in the proof of Propostion~\vref{proposition:DerivationSubstitution} is denoted $\Phi\{a^\phi/\gamma\}$.
\end{notation}
%-------------

%-------------
\begin{remark}
The notion of substitution can be extended to allow $\phi$ to contain interaction and cut vertices, but we shall not need that in this thesis.
\end{remark}
%-----------

%-----------------
\begin{convention}
To avoid ambiguity in Definitions~\vref{definition:FourBoxes}, \vref{definition:IsolatedSubflowRemoval}, \vref{definition:PathBreaker} and \vref{definition:MultipleIsolatedSubflowsRemoval} we have establish the following convention:
Let $\boldsymbol\epsilon=\epsilon_1,\dots,\epsilon_n$, $\boldsymbol\iota=\iota_1,\dots,\iota_m$, $\boldsymbol{\epsilon'}=\epsilon'_1,\dots,\epsilon'_n$ and $\boldsymbol{\iota'}=\iota'_1,\dots,\iota'_m$, then, when we write
\[
\atomicflow
{
(-10, 0)*{\invisiblemark};
(  0, 6)*{\afvjdm4{\boldsymbol\epsilon}{}};
(  0, 0)*{\affr{14}8};
(  0, 0)*{\copy\contrup};
( -6,-6)*{\afvjdm4{f_1(\boldsymbol\epsilon)}{}};
(  0,-6)*{\cdots};
(  6,-6)*{\afvjdm4{}{f_k(\boldsymbol\epsilon)}};
( 10, 0)*{\invisiblemark};
}
\qquad\mbox{and}\qquad
\atomicflow
{
(-10, 0)*{\invisiblemark};
(  0,-6)*{\afvjum4{\boldsymbol\iota}{}};
(  0, 0)*{\affr{14}8};
(  0, 0)*{\copy\contrdown};
( -6, 6)*{\afvjum4{f_1(\boldsymbol\iota)}{}};
(  0, 6)*{\cdots};
(  6, 6)*{\afvjum4{}{f_k(\boldsymbol\iota)}};
( 10, 0)*{\invisiblemark};
}
\]
we mean
\[
\atomicflow
{
(-10, 0)*{\invisiblemark};
(  0,6)*{\afvjd4{\epsilon_1}{}};
(  0,0)*{\affr{14}8};
(  0,0)*{\copy\contrup};
( -6,-6)*{\afvjd4{f_1(\epsilon_1)}{}};
(  0,-6)*{\cdots};
(  6,-6)*{\afvjd4{}{f_k(\epsilon_1)}};
( 10, 0)*{\invisiblemark};
}
\quad\cdots\quad
\atomicflow
{
(-10, 0)*{\invisiblemark};
(  0,6)*{\afvjd4{\epsilon_n}{}};
(  0,0)*{\affr{14}8};
(  0,0)*{\copy\contrup};
( -6,-6)*{\afvjd4{f_1(\epsilon_n)}{}};
(  0,-6)*{\cdots};
(  6,-6)*{\afvjd4{}{f_k(\epsilon_n)}};
( 10, 0)*{\invisiblemark};
}
\qquad\mbox{and}\qquad
\atomicflow
{
(-10, 0)*{\invisiblemark};
(  0,-6)*{\afvjd4{\iota_1}{}};
(  0, 0)*{\affr{14}8};
(  0, 0)*{\copy\contrdown};
( -6, 6)*{\afvjd4{f_1(\iota_1)}{}};
(  0, 6)*{\cdots};
(  6, 6)*{\afvjd4{}{f_k(\iota_1)}};
( 10, 0)*{\invisiblemark};
}
\quad\cdots\quad
\atomicflow
{
(-10, 0)*{\invisiblemark};
(  0,-6)*{\afvjd4{\iota_m}{}};
(  0, 0)*{\affr{14}8};
(  0, 0)*{\copy\contrdown};
( -6, 6)*{\afvjd4{f_1(\iota_m)}{}};
(  0, 6)*{\cdots};
(  6, 6)*{\afvjd4{}{f_k(\iota_m)}};
( 10, 0)*{\invisiblemark};
}
\quad,
\]
respectively. In other words, edges are not connected in unexpected ways.
\end{convention}
%-----------

%======================================
\section{Simplifier}\label{section:Simplifier}

\TODO{Alessio: You were right about the mistake in the following definition. I added a remark.}

%--------------------------------
\newcommand{\frfb}{{\mathsf{sf}}}
\begin{definition}\label{definition:FourBoxes}
We define the reduction $\to_\frfb$ (where $\frfb$ stands for \emph{simple form}\index{simplifier!reduction}) as follows, for any atomic flows $\phi$ and $\psi$ that do not contain any interaction or cut vertices:
\[
\atomicflow{
(-8, 6)*{\afvjdm4{\boldsymbol{\epsilon_1}}{}};
( 0, 8)*{\afaidm{\boldsymbol{\epsilon_2}}{}{}{\boldsymbol{\epsilon_3}}{}{}};
( 8, 6)*{\afvjdm4{}{\boldsymbol{\epsilon_4}}};
%-
(-5, 0)*{\affr88};
(-1, 2)*{\aflabelleft\phi};
( 5, 0)*{\affr88};
( 9, 2)*{\aflabelleft\psi};
%-
(-8,-6)*{\afvjum4{\boldsymbol{\iota_1}}{}};
( 0,-8)*{\afaium{\boldsymbol{\iota_2}}{}{}{\boldsymbol{\iota_3}}{}{}};
( 8,-6)*{\afvjum4{}{\boldsymbol{\iota_4}}};
}
\quad\to_\frfb\quad
\atomicflow{
%one
(-32,0)="A";
"A"+(-8,0)*{\invisiblemark};
"A"+( 1,18.5)*{\aflabelleft{\boldsymbol{\epsilon_1}}};
"A"+( 1,18.90)*{\afvjm{2.2}};
"A"+( 1,12.25)*{\afacumexsq{}{}{}{}{}{}84};
"A"+(-3, 5)*{\afvjm2};
"A"+( 3, 8)*{\afawdm{}{}{}{}};
%
"A"+(-3, 6.5)*{\aflabelleft{f_1(\boldsymbol{\epsilon_1})}};
"A"+( 3, 6.5)*{\aflabelleft{f_1(\boldsymbol{\epsilon_2})}};
"A"+( 0, 0)*{\affr88};
"A"+( 4, 2)*{\aflabelleft{f_1(\phi)}};
"A"+(-3,-5.5)*{\aflabelleft{f_1(\boldsymbol{\iota_1})}};
"A"+( 3,-5.5)*{\aflabelleft{f_1(\boldsymbol{\iota_2})}};
%
"A"+(-3, -5)*{\afvjm2};
"A"+( 1,-12)*{\afacdmexsq{}{}{}{}{}{}84};
"A"+( 1,-17.75)*{\aflabelleft{\boldsymbol{\iota_1}}};
"A"+( 1,-19)*{\afvjm2};
"A"+( 3, -8)*{\afawum{}{}{}{}};
%two
(-16,10)="A";
"A"+(-3,8)*{\afawdm{}{}{}{}};
"A"+(14,8)*{\afaidmex{}{}{}{}{}{}{11}2};
%
"A"+(-3, 6.5)*{\aflabelleft{f_2(\boldsymbol{\epsilon_1})}};
"A"+( 3, 6.5)*{\aflabelleft{f_2(\boldsymbol{\epsilon_2})}};
"A"+( 0, 0)*{\affr88};
"A"+( 4, 2)*{\aflabelleft{f_2(\phi)}};
"A"+(-4,-5.5)*{\aflabelleft{f_2(\boldsymbol{\iota_1})}};
"A"+( 3,-5.5)*{\aflabelleft{f_2(\boldsymbol{\iota_2})}};
%
"A"+(-7,-10)*{\afcjrm8{12}};
"A"+( 3,-8)*{\afawum{}{}{}{}};
%three
(-16,-10)="A";
"A"+( -7,10)*{\afcjlm8{12}};
"A"+(  3, 8)*{\afawdm{}{}{}{}};
%
"A"+(-4, 6.5)*{\aflabelleft{f_3(\boldsymbol{\epsilon_1})}};
"A"+( 3, 6.5)*{\aflabelleft{f_3(\boldsymbol{\epsilon_2})}};
"A"+( 0, 0)*{\affr88};
"A"+( 4, 2)*{\aflabelleft{f_3(\phi)}};
"A"+(-3,-5.5)*{\aflabelleft{f_3(\boldsymbol{\iota_1})}};
"A"+( 3,-5.5)*{\aflabelleft{f_3(\boldsymbol{\iota_2})}};
%
"A"+( 14,-8)*{\afaiumex{}{}{}{}{}{}{11}2};
"A"+(-3,-8)*{\afawum{}{}{}{}};
%four
(-2, 0)="A";
(-5, 8)*{\afawdm{}{}{}{}};
( 1, 8)*{\afvjm8};
%
"A"+(-3, 6.5)*{\aflabelleft{f_4(\boldsymbol{\epsilon_1})}};
"A"+( 3, 6.5)*{\aflabelleft{f_4(\boldsymbol{\epsilon_2})}};
"A"+( 0, 0)*{\affr88};
"A"+( 4, 2)*{\aflabelleft{f_4(\phi)}};
"A"+(-3,-5.5)*{\aflabelleft{f_4(\boldsymbol{\iota_1})}};
"A"+( 3,-5.5)*{\aflabelleft{f_4(\boldsymbol{\iota_2})}};
%
( 1,-8)*{\afvjm8};
(-5,-8)*{\afawum{}{}{}{}};
%psi
(13, 16)*{\afvjdm{8}{}{\boldsymbol{\epsilon_4}}};
(13,  8)*{\afvjum{8}{}{g(\boldsymbol{\epsilon_4})}};
( 9, 13)*{\afvjm2};
( 7,  8)*{\afacdm{}{}{}{}{}{}};
( 3, 12)*{\afaidnw{}{}};
%
(10,0)="A";
"A"+(-3, 6.5)*{\aflabelright{g(\boldsymbol{\epsilon_3})}};
"A"+( 0, 0)*{\affr88};
"A"+( 4, 2)*{\aflabelleft{g(\psi)}};
"A"+(-3,-5.5)*{\aflabelright{g(\boldsymbol{\iota_3})}};
%
( 3,-14)*{\afaiunw{}{}};
( 7, -8)*{\afacum{}{}{}{}{}{}};
( 9,-13)*{\afvjm2};
(13, -8)*{\afvjdm{8}{}{g(\boldsymbol{\iota_4})}};
(13,-16)*{\afvjum{8}{}{\boldsymbol{\iota_4}}};
(18,  0)*{\invisiblemark};
}\quad.
\]
\end{definition}
%---------------

%------------------------------------------------
\begin{remark}\label{remark:RestrictionFourBoxes}
The reduction $\to_\frfb$ would still be sound if we removed the restriction on the flows $\phi$ and $\psi$ in Definition~\ref{definition:FourBoxes}. However, such a reduction would no longer correspond to the intuition described above.

\TODO{Give intuition.}

\end{remark}
%-----------

\TODO{Alessio: Added statement about size.}

%--------------------------------------------
\begin{theorem}\label{theorem:SoundFourBoxes}
Reduction\/ $\to_\frfb$ is sound; moreover if $\Phi\to_\frfb\Psi$, then the size of $\Psi$ depends at most polynomially on the size of $\Phi$.
\end{theorem}

\begin{proof}
Let $\Phi$ be a derivation with flow $\phi'$, such that $\phi'\to_\frfb\psi'$. We show that there exists a derivation $\Psi$ with flow $\psi'$ and with the same premiss and conclusion as $\Phi$. In the following, we refer to the figure in Definition~\vref{definition:FourBoxes}.

Assume all the edges in $\phi$ are mapped to from occurrences of the atoms $a_1$, $\dots$, $a_n$, and let
\[
\vlder{\Phi'}{\{\aid,\aiu\}}
{
 \vlsbr[\beta\;.\;\vlinf{}{}{\fff}{\vlsmallbrackets\vls(a_n^\phi.\bar a_n^\psi)}\;.\;\cdots\;.\;\vlinf{}{}{\fff}{\vlsmallbrackets\vls(a_1^\phi.\bar a_1^\psi)}]
}
{
 \vlsbr(\vlinf{}{}{\vlsmallbrackets\vls[a_1^\phi.\bar a_1^\psi]}{\ttt}\;.\;\cdots\;.\;\vlinf{}{}{\vlsmallbrackets\vls[a_n^\phi.\bar a_n^\psi]}{\ttt}\;.\;\alpha)
}\quad,
\]
be the $\ai$-decomposed form of $\Phi$.

\TODO{Consider using $\epsilon$'s instead of $\phi$'s above.}

We show several intermediate derivations which will be used to build $\Psi$. To make it easier to verify the atomic flow of $\Psi$, we will, through a slight misuse of notation, label the atom occurrences of the intermediate derivations to indicate what atomic flow each atom occurrence will map to, once the derivations are combined to create $\Psi$.

Consider the substitution
\[
\sigma=\{a_1^\phi/\vlsmallbrackets\vlsbr([a_1^{f_1(\phi)}.a_1^{f_2(\phi)}].[a_1^{f_3(\phi)}.a_1^{f_4(\phi)}]),\dots,a_n^\phi/\vlsmallbrackets\vlsbr([a_n^{f_1(\phi)}.a_n^{f_2(\phi)}].[a_n^{f_3(\phi)}.a_n^{f_4(\phi)}])\}\;.
\]
We can then obtain, by Proposition~\vref{proposition:DerivationSubstitution}, the derivation $\Phi'\sigma$ with atomic flow
\[
\atomicflow
{
(0,0)="A";
"A"+(-3, 6)*{\afvjdm4{f_1(\boldsymbol{\epsilon_1})}{}};
"A"+( 3, 6)*{\afvjdm4{f_1(\boldsymbol{\epsilon_2})}{}};
"A"+( 0, 0)*{\affr88};
"A"+( 4, 2)*{\aflabelleft{f_1(\phi)}};
"A"+(-3,-6)*{\afvjum4{f_1(\boldsymbol{\iota_1})}{}};
"A"+( 3,-6)*{\afvjum4{f_1(\boldsymbol{\iota_2})}{}};
%---
"A"+(14, 0)="A";
"A"+(-3, 6)*{\afvjdm4{f_2(\boldsymbol{\epsilon_1})}{}};
"A"+( 3, 6)*{\afvjdm4{f_2(\boldsymbol{\epsilon_2})}{}};
"A"+( 0, 0)*{\affr88};
"A"+( 4, 2)*{\aflabelleft{f_2(\phi)}};
"A"+(-3,-6)*{\afvjum4{f_2(\boldsymbol{\iota_1})}{}};
"A"+( 3,-6)*{\afvjum4{f_2(\boldsymbol{\iota_2})}{}};
%---
"A"+(14, 0)="A";
"A"+(-3, 6)*{\afvjdm4{f_3(\boldsymbol{\epsilon_1})}{}};
"A"+( 3, 6)*{\afvjdm4{f_3(\boldsymbol{\epsilon_2})}{}};
"A"+( 0, 0)*{\affr88};
"A"+( 4, 2)*{\aflabelleft{f_3(\phi)}};
"A"+(-3,-6)*{\afvjum4{f_3(\boldsymbol{\iota_1})}{}};
"A"+( 3,-6)*{\afvjum4{f_3(\boldsymbol{\iota_2})}{}};
%---
"A"+(14, 0)="A";
"A"+(-3, 6)*{\afvjdm4{f_4(\boldsymbol{\epsilon_1})}{}};
"A"+( 3, 6)*{\afvjdm4{f_4(\boldsymbol{\epsilon_2})}{}};
"A"+( 0, 0)*{\affr88};
"A"+( 4, 2)*{\aflabelleft{f_4(\phi)}};
"A"+(-3,-6)*{\afvjum4{f_4(\boldsymbol{\iota_1})}{}};
"A"+( 3,-6)*{\afvjum4{f_4(\boldsymbol{\iota_2})}{}};
%---
"A"+(14, 0)="A";
"A"+(-3, 6)*{\afvjdm4{}{g(\boldsymbol{\epsilon_3})}{}};
"A"+( 3, 6)*{\afvjdm4{}{g(\boldsymbol{\epsilon_4})}{}};
"A"+( 0, 0)*{\affr88};
"A"+( 4, 2)*{\aflabelleft{g(\psi)}};
"A"+(-3,-6)*{\afvjum4{}{g(\boldsymbol{\iota_3})}{}};
"A"+( 3,-6)*{\afvjum4{}{g(\boldsymbol{\iota_4})}{}};
%---
}
\qquad.
\]
For every $1\le i\le n$, there exist derivations
\[
\vlinf{}{}
{
 \vls(
  [
   a_i^{f_1(\phi)}
  \;.\;
   \vlinf{}{}{a_i^{f_2(\phi)}}{\fff}
  ]
 \;.\;
  [
   a_i^{f_3(\phi)}
  \;.\;
   \vlinf{}{}{a_i^{f_4(\phi)}}{\fff}
  ]
 )  
}
{a_i}
\qquad\mbox{and}\qquad
\vls
(
 \vlinf{}{}
 {a_i}
 {
  \vlsmallbrackets\vls[a_i^{f_1(\phi)}.a_i^{f_2(\phi)}]
 }
\;.\;
 [
  \vlinf{}{}{\ttt}{a_i^{f_3(\phi)}}
 \;.\;
  \vlinf{}{}{\ttt}{a_i^{f_4(\phi)}}
 ]
)
\quad,
\]

\TODO{`build' is too vague}

which allow us to build
\[
\vlder{\Psi_\top}{}{\alpha\sigma}{\alpha}
\qquad\mbox{and}\qquad
\vlder{\Psi_\bot}{}{\beta}{\beta\sigma}
\quad,
\]
with atomic flows
\[
\atomicflow
{
( 0, 0)*{\afacumexsq{}{}{}{}{}{}52};
(-5,-3.75)*{\aflabelleft{f_1(\boldsymbol{\epsilon_1})}};
( 5,-3.75)*{\aflabelleft{f_3(\boldsymbol{\epsilon_1})}};
(12,-2)*{\afawdm{}{}{f_2(\boldsymbol{\epsilon_1})}{}};
(20,-2)*{\afawdm{}{}{f_4(\boldsymbol{\epsilon_1})}{}};
(24, 0)*{\afvjum{12}{}{g(\boldsymbol{\epsilon_4})}};
(28, 0)*{\invisiblemark};
}
\qquad\mbox{and}\qquad
\atomicflow
{
(-9,0)*{\invisiblemark};
( 0,0)*{\afacdmexsq{}{}{}{}{}{}52};
(-5,4.25)*{\aflabelleft{f_1(\boldsymbol{\iota_1})}};
( 5,4.25)*{\aflabelleft{f_2(\boldsymbol{\iota_1})}};
(12,2)*{\afawum{}{}{f_3(\boldsymbol{\iota_1})}{}};
(20,2)*{\afawum{}{}{f_4(\boldsymbol{\iota_1})}{}};
(24,0)*{\afvjdm{12}{}{g(\boldsymbol{\iota_4})}};
}
\qquad,
\]
respectively.
Furthermore, for every $1\le i\le n$, there exist derivations
\[
\Psi_{\ttt,i}\;=\;
\vlderivation
{
 \vlin{=}{}{\vls[([\vlinf{}{}{a_i^{f_1(\phi)}}{\fff}\;.\;a_i^{f_2(\phi)}]\;.\;[\vlinf{}{}{a_i^{f_3(\phi)}}{\fff}\;.\;a_i^{f_4(\phi)}])\;.\;\vlinf{}{}{\bar a_i^{g(\psi)}}{\vls[\bar a_i.\bar a_i]}]}
 {
  \vlin{\swi}{}{\vls[\vlinf{\swi}{}{\vls\vlsmallbrackets[(a_i^{f_2(\phi)}.a_i^{f_4(\phi)}).\bar a_i]}{\vls\vlsmallbrackets(a_i^{f_2(\phi)}.[a_i^{f_4(\phi)}.\bar a_i])}\;.\;\bar a_i]}
  {
   \vlhy{\vls(\vlinf{}{}{\vls[a_i^{f_2(\phi)}.\bar a_i]}{\ttt}\;.\;\vlinf{}{}{\vls[a_i^{f_4(\phi)}.\bar a_i]}{\ttt})}
  }
 }
}
\]
and
\[
\Psi_{\fff,i}\;=\;
\vlderivation
{
 \vlin{\swi}{}{\vls[\vlinf{}{}{\fff}{\vls(a_i^{f_3(\phi)}.\bar a_i)}\;.\;\vlinf{}{}{\fff}{\vls(a_i^{f_4(\phi)}.\bar a_i)}]}
 {
  \vlin{=}{}{\vls(\vlinf{\swi}{}{\vls[a_i^{f_3(\phi)}.(a_i^{f_4(\phi)}.\bar a_i)]}{\vls([a_i^{f_3(\phi)}.a_i^{f_4(\phi)}].\bar a_i)}\;.\;\bar a_i)}
  {
   \vlhy{\vls(([\vlinf{}{}{\ttt}{a_i^{f_1(\phi)}}\;.\;\vlinf{}{}{\ttt}{a_i^{f_2(\phi)}}]\;.\;\vlsmallbrackets[a_i^{f_3(\phi)}.a_i^{f_4(\phi)}])\;.\;\vlinf{}{}{\vls(\bar a_i.\bar a_i)}{\bar a_i^{g(\psi)}})}
  }
 }
}\quad,
\]
which allow us to build
\[
\Psi_\ttt\;=\;
\vlsbr
(
 \vlder{\Psi_{\ttt,1}}{}
 {
  \vlsmallbrackets\vlsbr[a^\phi_1.\bar a^\psi_1]\sigma
 }
 {
  \ttt
 }
\;\;.\;\;\cdots\;\;.\;\;
 \vlder{\Psi_{\ttt,n}}{}
 {
  \vlsmallbrackets\vlsbr[a^\phi_n.\bar a^\psi_n]\sigma
 }
 {
  \ttt
 }
)
\qquad\mbox{and}\qquad
\Psi_\fff\;=\;
\vlsbr
[
 \vlder{\Psi_{\fff,n}}{}
 {
  \fff
 }
 {
  \vlsmallbrackets\vlsbr(a^\phi_n.\bar a^\psi_n)\sigma
 }
\;\;.\;\;\cdots\;\;.\;\;
 \vlder{\Psi_{\fff,1}}{}
 {
  \fff
 }
 {
  \vlsmallbrackets\vlsbr(a^\phi_1.\bar a^\psi_1)\sigma
 }
]
\quad,
\]
with atomic flows
\[
\atomicflow
{
(-12,0)*{\afawdm{}{}{f_1(\boldsymbol{\epsilon_2})}{}};
(-6,0)*{\afvjum8{f_2(\boldsymbol{\epsilon_2})}{}};
(-4,4)*{\afaidnw{}{}};
( 0,0)*{\afacdm{}{}{}{}{g(\boldsymbol{\epsilon_3})}{}};
( 4,4)*{\afaidnw{}{}};
( 6,0)*{\afvjum8{f_4(\boldsymbol{\epsilon_2})}{}};
(12,0)*{\afawdm{}{}{f_3(\boldsymbol{\epsilon_2})}{}};
}
\qquad\mbox{and}\qquad
\atomicflow
{
(-22, 0)*{\invisiblemark};
(-18, 0)*{\afawum{}{}{f_1(\boldsymbol{\iota_2})}{}};
(-12, 0)*{\afawum{}{}{f_2(\boldsymbol{\iota_2})}{}};
(  6, 0)*{\afvjdm8{f_4(\boldsymbol{\iota_2})}{}};
(  4,-6)*{\afaiunw{}{}};
(  0, 0)*{\afacum{}{}{}{}{g(\boldsymbol{\iota_3})}{}};
( -4,-6)*{\afaiunw{}{}};
( -6, 0)*{\afvjdm8{f_3(\boldsymbol{\iota_2})}{}};
}
\qquad,
\]
respectively.
Combining these derivations we can build
\[
\Psi\;=\;
\vlderivation
{
 \vlde{\vls[\Psi_\bot.\Psi_\fff]}{}
 {
  \beta
 }
 {
  \vlde{\Phi'\sigma}{}
  {
   \vlsbr[\beta.(a^\phi_n.\bar a^\psi_n).\cdots.(a^\phi_1.\bar a^\psi_1)]\sigma
  }
  {
   \vlde{\vls(\Psi_\ttt.\Psi_\top)}{}
   {
    \vlsbr([a^\phi_1.\bar a^\phi_1].\cdots.[a^\phi_n.\bar a^\phi_n].\alpha)\sigma
   }
   {
    \vlhy
    {
     \alpha
    }
   }
  }
 }
}\quad,
\]
with the desired atomic flow.

We know that the size of $\Phi'\sigma$ depends at most polynomially on the size of $\Phi$ by Theorem~\vref{theorem:aiDecomposedForm} and Proposition~\vref{proposition:DerivationSubstitution}, and it is straightforward to observe that the sizes of $\Psi_\ttt$, $\Psi_\top$, $\Psi_\fff$ and $\Psi_\bot$ depend at most linearly on the size of $\Phi$, so the size of $\Psi$ depends at most polynomially on the size of $\Phi$.
\end{proof}
%----------

\TODO{Alessio: I simplified this section to not talk about polarities. Changed definition of operator and proved proposition. Does not affect the following subsections (only the final Main Result section (it makes it much simpler).}

%----------------------------------
\newcommand{\Simpl}{\mathsf{Si}}
\begin{definition}\label{definition:Simplifier}
The \emph{Simplifier}\index{Simplifier}, $\Simpl$, is an operator whose arguments are distinct and pairwise non-dual atoms $a_1$, $\dots$, $a_n$ and a derivation $\Phi$, with atomic flow
\[
\atomicflow{
(-8, 6)*{\afvjdm4{}{}};
( 0, 8)*{\afaidm{}{}{}{}{}{}};
( 8, 6)*{\afvjdm4{}{}};
%-
(-5, 0)*{\affr88};
(-1, 2)*{\aflabelleft\phi};
( 5, 0)*{\affr88};
( 9, 2)*{\aflabelleft\psi};
%-
(-8,-6)*{\afvjum4{}{}};
( 0,-8)*{\afaium{}{}{}{}{}{}};
( 8,-6)*{\afvjum4{}{}};
}\quad,
\]
such that all the edges in $\phi$ are mapped to from occurrences of $a_1$, $\dots$, $a_n$ and no edges in $\psi$ are mapped to from occurrences of $a_1$, $\dots$, $a_n$.
We then define $\Simpl(\Phi,a_1,\dots,a_n)$ to be such that $\Phi\to_\frfb\Simpl(\Phi,a_1,\dots,a_n)$, where $\phi$ and $\psi$ are the flows, by the same names, shown in Definition~\vref{definition:FourBoxes}.
\end{definition}
%---------------

%------------------------------------------------------------
\begin{proposition}\label{proposition:Simplifier}
Given distinct and pairwise non-dual atoms $a_1$, $\dots$, $a_n$, and a derivation $\Phi$,
\begin{enumerate}
\item $\Simpl(\Phi,a_1,\dots,a_n)$ is in simple form with respect to $a_1$, $\dots$, $a_n$;
\item for any atom $b$, if\/ $\Phi$ is weakly streamlined with respect to $b$, then\/ $\Simpl(\Phi,a_1,\dots,a_n)$ is weakly streamlined with respect to $b$; and
\item the size of\/ $\Simpl(\Phi,a_1,\dots,a_n)$ depends at most polynomially on the size of\/ $\Phi$.
\end{enumerate}
\end{proposition}

\begin{proof}
In the following we refer to the figure in Definition~\vref{definition:FourBoxes}:
\begin{itemize}
\item by case (\ref{definition:FlowNormalForms:item:SimpleForm}) of Definition~\vref{definition:FlowNormalForms};
\item by studying the atomic flows in Definition~\ref{definition:FourBoxes} we can observe that for every path from an interaction vertex to a cut vertex in the atomic flow of $\Simpl(\Phi,a_1,\dots,a_n)$ whose edges are mapped to from occurrences of $b$, there is a path from an interaction vertex to a cut vertex in the atomic flow of $\Phi$ whose edges are mapped to from occurrences of $b$; and
\item by Theorem~\vref{theorem:SoundFourBoxes}.
\end{itemize}
\end{proof}
%----------

%======================================
\section{Isolated Subflow Removal}\label{section:IsolatedSubflowRemoval}

\newcommand{\ISR}{\mathsf{ISR}}

Given a derivation $\Phi$ in simple form with respect to an atom $a$, the operator, $\ISR$, defined in this section produces a derivation with the same premiss and conclusion as $\Phi$, which is weakly streamlined with respect to $a$.

We will see later how a derivation containing $n$ atoms can be weakly streamlined by two applications of $\Simpl$ and $n$ applications of $\ISR$. This is the most basic procedure for obtaining a weakly streamlined derivation, in particular it only deals with one atom at a time. In the following sections we will see how we can deal with several atoms in parallell.

The operator is defined in terms of the following flow reduction.

%--------------------------------
\newcommand{\fris}{{\mathsf{is}}}
\begin{definition}\label{definition:IsolatedSubflowRemoval}
We define the reduction $\to_\fris$ (where $\fris$ stands for \emph{isolated subflow}\index{Isolated Subflow Remover!reduction}) as follows, for any atomic flow $\phi$ and any connected atomic flow $\psi$ that does not contain identity or cut vertices:
\[
\atomicflow{
(-8, 6)*{\afvjdm4{\boldsymbol\epsilon}{}};
( 1, 8)*{\afaidex{}{}{}{}{}{}{6}{4}};
(-5, 0)*{\affr{8}{8}};
(-4, 2)*{\aflabelright{\phi}};
( 4, 0)*{\affr{6}{8}};
( 4, 2)*{\aflabelright{\psi}};
( 1,-8)*{\afaiuex{}{}{}{}{}{}{6}{4}};
(-8,-6)*{\afvjum4{\boldsymbol\iota}{}};
}
\quad\to_\fris\quad
\atomicflow{
( 0, 16.5)*{\afacumexsq{f_1(\boldsymbol\epsilon)}{}{}{f_2(\boldsymbol\epsilon)}{\boldsymbol\epsilon}{}{12}{4}};
( 6,  4  )*{\afvjm{12}};
( 0, 14  )*{\afawd{}{}{}{}};
(-3,  6  )*{\affr{8}{8}};
( 1,  8  )*{\aflabelleft{f_1(\phi)}};
( 0,  0  )*{\afvj4};
( 3, -6  )*{\affr{8}{8}};
( 7, -4  )*{\aflabelleft{f_2(\phi)}};
( 0,-14  )*{\afawu{}{}{}{}};
(-6, -4  )*{\afvjm{12}};
( 0,-16  )*{\afacdmexsq{f_1(\boldsymbol\iota)}{}{}{f_2(\boldsymbol\iota)}{\boldsymbol\iota}{}{12}{4}};
}\quad.
\]
\end{definition}
%---------------

%-------------------------------------------------------------
\begin{remark}\label{remark:IsolatedSubflowRemovalRestriction}
The condition on the atomic flow $\psi$ in Definition~\ref{definition:IsolatedSubflowRemoval} ensures that all the edges in $\psi$ are mapped to from occurrences of the same atom. However, the reduction would still be sound if, at the expense of a slightly more verbose soundness proof, we relaxed the condition to say that the upper and lower edge of $\psi$ belong to the same connected component.
\end{remark}

%---------------------------------------------------------
\begin{theorem}\label{theorem:SoundIsolatedSubflowRemoval}
Reduction\/ $\to_\fris$ is sound; moreover, if\/ $\Phi\to_\fris\Psi$, then the size of\/ $\Psi$ depends  polynomially on the size of\/ $\Phi$.
\end{theorem}

\begin{proof}
Let $\Phi$ be a derivation with flow $\phi'$, such that $\phi'\to_\fris\psi'$. We show that there exists a derivation $\Psi$ with flow $\psi'$ and with the same premiss and conclusion as $\Phi$. In the following, we refer to the figure in Definition~\ref{definition:IsolatedSubflowRemoval}.

Since $\psi$ is connected, we assume, by Convention~\vref{convention:AlternativeAiDecomposedForm}, that the following derivation is an $\ai$-decomposed form of $\Phi$:
\[
\vlder{\Phi'}{}
{
 \vlsbr[\beta\;.\;\vlinf{}{}{\fff}{\vls(a^\psi.\bar a)}]
}
{
 \vlsbr(\vlinf{}{}{\vls[a^\psi.\bar a]}{\ttt}\;.\;\alpha)
}\quad,
\]
for some atom $a$ and formulae $\alpha$ and $\beta$.

We obtain the two derivations $\Phi_\ttt$ and $\Phi_\fff$ from $\Phi'$ as follows:
\[
\Phi_\ttt\;=\;
\vlder{\Phi'\{a^\psi/\ttt\}}{}
{
 \vls[\beta.\bar a]
}
{
 \vls([\ttt.\bar a].\alpha)
}
\qquad\mbox{and}\qquad
\Phi_\fff\;=\;
\vlder{\Phi'\{a^\psi/\fff\}}{}
{
 \vls[\beta.(\fff.\bar a)]
}
{
 \vls(\bar a.\alpha)
}\quad.
\]
Since $\psi$ is connected and contains no identity or cut vertices, the mapping from all the occurrences $a^\psi$ to edges of $\psi$ is surjective. Hence, we know that both derivation $\Phi_\ttt$ and $\Phi_\fff$ have a flow isomorphic to $\phi$. We combine $\Phi_\ttt$ and $\Phi_\fff$ to get the desired derivation $\Psi$ with flow $\psi'$ and the same premiss and conclusion as $\Phi$:
\[
\Psi\;=\;
\vlderivation
{
 \vlin{\cod}{}
 {
  \beta
 }
 {
  \vlin{\swi}{}
  {
   \vls
   [
    \beta
   \;\;\;.\;\;\;
    \vlder{\Phi_\fff}{}
    {
     \vlsbr[\beta\;.\;(\fff\;.\;\vlinf{}{}{\ttt}{\bar a})]
    }
    {
     \vls(\bar a.\alpha)
    }
   ]
  }
  {
   \vlin{\cou}{}
   {
    \vls
    (
     \vlder{\Phi_\ttt}{}
     {
      \vls[\beta.\bar a]
     }
     {
      \vlsbr([\ttt\;.\;\vlinf{}{}{\bar a}{\fff}]\;.\;\alpha)
     }
    \;\;\;.\;\;\;
     \alpha
    )
   }
   {
    \vlhy{\alpha}
   }
  }
 }
}\quad.
\]
We know that the size of $\Phi_\ttt$ and the size of $\Phi_\fff$ depend polynomially on the size of $\Phi$ by Theorem~\vref{theorem:aiDecomposedForm} and Proposition~\vref{proposition:DerivationSubstitution}, and that the size of $\Psi$ depends at most quadratically on the size of $\alpha$ and $\beta$ by Lemma~\vref{lemma:GenericContraction}, so the size of $\Psi$ depends polynomially on the size of $\Phi$.
\end{proof}
%----------

%-------------------------------------
\begin{lemma}\label{lemma:IsolatedSubflowRemovalPaths}
Given two atomic flows $\phi$ and $\psi$, such that $\phi\to_\fris\psi$; let $f_1$ and $f_2$ be the isomorphisms and let $\nu_\ai$ be the evidenced cut (resp., interaction) vertex described in Definition~\ref{definition:IsolatedSubflowRemoval}; then, given an interaction (resp., cut) vertex $\nu$ in $\psi$, there is an interaction (resp., cut) vertex $\nu'$ in $\phi$, such that
\begin{itemize}
\item $\nu=f_1(\nu')$ or $\nu=f_2(\nu')$;
\item if there is a path from $\nu$ to $\bot$ (resp., $\top$) in $\psi$, then there is a path from $\nu'$ to $\bot$ (resp., $\top$) in $\phi$; and
\item if there is a cut (resp., interaction) vertex $\hat\nu$ in $\psi$, such that there is a path from $\nu$ to $\hat\nu$ in $\psi$, then there is a cut (resp., interaction) vertex $\hat\nu'$ in $\phi$, such that $\hat\nu=f_1(\nu')$ or $\hat\nu=f_2(\nu')$, or $\hat\nu'=\nu_\ai$; and there is a path from $\nu'$ to $\hat\nu'$ in $\phi$.
\end{itemize}
\end{lemma}

\begin{proof}
In the following we refer to the figure in Definition~\ref{definition:IsolatedSubflowRemoval}:
\begin{itemize}
 \item by definition;
 \item any path from $\nu$ to $\top$ (resp., $\bot$) in $\psi$ must contain an edge $\epsilon$, such that, for some upper (resp., lower) edge $\epsilon'$ of $\phi$, $f_1(\epsilon')=\epsilon$ or $f_2(\epsilon')=\epsilon$. Hence, there is a path from $\nu'$ to $\top$ (resp., $\bot$) in $\phi$; and
 \item we have to consider two cases:
 \begin{itemize}
  \item $\nu=f_1(\nu')$ and $\hat\nu=f_1(\hat\nu')$, or $\nu=f_2(\nu')$ and $\hat\nu=f_2(\hat\nu')$, then there is a path from $\nu'$ to $\hat\nu'$ in $\phi$; or
  \item $\nu=f_1(\nu')$ and $\hat\nu=f_2(\hat\nu')$ (resp., $\nu=f_2(\nu')$ and $\hat\nu=f_1(\hat\nu')$),then there is a path from $\nu'$ to $\nu_\ai$ in $\phi$.
 \end{itemize}
\end{itemize}
\end{proof}
%----------

%------------------------------
\begin{definition}\label{definition:IsolatedSubflowRemover}
The \emph{Isolated Subflow Remover}\index{Isolated Subflow Remover!operator}, $\ISR$, is an operator whose arguments are an atom $a$ and a derivation $\Phi$ that is in simple form with respect to $a$. If $\Phi$ is weakly streamlined with respect to $a$, then $\ISR(\Phi,a)=\Phi$; otherwise, consider the following  $\ai$-decomposed form of $\Phi$:
\[
\vlder{\Phi'}{}
{
 \vlsbr
 [
  \beta
 \;.\;
  \vlinf{}{}
  {
   \fff
  }
  {
   \vls(a^{\psi'}.\bar a)
  }
 \;.\;\cdots\;.\;
  \vlinf{}{}
  {
   \fff
  }
  {
   \vls(a^{\psi'}.\bar a)
  }
 ]
}
{
 \vlsbr
 (
  \vlinf{}{}
  {
   \vls[a^{\psi'}.\bar a]
  }
  {
   \ttt
  }
 \;.\;\cdots\;.\;
  \vlinf{}{}
  {
   \vls[a^{\psi'}.\bar a]
  }
  {
   \ttt
  }
 \;.\;
  \alpha
 )
}\quad,
\]
with atomic flow
\[
\atomicflow{
(-8, 6)*{\afvjdm4{\boldsymbol\epsilon}{}};
( 1, 8)*{\afaidmex{}{}{}{}{}{}{6}{4}};
(-5, 0)*{\affr{8}{8}};
(-1, 2)*{\aflabelleft{\phi'}};
( 4, 0)*{\affr{6}{8}};
( 7, 2)*{\aflabelleft{\psi'}};
( 1,-8)*{\afaiumex{}{}{}{}{}{}{6}{4}};
(-8,-6)*{\afvjum4{\boldsymbol\iota}{}};
}
\quad,
\]
where $\psi'$ is the juxtaposition of all the isolated subflows mapped to from occurrences of $a$ in $\Phi$. Consider the derivation
\[
\Psi\;=\;
\vlder{\Phi'}{}
{
 \vlsbr
 [
  \beta
 \;\;\;.\;\;\;
  \vlderivation
  {
   \vlin{}{}
   {
    \fff
   }
   {
    \vlde{}{\{\cod\}}
    {
     \vls(a.\bar a)
    }
    {
     \vlhy
     {
      \vls[(a.\bar a).\cdots.(a.\bar a)]
     }
    }
   }
  }
 ]
}
{
 \vlsbr
 (
  \vlderivation
  {
   \vlde{}{\{\cou\}}
   {
    \vls([a.\bar a].\cdots.[a.\bar a])
   }
   {
    \vlin{}{}
    {
     \vls[a.\bar a]
    }
    {
     \vlhy
     {
      \ttt
     }
    }
   }
  }
 \;\;\;.\;\;\;
  \alpha
 )
}\quad,
\]
with atomic flow
\[
\psi''\;=\;
\atomicflow{
(-8, 11.5)*{\afvjdm{15}{\boldsymbol\epsilon}{}};
( 4, 18)*{\afaidex{}{}{}{}{}{}{12}{4}};
%-
(-2, 10)*{\affr68};
(-2, 10)*{\copy\contrup};
(-2,  5)*{\afvjm2};
(-3,  0)*{\affr{14}{30}};
( 4, 13)*{\aflabelleft{\phi}};
(-5,  0)*{\affr{8}{8}};
(-1,  2)*{\aflabelleft{\phi'}};
(-2, -5)*{\afvjm2};
(-2,-10)*{\affr68};
(-2,-10)*{\copy\contrdown};
%-
(10, 10)*{\affr68};
(10, 10)*{\copy\contrup};
(10,  5)*{\afvjm2};
(11,  0)*{\affr{10}{30}};
(16, 13)*{\aflabelleft{\psi}};
(10,  0)*{\affr{6}{8}};
(13,  2)*{\aflabelleft{\psi'}};
(10, -5)*{\afvjm2};
(10,-10)*{\affr68};
(10,-10)*{\copy\contrdown};
%-
( 4,-18)*{\afaiuex{}{}{}{}{}{}{12}{4}};
(-8,-11.5)*{\afvjum{15}{\boldsymbol\iota}{}};
}\quad.
\]
We then define $\ISR(\Phi,a)$ to be such that $\Psi\to_\fris\ISR(\Phi,a)$, where $\phi$ and $\psi$ are the flows, by the same names, shown in Definition~\vref{definition:IsolatedSubflowRemoval}.
\end{definition}
%---------------

%------------------------------------------------------------
\begin{proposition}\label{proposition:IsolatedSubflowRemover}
Given an atom $a$ and a derivation $\Phi$ that is in simple form with respect to $a$,
\begin{enumerate}
\item $\ISR(\Phi,a)$ is weakly streamlined with respect to $a$;
\item for any atom $b$,
\begin{itemize}
\item if $\Phi$ is weakly streamlined with respect to $b$, then $\ISR(\Phi,a)$ is weakly streamlined with respect to $b$, and
\item if $b$ is not the dual of $a$ and $\Phi$ is in simple form with respect to $b$, then $\ISR(\Phi,a)$ is in simple form with respect to $b$; and
\end{itemize}
\item the size of\/ $\ISR(\Phi,a)$ depends polynomially on the size of\/ $\Phi$.
\end{enumerate}
\end{proposition}

\TODO{Alessio said: `\emph{This should be OK despite a possible mistake in 6.1.11. In fact, I think it would be better to justify point 2 by something else than (whatever becomes of) Lemma 6.1.11. The reason is that independent copies are made of the flows of b's, and then they are joined by (co)contractions. (Right?)}'.}

\begin{proof}
If $\Phi$ is weakly streamlined with respect to $a$, the result is trivial. Assume $\Phi$ is not weakly streamlined with respect to $a$, and let $\phi$, $\psi$, $\phi'$, $\psi'$ and $\psi''$ be the atomic flows given in Definition~\ref{definition:IsolatedSubflowRemover}, then
\begin{enumerate}
\item by definition there is no path in $\phi$ from an interaction to a cut vertex whose edges are mapped to from instances of $a$. By Lemma~\vref{lemma:IsolatedSubflowRemovalPaths}, we know that if there is a path from an interaction to a cut vertex in the atomic flow of $\ISR(\Phi,a)$ whose edges are mapped to from instances of $a$, then there must be a path from an interaction to a cut vertex in $\phi$ whose edges are mapped to from instances $a$. Hence, the statement follows by contradiction;
\item
\begin{itemize}
 \item if there is a path from an interaction (resp., cut) vertex in the atomic flow of $\ISR(\Phi,a)$ whose edges are mapped to from instances of $b$, then, by \newline Lemma~\vref{lemma:IsolatedSubflowRemovalPaths}, there is a path from an interaction (resp., cut) vertex in $\phi$, so also in $\phi'$, whose edges are mapped to from instances of $b$. Hence, the statement follows by contradiction; and
 \item if there is an interaction (reps., cut) vertex $\nu$ and a cut (resp., interaction) vertex $\hat\nu$ in the atomic flow of $\ISR(\Phi,a)$ such that there is a path from $\nu$ to $\hat\nu$ and a path from $\nu$ to $\bot$ (resp., $\top$), both of whose edges are mapped to from instances of $b$, then, by Lemma~\vref{lemma:IsolatedSubflowRemovalPaths}, there is an interaction (reps., cut) vertex $\nu'$ and a cut (resp., interaction) vertex $\hat\nu'$ in $\phi$ such that there is a path from $\nu$ to $\hat\nu$ and a path from $\nu$ to $\bot$ (resp., $\top$), both of whose edges are mapped to from instances of $b$. Furthermore, since we can assume that $b$ is not $a$ or $\bar a$, $\phi$ restricted to $b$ equals $\phi'$ restricted to $b$. Hence, the statement follows by contradiction.
\end{itemize}
\item the statement follows by Theorem~\vref{theorem:SoundIsolatedSubflowRemoval}.
\end{enumerate}
\end{proof}
%----------

\TODO{Copy example from AF1.}

%======================================
\section{Path Breaker}\label{section:PathBreaker}

\newcommand{\PB}{\mathsf{PB}}

\TODO{Make sure we don't join $a$ with $\bar a$.}

Given a derivation $\Phi$ and an atom $a$, the operator, $\PB$, defined in this section produces a derivation with the same premiss and conclusion as $\Phi$, which is weakly streamlined with respect to both $a$ and $\bar a$. This operator is a strict improvement over $\ISR$, since it does not require the input derivation to be in simple form, and it deals with the dual atoms in parallell. We will see later how a derivation containing $n$ atoms can be weakly streamlined by $n/2$ applications of $\PB$.

The operator is defined in terms of the following flow reduction.

%--------------------------------
\newcommand{\frpb}{{\mathsf{pb}}}
\begin{definition}\label{definition:PathBreaker}
We define the reduction $\to_\frpb$ (where $\frpb$ stands for \emph{path breaker}\index{Path Breaker!reduction}) as follows, for any atomic flows $\phi$ and $\psi$:
\[
\atomicflow
{
(-8, 7)*{\afvjdm{6}{\boldsymbol\epsilon}{}};
( 0, 8)*{\afaid{}{}{}{}{}{}};
( 8, 7)*{\afvjdm{6}{}{\boldsymbol{\epsilon'}}};
(-5, 0)*{\affr{8}{8}};
(-4, 2)*{\aflabelright\phi};
%---
( 5, 0)*{\affr{8}{8}};
( 6, 2)*{\aflabelright{\psi}};
( 8,-7)*{\afvjum{6}{}{\boldsymbol{\iota'}}};
( 0,-8)*{\afaiu{}{}{}{}{}{}};
(-8,-7)*{\afvjum{6}{\boldsymbol\iota}{}};
}
\quad\to_\frpb\quad
\atomicflow
{
%%%%% RED %%%%%
(0,-20)="D";
(0,-10)="Dhalf";
%% contractions
"D"+"D"="A";
%left
"A"+(-14,-15.5)-"D"*{\afvjmcol{23}{Red}};
"A"+(-11,-17)*{\afvjumcol{4}{\boldsymbol\iota}{}{Red}};
%right
"A"+(11,-11.5)-"Dhalf"*{\afvjmcol{11}{Red}};
"A"+(14,-15.5)-"D"*{\afvjmcol{23}{Red}};
"A"+(11,-17)*{\afvjumcol{4}{}{\boldsymbol{\iota'}}{Red}};
% top boxes
(0,0)="A";
"A"+(-11,-14)*{\afcjrmcol{6}{20}{Red}};
"A"+(11,-14)*{\afcjlmcol{6}{20}{Red}};
"A"+( 0,  8)*{\afaidcol{}{}{}{}{}{}{Red}{Red}};
"A"+(-2, -8)*{\afawucol{}{}{}{}{}{Red}};
% join one
"A"+(2,-10)*{\afvjcol{12}{Red}};
% middle boxes
"A"+"D"="A";
"A"+(9.5,-10)*{\afcjlmcol{3}{12}{Red}};
"A"+( 2,-8)*{\afawucol{}{}{}{}{}{Red}};
%%%%% GREEN %%%%%
%% cocontractions
(0,0)="A";
%left
"A"+(-11,17)*{\afvjdmcol{4}{\boldsymbol\epsilon}{}{OliveGreen}};
"A"+"D"+(-14,15.5)*{\afvjmcol{23}{OliveGreen}};
"A"+"Dhalf"+(-11,11.5)*{\afvjmcol{11}{OliveGreen}};
%right
"A"+(11,17)*{\afvjdmcol{4}{}{\boldsymbol{\epsilon'}}{OliveGreen}};
"A"+"D"+(14,15.5)*{\afvjmcol{23}{OliveGreen}};
% middle boxes
"A"+"D"="A";
"A"+(-9.5,10)*{\afcjlmcol{3}{12}{OliveGreen}};
"A"+(-2, 8)*{\afawdcol{}{}{}{}{}{OliveGreen}};
% join two
"A"+(-2,-10)*{\afvjcol{12}{OliveGreen}};
% bottom boxes
"A"+"D"="A";
"A"+(-11,14)*{\afcjlmcol{6}{20}{OliveGreen}};
"A"+(11,14)*{\afcjrmcol{6}{20}{OliveGreen}};
"A"+( 0,-8)*{\afaiucol{}{}{}{}{}{}{OliveGreen}{OliveGreen}};
"A"+( 2, 8)*{\afawdcol{}{}{}{}{}{OliveGreen}};
%%%%% BLACK %%%%%
%% cocontractions
(0,0)="A";
%left
"A"+(-8,5.5)*{\afvjm3};
"A"+(-11,11)*{\affr88};
"A"+(-11,11)*{\copy\contrup};
%right
"A"+(8,5.5)*{\afvjm3};
"A"+"Dhalf"+(11,11.5)*{\afvjm{11}};
"A"+(11,11)*{\affr88};
"A"+(11,11)*{\copy\contrup};
%% contractions
"D"+"D"="A";
%left
"A"+(-11,-11.5)-"Dhalf"*{\afvjm{11}};
"A"+(-8,-5.5)*{\afvjm3};
"A"+(-11,-11)*{\affr88};
"A"+(-11,-11)*{\copy\contrdown};
%right
"A"+(8,-5.5)*{\afvjm3};
"A"+(11,-11)*{\affr88};
"A"+(11,-11)*{\copy\contrdown};
% top boxes
(0,0)="A";
"A"+(-8,5.9)*{\aflabelright{f_1(\boldsymbol\epsilon)}};
"A"+(-5,  0)*{\affr{8}{8}};
"A"+(-6,  2)*{\aflabelright{f_1(\phi)}};
"A"+(-8,-5.3)*{\aflabelright{f_1(\boldsymbol\iota)}};
%
"A"+(8,5.9)*{\aflabelleft{g_1(\boldsymbol{\epsilon'})}};
"A"+( 5,  0)*{\affr{8}{8}};
"A"+( 4,  2)*{\aflabelright{g_1(\psi)}};
"A"+(8,-5.3)*{\aflabelleft{g_1(\boldsymbol{\iota'})}};
% middle boxes
"A"+"D"="A";
"A"+(9.5,10)*{\afcjrm{3}{12}};
"A"+(-9.5,-10)*{\afcjrm{3}{12}};
"A"+(-8,5.9)*{\aflabelright{f_2(\boldsymbol\epsilon)}};
"A"+(-5, 0)*{\affr{8}{8}};
"A"+(-6, 2)*{\aflabelright{f_2(\phi)}};
"A"+(-8,-5.3)*{\aflabelright{f_2(\boldsymbol\iota)}};
%
"A"+(8,5.9)*{\aflabelleft{g_2(\boldsymbol{\epsilon'})}};
"A"+( 5, 0)*{\affr{8}{8}};
"A"+( 4, 2)*{\aflabelright{g_2(\psi)}};
"A"+(8,-5.3)*{\aflabelleft{g_2(\boldsymbol{\iota'})}};
% bottom boxes
"A"+"D"="A";
"A"+(-8,5.9)*{\aflabelright{f_3(\boldsymbol\epsilon)}};
"A"+(-5, 0)*{\affr{8}{8}};
"A"+(-6, 2)*{\aflabelright{f_3(\phi)}};
"A"+(-8,-5.3)*{\aflabelright{f_3(\boldsymbol\iota)}};
%
"A"+(8,5.9)*{\aflabelleft{g_3(\boldsymbol{\epsilon'})}};
"A"+( 5, 0)*{\affr{8}{8}};
"A"+( 4, 2)*{\aflabelright{g_3(\psi)}};
"A"+(8,-5.3)*{\aflabelleft{g_3(\boldsymbol{\iota'})}};
}\quad,
\]
where the evidenced interaction and cut vertices belong to the same connected component.
\end{definition}
%---------------

%----------------------------------------------
\begin{theorem}\label{theorem:SoundPathBreaker}
Reduction $\to_\frpb$ is sound; moreover, if\/ $\Phi\to_\frpb\Psi$, then the size of $\Psi$ depends polynomially on the size of $\Phi$.
\end{theorem}

\begin{proof}
Let $\Phi$ be a derivation with flow $\phi'$, such that $\phi'\to_\frpb\psi'$. We show that there exists a derivation $\Psi$ with flow $\psi'$ and with the same premiss and conclusion as $\Phi$. In the following, we refer to the figure in Definition~\vref{definition:PathBreaker}.

Since the evidenced interaction and cut vertices belong to the same connected component, we assume, by Convention~\vref{convention:AlternativeAiDecomposedForm}, that the following derivation is an $\ai$-decomposed form of $\Phi$:
\[
\vlder{\Phi'}{}
{
 \vlsbr[\beta\;.\;\vlinf{}{}{\fff}{\vls(a^\phi.\bar a^\psi)}]
}
{
 \vlsbr(\vlinf{}{}{\vls[a^\phi.\bar a^\psi]}{\ttt}\;.\;\alpha)
}\quad,
\]
for some atom $a$ and formulae $\alpha$ and $\beta$.

We combine three copies of $\Phi'$ to obtain the desired derivation $\Psi$ with flow $\psi'$ and the same premiss and conclusion as $\Phi$:

\newbox\DeltaTopK
\setbox\DeltaTopK=
\hbox{$
\vlder{\Phi'}{}
{
 \vlsbr[\beta\;.\;(\vlinf{}{}{\ttt}{a^{f_1(\phi)}}\;.\;\bar a^{g_1(\psi)})]
}
{
 \vlsbr(\vlinf{}{}{\vls[a^{f_1(\phi)}.\bar a^{g_1(\psi)}]}{\ttt}\;.\;\alpha)
}
$}
\newbox\DeltaK
\setbox\DeltaK=
\hbox{$
\vlder{\Phi'}{}
{
 \vlsbr[\beta\;.\;(a^{f_2(\phi)}\;.\;\vlinf{}{}{\ttt}{\bar a^{g_2(\psi)}})]
}
{
 \vlsbr([\vlinf{}{}{a^{f_2(\phi)}}{\fff}\;.\;\bar a^{g_2(\psi)}]\;.\;\alpha)
}
$}
\newbox\DeltaBotK
\setbox\DeltaBotK=
\hbox{$
\vlder{\Phi'}{}
{
 \vlsbr[\beta\;.\;\vlinf{}{}{\fff}{\vls(a^{f_3(\phi)}.\bar a^{g_3(\psi)})}]
}
{
 \vlsbr([a^{f_3(\phi)}\;.\;\vlinf{}{}{\bar a^{g_3(\psi)}}{\fff}]\;.\;\alpha)
}
$}
\[
\Psi\quad=\quad
\vlderivation
{
 \vlin{\cod}{}{\beta}
 {
  \vlin{\swi}{}
  {
   \vls
   [
    \vlinf{\cod}{}{\beta}{\vls[\beta.\beta]}
   \;\;\;\;.\;\;\;\;
    \box\DeltaBotK
   ]
  }
  {
   \vlin{\swi}{}
   {
    \vls
    (
%     \vlinf{\swi}{}
%     {
%      \vls
      [
       \beta
      \;\;\;\;.\;\;\;\;
       \box\DeltaK
      ]
%     }
%     {
%      \vls(\vlsmallbrackets[\beta.\bar a^\psi].\alpha)
%     }
    \;\;\;\;\;.\;\;\;\;\;
     \alpha
    )   
   }
   {
    \vlin{\cod}{}
    {
     \vls
     (
      \box\DeltaTopK
     \;\;\;\;.\;\;\;\;
      \vlinf{\cou}{}{\vls(\alpha.\alpha)}{\alpha}
     )
    }
    {
     \vlhy{\alpha}
    }
   }
  }
 } 
}\qquad.
\]
We know that the size of $\Phi'$ depends at most cubically on the size of $\Phi$ by \newline Theorem~\vref{theorem:aiDecomposedForm}, and that the size of $\Psi$ depends at most quadratically on the size of $\alpha$ and $\beta$ by Lemma~\vref{lemma:GenericContraction}, so $\Psi$ depends polynomially on the size of $\Phi$.
\end{proof}
%----------

%-------------------------------------
\begin{lemma}\label{lemma:PathBreaker}
Given two atomic flows $\phi$ and $\psi$, such that $\phi\to_\frpb\psi$; in the following we refer to Definition~\ref{definition:PathBreaker}:
\begin{itemize}
\item let $f_1$, $f_2$, $f_3$, $g_1$, $g_2$ and $g_3$ be the evidenced isomorphisms;
\item let $\nu'_\aiu$ be the evidenced cut and let $\nu'_\aid$ be the evidenced interaction vertex in the redex; and
\item let $\nu_\aiu$ be the evidenced cut and let $\nu_\aid$ be the evidenced interaction vertex in the contractum,
\end{itemize}
then, given an interaction (resp., cut) vertex $\nu$ in $\psi$, there is an interaction (resp., cut) vertex $\nu'$ in $\phi$, such that
\begin{itemize}
\item for some $1\le i\le 3$, $\nu=f_i(\nu')$ or $\nu=g_i(\nu')$, or $\nu=\nu_\aid$ and $\nu'=\nu'_\aid$ (resp., $\nu=\nu_\aiu$ and $\nu'=\nu'_\aiu$);
\item if there is a path from $\nu$ to $\bot$ (resp., $\top$) in $\psi$, then there is a path from $\nu'$ to $\bot$ (resp., $\top$) in $\phi$; and
\item if there is a cut (resp., interaction) vertex $\hat\nu$ in $\psi$, such that there is a path from $\nu$ to $\hat\nu$ in $\psi$, then there is a cut (resp., interaction) vertex $\hat\nu'$ in $\phi$, such that, for some $1\le i\le 3$, $\hat\nu=f_i(\nu')$ or $\hat\nu=g_i(\nu')$, or $\hat\nu=\nu_\aiu$ and $\hat\nu'=\nu'_\aiu$ (resp., $\hat\nu=\nu_\aid$ and $\hat\nu'=\nu'_\aid$); and there is a path from $\nu'$ to $\hat\nu'$ in $\phi$.
\end{itemize}
\end{lemma}

\TODO{Check commens from Alessio}

\begin{proof}
We consider each case separately:
\begin{itemize}
  \item by definition;
  \item any path from $\nu$ to $\bot$ (resp., $\top$) in $\psi$ must contain an edge $\epsilon$, such that, for some lower (resp., upper) edge $\epsilon'$ of $\phi$ and some $1\le i\le 3$, $f_i(\epsilon')=\epsilon$ or $g_i(\epsilon')=\epsilon$. Hence, there is a path from $\nu'$ to $\bot$ (resp., $\top$) in $\phi$; and
 \item we have to consider two cases:
 \begin{itemize}
  \item for some $1\le i\le 3$, $\nu=f_i(\nu')$ and $\hat\nu=f_i(\hat\nu')$, or $\nu=g_i(\nu')$ and $\hat\nu=g_i(\hat\nu')$, then there is a path from $\nu'$ to $\hat\nu'$ in $\phi$; or
  \item $\nu=g_1(\nu')$ and $\hat\nu=g_2(\hat\nu')$, or $\nu=f_2(\nu')$ and $\hat\nu=f_3(\hat\nu')$ (resp., $\nu=g_2(\nu')$ and $\hat\nu=g_1(\hat\nu')$, or $\nu=f_3(\nu')$ and $\hat\nu=f_2(\hat\nu')$), then there is a path from $\nu'$ to $\nu'_\aiu$ (resp., $\nu'_\aid$) in $\phi$.
 \end{itemize}
\end{itemize}
\end{proof}
%----------

%----------------------------
\begin{definition}\label{definition:DerPathBreaker}
The \emph{Path Breaker}\index{Path Breaker!operator}, $\PB$, is an operator whose arguments are an atom $a$ and a derivation $\Phi$. If $\Phi$ is weakly streamlined with respect to both $a$ and $\bar a$, then $\PB(\Phi,a)=\Phi$; otherwise, consider the following $\ai$-decomposed form of $\Phi$:
\[
\vlder{\Phi'}{}
{
 \vlsbr
 [
  \beta
 \;.\;
  \vlinf{}{}
  {
   \fff
  }
  {
   \vls(a^\psi.\bar a)
  }
 \;.\;\cdots\;.\;
  \vlinf{}{}
  {
   \fff
  }
  {
   \vls(a^\psi.\bar a)
  }
 ]
}
{
 \vlsbr
 (
  \vlinf{}{}
  {
   \vls[a^\psi.\bar a]
  }
  {
   \ttt
  }
 \;.\;\cdots\;.\;
  \vlinf{}{}
  {
   \vls[a^\psi.\bar a]
  }
  {
   \ttt
  }
 \;.\;
  \alpha
 )
}\quad,
\]
with atomic flow
\[
\phi''\;=\;
\atomicflow
{
(-8, 7)*{\afvjdm{6}{\boldsymbol\epsilon}{}};
( 0, 8)*{\afaidm{}{}{}{}{}{}};
( 8, 7)*{\afvjdm{6}{}{\boldsymbol{\epsilon'}}};
(-5, 0)*{\affr{8}{8}};
(-4, 2)*{\aflabelright{\phi'}};
%---
( 5, 0)*{\affr{8}{8}};
( 6, 2)*{\aflabelright{\psi'}};
( 8,-7)*{\afvjum{6}{}{\boldsymbol{\iota'}}};
( 0,-8)*{\afaium{}{}{}{}{}{}};
(-8,-7)*{\afvjum{6}{\boldsymbol\iota}{}};
}
\quad,
\]
such that occurrences of $a$ do not appear in an interaction or cut instance in $\Phi'$. Consider the derivation
\[
\Psi\;=\;
\vlder{\Phi'}{}
{
 \vlsbr
 [
  \beta
 \;\;\;.\;\;\;
  \vlderivation
  {
   \vlin{}{}
   {
    \fff
   }
   {
    \vlde{}{\{\cod\}}
    {
     \vls(a.\bar a)
    }
    {
     \vlhy
     {
      \vls[(a.\bar a).\cdots.(a.\bar a)]
     }
    }
   }
  }
 ]
}
{
 \vlsbr
 (
  \vlderivation
  {
   \vlde{}{\{\cou\}}
   {
    \vls([a.\bar a].\cdots.[a.\bar a])
   }
   {
    \vlin{}{}
    {
     \vls[a.\bar a]
    }
    {
     \vlhy
     {
      \ttt
     }
    }
   }
  }
 \;\;\;.\;\;\;
  \alpha
 )
}\quad,
\]
with atomic flow
\[
\psi''\;=\;
\atomicflow{
(-11, 11.5)*{\afvjdm{15}{\boldsymbol\epsilon}{}};
( 13, 11.5)*{\afvjdm{15}{}{\boldsymbol{\epsilon'}}};
(  1, 18)*{\afaidex{}{}{}{}{}{}{12}{4}};
%-
( -5, 10)*{\affr68};
( -5, 10)*{\copy\contrup};
( -5,  5)*{\afvjm2};
( -8,  0)*{\affr{8}{8}};
( -4,  2)*{\aflabelleft{\phi'}};
( -5, -5)*{\afvjm2};
( -5,-10)*{\affr68};
( -5,-10)*{\copy\contrdown};
%-
(  7, 10)*{\affr68};
(  7, 10)*{\copy\contrup};
(  7,  5)*{\afvjm2};
( 10,  0)*{\affr{8}{8}};
( 14,  2)*{\aflabelleft{\psi'}};
(  7, -5)*{\afvjm2};
(  7,-10)*{\affr68};
(  7,-10)*{\copy\contrdown};
%-
(  1,-18)*{\afaiuex{}{}{}{}{}{}{12}{4}};
(-11,-11.5)*{\afvjum{15}{\boldsymbol\iota}{}};
( 13,-11.5)*{\afvjum{15}{}{\boldsymbol{\iota'}}};
%----
( -6,  0)*{\affr{14}{30}};
(  1, 13)*{\aflabelleft{\phi}};
(9.5,  0)*{\affr{13}{30}};
( 16, 13)*{\aflabelleft{\psi}};
}\quad.
\]
We then define $\PB(\Phi,a)$ to be such that $\Psi\to_\frpb\PB(\Phi,a)$, where $\phi$ and $\psi$ are the flows, by the same names, shown in Definition~\vref{definition:PathBreaker}.
\end{definition}
%---------------

%------------------------------------------------------------
\begin{proposition}\label{proposition:PathBreaker}
Given an atom $a$ and a derivation $\Phi$,
\begin{enumerate}
\item $\PB(\Phi,a)$ is weakly streamlined with respect to both $a$ and $\bar a$;
\item for any atom $b$, if $\Phi$ is weakly streamlined with respect to $b$, then $\PB(\Phi,a)$ is weakly streamlined with respect to $b$; and
\item the size of\/ $\PB(\Phi,a)$ depends polynomially on the size of\/ $\Phi$.
\end{enumerate}
\end{proposition}

\begin{proof}
If $\Phi$ is weakly streamlined with respect to both $a$ and $\bar a$, the result is trivial. Assume $\Phi$ is not weakly streamlined with respect to both $a$ and $\bar a$, and let $\phi$, $\psi$, $\phi'$, $\psi'$, $\phi''$ and $\psi''$ be the atomic flows given in Definition~\ref{definition:DerPathBreaker} and let $\nu_\aid$ (resp., $\nu_\aiu$) be the evidenced interaction (resp., cut) vertex in $\psi''$, then
\begin{enumerate}
\item by Definition~\ref{definition:DerPathBreaker} all the paths from an interaction (resp., cut) vertex whose edges are mapped to from instances of $a$ or $\bar a$ must start from $\nu_\aid$ (resp., $\nu_\aiu$). In Definition~\ref{definition:PathBreaker} we have colored these edges in red (resp., green). Since the red and the green edges never coincide, there are no paths from $\nu_\aid$ to $\nu_\aiu$;
\item if there is a path from an interaction (resp., cut) vertex in the atomic flow of $\PB(\Phi,a)$ whose edges are mapped to from instances of $b$, then, by Lemma~\vref{lemma:PathBreaker}, there is a path from an interaction (resp., cut) vertex in $\phi$ or $\psi$, so also in $\phi'$ or $\psi'$, whose edges are mapped to from instances of $b$. Hence, the statement follows by contradiction; and
\item the statement follows by Theorem~\vref{theorem:SoundPathBreaker}.
\end{enumerate}
\end{proof}
%----------

%-----------------------------------------
\begin{example}\label{example:PathBreaker}
Given a derivation $\Phi$ where the atoms $a_1$ and $a_2$ occur, such that the atomic flow associated with $\Phi$ is
\[
\atomicflow
{
(-2,8)*{\afaid{}{}{}{}{}{}};
(4,6)*{\afvjm4};
(0,0)*{\affr{10}8};
(5,2)*{\aflabelleft{\phi_1}};
(-4,-6)*{\afvjm4};
(2,-8)*{\afaiu{}{}{}{}{}{}};
}\quad
\atomicflow
{
(-2,8)*{\afaid{}{}{}{}{}{}};
(4,6)*{\afvjm4};
(0,0)*{\affr{10}8};
(5,2)*{\aflabelleft{\phi_2}};
(-4,-6)*{\afvjm4};
(2,-8)*{\afaiu{}{}{}{}{}{}};
}\quad
\atomicflow
{
(0,6)*{\afvjm4};
(0,0)*{\affr88};
(4,2)*{\aflabelleft{\psi}};
(0,-6)*{\afvjm4};
}\quad,
\]

\TODO{Alessio said: `\emph{This is a long shot. Perhaps you can argue a bit more and more clearly about the red edges. Not clear what the latter and former subflows are.}'.}

where all the edges in $\phi_1$ are mapped to from $a_1$ and all the edges in $\phi_2$ are mapped to from $a_2$, and there are no edges in $\psi$ that are mapped to from $a_1$ or $a_2$, then the atomic flow associated with $\PB((\Phi,a_1),a_2)$ is the juxtaposition of the following three flows (where indications of the different isomorphisms are left out):

\TODO{adjust labels}

\TODO{state that we ignore isomorphisms}

\TODO{put flows on one page (how?)}

\[
\atomicflow
{
%cocontraction - top
(4,37.5)*{\afvjm{3}};
(4,32)*{\affr{50}8};
(4,32)*{\copy\contrup};
%contraction - bot
(-4,-32)*{\affr{50}8};
(-4,-32)*{\copy\contrdown};
(-4,-37.5)*{\afvjm{3}};
%---------------------
(4,-18)="D";
(0,-9)="Dhalf";
%----------------
%%first
(-20,0)="B";
% cocontractions
"B"-"D"-"D"-(-12,7)="A";
%left
"A"+"Dhalf"+"Dhalf"+(4,-4)*{\afvjm{42}};
"A"+"Dhalf"+(0,-4)*{\afvjm{24}};
"A"+(-4,-4)*{\afvjm6};
% contractions
"B"+"D"+"D"+(-12,7)="A";
%right
"A"+(4,4)*{\afvjm6};
"A"+(0,4)-"Dhalf"*{\afvjm{24}};
"A"+(-4,4)-"Dhalf"-"Dhalf"*{\afvjm{42}};
%---
% top boxes
"B"-"D"="A";
"A"+(-2, 8)*{\afaid{}{}{}{}{}{}};
"A"+( 0,-8)*{\afawu{}{}{}{}{}};
"A"+( 0, 0)*{\affr{10}{8}};
"A"+( 2, 2)*{\aflabelright{\phi_1}};
% join one
"A"+(4,-9)*{\afvjcol{10}{Red}};
% middle boxes
"B"="A";
"A"+(-4, 8)*{\afawd{}{}{}{}{}};
"A"+( 4,-8)*{\afawu{}{}{}{}{}};
"A"+( 0, 0)*{\affr{10}{8}};
"A"+( 2, 2)*{\aflabelright{\phi_1}};
% join two
"A"+(0,-9)*{\afvjcol{10}{Red}};
% bottom boxes
"B"+"D"="A";
"A"+(2,-8)*{\afaiu{}{}{}{}{}{}};
"A"+(0, 8)*{\afawd{}{}{}{}{}};
"A"+(0, 0)*{\affr{10}{8}};
"A"+( 2, 2)*{\aflabelright{\phi_1}};
%----------------
%%second
(0,0)="B";
% cocontractions
"B"-"D"-"D"-(-12,7)="A";
%left
"A"+"Dhalf"+"Dhalf"+(4,-4)*{\afvjm{42}};
"A"+"Dhalf"+(0,-4)*{\afvjm{24}};
"A"+(-4,-4)*{\afvjm6};
% contractions
"B"+"D"+"D"+(-12,7)="A";
%right
"A"+(4,4)*{\afvjm6};
"A"+(0,4)-"Dhalf"*{\afvjm{24}};
"A"+(-4,4)-"Dhalf"-"Dhalf"*{\afvjm{42}};
%---
% top boxes
"B"-"D"="A";
"A"+(-2, 8)*{\afaid{}{}{}{}{}{}};
"A"+( 0,-8)*{\afawu{}{}{}{}{}};
"A"+( 0, 0)*{\affr{10}{8}};
"A"+( 2, 2)*{\aflabelright{\phi_1}};
% join one
"A"+(4,-9)*{\afvjcol{10}{Red}};
% middle boxes
"B"="A";
"A"+(-4, 8)*{\afawd{}{}{}{}{}};
"A"+( 4,-8)*{\afawu{}{}{}{}{}};
"A"+( 0, 0)*{\affr{10}{8}};
"A"+( 2, 2)*{\aflabelright{\phi_1}};
% join two
"A"+(0,-9)*{\afvjcol{10}{Red}};
% bottom boxes
"B"+"D"="A";
"A"+(2,-8)*{\afaiu{}{}{}{}{}{}};
"A"+(0, 8)*{\afawd{}{}{}{}{}};
"A"+(0, 0)*{\affr{10}{8}};
"A"+( 2, 2)*{\aflabelright{\phi_1}};
%----------------
%%third
(20,0)="B";
% cocontractions
"B"-"D"-"D"-(-12,7)="A";
%left
"A"+"Dhalf"+"Dhalf"+(4,-4)*{\afvjm{42}};
"A"+"Dhalf"+(0,-4)*{\afvjm{24}};
"A"+(-4,-4)*{\afvjm6};
% contractions
"B"+"D"+"D"+(-12,7)="A";
%right
"A"+(4,4)*{\afvjm6};
"A"+(0,4)-"Dhalf"*{\afvjm{24}};
"A"+(-4,4)-"Dhalf"-"Dhalf"*{\afvjm{42}};
%---
% top boxes
"B"-"D"="A";
"A"+(-2, 8)*{\afaid{}{}{}{}{}{}};
"A"+( 0,-8)*{\afawu{}{}{}{}{}};
"A"+( 0, 0)*{\affr{10}{8}};
"A"+( 2, 2)*{\aflabelright{\phi_1}};
% join one
"A"+(4,-9)*{\afvjcol{10}{Red}};
% middle boxes
"B"="A";
"A"+(-4, 8)*{\afawd{}{}{}{}{}};
"A"+( 4,-8)*{\afawu{}{}{}{}{}};
"A"+( 0, 0)*{\affr{10}{8}};
"A"+( 2, 2)*{\aflabelright{\phi_1}};
% join two
"A"+(0,-9)*{\afvjcol{10}{Red}};
% bottom boxes
"B"+"D"="A";
"A"+(2,-8)*{\afaiu{}{}{}{}{}{}};
"A"+(0, 8)*{\afawd{}{}{}{}{}};
"A"+(0, 0)*{\affr{10}{8}};
"A"+( 2, 2)*{\aflabelright{\phi_1}};
}
\]
\[
\atomicflow
{
(12,-50)="D";
% cocontractions
(18,-11)="A";
%
"A"+(6,5.5)*{\afvjm3};
"A"+(6,0)*{\affr{34}8};
"A"+(6,0)*{\copy\contrup};
"A"+(2,-29)*{\afvjm{50}};
"A"+(6,-29)*{\afvjm{50}};
"A"+(10,-33)*{\afvjm{58}};
"A"+(14,-54)*{\afvjm{100}};
"A"+(18,-54)*{\afvjm{100}};
"A"+(22,-58)*{\afvjm{108}};
% === BOX ONE ===
(0,-27)="A";
%-
"A"+( -6,24)*{\afaidex{}{}{}{}{}{}31};
%-
"A"+(-12,16)*{\affr{10}8};
"A"+(-12,16)*{\copy\contrup};
"A"+(  0,16)*{\affr{10}8};
"A"+(  0,16)*{\copy\contrup};
%-
"A"+(-16,8)*{\afvj8};
"A"+(-8,8)*{\afcjr88};
"A"+(0,8)*{\afcjrm{16}8};
%
"A"+(-8,8)*{\afcjl88};
"A"+(0,8)*{\afvj8};
"A"+(8,8)*{\afcjrm88};
%
"A"+(0,8)*{\afcjl{16}8};
"A"+(8,8)*{\afcjl88};
"A"+(16,8)*{\afvjm8};
%-
"A"+(-12,0)*{\affr{10}8};
"A"+(0,0)*{\affr{10}8};
"A"+(12,0)*{\affr{10}8};
"A"+(-10,2)*{\aflabelright{\phi_2}};
"A"+(2,2)*{\aflabelright{\phi_2}};
"A"+(14,2)*{\aflabelright{\phi_2}};
%-
"A"+(-8,-8)*{\afcjl88};
"A"+(0,-8)*{\afcjlcol{16}8{Red}};
%-
"A"+(-8,-8)*{\afcjrm88};
"A"+(0,-8)*{\afvj8};
"A"+(8,-8)*{\afcjlcol88{Red}};
%
"A"+(0,-8)*{\afcjrm{16}8};
"A"+(8,-8)*{\afcjr88};
"A"+(16,-8)*{\afvjcol8{Red}};
%
"A"+(  0,-16)*{\affr{10}8};
"A"+(  0,-16)*{\copy\contrdown};
"A"+( 12,-16)*{\affr{10}8};
"A"+( 12,-16)*{\copy\contrdown};
%---
"A"+(0,-24)*{\afawu{}{}{}{}};
"A"+(12,-25)*{\afvjcol{10}{Red}};
% === BOX TWO ===
"A"+"D"="A";
%-
"A"+(-12,24)*{\afawd{}{}{}{}};
%-
"A"+(-12,16)*{\affr{10}8};
"A"+(-12,16)*{\copy\contrup};
"A"+(  0,16)*{\affr{10}8};
"A"+(  0,16)*{\copy\contrup};
%-
"A"+(-16,8)*{\afvj8};
"A"+(-8,8)*{\afcjrcol88{Red}};
"A"+(0,8)*{\afcjrm{16}8};
%
"A"+(-8,8)*{\afcjl88};
"A"+(0,8)*{\afvjcol8{Red}};
"A"+(8,8)*{\afcjrm88};
%
"A"+(0,8)*{\afcjl{16}8};
"A"+(8,8)*{\afcjlcol88{Red}};
%-
"A"+(-12,0)*{\affr{10}8};
"A"+(0,0)*{\affr{10}8};
"A"+(12,0)*{\affr{10}8};
"A"+(-10,2)*{\aflabelright{\phi_2}};
"A"+(2,2)*{\aflabelright{\phi_2}};
"A"+(14,2)*{\aflabelright{\phi_2}};
%-
"A"+(-8,-8)*{\afcjlcol88{Red}};
"A"+(0,-8)*{\afcjl{16}8};
%-
"A"+(-8,-8)*{\afcjrm88};
"A"+(0,-8)*{\afvjcol8{Red}};
"A"+(8,-8)*{\afcjl88};
%
"A"+(0,-8)*{\afcjrm{16}8};
"A"+(8,-8)*{\afcjrcol88{Red}};
"A"+(16,-8)*{\afvj8};
%
"A"+(  0,-16)*{\affr{10}8};
"A"+(  0,-16)*{\copy\contrdown};
"A"+( 12,-16)*{\affr{10}8};
"A"+( 12,-16)*{\copy\contrdown};
%---
"A"+(12,-24)*{\afawu{}{}{}{}};
"A"+(0,-25)*{\afvjcol{10}{Red}};
% === BOX THREE ===
"A"+"D"="A";
%-
"A"+(  0,24)*{\afawd{}{}{}{}};
%-
"A"+(-12,16)*{\affr{10}8};
"A"+(-12,16)*{\copy\contrup};
"A"+(  0,16)*{\affr{10}8};
"A"+(  0,16)*{\copy\contrup};
%-
"A"+(-16,8)*{\afvjcol8{Red}};
"A"+(-8,8)*{\afcjr88};
"A"+(0,8)*{\afcjrm{16}8};
%
"A"+(-8,8)*{\afcjlcol88{Red}};
"A"+(0,8)*{\afvj8};
"A"+(8,8)*{\afcjrm88};
%
"A"+(0,8)*{\afcjlcol{16}8{Red}};
"A"+(8,8)*{\afcjl88};
%-
"A"+(-12,0)*{\affr{10}8};
"A"+(0,0)*{\affr{10}8};
"A"+(12,0)*{\affr{10}8};
"A"+(-10,2)*{\aflabelright{\phi_2}};
"A"+(2,2)*{\aflabelright{\phi_2}};
"A"+(14,2)*{\aflabelright{\phi_2}};
%-
"A"+(-16,-8)*{\afvjm8};
"A"+(-8,-8)*{\afcjl88};
"A"+(0,-8)*{\afcjl{16}8};
%-
"A"+(-8,-8)*{\afcjrm88};
"A"+(8,-8)*{\afcjl88};
"A"+(0,-8)*{\afvj8};
%
"A"+(0,-8)*{\afcjrm{16}8};
"A"+(8,-8)*{\afcjr88};
"A"+(16,-8)*{\afvj8};
%
"A"+(  0,-16)*{\affr{10}8};
"A"+(  0,-16)*{\copy\contrdown};
"A"+( 12,-16)*{\affr{10}8};
"A"+( 12,-16)*{\copy\contrdown};
%-
"A"+(  6,-24)*{\afaiuex{}{}{}{}{}{}31};
%---
"A"+(-20,-16)="A";
"A"+(-20,58)*{\afvjm{108}};
"A"+(-16,54)*{\afvjm{100}};
"A"+(-12,54)*{\afvjm{100}};
"A"+(-8,33)*{\afvjm{58}};
"A"+(-4,29)*{\afvjm{50}};
"A"+( 0,29)*{\afvjm{50}};
"A"+(-4,0)*{\affr{34}8};
"A"+(-4,0)*{\copy\contrdown};
"A"+(-4,-5.5)*{\afvjm3};
}
\]
\[
\atomicflow{
(0,34.5)*{\afvjm3};
(0,29)*{\affr{82}8};
(0,29)*{\copy\contrup};
%
(0,-29)*{\affr{82}8};
(0,-29)*{\copy\contrdown};
(0,-34.5)*{\afvjm3};
%-------------------
(30,0)="B";
%---------------
(0,0)-"B"="A";
"A"+(-10,14.5)*{\afvjm{21}};
"A"+(0,14.5)*{\afvjm{21}};
"A"+(10,14.5)*{\afvjm{21}};
%---
"A"+(-10,0)*{\affr88};
"A"+( -9,2)*{\aflabelright\psi};
"A"+(  0,0)*{\affr88};
"A"+(  1,2)*{\aflabelright\psi};
"A"+( 10,0)*{\affr88};
"A"+( 11,2)*{\aflabelright\psi};
%---
"A"+(-10,-14.5)*{\afvjm{21}};
"A"+(0,-14.5)*{\afvjm{21}};
"A"+(10,-14.5)*{\afvjm{21}};
%---------------
(0,0)="A";
%---
"A"+(-10,14.5)*{\afvjm{21}};
"A"+(0,14.5)*{\afvjm{21}};
"A"+(10,14.5)*{\afvjm{21}};
%---
"A"+(-10,0)*{\affr88};
"A"+( -9,2)*{\aflabelright\psi};
"A"+(  0,0)*{\affr88};
"A"+(  1,2)*{\aflabelright\psi};
"A"+( 10,0)*{\affr88};
"A"+( 11,2)*{\aflabelright\psi};
%---
"A"+(-10,-14.5)*{\afvjm{21}};
"A"+(0,-14.5)*{\afvjm{21}};
"A"+(10,-14.5)*{\afvjm{21}};
%---------------
"A"+"B"="A";
%---
"A"+(-10,14.5)*{\afvjm{21}};
"A"+(0,14.5)*{\afvjm{21}};
"A"+(10,14.5)*{\afvjm{21}};
%---
"A"+(-10,0)*{\affr88};
"A"+( -9,2)*{\aflabelright\psi};
"A"+(  0,0)*{\affr88};
"A"+(  1,2)*{\aflabelright\psi};
"A"+( 10,0)*{\affr88};
"A"+( 11,2)*{\aflabelright\psi};
%---
"A"+(-10,-14.5)*{\afvjm{21}};
"A"+(0,-14.5)*{\afvjm{21}};
"A"+(10,-14.5)*{\afvjm{21}};
}\quad.
\]
We marked some edges in red to point out the fundamental difference between the subflows containing $\phi_1$ and the subflows containing $\phi_2$
\end{example}
%------------


%======================================
\section{Multiple Isolated Subflows Removal}\label{section:MultipleIsolatedSubflowsRemoval}

\newcommand{\MISR}{\mathsf{MISR}}

With the operator $\ISR$ we can produce weakly streamilend derivations with respect to one atom at a time, with the operator $\PB$ we can produce weakly streamlined derivations with respect to two dual atoms in parallell. In this section we see an operator, $\MISR_n$, for every $n>0$, which is a generalisation of $\ISR$, that can produce a weakly streamlined derivation with respect to $n$ number of atoms in parallell, as long as they are pairwise non-dual.

We will see later how a derivation containing $2n$ atoms can be weakly streamlined by two applications of $\Simpl$ and two applications of $\MISR_n$.

\Tom{Added explanation:}

The operator is defined in terms of the following flow reduction. Unlike the flow reductions of the preceding sections, we here present a reductions which depends on several parameters. It is important to note that these parameters are independent of the derivation to which we later apply the operator. In order to perform streamlining on an arbitrary number of atoms in parallel, we need find a class of atomic flows, $\eta_k$, which are used as a sort of sharing mechanism. We are at this stage not able to describe the flows $\eta_k$ without relying on their corresponding derivations. For this reason, it might help the understanding of Definition~\vref{definition:MultipleIsolatedSubflowsRemoval} to refer to the derivation given in the proof of Theorem~\vref{theorem:SoundMultipleIsolatedSubflowsRemoval}.

In Subsection~\vref{subsection:ThresholdFormulae}, we present one possible combination of valid parameters, which yields quasipolynomial streamlining. We conjecture that by finding different parameters we will be able to obtain more efficient versions of this reduction. In particular, we hope to be able to obtain polynomial streamlining.

\newcommand{\Gammasf}{\mathsf\Gamma}
%----------------------------------
\newcommand{\frmis}{{\mathsf{mis}}}
\begin{definition}\label{definition:MultipleIsolatedSubflowsRemoval}
For every $n>0$, given
\begin{itemize}
\item atoms $a_1$, $\dots$, $a_n$;
\item an $N>0$;
\item for $0\le k\le N$, formulae $\gamma_{k,1}$, $\dots$, $\gamma_{k,n}$, such that
\begin{itemize}
 \item $\gamma_{0,1}=\cdots=\gamma_{0,n}=\ttt$, and
 \item $\gamma_{N,1}=\cdots=\gamma_{N,n}=\fff$; and
\end{itemize}
\item for $1\le k\le N$, a derivation
\[
\Gammasf_k\quad=\quad
\vlder{}{\SKS\setminus\{\aid,\aiu\}}
{
 \vls([a_1.\gamma_{k,1}].\cdots.[a_n.\gamma_{k,n}])
}
{
 \vls[(a_1.\gamma_{k-1,1}).\cdots.(a_n.\gamma_{k-1,n})]
}\quad,
\]
\end{itemize}
let, for $1\le k\le N$, $\eta_k$ be the flow of\/ $\Gammasf_k$, and let
\[
\mu_k\;=\;
\vcenter{\hbox{\includegraphics{Figures/redMISRAux}}}
\quad,
\]
where, for $1\le i\le n$, $l_i$ is the number of atom occurrences in $\gamma_{k,i}$, we define the reduction $\to_{\frmis_n}$ (where $\frmis$ stands for \emph{multiple isolated subflows}\index{Multiple Isolated Subflows Remover!reduction}) as follows, for any flow $\phi$ and any connected components $\psi_1$, $\dots$, $\psi_n$ that do not contain interaction or cut vertices:
\[
\vcenter{\hbox{\includegraphics{Figures/redMISRRedex}}}
\quad\to_{\frmis_n}\quad
\vcenter{\hbox{\includegraphics{Figures/redMISRContractum}}}
\quad,
\]
where we call the evidenced interaction (resp., cut) vertices $\nu_{\aid,1}$, $\dots$, $\nu_{\aid,n}$ (resp., $\nu_{\aiu,1}$, $\dots$, $\nu_{\aiu,n}$).
\end{definition}
%---------------

\Tom{Added a missing `is':}

\begin{remark}
The reduction ${\to_{\frmis_n}}$ is denoted as if it only depends on $n$, this is a misuse of notation, and we will take it for granted that we also have the other parameters whenever we write ${\to_{\frmis_n}}$.
\end{remark}

%--------------------------------------------------------------
\begin{remark}\label{remark:BaseMultipleIsolatedSubflowRemoval}
If $N=1$ and
$
\Gammasf_1=\;
\vlinf{=}{}
{
 \vls[a_1.\fff]
}
{
 \vls(a_1.\ttt)
}\;,
$
then ${\to_{\frmis_1}}={\to_\fris}$.
\end{remark}
%-----------

\Tom{connected flow -> connected component}

%------------------------------------------------------------------
\begin{theorem}\label{theorem:SoundMultipleIsolatedSubflowsRemoval}
For every $n>0$, reduction\/ ${\to_{\frmis_n}}$ is sound; moreover, if\/ $\Phi\to_{\frmis_n}\Psi$, then the size of\/ $\Psi$ depends linearly on $N$, polynomially on the size of\/ $\Phi$ and at most polynomially on\/ $\max\{\size{\Gammasf_1},\dots,\size{\Gammasf_N}\}$.
\end{theorem}

\begin{proof}
Let $\Phi$ be a derivation with flow $\phi'$, such that $\phi'\to_{\frmis_n}\psi'$. We show that there exists a derivation $\Psi$ with flow $\psi'$ and with the same premiss and conclusion as $\Phi$. In the following, we refer to the figures in Definition~\vref{definition:MultipleIsolatedSubflowsRemoval}.

Since each of $\psi_1$, $\dots$, $\psi_n$ is connected, we assume, by Convention~\vref{convention:AlternativeAiDecomposedForm}, that the following derivation is an $\ai$-decomposed form of $\Phi$:
\[
\vlder{\Phi'}{}
{
 \vls[\beta\;.\;\vlinf{}{}{\fff}{\vls(a_1.\bar a_1^{\psi_1})}\;.\;\vlinf{}{}{\fff}{\vls(a_n.\bar a_n^{\psi_n})}]
}
{
 \vls(\vlinf{}{}{\vls[a_1.\bar a_1^{\psi_1}]}{\ttt}\;.\;\vlinf{}{}{\vls[a_n.\bar a_n^{\psi_n}]}{\ttt}\;.\;\alpha)
}\quad,
\]
for some atoms $a_1$, $\dots$, $a_n$ (that, without loss of generality, we assume coincide with the atoms given in Definition~\vref{definition:MultipleIsolatedSubflowsRemoval}) and formulae $\alpha$ and $\beta$.

For every $0\le k\le N$, we obtain the derivation $\Phi_k$ from $\Phi'$ as follows:
\[
\Phi_k\quad=\quad
\vlder{\Phi'\{\bar a_1^{\psi_1}/\gamma_{k,1},\dots,\bar a_n^{\psi_n}/\gamma_{k,n}\}}{}
{
 \vlsbr[\beta.(a_1.\gamma_{k,1}).\cdots.(a_n.\gamma_{k,n})]
}
{
 \vls([a_1.\gamma_{k,1}].\cdots.[a_n.\gamma_{k,n}].\alpha)
}
\]
Since each of $\psi_1$, $\dots$, $\psi_n$ is a connected component and contains no interaction or cut vertices, the mapping from occurrences of $\bar a_i^{\psi_i}$ to edges of $\psi_i$ is surjective. Hence, we know that $\Phi_k$ has flow
\[
\vcenter{\hbox{\includegraphics{Figures/thmMISR}}}
\quad.
\]
We combine $\Phi_0$, $\dots$, $\Phi_N$, $\Gammasf_1$, $\dots$, $\Gammasf_N$ to get the desired derivation $\Psi$ with flow $\psi'$ and the same premiss and conclusion as $\Phi$:
\[
\vlderivation
{
 \vlin{\cod}{}
 {
  \beta
 }
 {
  \vlin{\swi}{}
  {
   \vls
   [
    \vlinf{\cod}{}{\beta}{\vls[\beta.\beta]}
   \;\;\;.\;\;\;
    \vlder{\Phi_N}{}
    {
     \vlsbr[\beta\;.\;(\vlinf{}{}{\ttt}{a_1}\;.\;\fff)\;.\;\cdots\;.\;(\vlinf{}{}{\ttt}{a_n}\;.\;\fff)]
    }
    {
     \vls([a_1.\fff].\cdots.[a_n.\fff].\alpha)
    }
   ]
  }
  {
   \vlin{\swi}{}
   {
    \vls
    (
     [
      \vlinf{\cod}{}{\beta}{\vls[\beta.\beta]}
     \;\;\;\;.\;\;\;\;
      \vlder{\Phi_{N-1}}{}
      {
       \vlsbr
       [
        \beta
       \;\;.\;\;
        \vlder{\Gammasf_N}{}
        {
         \vlsmallbrackets\vls([a_1.\fff].\cdots.[a_n.\fff])
        }
        {
         \vlsmallbrackets\vls[(a_1.\gamma_{N-1,1}).\cdots.(a_n.\gamma_{N-1,n})]
        }
       ]
      }
      {
       \vlsmallbrackets\vls([a_1.\gamma_{N-1,1}].\cdots.[a_n.\gamma_{N-1,n}].\alpha)
      }
     ]
    \;\;\;\;.\;\;\;\;
     \alpha
    )
   }
   {
    \vlin{\swi}{}
    {
     \vdots
    }
    {
     \vlin{\swi}{}
     {
      \vls
      (
       [
        \beta
       \;\;\;\;.\;\;\;\;
        \vlder{\Phi_1}{}
        {
         \vlsbr
         [
          \beta
         \;\;.\;\;
          \vlder{\Gammasf_2}{}
          {
           \vlsmallbrackets\vls([a_1.\gamma_{2,1}].\cdots.[a_n.\gamma_{2,n}])
          }
          {
           \vlsmallbrackets\vls[(a_1.\gamma_{1,1}).\cdots.(a_n.\gamma_{1,n})]
          }
         ]
        }
        {
         \vlsmallbrackets\vls([a_1.\gamma_{1,1}].\cdots.[a_n.\gamma_{1,n}].\alpha)
        }
       ]
      \;\;\;\;.\;\;\;\;
       \vlinf{\cou}{}{\vls(\alpha.\alpha)}{\alpha}
      )
     }
     {
      \vlin{\cou}{}
      {
       \vls
       (
        \vlder{\Phi_0}{}
        {
         \vlsbr
         [
          \beta
         \;\;.\;\;
          \vlder{\Gammasf_1}{}
          {
           \vlsmallbrackets\vls([a_1.\gamma_{1,1}].\cdots.[a_n.\gamma_{1,n}])
          }
          {
           \vlsmallbrackets\vls[(a_1.\ttt).\cdots.(a_n.\ttt)]
          }
         ]
        }
        {
         \vlsbr([\vlinf{}{}{a_1}{\fff}\;.\;\ttt]\;.\;\cdots\;.\;[\vlinf{}{}{a_n}{\fff}\;.\;\ttt]\;.\;\alpha)
        }
       \;\;\;\;.\;\;\;\;
        \vlinf{\cou}{}{\vls(\alpha.\alpha)}{\alpha}
       )
      }
      {
       \vlhy{\alpha}
      }
     }
    }
   }
  }
 }
}
\qquad.
\]
Since $\max\{\size{\gamma_{0,1}},\dots,\size{\gamma_{N,n}}\}$ is less than or equal to $\max\{\size{\Gammasf_1},\dots,\size{\Gammasf_N}\}$, we know that the size of $\Phi_0$, $\dots$, $\Phi_N$ depend at most cubically on the size of $\Phi$ and at most quadratically on the size of $\max\{\size{\Gammasf_1},\dots,\size{\Gammasf_N}\}$ by Theorem~\vref{theorem:aiDecomposedForm} and Proposition~\vref{proposition:DerivationSubstitution}, and that the size of $\Psi$ depends at most cubically on the size of $\alpha$ and $\beta$ by Lemma~\vref{lemma:GenericContraction}, so the size of $\Psi$ depends linearly on $N$, polynomially on the size of $\Phi$ and at most polynomially on the size of $\max\{\size{\Gammasf_1},\dots,\size{\Gammasf_N}\}$.
\end{proof}
%----------

\Tom{Added explanation:}

We now show the basic properties of $\to_\frmis$. Namely, that the reduction does not create any `new' interaction or cut vertices, and that it does not create any `new' paths between interaction or $\top$ and cut or $\bot$ vertices.

\Tom{Changed from itemize to enumerate.}

\Tom{Exchanged $\top$ and $\bot$ in item 2, and clarified statement.}

%-------------------------------------
\begin{lemma}\label{lemma:MultipleIsolatedSubflowsRemovalPaths}
In the following we refer to the names given in Definition~\vref{definition:MultipleIsolatedSubflowsRemoval}. Given two flows $\phi$ and $\psi$ and an $n>0$, such that $\phi\to_{\frmis_n}\psi$ then, given an interaction (resp., cut) vertex $\nu$ in $\psi$, there is an interaction (resp., cut) vertex $\nu'$ in $\phi$, such that
\begin{enumerate}
\item for some $1\le i\le N+1$, $\nu=f_i(\nu')$;
\item if there is a path from $\nu$ to $\bot$ (resp., $\top$), then there is a path from $\nu'$ to $\bot$ (resp., $\top$); and
\item if there is a cut (resp., interaction) vertex $\hat\nu$ in $\psi$, such that there is a path from $\nu$ to $\hat\nu$, then there is a cut (resp., interaction) vertex $\hat\nu'$ in $\phi$, such that, for some $1\le i\le N+1$, $\hat\nu=f_i(\hat\nu')$, or, for some $1\le i\le n$, $\hat\nu'=\nu_{\aiu,i}$ (resp., $\hat\nu'=\nu_{\aid,i}$); and there is a path from $\nu'$ to $\hat\nu'$.
\end{enumerate}
\end{lemma}

\begin{proof}
We consider each case separately:
\begin{enumerate}
 \item the statement follows by definition;
 \item any path from $\nu$ to $\bot$ (resp., $\top$) must contain an edge $\epsilon$, such that, for some lower (resp., upper) edge $\epsilon'$ of $\phi$ and some $1\le i\le N+1$, $f_i(\epsilon')=\epsilon$. Hence, there is a path from $\nu'$ to $\bot$ (resp., $\top$); and
 \item we have to consider two cases:
 \begin{enumerate}
  \item for some $1\le i\le N+1$, $\nu=f_i(\nu')$ and $\hat\nu=f_i(\hat\nu')$, then there is a path from $\nu'$ to $\hat\nu'$; or
  \item for some $1\le i<j\le N+1$, $\nu=f_i(\nu')$ and $\hat\nu=f_j(\hat\nu')$ (resp., $\nu=f_j(\nu')$ and $\hat\nu=f_i(\hat\nu')$), then, for some $1\le i\le n$, there is a path from $\nu'$ to $\nu_{\aiu,i}$ (resp., $\nu_{\aiu,i}$).
 \end{enumerate}
\end{enumerate}
\end{proof}
%----------

%--------------------------------
\begin{definition}\label{definition:MultipleIsolatedSubflowsRemover}
For every $n>0$, given the atoms, formulae and derivations described in Definition~\vref{definition:MultipleIsolatedSubflowsRemoval}, the \emph{Multiple Isolated Subflow Remover}\index{Multiple Isolated Subflows Remover!operator}, $\MISR_n$, is an operator whose arguments are atoms $a_1$, $\dots$, $a_n$ (that, without loss of generality, we assume coincide with the atoms given in Definition~\ref{definition:MultipleIsolatedSubflowsRemoval}), and a derivation $\Phi$ that is in simple form with respect to $a_1$, $\dots$, $a_n$. If $n=1$ and $\Phi$ is weakly streamlined with respect to $a_1$, then $\MISR_1(\Phi,a_1)=\Phi$; if $n>1$ and, for some $1\le i\le n$, $\Phi$ is weakly streamlined with respect to $a_i$, then $\MISR(\Phi,a_1,\dots,a_n)=\MISR_{n-1}=(\Phi,a_1,\dots,a_{i-1},a_{i+1},\dots,a_n)$; otherwise, consider the following $\ai$-decomposed form of $\Phi$:
\[
\vlder{\Phi'}{}
{
 \vlsbr
 [
  \beta
 \;.\;
  \vlinf{}{}
  {
   \fff
  }
  {
   \vls(a_n^{\psi_n}.\bar a_n)
  }
 \;.\;\cdots\;.\;
  \vlinf{}{}
  {
   \fff
  }
  {
   \vls(a_n^{\psi_n}.\bar a_n)
  }
 \;.\;\cdots\;.\;
  \vlinf{}{}
  {
   \fff
  }
  {
   \vls(a_1^{\psi_1}.\bar a_1)
  }
 \;.\;\cdots\;.\;
  \vlinf{}{}
  {
   \fff
  }
  {
   \vls(a_1^{\psi_1}.\bar a_1)
  }
 ]
}
{
 \vlsbr
 (
  \vlinf{}{}
  {
   \vls[a_1^{\psi_1}.\bar a_1]
  }
  {
   \ttt
  }
 \;.\;\cdots\;.\;
  \vlinf{}{}
  {
   \vls[a_1^{\psi_1}.\bar a_1]
  }
  {
   \ttt
  }
 \;.\;
  \vlinf{}{}
  {
   \vls[a_n^{\psi_n}.\bar a_n]
  }
  {
   \ttt
  }
 \;.\;\cdots\;.\;
  \vlinf{}{}
  {
   \vls[a_n^{\psi_n}.\bar a_n]
  }
  {
   \ttt
  }
 \;.\;
  \alpha
 )
}\quad,
\]
with flow
\[
\vcenter{\hbox{\includegraphics{Figures/defMISR1}}}
\quad,
\]
where, for $1\le i\le n$, $\psi_i$ is the juxtaposition of all the isolated subflows mapped to from occurrences of $a_i$ in $\Phi$. Consider the derivation
\[
\Psi\;=\;
\vlder{\Phi'}{}
{
 \vlsbr
 [
  \beta
 \;\;\;.\;\;\;
  \vlderivation
  {
   \vlin{}{}
   {
    \fff
   }
   {
    \vlde{}{\{\cod\}}
    {
     \vls(a_n.\bar a_n)
    }
    {
     \vlhy
     {
      \vls[(a_n.\bar a_n).\cdots.(a_n.\bar a_n)]
     }
    }
   }
  }
 \;\;\;.\;\;\;\cdots\;\;\;.\;\;\;
  \vlderivation
  {
   \vlin{}{}
   {
    \fff
   }
   {
    \vlde{}{\{\cod\}}
    {
     \vls(a_1.\bar a_1)
    }
    {
     \vlhy
     {
      \vls[(a_1.\bar a_1).\cdots.(a_1.\bar a_1)]
     }
    }
   }
  }
 ]
}
{
 \vlsbr
 (
  \vlderivation
  {
   \vlde{}{\{\cou\}}
   {
    \vls([a_1.\bar a_1].\cdots.[a_1.\bar a_1])
   }
   {
    \vlin{}{}
    {
     \vls[a_1.\bar a_1]
    }
    {
     \vlhy
     {
      \ttt
     }
    }
   }
  }
 \;\;\;.\;\;\;\cdots\;\;\;.\;\;\;
  \vlderivation
  {
   \vlde{}{\{\cou\}}
   {
    \vls([a_n.\bar a_n].\cdots.[a_n.\bar a_n])
   }
   {
    \vlin{}{}
    {
     \vls[a_n.\bar a_n]
    }
    {
     \vlhy
     {
      \ttt
     }
    }
   }
  }
 \;\;\;.\;\;\;
  \alpha
 )
}\quad,
\]
with flow
\[
\psi''\;=\;
\vcenter{\hbox{\includegraphics{Figures/defMISR2}}}
\quad.
\]
We then define $\MISR_n(\Phi,a_1,\dots,a_n)$ to be such that $\Psi\to_\frmis\MISR(\Phi,a_1,\dots,a_n)$, where $\phi$, $\psi_1$, $\dots$, $\psi_n$ are the flows, by the same names, shown in Definition~\vref{definition:MultipleIsolatedSubflowsRemoval}.
\end{definition}
%---------------

\Tom{Changed itemize to enumerate.}

%--------------------------------------------------------------------
\begin{proposition}\label{proposition:MultipleIsolatedSubflowRemover}
Given the atoms, formulae and derivations described in Definition~\vref{definition:MultipleIsolatedSubflowsRemoval}, and atoms $a_1$, $\dots$, $a_n$ and a derivation $\Phi$ that is in simple form with respect to $a_1$, $\dots$, $a_n$,
\begin{enumerate}
\item $\MISR_n(\Phi,a_1,\dots,a_n)$ is weakly streamlined with respect to $a_1$, $\dots$, $a_n$;
\item for any atom $b$,
\begin{enumerate}
\item if $\Phi$ is weakly streamlined with respect to $b$, then $\MISR_n(\Phi,a_1,\dots,a_n)$ is weakly streamlined with respect to $b$, and
\item if $b$ is not the dual of any of $a_1$, $\dots$, $a_n$ and $\Phi$ is in simple form with respect to $b$, then $\MISR_n(\Phi,a_1,\dots,a_n)$ is in simple form with respect to $b$; and
\end{enumerate}
\item the size of\/ $\MISR_n(\Phi,a_1,\dots,a_n)$ depends linearly on $N$, polynomially on the size of\/ $\Phi$, and at most polynomially on $\max\{\size{\Gammasf_1},\dots,\size{\Gammasf_N}\}$.
\end{enumerate}
\end{proposition}

\Tom{Fixed typo in second case:}

\begin{proof}
If $\Phi$ is weakly streamlined with respect to some atom from $a_1$, $\dots$, $a_n$, the result follows by induction. Assume $\Phi$ is not weakly streamlined with respect to any atom from $a_1$, $\dots$, $a_n$, and let $\phi$, $\psi_1$, $\dots$, $\psi_n$, $\phi'$, $\psi'_1$, $\dots$, $\psi'_n$ and $\psi''$ be the flows given in Definition~\vref{definition:MultipleIsolatedSubflowsRemover}, then
\begin{enumerate}
\item by definition there is no path in $\phi$ from an interaction to a cut vertex whose edges are mapped to from instances of one of $a_1$, $\dots$, $a_n$. By Lemma~\vref{lemma:MultipleIsolatedSubflowsRemovalPaths}, we know that if there is a path from an interaction to a cut vertex in the flow of $\MISR_n(\Phi,a_1,\dots,a_n)$ whose edges are mapped to from instances of one of $a_1$, $\dots$, $a_n$, then there must be a path from an interaction to a cut vertex in $\phi$ whose edges are mapped to from instances of one of $a_1$, $\dots$, $a_n$. Hence, the statement follows by contradiction;
\item
\begin{enumerate}
 \item if the flow of $\MISR_n(\Phi,a_1,\dots,a_n)$ contains a path from an interaction vertex to a cut vertex whose edges are mapped to from $b$, then, by Lemma~\ref{lemma:MultipleIsolatedSubflowsRemovalPaths}, there is a path from an interaction vertex to a cut vertex in $\phi$, so also in $\phi'$, whose edges are mapped to from $b$. Hence, the statement follows by contradiction; and
 \item if there is an interaction (reps., cut) vertex $\nu$ and a cut (resp., interaction) vertex $\hat\nu$ in the flow of $\MISR_n(\Phi,a_1,\dots,a_n)$ such that there is a path from $\nu$ to $\hat\nu$ and a path from $\nu$ to $\bot$ (resp., $\top$), both of whose edges are mapped to from $b$, then, by Lemma~\ref{lemma:MultipleIsolatedSubflowsRemovalPaths}, there is an interaction (resp., cut) vertex $\nu'$ and a cut (resp., interaction) vertex $\hat\nu'$ in $\phi$ such that there is a path from $\nu$ to $\hat\nu$ and a path from $\nu$ to $\bot$ (resp., $\top$), both of whose edges are mapped to from $b$. Furthermore, since we can assume that $b$ is not any of $a_1$, $\dots$, $a_n$ or their duals, $\phi$ restricted to $b$ equals $\phi'$ restricted to $b$. Hence, the statement follows by contradiction.
\end{enumerate}
\item the statement follows by Theorem~\vref{theorem:SoundMultipleIsolatedSubflowsRemoval}.
\end{enumerate}
\end{proof}
%----------

%--------------------------------------------------
\begin{remark}\label{remark:FromGammasToThresholds}
Given the atoms, formulae and derivations described in Definition~\vref{definition:MultipleIsolatedSubflowsRemoval}, we can prove by induction on $k$, that, for every $1\le i\le n$ and every $0\le k\le N$, the formula $\gamma_{k,i}$ is
\begin{itemize}
 \item true if at least $k$ of the atoms $a_1$, $\dots$, $a_{i-1}$, $a_{i+1}$, $\dots$, $a_n$ are true; and
 \item false if at least $N-k$ of the atoms $a_1$, $\dots$, $a_{i-1}$, $a_{i+1}$, $\dots$, $a_n$ are false.
\end{itemize}
It follows by contradiction that $N\ge n$. Furthermore, if $N=n$, we know that $\gamma_{k,i}$ is true if and only if at least $k$ of the atoms $a_1$, $\dots$, $a_{i-1}$, $a_{i+1}$, $\dots$, $a_n$ are true. This makes $\gamma_{k,i}$ a \emph{threshold formula}\index{threshold!formulae}, as we will se in the next section.
\end{remark}
%-----------

%=============================
\subsection{Threshold Formulae}\label{subsection:ThresholdFormulae}

\TODO{Change this introduction}

We present here the main construction of this paper, \emph{i.e.}, a class of derivations $\Gammasf$ that only depend on a given set of atoms and that allow us to normalise any proof containing those atoms. The complexity of the $\Gammasf$ derivations dominates the complexity of the normal proof, and is due to the complexity of certain `threshold formulae', on which the $\Gammasf$ derivations are based. The $\Gammasf$ derivations are constructed in Definition~\vref{definition:ThresholdDerivations}; this directly leads to Theorem~\vref{theorem:ThresholdDerivations}, which states a crucial property of the $\Gammasf$ derivations and which is the main result of this section.

In the following, $\lfloor x\rfloor$ denotes the maximum integer $n$ such that $n\le x$.

\TODO{Define threshold function.}

There are several ways of encoding threshold functions into formulae, and the problem is to find, among them, an encoding that allows us to obtain Theorem~\vref{theorem:ThresholdDerivations}. Efficiently obtaining the property stated in Theorem~\vref{theorem:ThresholdDerivations} crucially depends also on the proof system we adopt.

Threshold formulae realise boolean threshold functions, which are defined as boolean functions that are true if and only if at least $k$ of $n$ inputs are true (see \cite{Wege:87:The-Comp:vn} for a thorough reference on threshold functions). 

The following class of threshold formulae, which we found to work for system $\SKS$, is a simplification of the one adopted in \cite{AtseGalePudl:02:Monotone:yu}.

\renewcommand{\th}[2]{\mathop{\thetaup_{#1}^{#2}}}
%-------------------------------------------------------------------------------
\begin{definition}\label{definition:ThresholdFormulae}
Consider $n>0$, distinct atoms $a_1$, \dots, $a_n$, and let $p=\lfloor n/2\rfloor$ and $q=n-p$; for $k\ge0$, we define the \emph{threshold formulae\/}\index{threshold!formulae} $\th kn\avec1n$ as follows:
\begin{itemize}
%---------------------------------------
\item for any $n>0$ let $\th0n\avec1n\equiv\ttt$;
%---------------------------------------
\item for any $n>0$ and $k>n$ let $\th kn\avec1n\equiv\fff$;
%---------------------------------------
\item $\th11(a_1)\equiv a_1$;
%---------------------------------------
\item for any $n>1$ and $0<k\le n$ let
$\th kn\avec1n\equiv\bigvee_{\begin{subarray}{l}i+j=k      \\ 
                                                0\le i\le p\\ 
                                                0\le j\le q
                             \end{subarray}}
\vlsbr(\th ip\avec1p.\th jq\avec{p+1}n)$.
%---------------------------------------
\end{itemize}
\end{definition}

See, in Figure~\vref{figure:ThresholdFormulae}, some examples of threshold formulae.

The only reason why we require atoms to be distinct in threshold formulae is to avoid certain technical problems with substitutions in Definitions~\vref{definition:AuxillaryThresholdDerivation} and \vref{definition:ThresholdDerivations}. However, there is no substantial difficulty in relaxing this definition to any set of atoms.

%-------------------------------------------------------------------------------
\begin{figure}
\vlsmallbrackets
\begin{eqnarray*}
%---------------------------------------
\th02(a,b)&\equiv&\ttt
\quad,\\
\noalign{\medskip}
%---------------------------------------
\th12(a,b)&\equiv&\vls[({\vlnos\th11(a)}.{\vlnos\th01(b)}).
                       ({\vlnos\th01(a)}.{\vlnos\th11(b)})]
           \equiv     [(a.\ttt).(\ttt.b)]\\
          &=     &\vls [a      .      b ]
\quad,\\
\noalign{\medskip}
%---------------------------------------
\th22(a,b)&\equiv&\vls({\vlnos\th11(a)}.{\vlnos\th11(b)})\\
          &\equiv&\vls(a.b)
\quad,\\
\noalign{\medskip}
%---------------------------------------
\th03(a,b,c)&\equiv&\ttt
\quad,\\
\noalign{\medskip}
%---------------------------------------
\th13(a,b,c)&\equiv&\vls[({\vlnos\th11(a)}.{\vlnos\th02(b,c)}).
                         ({\vlnos\th01(a)}.{\vlnos\th12(b,c)})]
             \equiv     [(a.\ttt).(\ttt.[(b.\ttt).(\ttt.c)])]\\
            &=     &\vls[a.b.c]
\quad,\\
\noalign{\medskip}
%---------------------------------------
\th23(a,b,c)&\equiv&\vls[({\vlnos\th11(a)}.{\vlnos\th12(b,c)}).
                    ({\vlnos\th01(a)}.{\vlnos\th22(b,c)})]\\
            &=     &\vls[(a.[b.c]).(b.c)]
\quad,\\
\noalign{\medskip}
%---------------------------------------
\th33(a,b,c)&\equiv&\vls({\vlnos\th11(a)}.{\vlnos\th22(b,c)})
             \equiv     [(a.(b.c))]\\
            &=     &\vls(a.b.c)
\quad,\\
\noalign{\medskip}
%---------------------------------------
\th05(a,b,c,d,e)&\equiv&\ttt
\quad,\\
\noalign{\medskip}
%---------------------------------------
\th15(a,b,c,d,e)&\equiv&\vls[({\vlnos\th12(a,b)}.{\vlnos\th03(c,d,e)}).
                             ({\vlnos\th02(a,b)}.{\vlnos\th13(c,d,e)})]\\
                &=     &\vls[a.b.c.d.e]
\quad,\\
\noalign{\medskip}
%---------------------------------------
\th25(a,b,c,d,e)&\equiv&\vls[({\vlnos\th22(a,b)}.{\vlnos\th03(c,d,e)}).
                             ({\vlnos\th12(a,b)}.{\vlnos\th13(c,d,e)}).
                             ({\vlnos\th02(a,b)}.{\vlnos\th23(c,d,e)})]\\
                &=     &\vls[(a.b                                    ).
                             ([a.b]             .[c.d.e]             ).
                                                 (c.[d.e]).(d.e)      ]
\quad,\\
\noalign{\medskip}
%---------------------------------------
\th35(a,b,c,d,e)&\equiv&\vls[({\vlnos\th22(a,b)}.{\vlnos\th13(c,d,e)}).
                             ({\vlnos\th12(a,b)}.{\vlnos\th23(c,d,e)}).
                             ({\vlnos\th02(a,b)}.{\vlnos\th33(c,d,e)})]\\
                &=     &\vls[(a.b               .[c.d.e]             ).
                             ([a.b]             .[(c.[d.e]).(d.e)]   ).
                                                 (c.d.e)              ]
\quad,\\
\noalign{\medskip}
%---------------------------------------
\th45(a,b,c,d,e)&\equiv&\vls[({\vlnos\th22(a,b)}.{\vlnos\th23(c,d,e)}).
                             ({\vlnos\th12(a,b)}.{\vlnos\th33(c,d,e)})]\\
                &=     &\vls[(a.b               .[(c.[d.e]).(d.e)]   ).
                             ([a.b]             .c.d.e               )]
\quad,\\
\noalign{\medskip}
%---------------------------------------
\th55(a,b,c,d,e)&\equiv&\vls({\vlnos\th22(a,b)}.{\vlnos\th33(c,d,e)})\\
                &=     &\vls(a.b.c.d.e)
\quad,\\
\noalign{\medskip}
%---------------------------------------
\th65(a,b,c,d,e)&\equiv&\fff
\quad.
\end{eqnarray*}
\caption{Examples of threshold formulae.}
\label{figure:ThresholdFormulae}
\end{figure}

The formulae for threshold functions adopted in \cite{AtseGalePudl:02:Monotone:yu} correspond, for each choice of $k$ and $n$, to $\bigvee_{i\ge k}\th in\avec1n$. We presume that \cite{AtseGalePudl:02:Monotone:yu} employs these more complicated formulae because the formalism adopted there, the sequent calculus, is less flexible than deep inference, requiring more information in threshold formulae in order to construct suitable derivations.

% \TODO{Remove?}
% 
% %-------------------------------------------------------------------------------
% \begin{remark}
% For $n>0$, we have $\th1n\avec1n=\vls[a_1.\vldots.a_n]$ and $\th nn\avec1n=\vls(a_1.\vldots.a_n)$.
% \end{remark}
% 
% 
% \TODO{Read again:}
% %-------------------------------------------------------------------------------
% \begin{remark}\label{remark:ThersholdSubstitution}
% Given a threshold formula $\th kn\avec1n$ and an atom $a_i$, both $\th kn\avec1n\{a_i/\fff\}$ and $\th kn\avec 1n\{a_i/\ttt\}$ are threshold formula as well. In particular, $\th kn\avec1n\{a_i/\fff\}$ (resp., $\th kn\avec 1n\{a_i/\ttt\}$) is true if and only if at least $k$ (resp., $k-1$) of the atoms $a_1$, $\dots$, $a_{i-1}$, $a_{i+1}$, $\dots$, $a_n$ are true.
% \end{remark}

The size of the threshold formulae dominates the cost of the normalisation procedure, so, we evaluate their size. We leave as an exercise the proof of the following proposition.

%-------------------------------------------------------------------------------
\begin{proposition}\label{proposition:LargestThresholdFormula}
For any $n>0$ and $k\ge0$, $\size{\th kn\avec1n}\le\size{\th{\lfloor n/2\rfloor+1}n\avec1n}$.
\end{proposition}

%-------------------------------------------------------------------------------
\begin{lemma}\label{lemma:SizeThresholdMax}
The size of\/ $\th{\lfloor n/2\rfloor+1}n\avec1n$ is $n^{\Ord{\log n}}$.
\end{lemma}

%-------------------------------------------------------------------------------
\begin{proof}
Observe that $\size{\th kn\avec1n}\le\size{\th k{n+1}\avec1{n+1}}$. Let $p=\lfloor n/2\rfloor$ and $q=n-p$ and consider:
\begin{equation}\label{PropQuasIneq}
\begin{split}
\size{\th{p+1}n\avec1n}
&=\textstyle\sum_{\begin{subarray}{l}i+j=p+1    \\
                                     0\le i\le p\\
                                     0\le j\le q
                  \end{subarray}}
  \left(\size{\th ip\avec1p}+
        \size{\th jq\avec{p+1}n}\right)             \\
&\le\textstyle\sum_{\begin{subarray}{l}i+j=p+1\\
                                       0\le i,j\le q
                    \end{subarray}}
  \left(\size{\th iq\avec1q}+
        \size{\th jq\avec1q}\right)                 \\
&\le2(q+1)
  \size{\th{\lfloor q/2\rfloor+1}q\avec1q}\quad,
\end{split}
\end{equation}
where we use Proposition~\vref{proposition:LargestThresholdFormula}. We show that, for $h=2/(\log3-\log2)$ and for any $n>0$, we have $\size{\th{\lfloor n/2\rfloor+1}n\avec1n}\le n^{h\log n}$. We reason by induction on $n$; the case $n=1$ trivially holds. By the inequality~\eqref{PropQuasIneq}, and for $n>1$, we have
\begin{equation*}
\begin{split}
\size{\th{\lfloor n/2\rfloor+1}n\avec1n}
&\le2(n-\lfloor n/2\rfloor+1)
     (n-\lfloor n/2\rfloor)^{h\log(n-\lfloor n/2\rfloor)}       \\
&\le n^2n^{h\log(2n/3)}=n^{h\log n-h(\log3-\log2)+2}=n^{h\log n}
\quad.
\end{split}
\end{equation*}
\end{proof}

%-------------------------------------------------------------------------------
\begin{theorem}\label{theorem:SizeThreshold}
For any $k\ge0$ the size of\/ $\th kn\avec1n$ is $n^{\Ord{\log n}}$.
\end{theorem}

%-------------------------------------------------------------------------------
\begin{proof}
It immediately follows from Proposition~\vref{proposition:LargestThresholdFormula} and Lemma~\vref{lemma:SizeThresholdMax}.
\end{proof}

%-------------------------------------------------------------------------------
\begin{remark}\label{remark:UpsideDownCoweakening}
Given $n>1$, let $p=\lfloor n/2\rfloor$ and $q=n-p$. For $0\le k\le q$ and $1\le l\le p$, the following derivation is well defined:
\[
\vlinf{\gwu}
      {}
      {\fff}
      {\vls({\vlnos(\th pp\avec1p)}\{a_l/\fff\}.\th kq\avec{p+1}n)}
=
\vls(
\vlinf{\gwu}
      {}
      {\vls(\ttt)}
      {\vls(a_1.\cdots.a_{l-1}.a_{l+1}.\cdots.a_p.\th kq\avec{p+1}n)}
.\fff)
\quad.
\]
Analogously, for $0\le k\le p$ and $p+1\le l\le n$, we can define the following derivation:
\[
\vlinf{\gwu}
      {}
      {\fff}
      {\vls(\th kp\avec1p.{\vlnos(\th qq\avec{p+1}n)}\{a_l/\fff\})}
=
\vls(
\vlinf{\gwu}
      {}
      {\vls(\ttt)}
      {\vls(\th kp\avec1p.a_{p+1}.\cdots.a_{l-1}.a_{l+1}.\cdots.a_n)}
.\fff)
\quad.
\]
Both classes of derivations are used in Definition~\vref{definition:AuxillaryThresholdDerivation}.
\end{remark}

\newcommand{\Uth}[3]{\mathop{\mathsf\Upsilon_{#1,#2}^{#3}}}
\newcommand{\Dth}[3]{\mathop{\mathsf\Delta_{#1,#2}^{#3}}}
\newcommand{\Gth}[3]{\mathop{\Gammasf_{#1,#2}^{#3}}}
%-------------------------------------------------------------------------------
\begin{definition}\label{definition:AuxillaryThresholdDerivation}
Consider $n>0$, distinct atoms $a_1$, \dots, $a_n$, and let $p=\lfloor n/2\rfloor$ and $q=n-p$.
\begin{itemize}
%---------------------------------------
%---------------------------------------
\item
For $n>1$ and $1\le l\le n$, we define the derivations $\Uth kln\avec1n$ and $\Dth kln\avec1n$ as follows:
\[
\Uth kln\avec1n=\begin{cases}
\vlinf{\gwu}
      {}
      {\fff}
      {\vls({\vlnos(\th pp\avec1p)}\{a_l/\fff\}.\th{k-p}q\avec{p+1}n)}
             &\text{if $p\le k\le n$ and $l\le p$}\\
\noalign{\medskip}
\vlinf{\gwu}
      {}
      {\fff}
      {\vls(\th{k-q}p\avec1p.{\vlnos(\th qq\avec{p+1}n)}\{a_l/\fff\})}
             &\text{if $q\le k\le n$ and $p<l$}\\
\noalign{\medskip}
\fff         &\text{otherwise}
              \end{cases}
\]
and
\[
\Dth kln\avec1n=\begin{cases}
\vlinf{\gwd}
      {}
      {\th kq\avec{p+1}n}
      {\fff}
             &\text{if $0<k\le q$ and $l\le p$}\\
\noalign{\medskip}
\vlinf{\gwd}
      {}
      {\th kp\avec1p}
      {\fff}
             &\text{if $0<k\le p$ and $p<l$}\\
\noalign{\medskip}
\fff         &\text{otherwise}
              \end{cases}\quad.
\]
%---------------------------------------
%---------------------------------------
\item
For $k\ge0$ and $1\le l\le n$, we define the derivations $\vlsmash{\Gth kln\avec1n}$, recursively on $n$, as follows:
\begin{itemize}
%---------------------------------------
\item $\Gth 011(a_1)=\ttt$;
%---------------------------------------
\item for $k>0$, $\Gth k11(a_1)=\fff$;
%---------------------------------------
\item for $k>n$, $\Gth kln\avec1n=\fff$;
%---------------------------------------
\item for $n>1$ and $k\le n$, let
\[
\Gth kln\avec1n=\begin{cases}
%---------------------------------------
\vls[
\bigvee_{\begin{subarray}{l}i+j=k      \\ 
                            0\le i<p   \\ 
                            0\le j\le q
         \end{subarray}}(
\Gth ilp\avec1p.
\th jq\avec{p+1}n).
\Uth kln\avec1n.\Dth{k+1}ln\avec1n]
&\text{if $l\le p$}\\
\noalign{\medskip}
%---------------------------------------
\vls[
\bigvee_{\begin{subarray}{l}i+j=k      \\
                            0\le i\le p\\ 
                            0\le j<q
         \end{subarray}}(
\th ip\avec1p.
\Gth j{l-p}q\avec{p+1}n).
\Uth kln\avec1n.\Dth{k+1}ln\avec1n]
&\text{if $p<l$}
\end{cases}
\quad.
\]
%---------------------------------------
\end{itemize}
%---------------------------------------
%---------------------------------------
\end{itemize}
\end{definition}


%-------------------------------------------------------------------------------
\begin{example}\label{example:AuxillaryThresholdDerivations}
See, in Figure~\vref{figure:AuxillaryThresholdDerivations}, some examples of derivations $\vlsmash{\Gth kln\avec1n}$. Note that, for clarity, we removed all instances of the trivial derivations $\Uth112\avec12=\Uth122\avec12=\Uth113\avec13=\vldownsmash{\vlinf\gwu{}\fff\fff}$. We can do so because these derivation instances appear as disjuncts.
\end{example}

%-------------------------------------------------------------------------------
\begin{figure}
\begin{eqnarray*}
%---------------------------------------
\Gth 015\avecletter&=&
\vls [\ttt.\vlderivation{
\vlin{}{}{b}{
\vlhy{\vls \fff}
}}
.\vlderivation{
\vlin{}{}{\vls [c.d.e]}{
\vlhy{\vls \fff}
}}
]\quad,\\
\noalign{\smallskip}
%---------------------------------------
\Gth 115\avecletter&=&
\vls [b.([\ttt.\vlderivation{
\vlin{}{}{b}{
\vlhy{\vls \fff}
}}
].[c.d.e]).\vlderivation{
\vlin{}{}{\vls [(c.[d.e]).(d.e)]}{
\vlhy{\vls \fff}
}}
]\quad,\\
\noalign{\smallskip}
%---------------------------------------
\Gth 215\avecletter&=&
\vls [(b.[c.d.e]).([\ttt.\vlderivation{
\vlin{}{}{b}{
\vlhy{\vls \fff}
}}
].[(c.[d.e]).(d.e)]).\vlderivation{
\vlin{}{}{\vls \fff}{
\vlhy{\vls (\fff.b)}
}}
.\vlderivation{
\vlin{}{}{\vls (c.d.e)}{
\vlhy{\vls \fff}
}}
]\quad,\\
\noalign{\smallskip}
%---------------------------------------
\Gth 315\avecletter&=&
\vls [(b.[(c.[d.e]).(d.e)]).([\ttt.\vlderivation{
\vlin{}{}{b}{
\vlhy{\vls \fff}
}}
].c.d.e).\vlderivation{
\vlin{}{}{\vls \fff}{
\vlhy{\vls (\fff.b.[c.d.e])}
}}
]\quad,\\
\noalign{\smallskip}
%---------------------------------------
\Gth 415\avecletter&=&
\vls [(b.c.d.e).\vlderivation{
\vlin{}{}{\vls \fff}{
\vlhy{\vls (\fff.b.[(c.[d.e]).(d.e)])}
}}
]\quad,\\
\noalign{\smallskip}
%---------------------------------------
\Gth 515\avecletter&=&
\vlderivation{
\vlin{}{}{\vls \fff}{
\vlhy{\vls (\fff.b.c.d.e)}
}}
\quad,\\
\noalign{\smallskip}
%---------------------------------------
\Gth 035\avecletter&=&
\vls [\ttt.\vlderivation{
\vlin{}{}{\vls [d.e]}{
\vlhy{\vls \fff}
}}
.\vlderivation{
\vlin{}{}{\vls [a.b]}{
\vlhy{\vls \fff}
}}
]\quad,\\
\noalign{\smallskip}
%---------------------------------------
\Gth 135\avecletter&=&
\vls [([a.b].[\ttt.\vlderivation{
\vlin{}{}{\vls [d.e]}{
\vlhy{\vls \fff}
}}
]).d.e.\vlderivation{
\vlin{}{}{\vls (d.e)}{
\vlhy{\vls \fff}
}}
.\vlderivation{
\vlin{}{}{\vls (a.b)}{
\vlhy{\vls \fff}
}}
]\quad,\\
\noalign{\smallskip}
%---------------------------------------
\Gth 235\avecletter&=&
\vls [(a.b.[\ttt.\vlderivation{
\vlin{}{}{\vls [d.e]}{
\vlhy{\vls \fff}
}}
]).([a.b].[d.e.\vlderivation{
\vlin{}{}{\vls (d.e)}{
\vlhy{\vls \fff}
}}
]).(d.e).\vlderivation{
\vlin{}{}{\vls \fff}{
\vlhy{\vls (\fff.[d.e])}
}}
]\quad,\\
\noalign{\smallskip}
%---------------------------------------
\Gth 335\avecletter&=&
\vls [(a.b.[d.e.\vlderivation{
\vlin{}{}{\vls (d.e)}{
\vlhy{\vls \fff}
}}
]).([a.b].[(d.e).\vlderivation{
\vlin{}{}{\vls \fff}{
\vlhy{\vls (\fff.[d.e])}
}}
]).\vlderivation{
\vlin{}{}{\vls \fff}{
\vlhy{\vls (\fff.d.e)}
}}
]\quad,\\
\noalign{\smallskip}
%---------------------------------------
\Gth 435\avecletter&=&
\vls [(a.b.[(d.e).\vlderivation{
\vlin{}{}{\vls \fff}{
\vlhy{\vls (\fff.[d.e])}
}}
]).\vlderivation{
\vlin{}{}{\vls \fff}{
\vlhy{\vls ([a.b].\fff.d.e)}
}}
]\quad,\\
\noalign{\smallskip}
%---------------------------------------
\Gth 535\avecletter&=&
\vlderivation{
\vlin{}{}{\vls \fff}{
\vlhy{\vls (a.b.\fff.d.e)}
}}\quad,\\
\noalign{\smallskip}
%---------------------------------------
\Gth 055\avecletter&=&
\vls [\ttt.\vlderivation{
\vlin{}{}{d}{
\vlhy{\vls \fff}
}}
.\vlderivation{
\vlin{}{}{c}{
\vlhy{\vls \fff}
}}
.\vlderivation{
\vlin{}{}{\vls [a.b]}{
\vlhy{\vls \fff}
}}
]\quad,\\
\noalign{\smallskip}
%---------------------------------------
\Gth 155\avecletter&=&
\vls [([a.b].[\ttt.\vlderivation{
\vlin{}{}{d}{
\vlhy{\vls \fff}
}}
.\vlderivation{
\vlin{}{}{c}{
\vlhy{\vls \fff}
}}
]).(c.[\ttt.\vlderivation{
\vlin{}{}{d}{
\vlhy{\vls \fff}
}}
]).d.\vlderivation{
\vlin{}{}{\vls (a.b)}{
\vlhy{\vls \fff}
}}
]\quad,\\
\noalign{\smallskip}
%---------------------------------------
\Gth 255\avecletter&=&
\vls [(a.b.[\ttt.\vlderivation{
\vlin{}{}{d}{
\vlhy{\vls \fff}
}}
.\vlderivation{
\vlin{}{}{c}{
\vlhy{\vls \fff}
}}
]).([a.b].[(c.[\ttt.\vlderivation{
\vlin{}{}{d}{
\vlhy{\vls \fff}
}}
]).d]).(c.d).\vlderivation{
\vlin{}{}{\vls \fff}{
\vlhy{\vls (d.\fff)}
}}
]\quad,\\
\noalign{\smallskip}
%---------------------------------------
\Gth 355\avecletter&=&
\vls [(a.b.[(c.[\ttt.\vlderivation{
\vlin{}{}{d}{
\vlhy{\vls \fff}
}}
]).d]).([a.b].[(c.d).\vlderivation{
\vlin{}{}{\vls \fff}{
\vlhy{\vls (d.\fff)}
}}
]).\vlderivation{
\vlin{}{}{\vls \fff}{
\vlhy{\vls (c.d.\fff)}
}}
]\quad,\\
\noalign{\smallskip}
%---------------------------------------
\Gth 455\avecletter&=&
\vls [(a.b.[(c.d).\vlderivation{
\vlin{}{}{\vls \fff}{
\vlhy{\vls (d.\fff)}
}}
]).\vlderivation{
\vlin{}{}{\vls \fff}{
\vlhy{\vls ([a.b].c.d.\fff)}
}}
]\quad,\\
\noalign{\smallskip}
%---------------------------------------
\Gth 555\avecletter&=&
\vlderivation{
\vlin{}{}{\vls \fff}{
\vlhy{\vls (a.b.c.d.\fff)}
}}\quad.
\end{eqnarray*}
\caption{Examples of $\Gth kl5\avecletter$, where $\avecletter=(a,b,c,d,e)$.}
\label{figure:AuxillaryThresholdDerivations}
\end{figure}


%-------------------------------------------------------------------------------
\begin{theorem}\label{theorem:AuxillaryThresholdDerivations}
For any $n>0$, $k\ge0$ and\/ $1\le l\le n$, the derivation\/ $\vlsmash{\Gth kln\avec1n}$ has shape
\[
\vlder{}{\{\awd,\awu\}}{(\th{k+1}n\avec1n)\{a_l/\ttt\}}
                       {(\th kn\avec1n)\{a_l/\fff\}}
\quad,
\]
and\/ $\size{\Gth kln\avec1n}$ is $n^{\Ord{\log n}}$.
\end{theorem}

%-------------------------------------------------------------------------------
\begin{proof}
The shape of $\Gth kln\avec1n$ can be verified by inspecting Definition~\vref{definition:AuxillaryThresholdDerivation}. For example, this is the case when $n>1$ and $l\le p\le k<q$, where $p=\lfloor n/2\rfloor$ and $q=n-p$:
\vlstore{\noalign{\medskip}
\vls[
\textstyle\bigvee_{\begin{subarray}{l}i+j=k      \\
                                      0\le i<p   \\
                                      0\le j\le q
                   \end{subarray}}(
\vlder{\Gth ilp\avec1p}
      {}
      {(\th{i+1}p\avec1p)\{a_l/\ttt\}}
      {(\th ip\avec1p)\{a_l/\fff\}}
.
\th jq\avec{p+1}n)
.
\vlinf{\gwu}
      {}
      {\fff}
      {\vls({\vlnos(\th pp\avec1p)}\{a_l/\fff\}.\th{k-p}q\avec{p+1}n)}
.
\vlinf{\gwd}
      {}
      {\th{k+1}q\avec{p+1}n}
      {\fff}
]}
\begin{multline*}
\vlder{\Gth kln\avec1n}
      {}
      {(\th{k+1}n\avec1n)\{a_l/\ttt\}}
      {(\th kn\avec1n)\{a_l/\fff\}}
={}\\
\vlread
\quad.
\end{multline*}
(Remember that
\[
\th kn\avec1n\equiv\bigvee_{\begin{subarray}{l}
                            i+j=k\\ 
                            0\le i\le p\\ 
                            0\le j\le q
                            \end{subarray}}
                   \vlsbr(\th ip\avec1p.\th jq\avec{p+1}n)
\]
and $\th0p\avec1p\equiv\ttt$.) General (co)weak\-en\-ing rule instances can be replaced by atomic ones because of Lemma~\vref{lemma:GenericWeakening}. The size bound on $\Gth kln\avec1n$ follows from Proposition~\vref{proposition:DerivationSubstitution} and Theorem~\vref{theorem:SizeThreshold}.
\end{proof}

\begin{definition}\label{definition:ThresholdDerivations}
Consider $n>0$, distinct atoms $a_1$, \dots, $a_n$. For $k\ge0$, we define the derivations $\vlsmash{\Gammasf^n_k\avec1n}$ as follows:
\[
\Gammasf^n_k\avec1n\quad=\quad
\vlderivation
{
 \vlde{}{\{\cou\}}
 {
  \vlsbr
  (
   \vlder{}{\{\ssd\}}
   {
    \vlsbr
    [
     \vlder{}{\{\acd\}}
     {
      a_1
     }
     {
      \vls[a_1.\cdots.a_1]
     }
    \;\;.\;\;
     (\th {k+1}n\avec1n)\{a_1/\fff\}
    ]
   }
   {
    \th {k+1}n\avec1n
   }
  \;\;.\;\;\cdots\;\;.\;\;
   \vlder{}{\{\ssd\}}
   {
    \vlsbr
    [
     \vlder{}{\{\acd\}}
     {
      a_n
     }
     {
      \vls[a_n.\cdots.a_n]
     }
    \;\;.\;\;
     (\th {k+1}n\avec1n)\{a_n/\fff\}
    ]
   }
   {
    \th {k+1}n\avec1n
   }
  )
 }
 {
  \vlde{}{\{\cod\}}
  {
   \th {k+1}n\avec1n
  }
  {
   \vlhy
   {
    \vlsbr
    [
     \vlder{}{\{\ssu\}}
     {
      \th {k+1}n\avec1n
     }
     {
      \vlsbr
      (
       \vlder{}{\acu}
       {
        \vls(a_1.\cdots.a_1)
       }
       {
        a_1
       }
      \;\;.\;\;
       \vlder{\Gth kin\avec1n}{}
       {
        (\th {k+1}n\avec1n)\{a_1/\ttt\}
       }
       {
        (\th kn\avec1n)\{a_1/\fff\}
       }
      )
     }
    \;\;.\;\;\cdots\;\;.\;\;
     \vlder{}{\{\ssu\}}
     {
      \th {k+1}n\avec1n
     }
     {
      \vlsbr
      (
       \vlder{}{\acu}
       {
        \vls(a_n.\cdots.a_n)
       }
       {
        a_n
       }
      \;\;.\;\;
       \vlder{\Gth knn\avec1n}{}
       {
        (\th {k+1}n\avec1n)\{a_n/\ttt\}
       }
       {
        (\th kn\avec1n)\{a_n/\fff\}
       }
      )
     }
    ]
   }
  }
 }
}\quad.
\]
\end{definition}

%-------------------------------------------------------------------------------
\begin{theorem}\label{theorem:ThresholdDerivations}
For any $n>0$ and $k\ge0$, the derivation\/ $\vlsmash{\Gammasf^n_k\avec1n}$ has shape
\[
\vlder{}{\SKS\setminus\{\aid,\aiu\}}
{
 \vls([a_1.(\th{k+1}n\avec1n)\{a_1/\fff\}].\cdots.[a_n.(\th{k+1}n\avec1n)\{a_n/\fff\}])
}
{
 \vls[(a_1.(\th{k}n\avec1n)\{a_1/\fff\}).\cdots.(a_n.(\th{k}n\avec1n)\{a_n/\fff\})]
}
\quad,
\]
and\/ $\size{\Gammasf^n_k\avec1n}$ is $n^{\Ord{\log n}}$.
\end{theorem}
%------------


\newcommand{\frqmis}{{\mathsf{qmis}}}
\newcommand{\QMISR}{\mathsf{QMISR}}
%---------------------------------------
\begin{definition}\label{definition:QuasipolynomialMultipleSubflowRemoval}
For every $n>0$, we define
\begin{itemize}
\item the reduction ${\to_{\frqmis_n}}$ (where $\frqmis$ stands for \emph{quasipolynomial multiple isolated subflows}\index{Quasipolynomial Multiple Isolated Subflows Remover!reduction}); and
\item and the operator the \emph{Quasipolynomial Multiple Isolated Subflows Remover}\index{Quasipolynomial Multiple Isolated Subflows Remover!operator}, $\QMISR_n$,
\end{itemize}
to be special cases of ${\to_{\frmis_n}}$ and $\MISR_n$, respectively, such that, given atoms $\avec 1n$,
\begin{itemize}
\item $N=n$;
\item for $0\le k\le n$ and $1\le i\le n$, $\gamma_{k,i}=(\th kn\avec1n)\{a_i/\fff\}$; and
\item for $1\le k\le n$, $\Gammasf_k=\Gammasf^n_k\avec1n$.
\end{itemize}
\end{definition}
%---------------

%-------------------------------------------------------------------------
\begin{theorem}\label{theorem:SoundQuasipolynomialMultipleSubflowsRemoval}
For every $n>0$, $\to_{\frqmis_n}$ is sound; moreover, if\/ $\Phi\to_{\frqmis_n}\Psi$, then the size of $\Psi$ depends polynomially on the size of $\Phi$ and quasipolynomially on $n$.
\end{theorem}

\begin{proof}
The result follows by Theorem~\vref{theorem:SoundMultipleIsolatedSubflowsRemoval}, Definition~\vref{definition:ThresholdFormulae} and Theorem~\ref{theorem:ThresholdDerivations}.
\end{proof}
%----------

%------------------------------------------------------------
\begin{proposition}\label{proposition:QuasipolynomialMultipleIsolatedSubflowRemover}
Given atoms $a_1$, $\dots$, $a_n$ and a derivation $\Phi$ that is in simple form with respect to $a_1$, $\dots$, $a_n$,
\begin{enumerate}
\item $\QMISR_n(\Phi,a_1,\dots,a_n)$ is weakly streamlined with respect to $a_1$, $\dots$, $a_n$;
\item for any atom $b$,
\begin{itemize}
\item if $\Phi$ is weakly streamlined with respect to $b$, then $\QMISR_n(\Phi,a_1,\dots,a_n)$ is weakly streamlined with respect to $b$, and
\item if $b$ is not the dual of any of $a_1$, $\dots$, $a_n$ and $\Phi$ is in simple form with respect to $b$, then $\QMISR_n(\Phi,a_1,\dots,a_n)$ is in simple form with respect to $b$; and
\end{itemize}
\item the size of\/ $\QMISR_n(\Phi,a_1,\dots,a_n)$ depends polynomially on the size of\/ $\Phi$, and quasipolynomially on $n$.
\end{enumerate}
\end{proposition}

\begin{proof}
The statements follow by Proposition~\vref{proposition:MultipleIsolatedSubflowRemover} and Theorem~\vref{theorem:ThresholdDerivations}.
\end{proof}
%----------

