\chapter{Transformations}

\section{Local Transformations}

\begin{definition}
In Figure~\ref{FigRed}, we define graphical expressions of the kind $r\colon\phi'\to\psi'$, where $r$ is a name and $\phi'$ and $\psi'$ are flows.
\end{definition}

\TODO{In Figures~\ref{FigRed}, \ref{FigRedW} and \ref{FigRedC} extend the edges so the atomic flows on either side of the arrow has equal height.}

\newcommand{\rwdcd}{{{\mathsf w}{\downarrow}{\hbox{-}}{\mathsf c}{\downarrow}}}
\newcommand{\rwdiu}{{{\mathsf w}{\downarrow}{\hbox{-}}{\mathsf i}{\uparrow  }}}
\newcommand{\rwdwu}{{{\mathsf w}{\downarrow}{\hbox{-}}{\mathsf w}{\uparrow  }}}
\newcommand{\rwdcu}{{{\mathsf w}{\downarrow}{\hbox{-}}{\mathsf c}{\uparrow  }}}
\newcommand{\rcuwu}{{{\mathsf c}{\uparrow  }{\hbox{-}}{\mathsf w}{\uparrow  }}}
\newcommand{\rcdwu}{{{\mathsf c}{\downarrow}{\hbox{-}}{\mathsf w}{\uparrow  }}}
\newcommand{\rcdiu}{{{\mathsf c}{\downarrow}{\hbox{-}}{\mathsf i}{\uparrow  }}}
\newcommand{\rcdcu}{{{\mathsf c}{\downarrow}{\hbox{-}}{\mathsf c}{\uparrow  }}}
\newcommand{\ridwu}{{{\mathsf i}{\downarrow}{\hbox{-}}{\mathsf w}{\uparrow  }}}
\newcommand{\ridcu}{{{\mathsf i}{\downarrow}{\hbox{-}}{\mathsf c}{\uparrow  }}}
\newcommand{\rcd}{{\mathsf c}{\downarrow  }}
\newcommand{\rcu}{{\mathsf c}{\uparrow  }}
%---------------------------------------
\begin{figure}[tbp]
\[
\begin{array}{@{}c@{}c@{}}
%-------------------
\rwdcd\colon\quad\afraise{\atomicflow{
( 0  ,0)*{\afacd{}{}{}\one{}\two};
(-2  ,4)*{\afawdnw{}{}};
(-3.5,0)*{\invisiblemark};
( 3.5,0)*{\invisiblemark}}}
\quad\to\quad
\atomicflow{
( 0  ,3.3)*{\aflabelright{\one,\two}};
( 0  ,3  )*{\afvj6};
(-1.5,0  )*{\invisiblemark};
( 3  ,0  )*{\invisiblemark}}
&\qquad
%-------------------
\rcuwu\colon\quad\aflower{\atomicflowinv{
( 0  ,0)*{\afacu{}{}{}\two{}\one};
(-2  ,6)*{\afawunw{}{}};
( 3.5,0)*{\invisiblemark}}}
\quad\to\quad
\atomicflow{
(0,3  )*{\afvj6};
(0,3.3)*{\aflabelright{\one,\two}};
(3,0  )*{\invisiblemark}}
\\
%-------------------
\rwdiu\colon\quad\atomicflow{
( 0  ,0)*{\afaiu{}{}{}\one{}{}};
(-2  ,4)*{\afawdnw{}{}};
(-3.5,0)*{\invisiblemark};
( 3.5,0)*{\invisiblemark}}
\quad\to\quad
\aflower{\atomicflow{
( 0  ,0)*{\afawu{}{}{}\one};
(-1.5,0)*{\invisiblemark};
( 1.5,0)*{\invisiblemark}}}
&\qquad
%-------------------
\ridwu\colon\quad\atomicflowinv{
( 0  ,0)*{\afaid{}{}{}\one{}{}};
(-2  ,6)*{\afawunw{}{}};
( 3.5,0)*{\invisiblemark}}
\quad\to\quad
\afraise{\atomicflowinv{
(0  ,0)*{\afawd{}{}{}\one};
(1.5,0)*{\invisiblemark}}}
\\
%-------------------
\multispan2{\hfil$
\rwdwu\colon\quad\atomicflow{
( 0  ,6)*{\afawd{}{}{}{}};
( 0  ,0)*{\afawunw{}{}{}{}};
(-1.5,0)*{\invisiblemark};
( 1.5,0)*{\invisiblemark}}
\quad\to\quad
\atomicflow{}
$\hfil}\\
%-------------------
\rwdcu\colon\quad\afraise{\atomicflow{
( 0  ,-4)*{\afacu\one{}{}\two{}{}};
( 0  , 0)*{\afawdnw{}{}};
(-3.5, 0)*{\invisiblemark};
( 3.5, 0)*{\invisiblemark}}}
\quad\to\quad
\atomicflow{
(-2  ,-4)*{\afawd{}{}\one{}};
( 2  ,-4)*{\afawd{}{}{}\two};
(-3.5, 0)*{\invisiblemark};
( 3.5, 0)*{\invisiblemark}}
&\qquad
%-------------------
\rcdwu\colon\quad\aflower{\atomicflowinv{
( 0  ,-4)*{\afacd\one{}{}\two{}{}};
( 0  , 2)*{\afawunw{}{}};
(-3.5, 0)*{\invisiblemark};
( 3.5, 0)*{\invisiblemark}}}
\quad\to\quad
\atomicflowinv{
(-2  ,-4)*{\afawu{}{}\one{}};
( 2  ,-4)*{\afawu{}{}{}\two};
(-3.5, 0)*{\invisiblemark};
( 3.5, 0)*{\invisiblemark}}
\\
%-------------------
\rcdiu\colon\quad\aflower{\atomicflow{
(   6, 6)*{\afvjd4{}\three{}{}};
(   3, 0)*{\afaiuex{}{}{}{}{}{}32};
(   0, 4)*{\afacdnw\one{}{}\two};
(-3.5, 0)*{\invisiblemark};
( 7.5, 0)*{\invisiblemark}}}
\quad\to\quad
\aflower{\atomicflow{
(  10,8)*{\afacu{}{}{}{}{}\three};
(   0,8)*{\afvjd8\one{}};
(   4,8)*{\afvjd8{}\two};
(   6,2)*{\afaiunw{}{}};
(   6,0)*{\afaiuex{}{}{}{}{}{}31}}}
&\qquad
%-------------------
\ridcu\colon\quad\afraise{\atomicflowinv{
(   6,6)*{\afvju4{}\three{}{}};
(   3,0)*{\afaidex {}{}{}{}{}{}32};
(   0,6)*{\afacunw\one{}{}\two};
(-3.5,0)*{\invisiblemark};
( 7.5,0)*{\invisiblemark}}}
\quad\to\quad
\afraise{\atomicflowinv{
(  10,8)*{\afacd{}{}{}{}{}\three};
(   0,8)*{\afvju8\one{}};
(   4,8)*{\afvju8{}\two};
(   6,4)*{\afaidnw{}{}};
(   6,0)*{\afaidex{}{}{}{}{}{}31}}}
\\
%-------------------
\multispan2{\hfil$
\rcdcu\colon\quad\atomicflow{
( 0,6)*{\afacd\one{}{}\two{}{}};
( 0,0)*{\afacunw\three{}{}\four};
(-4,0)*{\invisiblemark};
( 4,0)*{\invisiblemark}}
\quad\to\quad
\atomicflow{
(0,12)*{\afacu{}{}{}{}\one{}};
(6,12)*{\afacu{}{}{}{}{}\two};
( 0,0)*{\afacd{}{}{}{}\three{}};
( 6,0)*{\afacd{}{}{}{}{}\four};
(-2,6)*{\afvj4};
( 8,6)*{\afvj4};
( 3,6)*{\afex24}}
$\hfil}\\
%-------------------
\rcd\colon\quad\aflower{\atomicflow{
(   0, 4)*{\afacd\one{}{}\two{}\three};
(-3.5, 0)*{\invisiblemark};
( 3.5, 0)*{\invisiblemark}
}}
\quad\to\quad
\aflower{\atomicflow{
(13,12)*{\afaidex{}{}{}{}{}{}32};
(16, 4)*{\afvju8{}\three};
(10, 6)*{\afacunw{}{}{}{}};
( 0, 8)*{\afvjd8\one{}};
( 4, 8)*{\afvjd8{}\two};
( 6, 2)*{\afaiunw{}{}};
( 6, 0)*{\afaiuex{}{}{}{}{}{}31}}}
&\qquad
%-------------------
\rcu\colon\quad\afraise{\atomicflowinv{
(   0,-4)*{\afacu\one{}{}\two{}\three};
(-3.5, 0)*{\invisiblemark};
( 3.5, 0)*{\invisiblemark}
}}
\quad\to\quad
\aflower{\atomicflow{
(13,-12)*{\afaiuex{}{}{}{}{}{}32};
(16, -4)*{\afvjd8{}\three};
(10,-8.25)*{\afacdnw{}{}{}{}};
( 0, -8)*{\afvju8\one{}};
( 4, -8)*{\afvju8{}\two};
( 6, -4)*{\afaidnw{}{}};
( 6,  0)*{\afaidex{}{}{}{}{}{}31}}}
\\
\end{array}
\]
\caption{Atomic-flow reduction rules.}
\label{FigRed}
\end{figure}%

%---------------------------------------
\begin{example}
The `reduction' on the left, when used inside a larger atomic flow, might create a situation as on the right:
\nopagebreak[4]\medskip\afnegspace
\[
\atomicflow{
( 0  ,0)*{\afacu{}{}{}{}{}{}};
( 0  ,4)*{\afawdnw{}{}}}
\quad\to\quad
\atomicflow{
( 0  ,0)*{\afaid{}{}{}{}{}{}}}
\qquad\qquad
\atomicflow{
( 0  , 8)*{\afacu{}{}{}{}\ppl{}};
( 0  ,12)*{\afawdnw{}{}};
(-2  , 4)*{\aflabelleft\ppl};
( 2  , 4)*{\aflabelright\ppl};
( 0  , 0)*{\afacd{}{}{}{}\ppl{}};
(-3.5, 0)*{\invisiblemark};
( 3.5, 0)*{\invisiblemark}}
\quad\to\quad
\atomicflow{
( 0  ,4)*{\afaidnw{}{}};
( 0  ,0)*{\afacd\ppl{}{}{\scriptstyle?}\ppl{}};
(-3.5,0)*{\invisiblemark};
( 3.5,0)*{\invisiblemark}}
\quad,
\] 
where the graph at the right is not an atomic flow, for lack of a polarity assignment.
\end{example}

This prompts us to define reduction rules and reductions for atomic flows as follows.

%---------------------------------------
\begin{definition}
An (\emph{atomic-flow}) \emph{reduction rule $r$ from flow $\phi'$ to flow $\psi'$} is a quadruple $(\phi',\psi',f,g)$ such that:
\begin{enumerate}
\item $f$ is a one-to-one map from the upper edges of $\phi'$ to the upper edges of $\psi'$,
\item $g$ is a one-to-one map from the lower edges of $\phi'$ to the lower edges of $\psi'$,
\item for every polarity assignment $\pi$ for $\phi'$, there is a polarity assignment $\pi'$ for $\psi'$ such that $\pi'(f(\epsilon))=\pi(\epsilon)$ and $\pi'(g(\epsilon'))=\pi(\epsilon')$, for any upper edge $\epsilon$ and any lower edge $\epsilon'$ of $\phi'$;
\end{enumerate}
we define reduction rules with graphical expressions $r\colon\phi'\to\psi'$, where $f$ and $g$ are indicated by labelling edges. A binary relation $R$ on the set of atomic flows is called an (\emph{atomic-flow}) \emph{reduction} if, whenever $\phi\mathrel{R}\psi$, there is a one-to-one map from the upper edges of $\phi$ to the upper edges of $\psi$ and a one-to-one map from the lower edges of $\phi$ to the lower edges of $\psi$. For every reduction rule $r\colon\phi'\to\psi'$, the reduction ${\to_r}$ is defined, such that $\phi\to_r\psi$ if and only if $\phi'$ appears as a subgraph in $\phi$ and we obtain $\psi$ by replacing $\phi'$ with $\psi'$ in $\phi$, while respecting the correspondence of edges; we call this operation a \emph{reduction by $r$}.
\end{definition}

\begin{remark}
The condition on polarity assignments for a reduction rule $r$ guarantees that the $\psi$ in $\phi\to_r\psi$ is a proper atomic flow, if $\phi$ is one.
\end{remark}

\begin{remark}
Because of the condition on polarity assignments for reduction rules, two distinct connected components in a flow cannot be connected by a reduction. To see that this is impossible, consider the following `reduction rule', which violates the condition on polarity assignments:
\[
\aflower{\atomicflow{
(-2,0)*{\afawu{}{}{}{}};
( 2,0)*{\afawu{}{}{}{}}}}
\quad\to\quad
\aflower{\atomicflow{
( 0  , 2)*{\afaiu{}{}{}{}{}{}}}}
\quad.
\]
\afnegspace
For this `reduction rule' there exist both valid (left) and invalid (right) polarity assignments:
\[
\aflower{\atomicflow{
(-2  ,0)*{\afawu{}{}\ppl{}};
( 2  ,0)*{\afawu{}{}{}\pmi};
(-3.5,0)*{\invisiblemark};
( 3.5,0)*{\invisiblemark}}}
\quad\to\quad
\aflower{\atomicflow{
( 0  ,2)*{\afaiu\ppl{}{}\pmi{}{}};
(-3.5,0)*{\invisiblemark};
( 3.5,0)*{\invisiblemark}}}
\qquad\qquad
\aflower{\atomicflow{
(-2,0)*{\afawu{}{}\ppl{}};
( 2,0)*{\afawu{}{}{}\ppl};
(-3.5,0)*{\invisiblemark};
( 3.5,0)*{\invisiblemark}}}
\quad\to\quad
\aflower{\atomicflow{
( 0  , 2)*{\afaiu\ppl{}{}{\scriptstyle?}{}{}};
(-3.5, 0)*{\invisiblemark};
( 3.5, 0)*{\invisiblemark}}}
\quad.
\]
\afnegspace
\end{remark}

It is immediate to check:

\begin{proposition}
The graphical expressions in Figure~\ref{FigRed} are atomic-flow reduction rules.
\end{proposition}

%---------------------------------------
\begin{definition}
A finite set of reduction rules is a \emph{flow rewriting system}. For every flow rewriting system $F=\{r_1,\dots,r_h\}$ we define ${\to_F}={\to_{r_1}\cup\cdots\cup{\to_{r_h}}}$. The reflexive transitive closure of $\to_F$ is denoted by $\to_F^\star$. Given a set of atomic flows $S$, we say that a flow rewriting system $F$ is \emph{terminating on $S$} if there is no infinite chain $\phi_1\to_F\phi_2\to_F\cdots$, for every $\phi_1\in S$; if $F$ is terminating on the set of atomic flows, we say that it is \emph{terminating}. We say that atomic flow $\phi$ is \emph{normal} for flow rewriting system $F$ if there is no atomic flow $\psi$ such that $\phi\to_F\psi$.
\end{definition}

\newcommand{\frw}{{\mathsf w}}
%---------------------------------------
\begin{definition}
The following flow rewriting system is called $\frw$:
\[
\{\;\rwdcd\;,\;\rcuwu\;,\;\rwdiu\;,\;\ridwu\;,\;\rwdwu\;,\;\rwdcu\;,\;\rcdwu\;\}
\quad.
\]
\end{definition}

\newcommand{\frc}{{\mathsf c}}
%---------------------------------------
\begin{definition}
The following flow rewriting system is called $\frc$:
\[
\{\;\rcdiu\;,\;\ridcu\;,\;\rcdcu\;\}\quad.
\]
\end{definition}

\TODO{Change the definition of reduction rule so they can apply under a given polarity assignment.}
%---------------------------------------
\begin{definition}
The reduction rules $\rcdiu^\ppl$, $\ridcu^\ppl$, $\rcdcu^\ppl$, $\rcd^\ppl$ and $\rcu^\ppl$ are the rules $\rcdiu$, $\ridcu$, $\rcdcu$, $\rcd$ and $\rcu$, respectively, where the (co)contraction vertex in the redex is positive under a given polarity assignment.
\end{definition}

\newcommand{\frcp}{{\frc^\ppl}}
%---------------------------------------
\begin{definition}
The following flow rewriting system is called $\frcp$:
\[
\{\;\rcdiu^\ppl\;,\;\ridcu^\ppl\;,\;\rcdcu^\ppl\;\}\quad.
\]
\end{definition}

\subsection{Soundness}

%---------------------------------------
\begin{definition}
A reduction $R$ is \emph{sound} if, for every $\phi$ and $\psi$ such that $\phi\mathrel{R}\psi$ and for every derivation $\Phi$ with atomic flow $\phi$, there is a derivation $\Psi$ with atomic flow $\psi$ such that $\Phi$ and $\Psi$ have the same premiss and conclusion; in this case we write $\Phi\mathrel{R}\Psi$. A reduction rule $r$ is \emph{sound} if $\to_r$ is sound.
\end{definition}

The proof of the following theorem is essentially contained in Figures~\ref{FigRedW} and \ref{FigRedC}.

%TODO: define \ot (should use \shortleftarrow, but it gives errors)
\newcommand{\ot}{\mathbin\leftarrow}

%---------------------------------------
\begin{figure}[tbp]
\[
\begin{array}{@{}l@{}c@{}}
%---------------------------------------
\rwdcd\colon\hfil\afraise{\atomicflow{
( 0  ,0)*{\afacd\three{}{}\one{}\two};
(-2  ,4)*{\afawdnw{}{}};
(-3.5,0)*{\invisiblemark};
( 3.5,0)*{\invisiblemark}}}
\quad\to\quad
\atomicflow{
( 0  ,3.3)*{\aflabelright{\one,\two}};
( 0  ,3  )*{\afvj6};
(-1.5,0  )*{\invisiblemark};
( 3  ,0  )*{\invisiblemark}}
&\qquad
\vlder{\Phi}{}{\zeta\left\{\vlinf{}{}{a^\two}{\vls[a^\three.a^\one]}\right\}}
              {\xi\left\{\vlinf{}{}{a^\three}{\fff}\right\}}
\quad\to_\rwdcd\quad
\vlder{\Phi\{a^\three\ot\fff\}}{}{\zeta\left\{\vlinf{=}{}{a^{\one,\two}}{\vls[\fff.a^{\one,\two}]}\right\}}
              {\xi\{\fff\}}
\\
\noalign{\bigskip}
%---------------------------------------
\rwdiu\colon\hfil\atomicflow{
( 0  ,0)*{\afaiu\two {}{}\one{}{}};
(-2  ,4)*{\afawdnw{}{}};
(-3.5,0)*{\invisiblemark};
( 3.5,0)*{\invisiblemark}}
\quad\to\quad
\aflower{\atomicflow{
( 0  ,0)*{\afawu{}{}{}\one};
(-1.5,0)*{\invisiblemark};
( 1.5,0)*{\invisiblemark}}}
&\qquad
\vlder{\Phi}{}{\zeta\left\{\vlinf{}{}{\fff}{\vls(a^\two.\bar a^\one)}\right\}}
              {\xi\left\{\vlinf{}{}{a^\two}{\fff}\right\}}
\quad\to_\rwdiu\quad
\vlder{\Phi\{a^\two\ot\fff\}}{}{\zeta\left\{\vlinf{=}{}{\fff}{\vls(\fff\;.\;\vlinf{}{}{\ttt}{\bar a^\one})}\right\}}
              {\xi\{\fff\}}
\\
\noalign{\bigskip}
%---------------------------------------
\rwdwu\colon\hfil\atomicflow{
( 0  ,6)*{\afawd{}{}{}\one};
( 0  ,0)*{\afawunw{}{}{}{}};
(-1.5,0)*{\invisiblemark};
( 1.5,0)*{\invisiblemark}}
\quad\to\quad
\atomicflow{}
&\qquad
\vlder{\Phi}{}{\zeta\left\{\vlinf{}{}{\ttt}{a^\one}\right\}}
              {\xi\left\{\vlinf{}{}{a^\one}{\fff}\right\}}
\quad\to_\rwdwu\quad
\vlder{\Phi\{a^\one\ot\fff\}}{}{\zeta\left\{\vlderivation
                           {
                            \vlin{=}{}{\ttt}
                            {
                             \vlin{\swi}{}{\vls[(\fff.\fff).\ttt]}
                             {
                              \vlin{=}{}{\vls(\fff.[\fff.\ttt])}
                              {
                               \vlhy{\fff}
                              }
                             }
                            }
                           }\right\}}
              {\xi\{\fff\}}
\\
\noalign{\bigskip}
%---------------------------------------
\rwdcu\colon\hfil\afraise{\atomicflow{
( 0  ,-4)*{\afacu\one{}{}\two{}\three};
( 0  , 0)*{\afawdnw{}{}};
(-3.5, 0)*{\invisiblemark};
( 3.5, 0)*{\invisiblemark}}}
\quad\to\quad
\atomicflow{
(-2  ,-4)*{\afawd{}{}\one{}};
( 2  ,-4)*{\afawd{}{}{}\two};
(-3.5, 0)*{\invisiblemark};
( 3.5, 0)*{\invisiblemark}}
&\qquad
\vlder{\Phi}{}{\zeta\left\{\vlinf{}{}{\vls(a^\one.a^\two)}{a^\three}\right\}}
              {\xi\left\{\vlinf{}{}{a^\three}{\fff}\right\}}
\quad\to_\rwdcu\quad
\vlder{\Phi\{a^\three\ot\fff\}}{}{\zeta\left\{\vlinf{=}{}{\vls(\vlinf{}{}{a^\one}{\fff}\;.\;\vlinf{}{}{a^\two}{\fff})}{\fff}\right\}}
              {\xi\{\fff\}}
\\
%---------------------------------------
\end{array}
\]
\caption{`Downwards' reduction rules for weakening and their soundness.}
\label{FigRedW}
\end{figure}%

\newbox\ContDownIntUp
\setbox\ContDownIntUp=
\hbox{$
\vlderivation
{
 \vlin{}{}{\fff}
 {
  \vlin{=}{}{\vlsbr(\vlinf{\swi}{}{\vlinf{=}{}{a^\one}{\vlsbr[\vlinf{}{}{\fff}{\vls(\bar a.a^\two)}\;.\;a^\one]}}{\vls(\bar a.[a^\two.a^\one])}\;\;\;.\;\;\;\bar a)}
  {
   \vlhy{\vlsbr([a^\one.a^\two]\;.\;\vlinf{}{}{\vls(\bar a.\bar a)}{\bar a^\three})}
  }
 }
}
$}
\newbox\ContDownContUp
\setbox\ContDownContUp=
\hbox{$
\vlinf{\med}{}{\vlsbr(\vlinf{}{}{a^\three}{\vls[a.a]}\;.\;\vlinf{}{}{a^\four}{\vls[a.a]})}
              {\vlsbr[\vlinf{}{}{\vls(a.a)}{a^\one}\;.\;\vlinf{}{}{\vls(a.a)}{a^\two}]}
$}
\newbox\ContDown
\setbox\ContDown=
\hbox{$
\vlinf\swi{}{
			 \vlsbr[\vlderivation
			        {
			         \vlin\swi{}
			         {
			          \vls[\vlinf{}{}
                           {\fff}
                           {\vls(\bar a.a^\one)}
                          .
                           \vlinf{}{}
                           {\fff}
                           {\vls(\bar a.a^\two)}
                          ]
                     }
                     {
                      \vlin\swi{}
                      {
                       \vls(\bar a.[(\bar a.a^\one).a^\two])
                      }
                      {
                       \vlhy
                       {\vls(\vlinf{}{}
                             {\vls(\bar a.\bar a)}
                             {\bar a}          
                            .
                             [a^\one.a^\two]
                            )
                       }
                      }
                     }
                    }
                   \;\;\;.
                    {a^\three}
                   ]
            }
            {
             \vls(\vlinf{}{}
                  {\vls[\bar a.a^\three]}
                  {\ttt}
                 .
                  [a^\one.a^\two]
                 )
            }
$}
%---------------------------------------
\begin{figure}[tbp]
\[
\begin{array}{@{}l@{}c@{}}
%---------------------------------------
\rcdiu\colon\aflower{\atomicflow{
(   6, 6)*{\afvjd4{}\three{}{}};
(   3, 0)*{\afaiuex{}\four{}{}{}{}32};
(   0, 4)*{\afacdnw\one{}{}\two};
(-3.5, 0)*{\invisiblemark};
( 7.5, 0)*{\invisiblemark}}}
\quad\to\quad
\aflower{\atomicflow{
(10,8)*{\afacu{}{}{}{}{}\three};
( 0,8)*{\afvjd8\one{}};
( 4,8)*{\afvjd8{}\two};
( 6,2)*{\afaiunw{}{}};
( 6,0)*{\afaiuex{}{}{}{}{}{}31}}}
&\qquad
\vlder{\Phi}{}{\zeta\left\{\vlinf{}{}{\fff}{\vls(a^\four.\bar a^\three)}\right\}}
              {\xi\left\{\vlinf{}{}{a^\four}{\vls[a^\one.a^\two]}\right\}}
\quad\to_\rcdiu\quad
\vlder{\Phi\{a^\four\ot\vls[a^\one.a^\two]\}}{}{\zeta\left\{\box\ContDownIntUp\right\}}
                                     {\xi\left\{\vls[a^\one.a^\two]\right\}}
\\
\noalign{\bigskip}
%---------------------------------------
\rcdcu\colon\hfil\atomicflow{
( 0,6)*{\afacd\one{}{}\two{}\five};
( 0,0)*{\afacunw\three{}{}\four};
(-4,0)*{\invisiblemark};
( 4,0)*{\invisiblemark}}
\quad\to\quad
\atomicflow{
( 0,12)*{\afacu{}{}{}{}\one{}};
( 6,12)*{\afacu{}{}{}{}{}\two};
( 0, 0)*{\afacd{}{}{}{}\three{}};
( 6, 0)*{\afacd{}{}{}{}{}\four};
(-2, 6)*{\afvj4};
( 8, 6)*{\afvj4};
( 3, 6)*{\afex24}}
&\qquad
\vlder{\Phi}{}{\zeta\left\{\vlinf{}{}{\vls(a^\three.a^\four)}{a^\five}\right\}}
              {\xi\left\{\vlinf{}{}{a^\five}{\vls[a^\one.a^\two]}\right\}}
\quad\to_\rcdcu\quad
\vlder{\Phi\{a^\five\ot\vls[a^\one.a^\two]\}}{}{\zeta\left\{\box\ContDownContUp\right\}}
                                     {\xi\left\{\vls[a^\one.a^\two]\right\}}
\\
%---------------------------------------
\rcd\colon\hfil\atomicflow{
(   0, 4)*{\afacd\one{}{}\two{}\three};
(-3.5, 0)*{\invisiblemark};
( 3.5, 0)*{\invisiblemark}}
\quad\to\quad
\atomicflow{
(13,12)*{\afaidex{}{}{}{}{}{}32};
(16, 4)*{\afvju8{}\three};
(10, 6)*{\afacunw{}{}{}{}};
( 0, 8)*{\afvjd8\one{}};
( 4, 8)*{\afvjd8{}\two};
( 6, 2)*{\afaiunw{}{}};
( 6, 0)*{\afaiuex{}{}{}{}{}{}31}}
&\qquad
\xi\left\{\vlinf{}{}{a^\three}{\vls[a^\one.a^\two]}\right\}
\quad\to_\rcd\quad
\xi\left\{\box\ContDown\right\}
\\
%---------------------------------------
\end{array}
\]
\caption{`Downwards' reduction rules for contraction and their soundness.}
\label{FigRedC}
\end{figure}%

%---------------------------------------
\begin{theorem}\label{TheoSound}
The reduction rules\/ $\rwdcd$, $\rwdiu$, $\rwdwu $, $\rwdcu$, $\rcdiu$, $\rcdcu$, $\rcuwu$, $\ridwu$, $\rcdwu$, $\ridcu$, $\rcd$ and\/ $\rcu$ are sound.
\end{theorem}

\begin{proof}
For $r\in\{\rwdcd,\rwdiu,\rwdwu,\rwdcu,\rcdiu,\rcdcu,\rcd\}$ and $r\colon\phi'\to\psi'$ as in the left columns of Figures~\ref{FigRedW} and \ref{FigRedC}, for every $\phi$ and $\psi$ such that $\phi\to_r\psi$ and for every $\Phi$ with flow $\phi$, the right columns of the tables provide reductions $\Phi\to_r\Psi$, where $\Psi$ has flow $\psi$, as follows. If $\Phi'\to_r\Psi'$ is the reduction provided by the table, then
\[
\Phi=
\vlderivation              {
\vlde{\Psi_2}{}{\beta  }  {
\vlde{\Phi' }{}{\beta' } {
\vlde{\Psi_1}{}{\alpha'}{
\vlhy          {\alpha }}}}}
\qquad\hbox{and}\qquad
\Psi=
\vlderivation              {
\vlde{\Psi_2}{}{\beta  }  {
\vlde{\Psi'}{}{\beta' } {
\vlde{\Psi_1}{}{\alpha'}{
\vlhy          {\alpha }}}}}
\quad.
\]
We can deal with the remaining rules by employing dual derivations to the ones shown.
\end{proof}

\TODO{Check why the inference rule $\swi'$ taken from AFI used atoms instead of formulae.}

%---------------------------------------
\begin{remark}\label{RemIndep}
The previous soundness theorem only depends on the switch and medial rules for the reductions in Figure~\ref{FigRedC}. Any system obtained from $\SKS$ by replacing $\swi$ and $\med$ with linear rules that can derive them would support a soundness theorem like the one above, for the same reduction rules. For example, we could think of replacing $\swi$ with the rule $\vlinf{\swi'}{}{\vls[(\alpha.\gamma).[\beta.\delta]]}{\vls([\alpha.\beta].[\gamma.\delta])}$, from which $\swi$ is derivable.
\end{remark}


\subsection{Non-termination}

%---------------------------------------
\begin{remark}\label{RemCycle}
Flow rewriting system $\frc$ is not terminating:
\nopagebreak[4]\medskip\afnegspace
\[
\atomicflow{
( 4  ,8)*{\afaidnw{}{}};
( 6  ,6)*{\afvj4};
( 3  ,0)*{\afaiuex{\ppl}{}{}{\pmi}{}{}32};
( 0  ,4)*{\afacd{\ppl}{}{}{}{}{}};
(-3.5,0)*{\invisiblemark};
( 7.5,0)*{\invisiblemark}}
\quad\to_\frc\quad
\atomicflow{
( 9  ,12)*{\afaidex{\pmi}{}{}{\ppl}{}{}32};
( 6  , 8)*{\afacu{}{}{}{}{}{}};
(12  , 6)*{\afvj4};
( 0  , 6)*{\afvjd4\ppl{}};
( 2  , 2)*{\afaiunw{}{}};
(10  , 2)*{\afaiunw{}{}{}{}{}{}};
(-1.5, 0)*{\invisiblemark};
(13.5, 0)*{\invisiblemark}}
\quad\to_\frc\quad
\atomicflow{
( 6  ,8)*{\afaidnw{}{}};
(14  ,8)*{\afaidnw{}{}};
( 4  ,6)*{\afvj4};
(16  ,6)*{\afvj4};
( 2  ,0)*{\afaiu{\ppl}{}{}{}{}{}};
(13  ,0)*{\afaiuex{\ppl}{}{}{\pmi}{}{}32};
(10  ,4)*{\afacd{}{}{}{}{}{}};
(-1.5,0)*{\invisiblemark};
(17.5,0)*{\invisiblemark}}
\quad\to_\frc\quad\cdots\quad.
\]
\afnegspace
We see that if a contraction vertex belongs to an $\ai$-cycle, reductions by $\frc$ make it `bounce' in the $\ai$-cycle and create a trail; while bouncing, the vertex alternates between contraction and cocontraction; if we assign a polarity to the flow, the vertex alternates between being positive and negative.
\end{remark}

\subsection{Termination and Confluence}
%---------------------------------------
\begin{theorem}\label{TheoWTerm}
Flow rewriting system\/ $\frw$ is terminating.
\end{theorem}

%---------------------------------------
\begin{proof}
At every reduction, either the number of vertices decreases, or it stays the same but the number of contraction and cocontraction vertices decreases.
\end{proof}

%---------------------------------------
\begin{theorem}\label{TheoCPTerm}
Flow rewriting system\/ $\frcp$ is terminating.
\end{theorem}

%---------------------------------------
\begin{theorem}\label{TheoCdPCuPTerm}
Flow rewriting system\/ $\{\rcd^\ppl,\rcu^\ppl\}$ is terminating.
\end{theorem}

%---------------------------------------
\begin{theorem}\label{TheoCTerm}
Flow rewriting system\/ $\frc$ is terminating on the set of cycle-free atomic flows.
\end{theorem}

%---------------------------------------
\begin{theorem}\label{TheoWCConf}
Flow rewriting system\/ $\frw\cup\frc$ is confluent.
\end{theorem}

%======================================
\section{Global Transformations}

\TODO{Do not define a global transformation, move the restrictions here to the definition of reduction.}
%---------------------------------------
\begin{definition}
A \emph{global} (\emph{atomic-flow}) \emph{transformation $r$ from flow $\phi$ to flow $\psi$} is a quadruple $(\phi,\psi,f,g,h)$ such that:
\begin{enumerate}
\item $f$ is an injective, partial map from the edges of $\psi$ to the edges of $\phi$, such that for any edges $\epsilon'$ in $\psi$ and $\epsilon$ in $\phi$, if $\epsilon'$ is a \emph{copy} of $\epsilon$ then $f(\epsilon')=\epsilon$,
\item $g$ is a one-to-one map from the upper edges of $\phi$ to the upper edges of $\psi$, such that for any edges $\epsilon$ of $\phi$ and $\epsilon'$ and $\epsilon''$ of $\psi$ such that $\epsilon$ and $\epsilon''$ are upper edges, $f(\epsilon')=\epsilon$ and there is a path from $\epsilon'$ to $\epsilon''$, then $g(\epsilon)=\epsilon''$,
\item $h$ is a one-to-one map from the lower edges of $\phi$ to the lower edges of $\psi$, such that for any edges $\epsilon$ of $\phi$ and $\epsilon'$ and $\epsilon''$ of $\psi$ such that $\epsilon$ and $\epsilon''$ are lower edges, $f(\epsilon')=\epsilon$ and there is a path from $\epsilon''$ to $\epsilon'$, then $h(\epsilon)=\epsilon''$,
\item for every polarity assignment $\pi$ for $\phi$, there is a polarity assignment $\pi'$ for $\psi$ such that $\pi'(g(\epsilon))=\pi(\epsilon)$ and $\pi'(h(\epsilon'))=\pi(\epsilon')$, for any upper edge $\epsilon$ and any lower edge $\epsilon'$ of $\phi$;
\end{enumerate}
we define global transformations with graphical expressions $r\colon\phi\to\psi$, where $g$ and $h$ are indicated by labeling edges.
\end{definition}

%======================================
\subsection{Four Boxes}

\newcommand{\frfb}{{\mathsf{fb}}}
%---------------------------------------
\begin{definition}\label{DefFourBox}
\[
\atomicflow{
(-8, 6)*{\afvjm4};
( 0, 8)*{\afaidm{}{}{}{}{}{}};
( 8, 6)*{\afvjm4};
%-
(-5, 0)*{\affr88};
(-4, 2)*{\aflabelright\phi};
( 5, 0)*{\affr88};
( 6, 2)*{\aflabelright\psi};
%-
(-8,-6)*{\afvjm4};
( 0,-8)*{\afaium{}{}{}{}{}{}};
( 8,-6)*{\afvjm4};
}
\quad\to_\frfb\quad
\atomicflow{
%left
(-27, 8)*{\afawdm{}{}{}{}};
(-19,15)*{\afvjdm6{}{}};
(-19, 8)*{\afacum{}{}{}{}{}{}};
(-24, 0)*{\affr88};
(-20, 2)*{\aflabelleft\phi};
(-27,-8)*{\afawum{}{}{}{}};
(-19,-8)*{\afacdm{}{}{}{}{}{}};
(-19,-15)*{\afvjum6{}{}};
%top
(-15,18)*{\afawdm{}{}{}{}};
( -7,18)*{\afaidnw{}{}};
( -9,16)*{\afvjm4};
( -2,15)*{\afcjlm66};
(-12,10)*{\affr88};
( -8,12)*{\aflabelleft\phi};
(-16, 1)*{\afcjrm2{10}};
( -9, 2)*{\afawum{}{}{}{}};
%bot
(-16,-1)*{\afcjlm2{10}};
( -9,-2)*{\afawdm{}{}{}{}};
(-12,-10)*{\affr88};
( -8, -8)*{\aflabelleft\phi};
( -9,-16)*{\afvjm4};
( -2,-15)*{\afcjrm66};
( -7,-20)*{\afaiunw{}{}};
(-15,-18)*{\afawum{}{}{}{}};
%center
(-3, 11)*{\afvjm{14}};
( 3,  8)*{\afacdm{}{}{}{}{}{}};
( 7, 12)*{\afaidnw{}{}};
( 0,  0)*{\affr88};
( 4,  2)*{\aflabelleft\psi};
( 7,-14)*{\afaiunw{}{}};
( 3, -8)*{\afacum{}{}{}{}{}{}};
(-3,-11)*{\afvjm{14}};
%right
(15, 8)*{\afawdm{}{}{}{}};
( 9, 8)*{\afvjm8};
(12, 0)*{\affr88};
(16, 2)*{\aflabelleft\phi};
( 9,-8)*{\afvjm8};
(15,-8)*{\afawum{}{}{}{}};
}
\]
\end{definition}

%---------------------------------------
\begin{theorem}\label{ThFBSound}
Reduction\/ $\to_\frfb$ is sound.
\end{theorem}

\TODO{Redo proof to be in line with the rest of the thesis.}

\TODO{Use Formalism A notation.}

\TODO{B -> $\phi$ and A -> $\psi$.}

%---------------------------------------
\begin{proof}
For every atom occurence, $a$, which maps to an edge in $B$. We substitute $a$ with the formula $\vls([a_1.a_2].[a_3.a_4])$, where $a=a_1=a_2=a_3=a_4$ and the subscripts denote which of the $B_1,B_2,B_3,B_4$ the atom occurence maps to. Rule instances $\vlinf{\aid}{}{\vls[\bar{a}.a]}{\ttt}$ and $\vlinf{\aiu}{}{\fff}{\vls(\bar{a}.a)}$ are substituted with derivations
\[
\vlderivation
{
\vlin{\awd\times2}{}{\vls[\bar{a}.([a_1.a_2].[a_3.a_4])]}
 {
 \vlin{\acd}{}{\vls[\bar{a}.(a_2.a_4)]}
  {
  \vlin{\swi\times2}{}{\vls[\bar{a}.\bar{a}.(a_2.a_4)]}
   {
   \vlin{\aid\times2}{}{\vls([\bar{a}.a_2].[\bar{a}.a_4])}
    {
    \vlhy{\ttt}
    }
   }
  }
 }
}\qquad
\vlderivation
{
\vlin{\aiu\times2}{}{\fff}
 {
 \vlin{\swi\times2}{}{\vls[(\bar{a}.a_3).(\bar{a}.a_4)]}
  {
  \vlin{\acu}{}{\vls(\bar{a}.\bar{a}.[a_3.a_4])}
   {
   \vlin{\awu\times2}{}{\vls(\bar{a}.[a_3.a_4])}
    {
    \vlhy{\vls(\bar{a}.[a_1.a_2].[a_3.a_4])}
    }
   }
  }
 }
}\quad,
\]
respectively. Rule instances $\acd,\acu,\awd,\awu$ which apply to $a$ are substituted with their generic counterparts applied to $\vls([a_1.a_2].[a_3.a_4])$. Finally any occurence of $\vls([a_1.a_2].[a_3.a_4])$ in the premiss or conclusion are substituted with the derivations
\[
\vlderivation
{
 \vlin{\awd\times2}{}{\vls([a_1.a_2].[a_3.a_4])}
 {
  \vlin{\acu}{}{\vls(a_1.a_3)}
  {
   \vlhy{a}
  }
 }
}\qquad
\vlderivation
{
 \vlin{\acd}{}{a}
 {
  \vlin{\awu\times2}{}{\vls[a_1.a_2]}
  {
   \vlhy{\vls([a_1.a_2].[a_3.a_4])}
  }
 }
}\quad,
\]
respectively. It is now straightforward to verify that the derivation so obtained has the aforementioned atomic flow and the result follows.
\end{proof}


%======================================
\subsection{Simple Flow Removal}

\TODO{Change the following definition to be inline with notation and conventions in the rest of the thesis.}

\TODO{Change the definition to show a simple sub flow instead of a simple edge.}

\TODO{Define simple subflow and make sure we use that instead of simple edge elsewhere.}

\newcommand{\frsf}{{\mathsf{sf}}}
%---------------------------------------
\begin{definition}\label{DefSimSubRem}
We define the reduction $\to_\frsf$ (where $\frsf$ stands for \emph{simple flow}) as follows, for every atomic flow $A$:
\[
\atomicflow{
(10,16)*{\afaidnw{}{}};
( 0,14)*{\afvju4{\epsilon_1}{}};
( 2,14)*{\cdots};
( 4,14)*{\afvju4{}{\epsilon_h}};
( 8,14)*{\afvju4{}\two};
( 9,10)*{\aflabelleft A};
( 4, 9)*{\affr{10}6};
(12,11)*{\afvj{10}};
( 0, 4)*{\afvjd4{\epsilon'_1}{}};
( 2, 4)*{\cdots};
( 4, 4)*{\afvjd4{}{\epsilon'_k}};
(10, 2)*{\afaiu{}\three{}\one{}{}};
(-3, 0)*{\invisiblemark};
(14, 0)*{\invisiblemark}}
\quad\to_\frsf\quad
\atomicflow{
( 0,24   )*{\afvju{16}{\hat\epsilon_1}{}};
( 2,18   )*{\cdots};
( 4,24   )*{\afvju{16}{}{\hat\epsilon_h}};
( 8,18.25)*{\aflabelright{\hat\two}};
( 9,14   )*{\aflabelleft {\hat A}};
( 4,13   )*{\affr{10}6};
( 0, 8.4 )*{\afvjd2{\hat\epsilon'_1}{}};
( 2, 8   )*{\cdots};
( 4, 8.4 )*{\afvjd2{}{\hat\epsilon'_k}};
(12,32   )*{\afvju2{\tilde\epsilon_1}{}};
(14,32   )*{\cdots};
(16,32   )*{\afvju2{}{\tilde\epsilon_h}};
(17,28   )*{\aflabelleft {\tilde A}};
(12,27   )*{\affr{10}6};
(12,16   )*{\afvjd{16}{\tilde\epsilon'_1}{}};
(14,22   )*{\cdots}; 
(16,16   )*{\afvjd{16}{}{\tilde\epsilon'_k}};
( 8,22.25)*{\aflabelleft{\tilde\three}};
( 8,34   )*{\afawd{}{}{\tilde\two}{}};
( 8,42   )*{\cdots};
( 6,38   )*{\afacuexsq{}{}{}{}{\epsilon_1}{}31};
(10,38   )*{\afacuexsq{}{}{}{}{}{\epsilon_h}31};
( 8,20   )*{\afvj8};
( 8, 6   )*{\afawu{}{}{}{\hat\three}};
( 6, 2   )*{\afacdexsq{}{}{}{}{\epsilon'_1}{}31};
(10, 2   )*{\afacdexsq{}{}{}{}{}{\epsilon'_k}31};
( 8,-2   )*{\cdots};
(-3, 0   )*{\invisiblemark};
(19, 0   )*{\invisiblemark}}
\quad,
\]
where $h,k\ge0$, edges have been renamed with $\hat{\enspace}$ and $\tilde{\enspace}$ accents, flows $\tilde A$ and $\hat A$ are both isomorphic to $A$, and edges $\hat\two$ and $\tilde\three$ are identified.
\end{definition}

%---------------------------------------
\begin{theorem}\label{ThSFSound}
Reduction\/ $\to_\frsf$ is sound.
\end{theorem}

\TODO{Redo proof to be in line with the rest of the thesis.}

\TODO{Use Formalism A notation.}

\TODO{Define super switch, super switch down and super switch up.}

%---------------------------------------
\begin{proof}
Let $\Phi$ be a derivation with flow $B$, such that $B\to_\frsf C$. We show that there exists a derivation $\Psi$ with flow $C$ and with the same premiss and conclusion as $\Phi$. In the following, we refer to the figure in Definition~\ref{DefRedS}. We assume that $\Phi$ has premiss $\xi\{\ttt\}$ and conclusion $\zeta\{\fff\}$, where the evidenced and labelled $\ttt$ and $\fff$ can be traced to the interaction and cointeraction vertices eliminated by $\to_\frsf$, respectively (this can always be done by using switches and unit equations). Intuitively, we can think of $\ttt$ and $\fff$ as mapping to special `unit edges', which can be substituted just like normal edges. So, we assume that $\Phi$ is
\[
\vlderivation                                           {
\vlde{\Phi_3}{}{\zeta\{ \fff\}                 }    {
\vlin{\aiu  }{}{\zeta'\{ \fff\}                }   {
\vlde{\Phi_2}{}{\zeta'\vlscn(\bar a^\three.a^\one)}  {
\vlin{\aid  }{}{\xi'\vlscn[\bar a^\two.a^\one]    } {
\vlde{\Phi_1}{}{\xi'\{ \ttt\}                  }{
\vlhy          {\xi\{ \ttt\}                   }}}}}}}
\quad.
\]
We obtain the two derivations $\Psi'$ and $\Psi''$ from $\Phi$ as follows:
\[
\Psi'=\;\;
\vlderivation                                                  {
\vlde{\Phi_3\{\fff\ot\bar a\}}
             {}{\zeta\{\bar a^{\tilde\three}\}          }     {
\vlin{=     }{}{\zeta'\{\bar a^{\tilde\three}\}         }    {
\vlde{\Phi_2\{a^\one\ot\ttt\}}
             {}{\zeta'\vlscn(\bar a^{\tilde\three}.\ttt)}   {
\vlin{\awd  }{}{\xi'\vlscn[\bar a^{\tilde\two}.\ttt]    }  {
\vlin{=     }{}{\xi'\vlscn[\fff.\ttt]                   } {
\vlde{\Phi_1}{}{\xi'\{\ttt\}                            }{
\vlhy          {\xi\{\ttt\}                             }}}}}}}}
\qquad\hbox{and}\qquad
\Psi''=\;\;
\vlderivation                                                                  {
\vlde{\Phi_3                  }{}{\zeta\{\fff\}                         }     {
\vlin{=                       }{}{\zeta'\{\fff\}                        }    {
\vlin{\awu                    }{}{\zeta'\vlscn(\ttt.\fff)               }   {
\vlde{\Phi_2\{a^\one\ot \fff\}}{}{\zeta'\vlscn(\bar a^{\hat\three}.\fff)}  {
\vlin{=                       }{}{\xi'\vlscn[\bar a^{\hat\two}.\fff]    } {
\vlde{\Phi_1\{\ttt\ot\bar a\}}
                               {}{\xi'\{\bar a^{\hat\two}\}             }{
\vlhy                            {\xi\{\bar a^{\hat\two}\}              }}}}}}}}
\quad.
\]
Derivation $\Psi'$ has flow $B'$ and $\Psi''$ has $B''$:
\nopagebreak[4]\bigskip\afnegspace
\[
B'=\raise2\atflowelheight\hbox{$
\atomicflow{
( 4,14)*{\afvju4{\tilde\epsilon_1}{}};
( 6,14)*{\cdots};
( 8,14)*{\afvju4{}{\tilde\epsilon_h}};
( 0, 4)*{\afvjd4{\tilde\three}{}};
( 9,10)*{\aflabelleft {\tilde A}};
( 4, 9)*{\affr{10}6};
( 4, 4)*{\afvjd4{\tilde\epsilon'_1}{}};
( 6, 4)*{\cdots};
( 8, 4)*{\afvjd4{}{\tilde\epsilon'_k}};
( 0,16)*{\afawd{}{}{\tilde\two}{}};
(-2, 4)*{\invisiblemark};
(11, 4)*{\invisiblemark}}$}
\qquad\hbox{and}\qquad
B''=\lower2\atflowelheight\hbox{$
\atomicflow{
( 0,14)*{\afvju4{\hat\epsilon_1}{}};
( 2,14)*{\cdots};
( 4,14)*{\afvju4{}{\hat\epsilon_h}};
( 8,14)*{\afvju4{}{\hat\two}};
( 9,10)*{\aflabelleft {\hat A}};
( 4, 9)*{\affr{10}6}; 
( 0, 4)*{\afvjd4{\hat\epsilon'_1}{}};
( 2, 4)*{\cdots};
( 4, 4)*{\afvjd4{}{\hat\epsilon'_k}};
( 8, 2)*{\afawu{}{}{}{\hat\three}};
(-3, 4)*{\invisiblemark}}$}
\quad.
\]
\afnegspace
We combine $\Psi'$ and $\Psi''$ to get the desired derivation $\Psi$ with flow $C$ and the same premiss and conclusion as $\Phi$:
\[
\Psi=\;\;
\vlderivation                                                              {
\vlin{\cod                      }{}{\zeta\{\fff\}                    }    {
\vlde{\vls[\Psi''.\zeta\{\fff\}]}{}{\vls[\zeta\{\fff\}.\zeta\{\fff\}]}   {
\vlin{\swi                      }{}{\vls[\xi\{\bar a\}.\zeta\{\fff\}]}  {
\vlde{\vls(\xi\{\ttt\}.\Psi')   }{}{\vls(\xi\{\ttt\}.\zeta\{\bar a\})} {
\vlin{\cou                      }{}{\vls(\xi\{\ttt\}.\xi\{\ttt\})    }{
\vlhy                              {\xi\{\ttt\}                      }}}}}}}
\quad,
\]
where $\swi$, $\cod$ and $\cou$ are `macro' rules introduced in Remarks~\ref{RemSupSwitch} and \ref{RemGenContr}.
\end{proof}

%======================================
\subsection{Path Breaker}

\TODO{Import from AFII when the paper is final.}

\TODO{Consider changing into the style of simple flow remover above: first define a reduction on atomic flows then show it is sound. It fits better with the message of the thesis.}

\newcommand{\Break}{\mathsf{Break}}
\begin{definition}\label{DefPathBreak}
The \emph{path breaker}, $\Break$, is an operator whose arguments are a derivation $\vlder{\Phi}{}{\beta}{\alpha}$ and the atom occurrences $a^\lambda$ and $a^\mu$, such that $\alpha=\vlsmallbrackets\vls([a^\lambda.\bar a].\gamma)$, $\beta=\vls[\delta.(a^\mu.\bar a)]$, and the results are a derivation and two atom occurrences $a^\epsilon$ and $a^\iota$ such that $\Break(\Phi,a^\lambda,a^\mu)=(\Psi,a^\epsilon,a^\iota)$, where:
\newbox\DeltaTopK
\setbox\DeltaTopK=
\hbox{$
\vlderivation
{
 \vlin{=}{}{\vls[\delta.(\vlinf{}{}{\ttt}{a^{\mu'}}.\bar a)]}
 {
  \vlde{\Phi}{}{\beta}
  {
   \vlhy{\alpha}
  }
 }
}$
}
\newbox\DeltaK
\setbox\DeltaK=
\hbox{$
\vlderivation
{
 \vlin{=}{}{\vls[\delta.(a^{\epsilon'}.\vlinf{}{}{\ttt}{\bar a})]}
 {
  \vlde{\Phi}{}{\beta}
  {
   \vlin{=}{}{\alpha}
   {
    \vlhy{\vls([\vlinf{}{}{a^{\lambda'}}{\fff}.\bar a].\gamma)}
   }
  }
 }
}$
}
\newbox\DeltaBotK
\setbox\DeltaBotK=
\hbox{$
\vlderivation
{
 \vlde{\Phi}{}{\beta}
 {
  \vlin{=}{}{\alpha}
  {
   \vlhy{\vls([a^{\epsilon'}.\vlinf{}{}{\bar a}{\fff}].\gamma)}
  }
 }
}$
}
\[
\Psi\quad=\quad
\vlderivation
{
 \vlin{=}{}{\vls[\vlinf{2\cdot\cod}{}{\delta}{\vls[\delta.\delta.\delta]}\;.\;(a^\iota.\bar a)]}
 {
  \vlin{=}{}{\vlsbr[\delta\;\;\;.\;\;\;\delta\;\;\;.\;\;\;\box\DeltaBotK]}
  {
   \vlin{\swi}{}{\vlsmallbrackets\vls[(\gamma.a^{\epsilon'}).[\delta.\delta]]}
   {
    \vlin{=}{}{\vlsmallbrackets\vls(\gamma.[a^{\epsilon'}.[\delta.\delta]])}
    {
     \vlin{=}{}{\vlsbr([\delta\;\;\;\;\;.\;\;\;\;\;\box\DeltaK]\;\;\;\;\;.\;\;\;\;\;\gamma)}
     {
      \vlin{=}{}{\vls(\vlinf{\swi}{}{\vls[(\gamma.\bar a).\delta]}{\vls(\gamma.[\bar a.\delta])}\;.\;\gamma)}
      {
       \vlin{=}{}{\vlsbr(\box\DeltaTopK\;\;\;.\;\;\;\gamma\;\;\;.\;\;\;\gamma)}
       {
        \vlhy{\vls([a^\epsilon.\bar a]\;.\;\vlinf{2\cdot\cou}{}{\vls(\gamma.\gamma.\gamma)}{\gamma})}
       }
      }
     }
    }
   }
  }
 } 
}\quad,
\]
where $\epsilon$, $\lambda'$ and $\epsilon'$ correspond to $\lambda$ and $\mu'$, $\epsilon'$ and $\iota$ correspond to $\mu$ in the atomic flow of $\Phi$.
\end{definition}

\begin{proposition}\label{PropPathBreak}
If, for some flows $\phi$ and $\psi$, the atomic flow of $\vlder{\Phi}{}{\vlsmallbrackets\vls[\beta.(a^{\mu}.\bar a^{\mu'})]}{\vlsmallbrackets\vls([a^{\lambda}.\bar a^{\lambda'}].\alpha)}$ is of shape
\[
\atomicflow
{
(-8, 6)*{\afvjm{4}};
(-2, 6)*{\afvju{4}{\lambda}{}};
( 2, 6)*{\afvju{4}{}{\lambda'}};
( 8, 6)*{\afvjm{4}};
(-5, 0)*{\affr{8}{8}};
(-4, 2)*{\aflabelright\phi};
%---
( 5, 0)*{\affr{8}{8}};
( 6, 2)*{\aflabelright{\psi}};
( 8,-6)*{\afvjm{4}};
(-2,-6)*{\afvjd{4}{\mu}{}};
( 2,-6)*{\afvjd{4}{}{\mu'}};
(-8,-6)*{\afvjm{4}};
}\quad,
\]
and if\/ $\Break(\Phi,a^\lambda,a^\mu)=\left(\vlder{\Psi}{}{\beta}{\alpha},a^\epsilon,a^\iota\right)$, where $\alpha=\vlsmallbrackets\vls([a^\epsilon.\bar a^{\epsilon'}].\gamma)$ and $\beta=\vlsmallbrackets\vls[\delta.(a^\iota.\bar a^{\iota'})]$, then
\begin{itemize}
	\item there are no paths from $\epsilon$ to $\iota$ and no paths from $\epsilon'$ to $\iota'$ in the atomic flow of\/ $\Psi$; and
	\item there are no paths from an upper path $\epsilon''$ to a lower path $\iota''$ in the atomic flow of $\Psi$ unless there is a path from the corresponding upper path $\lambda''$ to the lower path $\mu''$ in the atomic flow of $\Phi$.
\end{itemize}
\end{proposition}

\begin{figure}
\[
\atomicflow
{
(-8, 6)*{\afvjm{4}};
(-2, 6)*{\afvju{4}{\lambda}{}};
( 2, 6)*{\afvju{4}{}{\lambda'}};
( 8, 6)*{\afvjm{4}};
(-5, 0)*{\affr{8}{8}};
(-4, 2)*{\aflabelright\phi};
%---
( 5, 0)*{\affr{8}{8}};
( 6, 2)*{\aflabelright{\psi}};
( 8,-6)*{\afvjm{4}};
(-2,-6)*{\afvjd{4}{\mu}{}};
( 2,-6)*{\afvjd{4}{}{\mu'}};
(-8,-6)*{\afvjm{4}};
}\qquad\qquad
\atomicflow
{
%%%%% RED %%%%%
(0,-20)="D";
(0,-10)="Dhalf";
%% contractions
"D"+"D"="A";
%left
"A"+(-14,-15.5)-"D"*{\afvjmcol{23}{Red}};
"A"+(-11,-16.5)*{\afvjmcol{3}{Red}};
%right
"A"+(11,-11.5)-"Dhalf"*{\afvjmcol{11}{Red}};
"A"+(14,-15.5)-"D"*{\afvjmcol{23}{Red}};
"A"+(11,-16.5)*{\afvjmcol{3}{Red}};
% top boxes
(0,0)="A";
"A"+(-11,-14)*{\afcjrmcol{6}{20}{Red}};
"A"+(11,-14)*{\afcjlmcol{6}{20}{Red}};
"A"+(-2, 11)*{\afvjdcol{14}{\epsilon}{}{Red}};
"A"+( 2, 11)*{\afvjdcol{14}{}{\epsilon'}{Red}};
"A"+(-2, -8)*{\afawucol{}{}{\mu}{}{}{Red}};
% join one
"A"+(2,-7)*{\afvjdcol6{}{\mu'}{Red}};
"A"+(2,-13)*{\afvjucol6{}{\lambda'}{Red}};
% middle boxes
"A"+"D"="A";
"A"+(9.5,-10)*{\afcjlmcol{3}{12}{Red}};
"A"+( 2,-8)*{\afawucol{}{}{}{\mu'}{}{Red}};
%%%%% GREEN %%%%%
%% cocontractions
(0,0)="A";
%left
"A"+(-11,16.5)*{\afvjmcol{3}{OliveGreen}};
"A"+"D"+(-14,15.5)*{\afvjmcol{23}{OliveGreen}};
"A"+"Dhalf"+(-11,11.5)*{\afvjmcol{11}{OliveGreen}};
%right
"A"+(11,16.5)*{\afvjmcol{3}{OliveGreen}};
"A"+"D"+(14,15.5)*{\afvjmcol{23}{OliveGreen}};
% middle boxes
"A"+"D"="A";
"A"+(-9.5,10)*{\afcjlmcol{3}{12}{OliveGreen}};
"A"+(-2, 8)*{\afawdcol{}{}{\lambda}{}{}{OliveGreen}};
% join two
"A"+(-2, -7)*{\afvjdcol6{\mu}{}{OliveGreen}};
"A"+(-2,-13)*{\afvjucol6{\lambda}{}{OliveGreen}};
% bottom boxes
"A"+"D"="A";
"A"+(-11,14)*{\afcjlmcol{6}{20}{OliveGreen}};
"A"+(11,14)*{\afcjrmcol{6}{20}{OliveGreen}};
"A"+(-2,-11)*{\afvjucol{14}{\iota}{}{OliveGreen}};
"A"+( 2,-11)*{\afvjucol{14}{}{\iota'}{OliveGreen}};
"A"+( 2, 8)*{\afawdcol{}{}{}{\lambda'}{}{OliveGreen}};
%%%%% BLACK %%%%%
%% cocontractions
(0,0)="A";
%left
"A"+(-8,5.5)*{\afvjm3};
"A"+(-11,11)*{\affr88};
"A"+(-11,11)*{\copy\contrup};
%right
"A"+(8,5.5)*{\afvjm3};
"A"+"Dhalf"+(11,11.5)*{\afvjm{11}};
"A"+(11,11)*{\affr88};
"A"+(11,11)*{\copy\contrup};
%% contractions
"D"+"D"="A";
%left
"A"+(-11,-11.5)-"Dhalf"*{\afvjm{11}};
"A"+(-8,-5.5)*{\afvjm3};
"A"+(-11,-11)*{\affr88};
"A"+(-11,-11)*{\copy\contrdown};
%right
"A"+(8,-5.5)*{\afvjm3};
"A"+(11,-11)*{\affr88};
"A"+(11,-11)*{\copy\contrdown};
% top boxes
(0,0)="A";
"A"+(-5,  0)*{\affr{8}{8}};
"A"+(-4,  2)*{\aflabelright{\phi}};
"A"+( 5,  0)*{\affr{8}{8}};
"A"+( 6,  2)*{\aflabelright{\psi}};
% middle boxes
"A"+"D"="A";
"A"+(9.5,10)*{\afcjrm{3}{12}};
"A"+(-9.5,-10)*{\afcjrm{3}{12}};
"A"+(-5, 0)*{\affr{8}{8}};
"A"+(-4, 2)*{\aflabelright{\phi}};
"A"+( 5, 0)*{\affr{8}{8}};
"A"+( 6, 2)*{\aflabelright{\psi}};
% bottom boxes
"A"+"D"="A";
"A"+(-5, 0)*{\affr{8}{8}};
"A"+(-4, 2)*{\aflabelright{\phi}};
"A"+( 5, 0)*{\affr{8}{8}};
"A"+( 6, 2)*{\aflabelright{\psi}};
}
\]
\caption{The atomic flows of $\Phi$ (left) and $\Psi$ (right) such that $\Break(\Phi,a^\lambda,a^\mu)=(\Psi,a^\epsilon,a^\iota)$.}
\label{FigFlowBreak}
\end{figure}

\begin{proof}
Refer to the atomics in Figure~\ref{FigFlowBreak}.
\begin{itemize}
	\item All the edges that might be in paths from $\epsilon$ or $\epsilon'$ are coloured in red and all the edges that might be in paths to $\iota$ or $\iota'$ are coloured in green. Since the red and the green paths do not overlap, there are no paths from $\epsilon$ to $\iota$ and no paths from $\epsilon'$ to $\iota'$; and
	\item If there is a path from an upper edge $\epsilon''$ to a lower edge $\mu''$ in the atomic flow of $\Psi$ the path must pass through one of the three copies of either $\phi$ or $\psi$, hence there must be a path from the corresponding upper edge $\lambda''$ to the corresponding lower edge $\mu''$ in the atomic flow of $\Phi$.
\end{itemize}
\end{proof}

\subsection{Unnamed Transformation}

\TODO{Either rephrase this in terms of an atomic flow transformation and a soundness proof or move it away from here.}

\TODO{Define for derivations instead of proofs. Use simple form of derivations with respect to a polarity assignment. Use the decomposed normal form, yet to be given a name, for derivations.}

\TODO{Come up with a better name than cut-free form (as it is not an accurate description if we talk about derivations.).}

\newcommand{\frqp}{{\mathsf{qp}}}

\TODO{Use the symbol $\frqp$ to denote the following reduction.}

%-------------------------------------------------------------------------------
\begin{definition}\label{DefNorm}
For $n>0$, let $\Pi$ be a proof in simple form over $\avec1n$, such that it and its atomic flow have shape
\[
\hbox{\phantom{$\vls[\alpha.{}]$}}
\vlder\Psi
      {}
      {\vls[\llap{$\vls[\alpha.{}]$}
            \vlinf{}{}\fff{\vls(a_1.\bar a_1^{\phi_1})}.\cdots.
            \vlinf{}{}\fff{\vls(a_n.\bar a_n^{\phi_n})}]}
      {\vls(\vlinf{}{}{\vls[a_1.\bar a_1^{\phi_1}]}\ttt.\cdots.
            \vlinf{}{}{\vls[a_n.\bar a_n^{\phi_n}]}\ttt)}
\qquad\text{and}\qquad
\atomicflow{
(17.75 ,16  )*{\afaidex{}{}{}{}{}{}{43}8};
(14.25 ,14  )*{\afaidex{}{}{}{}{}{}{13}8};
( 9    ,12  )*{\cdots};
( 7    ,11  )*{\afvj2};
(28.5  ,11  )*{\afvj2};
(18.5  , 9  )*{\aflabelright{\phi_1}};
(29.5  , 9  )*{\aflabelright{\phi_n}};
( 1    , 6  )*{\copy\interdown};
( 4    , 6  )*{\copy\weakdown};
( 5.5  , 6  )*{\affr{13}8};
( 6    , 6  )*{\copy\weakup};
( 8    , 6  )*{\copy\contrdown};
(10    , 6  )*{\copy\contrup};
(16    , 6  )*{\copy\contrdown};
(17.5  , 6  )*{\affr78};
(18    , 6  )*{\copy\contrdown};
(23    , 6  )*{\cdots};
(27    , 6  )*{\copy\contrup};
(28.5  , 6  )*{\affr78};
(29    , 6  )*{\copy\contrup};
( 7    , 1  )*{\afvj2};
(28.5  , 1  )*{\afvj2};
( 9    , 0  )*{\cdots};
( 1    ,-1.5)*{\afvjm7};
(14.25 ,-2  )*{\afaiuex{}{}{}{}{}{}{13}8};
(17.75 ,-4  )*{\afaiuex{}{}{}{}{}{}{43}8}}
\quad,
\]

for some derivation $\Psi$. For $0\le i\le n+1$, let $\theta_i\equiv\th in\avec1n$. For $0\le k\le n$, we define the derivations $\vlder{\Phi_k}{\SKS\setminus\{\aiu\}}{\vls[\alpha.\theta_{k+1}]}{\theta_k}$ as
\[ %%%%% The \dimen's must be adjusted if fonts and layout parameters are changed
\dimen0=2560000sp
\kern\dimen0
\vlder{\Psi_k}
      {\SKS\setminus\{\aiu\}}
      {\kern-\dimen0\vlsbr[\alpha.
       \vlinf{\scriptstyle(n-1)\cdot\cod}
             {}
             {\theta_{k+1}}
             {\vls[\vlder{}
                         {\{\awu,\acu,\swi\}}
                         {\theta_{k+1}}
                         {\vlsbr(a_1
                                .\;\;
                                \vlder{\Gth k1n\avec1n}
                                      {\{\awd,\awu\}}
                                      {\theta_{k+1}\{a_1/\ttt\}}
                                      {\theta_k    \{a_1/\fff\}})}
                  \;\;\;.\vldots.\;\;\;
                   \vlder{}
                         {\{\awu,\acu,\swi\}}
                         {\theta_{k+1}}
                         {\vlsbr(a_n
                                .\;\;
                                \vlder{\Gth knn\avec1n}
                                      {\{\awd,\awu\}}
                                      {\theta_{k+1}\{a_n/\ttt\}}
                                      {\theta_k    \{a_n/\fff\}})}
                 ]}]                                              }  
      {\vlinf{\llap{$\scriptstyle(n-1)\cdot\cou$}}
             {}
             {\vlsbr(\vlder{}
                           {\{\awd,\acd,\swi\}}
                           {\vls[a_1.\theta_k\{a_1/\fff\}]}
                           {\theta_k                      }
                    \;\;\;.\vldots.\;\;\;
                     \vlder{}
                           {\{\awd,\acd,\swi\}}
                           {\vls[a_n.\theta_k\{a_n/\fff\}]}
                           {\theta_k                      })}
             {\theta_k                                    }}
\quad,
\]
where $\Psi_k=\Psi\{\bar a_1^{\phi_1}/\theta_k\{a_1/\fff\},\dots,\bar a_n^{\phi_n}/\theta_k\{a_n/\fff\}\}$ and where we use Proposition~\ref{PropAuxNorm}. We define the \emph{cut-free form of\/ $\Pi$} as the following proof in $\SKS\setminus\{\aiu\}$:
\[ %%%%% The \dimen's must be adjusted if fonts and layout parameters are changed
\dimen0=910000sp
\dimen1=2970000sp
\dimen2=3870000sp
\vlinf{n\cdot\gcd}
      {\quad.}
      \alpha
      {\kern\dimen2\vlder{\Phi_0}
             {}
             {\kern-\dimen2
              \vlsbr[\alpha
                    .
                    \kern\dimen1
                    \vlder{\Phi_1}
                          {}
                          {\kern-\dimen1
                           \vlsbr[\alpha
                                 .
                                 \cdots.\kern\dimen0
                                 \begin{tabular}{@{}c@{}}
                                 $\theta_2$\\
                                 $\vdots$\\
                                 $\kern-\dimen0
                                  \vlsbr[\alpha
                                        .
                                        \vlder{\Phi_n}
                                              {}
                                              {\vlsbr[\alpha
                                                     .\theta_{n+1}]}
                                              {\theta_n}]$
                                 \end{tabular}]}
                          {\theta_1}]}
            {\theta_0}}
\]
(We recall that $\theta_0\equiv\ttt$ and $\theta_{n+1}\equiv\fff$.)
\end{definition}
