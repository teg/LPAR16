\chapter{Transformations}

\TODO{Polarity assignment: `to' -> `for'}

%-------------------------------------------------
\begin{definition}\label{definition:FlowReduction}
An (\emph{atomic-flow}) \emph{reduction $R$}\index{reduction} is a binary relation on the set of atomic flows, such that $\phi\mathrel{R}\psi$ if
\begin{enumerate}
\item
there is a one-to-one map, $f$, from the upper edges of $\phi$ to the upper edges of $\psi$;
\item
there is a one-to-one map, $g$, from the lower edges of $\phi$ to the lower edges of $\psi$; and
\item\label{definition:FlowReduction:item:Polarity}
for every polarity assignment $\pi$ for $\phi$, there is a polarity assignment $\pi'$ for $\psi$ such that $\pi'(f(\epsilon))=\pi(\epsilon)$ and $\pi'(g(\iota))=\pi(\iota)$, for any upper edge $\epsilon$ and any lower edge $\iota$ of $\phi'$.
\end{enumerate}
We call $\phi$ a \emph{redex} and $\psi$ a \emph{contractum} of $\mathrel R$.
\end{definition}
%---------------

%-----------------------------------------------
\begin{remark}\label{remark:LabelBijectionEdges}
Given a reduction $R$ and two atomic flows $\phi$ and $\psi$, such that $\phi\mathrel{R}\psi$, we indicate the bijections $f$ and $g$ by labeling the upper (resp., lower) edge $f(\epsilon)$ (resp., $g(\epsilon)$) of $\psi$ by $\epsilon$, for every upper (resp., lower) edge $\epsilon$ of $\phi$.
\end{remark}
%-----------

\TODO{Add example.}

%--------------------------------------------------
\begin{definition}\label{definition:SoundRedcution}
A reduction $R$ is \emph{sound}\index{reduction!sound} if, for every $\phi$ and $\psi$, such that $\phi\mathrel{R}\psi$, and for every derivation $\Phi$ with atomic flow $\phi$, there is a derivation $\Psi$ with atomic flow $\psi$ such that $\Phi$ and $\Psi$ have the same premiss and conclusion; in this case we write $\Phi\mathrel{R}\Psi$.
\end{definition}
%---------------

\TODO{Explain that we force the contractum to be unique (by convention).}

%---------------------------------------------
\begin{remark}\label{remark:ReductionPolarity}
It is straightforward to verify that the $\pi'$ of Definition~\ref{definiton:ReductionPolarity} is unique for every reduction we define in this thesis. We will therefore refer to it as $\pi$ throughout, without further comment. 
\end{remark}
%-----------

\TODO{Alessio: I want to allow identity and cuts as long as they do not map to vertices in $\phi$. Since $\phi$ only contains $a$ and not $\bar a$, it can not contain any identity or cut vertices. I added a sentence to the proof to make this clear.}

\TODO{Make sure that it is always clear that $f_i$ is a flow isomorphism.}

%------------------------------------------------------------
\begin{proposition}\label{proposition:DerivationSubstitution}
Given a derivation\/ $\vlder\Phi\SKS{\beta}\alpha$, let its associated flow have shape
\[
\atomicflow{
(-5, 6)*{\afvjdm4{}{}};
( 5, 6)*{\afvjdm4{}{}};
(-1, 2)*{\aflabelleft\phi};
( 9, 2)*{\aflabelleft\psi};
(-5, 0)*{\affr88};
( 5, 0)*{\affr88};
(-5,-6)*{\afvjum4{}{}};
( 5,-6)*{\afvjum4{}{}}
}
\quad,
\]
such that $\phi$ is a connected component whose edges are each associated with atom $a$; then, for any formula $\gamma$, there exists a derivation
\[
\vlder\Psi\SKS{\beta \{a^\phi/\gamma\}}
              {\alpha\{a^\phi/\gamma\}}
\]
whose associated flow is
\[
\atomicflow{
(-17, 6)*{\afvjdm4{}{}};
( -5, 6)*{\afvjdm4{}{}};
(  5, 6)*{\afvjdm4{}{}};
(-13, 2)*{\aflabelleft{f_1(\phi)}};
( -1, 2)*{\aflabelleft{f_n(\phi)}};
(  9, 2)*{\aflabelleft\psi};
(-17, 0)*{\affr88};
(-11, 0)*{\cdots};
( -5, 0)*{\affr88};
(  5, 0)*{\affr88};
(-17,-6)*{\afvjum4{}{}};
( -5,-6)*{\afvjum4{}{}};
(  5,-6)*{\afvjum4{}{}};
}
\quad,
\]
where $n$ is the number of atom occurrences in $\gamma$; moreover, the size of\/ $\Psi$ depends linearly on the size of\/ $\Phi$ and quadratically on the size of $\gamma$.
\end{proposition}
%----------------

%-------------------------------------------------------------------------------
\begin{proof}
We can proceed by structural induction on $\Phi$. For every formula in $\Phi$ we substitute $a^\phi$ with $\gamma$. Since all the edges in $\phi$ are mapped to from $a$ (and not $\bar a$), we know that all the vertices in $\phi$ are mapped to from instances of $\acd$, $\acu$, $\awd$ and $\awu$. We substitute every instance of $\acd$, $\acu$, $\awd$ and $\awu$ where $a^\phi$ appears, by $\cod$, $\cou$, $\wed$, $\weu$, respectively, with $\gamma$ in the place of $a^\phi$. The result then follows by Lemma~\vref{lemma:GenericWeakening} and Lemma~\vref{lemma:GenericContraction}.
\end{proof}
%----------

\TODO{Put reference.}

%---------------
\begin{notation}
The derivation $\Psi$ obtained in the proof of Propostion~\vref{proposition:DerivationSubstitution} is denoted $\Phi\{a^\phi/\gamma\}$.
\end{notation}
%-------------

%-------------
\begin{remark}
The notion of substitution can be extended to allow $\phi$ to contain interaction and cut vertices, but we shall not need that in this thesis.
\end{remark}
%-----------

%====================================================================
\section{Global Transformations}\label{section:GlobalTransformations}

\TODO{Alessio: $\boldsymbol\epsilon$ is used to denote the upper edges of a flow. The definition of upper edges is given as a set (from AF1). Why should they be sequences? Wouldn't that give the impression that they are not commutative?}

\TODO{The labels on upper/lower edges do not fit with the labels used in the definition of four boxes.}

%-------------
\begin{convention}
To avoid ambiguity in Definitions~\vref{definition:FourBoxes}, \vref{definition:IsolatedSubflowRemoval}, \vref{definition:PathBreaker} and \vref{definition:MultipleIsolatedSubflowsRemoval} we have establish the following convention:
Let $\boldsymbol\epsilon=\{\epsilon_1,\dots,\epsilon_n\}$, $\boldsymbol\iota=\{\iota_1,\dots,\iota_m\}$, $\boldsymbol{\epsilon'}=\{\epsilon'_1,\dots,\epsilon'_n\}$ and $\boldsymbol{\iota'}=\{\iota'_1,\dots,\iota'_m\}$, then, when we write
\[
\atomicflow
{
(-10, 0)*{\invisiblemark};
(  0, 6)*{\afvjdm4{\boldsymbol\epsilon}{}};
(  0, 0)*{\affr{14}8};
(  0, 0)*{\copy\contrup};
( -6,-6)*{\afvjdm4{f_1(\boldsymbol\epsilon)}{}};
(  0,-6)*{\cdots};
(  6,-6)*{\afvjdm4{}{f_k(\boldsymbol\epsilon)}};
( 10, 0)*{\invisiblemark};
}
\qquad\mbox{and}\qquad
\atomicflow
{
(-10, 0)*{\invisiblemark};
(  0,-6)*{\afvjum4{\boldsymbol\iota}{}};
(  0, 0)*{\affr{14}8};
(  0, 0)*{\copy\contrdown};
( -6, 6)*{\afvjum4{f_1(\boldsymbol\iota)}{}};
(  0, 6)*{\cdots};
(  6, 6)*{\afvjum4{}{f_k(\boldsymbol\iota)}};
( 10, 0)*{\invisiblemark};
}
\]
we mean
\[
\atomicflow
{
(-10, 0)*{\invisiblemark};
(  0,6)*{\afvjd4{\epsilon_1}{}};
(  0,0)*{\affr{14}8};
(  0,0)*{\copy\contrup};
( -6,-6)*{\afvjd4{f_1(\epsilon_1)}{}};
(  0,-6)*{\cdots};
(  6,-6)*{\afvjd4{}{f_k(\epsilon_1)}};
( 10, 0)*{\invisiblemark};
}
\quad\cdots\quad
\atomicflow
{
(-10, 0)*{\invisiblemark};
(  0,6)*{\afvjd4{\epsilon_n}{}};
(  0,0)*{\affr{14}8};
(  0,0)*{\copy\contrup};
( -6,-6)*{\afvjd4{f_1(\epsilon_n)}{}};
(  0,-6)*{\cdots};
(  6,-6)*{\afvjd4{}{f_k(\epsilon_n)}};
( 10, 0)*{\invisiblemark};
}
\qquad\mbox{and}\qquad
\atomicflow
{
(-10, 0)*{\invisiblemark};
(  0,-6)*{\afvjd4{\iota_1}{}};
(  0, 0)*{\affr{14}8};
(  0, 0)*{\copy\contrdown};
( -6, 6)*{\afvjd4{f_1(\iota_1)}{}};
(  0, 6)*{\cdots};
(  6, 6)*{\afvjd4{}{f_k(\iota_1)}};
( 10, 0)*{\invisiblemark};
}
\quad\cdots\quad
\atomicflow
{
(-10, 0)*{\invisiblemark};
(  0,-6)*{\afvjd4{\iota_m}{}};
(  0, 0)*{\affr{14}8};
(  0, 0)*{\copy\contrdown};
( -6, 6)*{\afvjd4{f_1(\iota_m)}{}};
(  0, 6)*{\cdots};
(  6, 6)*{\afvjd4{}{f_k(\iota_m)}};
( 10, 0)*{\invisiblemark};
}
\quad,
\]
respectively. Furthermore, in Definition~\ref{definition:FourBoxes}, when we write
\[
\atomicflow
{
(-9,0)*{\invisiblemark};
(-6,0)*{\afvjum8{f_2(\boldsymbol{\epsilon})}{}};
(-4,4)*{\afaidnw{}{}};
( 0,0)*{\afacdm{}{}{}{}{g(\boldsymbol{\epsilon'})}{}};
( 4,4)*{\afaidnw{}{}};
( 6,0)*{\afvjum8{}{f_4(\boldsymbol{\epsilon})}};
( 9,0)*{\invisiblemark};
}
\quad\mbox{and}\quad
\atomicflow
{
(-9, 0)*{\invisiblemark};
(-6, 0)*{\afvjdm8{f_3(\boldsymbol{\iota})}{}};
(-4,-6)*{\afaiunw{}{}};
( 0, 0)*{\afacum{}{}{}{}{g(\boldsymbol{\iota'})}{}};
( 4,-6)*{\afaiunw{}{}};
( 6, 0)*{\afvjdm8{}{f_4(\boldsymbol{\iota})}};
( 9,0)*{\invisiblemark};
}\quad,
\]
we mean
\[
\atomicflow
{
(-9,0)*{\invisiblemark};
(-6,0)*{\afvju8{f_2(\epsilon_1)}{}};
(-4,4)*{\afaidnw{}{}};
( 0,0)*{\afacd{}{}{}{}{g(\epsilon_1')}{}};
( 4,4)*{\afaidnw{}{}};
( 6,0)*{\afvju8{}{f_4(\epsilon_1)}};
( 9,0)*{\invisiblemark};
}
\quad\cdots\quad
\atomicflow
{
(-9,0)*{\invisiblemark};
(-6,0)*{\afvju8{f_2(\epsilon_n)}{}};
(-4,4)*{\afaidnw{}{}};
( 0,0)*{\afacd{}{}{}{}{g(\epsilon_n')}{}};
( 4,4)*{\afaidnw{}{}};
( 6,0)*{\afvju8{}{f_4(\epsilon_n)}};
( 9,0)*{\invisiblemark};
}
\quad\mbox{and}\quad
\atomicflow
{
(-9, 0)*{\invisiblemark};
(-6, 0)*{\afvjd8{f_3(\iota_1)}{}};
(-4,-6)*{\afaiunw{}{}};
( 0, 0)*{\afacu{}{}{}{}{g(\iota_1')}{}};
( 4,-6)*{\afaiunw{}{}};
( 6, 0)*{\afvjd8{}{f_4(\iota_1)}};
( 9,0)*{\invisiblemark};
}
\quad\cdots\quad
\atomicflow
{
(-9, 0)*{\invisiblemark};
(-6, 0)*{\afvjd8{f_3(\iota_m)}{}};
(-4,-6)*{\afaiunw{}{}};
( 0, 0)*{\afacu{}{}{}{}{g(\iota_m')}{}};
( 4,-6)*{\afaiunw{}{}};
( 6, 0)*{\afvjd8{}{f_4(\iota_m)}};
( 9,0)*{\invisiblemark};
}\quad,
\]
respectively. In other words, edges are not connected in unexpected ways.
\end{convention}
%-----------

%======================================
\subsection{Four Boxes}\label{subsection:FourBoxes}

\newcommand{\frfb}{{\mathsf{fb}}}
%---------------------------------------
\begin{definition}\label{definition:FourBoxes}
We define the reduction $\to_\frfb$ (where $\frfb$ stands for \emph{four boxes}) as follows, for any atomic flows $\phi$ and $\psi$ that do not contain interaction or cut vertices:
\[
\atomicflow{
(-8, 6)*{\afvjdm4{\boldsymbol{\epsilon}}{}};
( 0, 8)*{\afaidm{}{}{}{}{}{}};
( 8, 6)*{\afvjdm4{}{\boldsymbol{\epsilon'}}};
%-
(-5, 0)*{\affr88};
(-4, 2)*{\aflabelright\phi};
( 5, 0)*{\affr88};
( 6, 2)*{\aflabelright\psi};
%-
(-8,-6)*{\afvjum4{\boldsymbol{\iota}}{}};
( 0,-8)*{\afaium{}{}{}{}{}{}};
( 8,-6)*{\afvjum4{}{\boldsymbol{\iota'}}};
}
\quad\to_\frfb\quad
\atomicflow{
%left
(-27, 8)*{\afawdm{}{}{}{}};
(-19,16)*{\afvjdm8{\boldsymbol{\epsilon}}{}};
(-19, 8)*{\afacum{}{}{}{}{}{}};
(-24, 0)*{\affr88};
(-20, 2)*{\aflabelleft{f_1(\phi)}};
(-27,-8)*{\afawum{}{}{}{}};
(-19,-8)*{\afacdm{}{}{}{}{}{}};
(-19,-16)*{\afvjum8{\boldsymbol{\iota}}{}};
%top
(-15,18)*{\afawdm{}{}{}{\boldsymbol{\tilde\epsilon}}};
( -7,18)*{\afaidnw{}{}};
( -9,16)*{\afvjm4};
( -2,15)*{\afcjlm66};
(-12,10)*{\affr88};
( -8,12)*{\aflabelleft{f_2(\phi)}};
(-16, 1)*{\afcjrm2{10}};
( -9, 2)*{\afawum{}{}{}{}};
%bot
(-16,-1)*{\afcjlm2{10}};
( -9,-2)*{\afawdm{}{}{}{}};
(-12,-10)*{\affr88};
( -8, -8)*{\aflabelleft{f_3(\phi)}};
( -9,-16)*{\afvjm4};
( -2,-15)*{\afcjrm66};
( -7,-20)*{\afaiunw{}{}};
(-15,-18)*{\afawum{}{}{}{\boldsymbol{\tilde\iota}}};
%center
(-3, 12)*{\afvjdm{16}{}{\boldsymbol{\epsilon'}}};
( 3,  8)*{\afacdm{}{}{}{}{}{}};
( 7, 12)*{\afaidnw{}{}};
( 0,  0)*{\affr88};
( 4,  2)*{\aflabelleft{g(\psi)}};
( 7,-14)*{\afaiunw{}{}};
( 3, -8)*{\afacum{}{}{}{}{}{}};
(-3,-12)*{\afvjum{16}{}{\boldsymbol{\iota'}}};
%right
(15, 8)*{\afawdm{}{}{}{\boldsymbol{\hat\epsilon}}};
( 9, 8)*{\afvjm8};
(12, 0)*{\affr88};
(16, 2)*{\aflabelleft{f_4(\phi)}};
( 9,-8)*{\afvjm8};
(15,-8)*{\afawum{}{}{}{\boldsymbol{\hat\iota}}};
}\quad.
\]
\end{definition}

%---------------------------------------
\begin{theorem}\label{theorem:SoundFourBoxes}
Reduction\/ $\to_\frfb$ is sound.
\end{theorem}

%---------------------------------------
\begin{proof}
Let $\Phi$ be a derivation with flow $\phi'$, such that $\phi'\to_\frfb\psi'$. We show that there exists a derivation $\Psi$ with flow $\psi'$ and with the same premiss and conclusion as $\Phi$. In the following, we refer to the figure in Definition~\vref{definition:FourBoxes}.

We obtain $\Psi$, with atomic flow $\psi'$, from $\Phi$ by, for every atom occurrence $a^\phi$ in $\Phi$:
\begin{enumerate}
	\item If $a^\phi$ is not in the premiss or conclusion of $\Phi$, substitute it with the formula
	\[
	 \vls([a^{f_1(\phi)}.a^{f_2(\phi)}].[a^{f_3(\phi)}.a^{f_4(\phi)}])\quad.
	\]
	\item If $a^\phi$ is in the premiss or conclusion of $\Phi$, substitute it with the derivation
\[
\vlinf{}{}
{
 \vls(
  [
   a^{f_1(\phi)}
  \;.\;
   \vlinf{}{}{a^{f_2(\phi)}}{\fff}
  ]
 \;.\;
  [
   a^{f_3(\phi)}
  \;.\;
   \vlinf{}{}{a^{f_4(\phi)}}{\fff}
  ]
 )  
}
{a}
\quad\mbox{or}\quad
\vlinf{}{}
{a}
{
 \vls
 (
  [a^{f_1(\phi)}.a^{f_2(\phi)}]
 \;.\;
  [
   \vlinf{}{}{\ttt}{a^{f_3(\phi)}}
  \;.\;
   \vlinf{}{}{\ttt}{a^{f_4(\phi)}}
  ]
 )
}
\quad,
\]
respectively.
	\item Substitute rule instances $\vlinf{}{}{\vls[a^\phi.\bar a^\psi]}{\ttt}$ and $\vlinf{}{}{\fff}{\vls(a^\phi.\bar a^\psi)}$ with the derivations
\[
\vlderivation
{
 \vlin{=}{}{\vls[([\vlinf{}{}{a^{f_1(\phi)}}{\fff}\;.\;a^{f_2(\phi)}]\;.\;[\vlinf{}{}{a^{f_3(\phi)}}{\fff}\;.\;a^{f_4(\phi)}])\;.\;\vlinf{}{}{\bar a^\psi}{\vls[\bar a.\bar a]}]}
 {
  \vlin{\swi}{}{\vls[\vlinf{\swi}{}{\vls\vlsmallbrackets[(a^{f_2(\phi)}.a^{f_4(\phi)}).\bar a]}{\vls\vlsmallbrackets(a^{f_2(\phi)}.[a^{f_4(\phi)}.\bar a])}\;.\;\bar a]}
  {
   \vlhy{\vls(\vlinf{}{}{\vls[a^{f_2(\phi)}.\bar a]}{\ttt}\;.\;\vlinf{}{}{\vls[a^{f_4(\phi)}.\bar a]}{\ttt})}
  }
 }
}\qquad\mbox{and}
\]
\[
\vlderivation
{
 \vlin{\swi}{}{\vls[\vlinf{}{}{\fff}{\vls(a^{f_3(\phi)}.\bar a)}\;.\;\vlinf{}{}{\fff}{\vls(a^{f_4(\phi)}.\bar a)}]}
 {
  \vlin{=}{}{\vls(\vlinf{\swi}{}{\vls[a^{f_3(\phi)}.(a^{f_4(\phi)}.\bar a)]}{\vls([a^{f_3(\phi)}.a^{f_4(\phi)}].\bar a)}\;.\;\bar a)}
  {
   \vlhy{\vls(([\vlinf{}{}{\ttt}{a^{f_1(\phi)}}\;.\;\vlinf{}{}{\ttt}{a^{f_2(\phi)}}]\;.\;\vlsmallbrackets[a^{f_3(\phi)}.a^{f_4(\phi)}])\;.\;\vlinf{}{}{\vls(\bar a.\bar a)}{\bar a^\psi})}
  }
 }
}\quad,
\]
respectively.
	\item Substitute rule instances $\acd$, $\acu$, $\awd$, $\awu$ which apply to $a^\phi$ with $\cod$, $\cou$, $\wed$, $\weu$, respectively, applied to $\vls\vlsmallbrackets([a^{f_1(\phi)}.a^{\phi_2}].[a^{f_3(\phi)}.a^{f_4(\phi)}])$.
\end{enumerate}
\end{proof}

\TODO{Make a proposition? Rephrase?}

\begin{lemma}\label{lemma:FourBoxesSize}
For any two atomic flows $\phi$ and $\psi$ such that $\phi\to_\frfb\psi$ the size of $\psi$ is at most four times the size of $\phi$.
\end{lemma}

\begin{lemma}\label{lemma:FourBoxesStreamlining}
Given a polarity assignment $\pi$, and two atomic flows $\phi$ and $\psi$ such that $\phi\to_\frfb\psi$ and $\phi$ is weakly streamlined with respect to $\pi$ then $\psi$ is weakly streamlined with respect to $\pi$.
\end{lemma}

\begin{proof}
By studying the atomic flows in Definition~\vref{definition:FourBoxes} we can observe that for every path from an interaction vertex to a cut vertex in $\psi$ there is a path from an interaction vertex to a cut vertex in $\phi$ with the same polarity assignment.
\end{proof}

\begin{lemma}\label{lemma:FlowExistsSimpleForm}
Given an atomic flow $\phi$ and a polarity assignment $\pi$ there exists an atomic flow $\psi$, which is on simple form with respect to $\pi$, such that $\phi\to_\frfb\psi$.
\end{lemma}

\begin{proof}
Since we can represent $\phi$ with polarity assignment $\pi$ as follows:
\[
\atomicflow{
(-8, 6)*{\afvjdm4{\boldsymbol{\epsilon}}{}};
( 0, 8)*{\afaidm{}{}{}{}{}{}};
( 8, 6)*{\afvjdm4{}{\boldsymbol{\epsilon'}}};
%-
(-5, 0)*{\affr88};
(-6, 2)*{\aflabelleft\ppl};
( 5, 0)*{\affr88};
( 4, 2)*{\aflabelleft\pmi};
%-
(-8,-6)*{\afvjum4{\boldsymbol{\iota}}{}};
( 0,-8)*{\afaium{}{}{}{}{}{}};
( 8,-6)*{\afvjum4{}{\boldsymbol{\iota'}}};
}\quad.
\]
It is routine to check Definition~\vref{definition:FlowSimpleForm} and Definition~\vref{definition:FourBoxes} to verify that $\psi$ is on simple form with respect to $\pi$.
\end{proof}

\TODO{Reread this:}

\newcommand{\Simpl}{\mathsf{Simpl}}
\begin{definition}\label{definition:TheSimpleForm}
Given an atomic flow $\phi$ and a polarity assignment $\pi$, let $\psi$ be the atomic flow obtained in the proof of Lemma~\vref{lemma:FlowExistsSimpleForm}, such that $\phi\to_\frfb\psi$ and $\psi$ is on simple form with respect to $\pi$, then we say that $\psi$ is \emph{the simple form of $\phi$ with respect to $\pi$}, denoted $\Simpl(\phi,\pi)$. If $\phi$ is the atomic flow of the derivation $\Phi$ and $\psi$ is the atomic flow of the derivation $\Psi$, then we say that $\Psi$ is \emph{the simple form of $\Phi$ with respect to $\pi$}, denoted $\Simpl(\Phi,\pi)$.
\end{definition}

\TODO{Check:}

\begin{remark}\label{remark:FourBoxesDestroySimpleForm}
Given an atomic flow $\phi$ and a polarity assignment $\pi$, such that $\phi$ is not weakly streamlined with respect to $\bar\pi$, then $\Simpl(\phi,\pi)$ is not on simple form with respect to $\bar\pi$.
\end{remark}


%======================================
\subsection{Isolated Subflow Removal}\label{subsection:IsolatedSubflowRemoval}

\TODO{Throughout: use `isolated subflow' and define it somewhere else.}

\newcommand{\fris}{{\mathsf{is}}}
%---------------------------------------
\begin{definition}\label{definition:IsolatedSubflowRemoval}
We define the reduction $\to_\fris$ (where $\fris$ stands for \emph{isolated subflow}) as follows, for any atomic flow $\phi$ and any connected atomic flow $\psi$:
\[
\atomicflow{
(-8, 6)*{\afvjdm4{\boldsymbol\epsilon}{}};
( 1, 8)*{\afaidex{}{}{}{}{}{}{6}{4}};
(-5, 0)*{\affr{8}{8}};
(-4, 2)*{\aflabelright{\phi}};
( 4, 0)*{\affr{6}{8}};
( 4, 2)*{\aflabelright{\psi}};
( 1,-8)*{\afaiuex{}{}{}{}{}{}{6}{4}};
(-8,-6)*{\afvjum4{\boldsymbol\iota}{}};
}
\quad\to_\fris\quad
\atomicflow{
( 0, 16.5)*{\afacumexsq{f_2(\boldsymbol\epsilon)}{}{}{f_1(\boldsymbol\epsilon)}{\boldsymbol\epsilon}{}{12}{4}};
(-6,  4  )*{\afvjm{12}};
( 0, 14  )*{\afawd{}{}{}{}};
( 3,  6  )*{\affr{8}{8}};
( 2,  8  )*{\aflabelright{f_1(\phi)}};
( 0,  0  )*{\afvj4};
(-3, -6  )*{\affr{8}{8}};
(-4, -4  )*{\aflabelright{f_2(\phi)}};
( 0,-14  )*{\afawu{}{}{}{}};
( 6, -4  )*{\afvjm{12}};
( 0,-16  )*{\afacdmexsq{f_2(\boldsymbol\iota)}{}{}{f_1(\boldsymbol\iota)}{\boldsymbol\iota}{}{12}{4}};
}\quad.
\]
\end{definition}

%---------------------------------------
\begin{theorem}\label{theorem:SoundIsolatedSubflowRemoval}
Reduction\/ $\to_\fris$ is sound.
\end{theorem}

%---------------------------------------
\begin{proof}
Let $\Phi$ be a derivation with flow $\phi'$, such that $\phi'\to_\fris\psi'$. We show that there exists a derivation $\Psi$ with flow $\psi'$ and with the same premiss and conclusion as $\Phi$. In the following, we refer to the figure in Definition~\ref{definition:IsolatedSubflowRemoval}.

Since $\psi$ is connected, we assume, by Remark~\vref{remark:AlternativeAiDecomposedForm}, that the following derivation is an $\ai$-decomposed form of $\Phi$:
\[
\vlder{\Phi'}{}
{
 \vlsbr[\beta\;.\;\vlinf{}{}{\fff}{\vls(a^\psi.\bar a)}]
}
{
 \vlsbr(\vlinf{}{}{\vls[a^\psi.\bar a]}{\ttt}\;.\;\alpha)
}\quad,
\]
for some atom $a$ and formulae $\alpha$ and $\beta$.

We obtain the two derivations $\Phi_\ttt$ and $\Phi_\fff$ from $\Phi'$ as follows:
\[
\Phi_\ttt\;=\;
\vlder{\Phi'\{a^\psi/\fff, \bar a^\psi/\ttt\}}{}
{
 \vls[\beta.\bar a]
}
{
 \vls([\ttt.\bar a].\alpha)
}
\qquad\mbox{and}\qquad
\Phi_\fff\;=\;
\vlder{\Phi'\{a^\psi/\ttt, \bar a^\psi/\fff\}}{}
{
 \vls[\beta.(\fff.\bar a)]
}
{
 \vls(\bar a.\alpha)
}\quad.
\]
Since $\psi$ is connected, the mapping from all the occurrences $a^\psi$ and $\bar a^\psi$ to edges of $\psi$ is surjective. Hence, we know that both derivation $\Phi_\ttt$ and $\Phi_\fff$ have a flow isomorphic to $\phi$. We combine $\Phi_\ttt$ and $\Phi_\fff$ to get the desired derivation $\Psi$ with flow $\psi'$ and the same premiss and conclusion as $\Phi$:
\[
\Psi\;=\;
\vlderivation
{
 \vlin{\cod}{}
 {
  \beta
 }
 {
  \vlin{\swi}{}
  {
   \vls
   [
    \beta
   \;\;\;.\;\;\;
    \vlder{\Phi_\fff}{}
    {
     \vlsbr[\beta\;.\;(\fff\;.\;\vlinf{}{}{\ttt}{\bar a})]
    }
    {
     \vls(\bar a.\alpha)
    }
   ]
  }
  {
   \vlin{\cou}{}
   {
    \vls
    (
     \vlder{\Phi_\ttt}{}
     {
      \vls[\beta.\bar a]
     }
     {
      \vlsbr([\ttt\;.\;\vlinf{}{}{\bar a}{\fff}]\;.\;\alpha)
     }
    \;\;\;.\;\;\;
     \alpha
    )
   }
   {
    \vlhy{\alpha}
   }
  }
 }
}\quad.
\]
\end{proof}

\begin{lemma}\label{lemma:IsolatedSubflowRemovalStreamlining}
Given a polarity assignment $\pi$, and two atomic flows $\phi$ and $\psi$ such that $\phi\to_\fris\psi$ and $\phi$ is weakly streamlined with respect to $\pi$ then $\psi$ is weakly streamlined with respect to $\pi$.
\end{lemma}

\newcommand{\ISR}{\mathsf{ISR}}
\begin{definition}
The \emph{isolated subflow remover}, $\ISR$, is an operator whose arguments are an atom $a$, and a derivation of shape
\[
\Phi\;=\;
\vlder{\Phi'}{}
{
 \vlsbr
 [
  \beta
 \;.\;
  \vlinf{}{}
  {
   \fff
  }
  {
   \vls(a^\psi.\bar a)
  }
 \;.\;\cdots\;.\;
  \vlinf{}{}
  {
   \fff
  }
  {
   \vls(a^\psi.\bar a)
  }
 ]
}
{
 \vlsbr
 (
  \vlinf{}{}
  {
   \vls[a^\psi.\bar a]
  }
  {
   \ttt
  }
 \;.\;\cdots\;.\;
  \vlinf{}{}
  {
   \vls[a^\psi.\bar a]
  }
  {
   \ttt
  }
 \;.\;
  \alpha
 )
}\quad,
\]
with atomic flow
\[
\atomicflow{
(-8, 6)*{\afvjdm4{}{\boldsymbol\epsilon}};
( 1, 8)*{\afaidmex{}{}{}{}{}{}{6}{4}};
(-5, 0)*{\affr{8}{8}};
(-1, 2)*{\aflabelleft{\phi}};
( 4, 0)*{\affr{6}{8}};
( 7, 2)*{\aflabelleft{\psi}};
( 1,-8)*{\afaiumex{}{}{}{}{}{}{6}{4}};
(-8,-6)*{\afvjum4{}{\boldsymbol\iota}};
}
\quad,
\]
such $\psi$ is the juxtraposition of all the isolated subflows mapped to from $a$. Consider the derivation
\[
\Psi'\;=\;
\vlder{\Phi'}{}
{
 \vlsbr
 [
  \beta
 \;\;\;.\;\;\;
  \vlderivation
  {
   \vlin{}{}
   {
    \fff
   }
   {
    \vlde{}{\{\cod\}}
    {
     \vls(a.\bar a)
    }
    {
     \vlhy
     {
      \vls[(a.\bar a).\cdots.(a.\bar a)]
     }
    }
   }
  }
 ]
}
{
 \vlsbr
 (
  \vlderivation
  {
   \vlde{}{\{\cou\}}
   {
    \vls([a.\bar a].\cdots.[a.\bar a])
   }
   {
    \vlin{}{}
    {
     \vls[a.\bar a]
    }
    {
     \vlhy
     {
      \ttt
     }
    }
   }
  }
 \;\;\;.\;\;\;
  \alpha
 )
}\quad,
\]
with atomic flow
\[
\psi'\;=\;
\atomicflow{
(-8, 11)*{\afvjdm{14}{\boldsymbol\epsilon}{}};
( 2, 18)*{\afaidex{}{}{}{}{}{}{8}{4}};
%-
(-2, 10)*{\affr68};
(-2, 10)*{\copy\contrup};
(-2,  5)*{\afvjm2};
(-5,  0)*{\affr{8}{8}};
(-1,  2)*{\aflabelleft{\phi}};
(-2, -5)*{\afvjm2};
(-2,-10)*{\affr68};
(-2,-10)*{\copy\contrdown};
%-
( 6, 10)*{\affr68};
( 6, 10)*{\copy\contrup};
( 6,  5)*{\afvjm2};
( 6,  0)*{\affr{6}{8}};
( 9,  2)*{\aflabelleft{\psi}};
( 6, -5)*{\afvjm2};
( 6,-10)*{\affr68};
( 6,-10)*{\copy\contrdown};
%-
( 2,-18)*{\afaiuex{}{}{}{}{}{}{8}{4}};
(-8,-11)*{\afvjum{14}{\boldsymbol\iota}{}};
}\quad.
\]
We then define $\ISR(\Phi,a)$ to be such that $\Psi'\to_\fris\ISR(\Phi,a)$.
\end{definition}

\begin{proposition}
Given an atom $a$, and a derivation
\[
\Phi\;=\;
\vlder{}{}
{
 \vlsbr
 [
  \beta
 \;.\;
  \vlinf{}{}
  {
   \fff
  }
  {
   \vls(a^\psi.\bar a)
  }
 \;.\;\cdots\;.\;
  \vlinf{}{}
  {
   \fff
  }
  {
   \vls(a^\psi.\bar a)
  }
 ]
}
{
 \vlsbr
 (
  \vlinf{}{}
  {
   \vls[a^\psi.\bar a]
  }
  {
   \ttt
  }
 \;.\;\cdots\;.\;
  \vlinf{}{}
  {
   \vls[a^\psi.\bar a]
  }
  {
   \ttt
  }
 \;.\;
  \alpha
 )
}\quad,
\]
with atomic flow
\[
\atomicflow{
(-8, 6)*{\afvjdm4{}{\boldsymbol\epsilon}};
( 1, 8)*{\afaidmex{}{}{}{}{}{}{6}{4}};
(-5, 0)*{\affr{8}{8}};
(-1, 2)*{\aflabelleft{\phi}};
( 4, 0)*{\affr{6}{8}};
( 7, 2)*{\aflabelleft{\psi}};
( 1,-8)*{\afaiumex{}{}{}{}{}{}{6}{4}};
(-8,-6)*{\afvjum4{}{\boldsymbol\iota}};
}
\quad,
\]
such that all the edges in $\psi$ are mapped to from $a$.
\begin{itemize}
\item $\ISR(\Phi,a)$ is weakly streamlined with respect to $a$;
\item for any atom $b$, if $\Phi$ is weakly streamlined with respect to $b$, then $\ISR(\Phi,a)$ is weakly streamlined with respect to $b$;
\item if $\Phi$ is on simple form with respect to $\pi$ then $\ISR(\Phi,a)$ is on simple form with respect to $\pi$; and
\item $\ISR(\Phi,a)$ depends at most linearly on the size of $\Phi$.
\end{itemize}
\end{proposition}

%======================================
\subsection{Path Breaker}\label{subsection:PathBreaker}

\TODO{Add labels to edges}

%--------------------------------
\newcommand{\frpb}{{\mathsf{pb}}}
\begin{definition}\label{definition:PathBreaker}
We define the reduction $\to_\frpb$ (where $\frpb$ stands for \emph{path breaker}) as follows, for any atomic flows $\phi$ and $\psi$:
\[
\atomicflow
{
(-8, 7)*{\afvjdm{6}{\boldsymbol\epsilon}{}};
( 0, 8)*{\afaid{}{}{}{}{}{}};
( 8, 7)*{\afvjdm{6}{}{\boldsymbol{\epsilon'}}};
(-5, 0)*{\affr{8}{8}};
(-4, 2)*{\aflabelright\phi};
%---
( 5, 0)*{\affr{8}{8}};
( 6, 2)*{\aflabelright{\psi}};
( 8,-7)*{\afvjum{6}{}{\boldsymbol{\iota'}}};
( 0,-8)*{\afaiu{}{}{}{}{}{}};
(-8,-7)*{\afvjum{6}{\boldsymbol\iota}{}};
}
\quad\to_\frpb\quad
\atomicflow
{
%%%%% RED %%%%%
(0,-20)="D";
(0,-10)="Dhalf";
%% contractions
"D"+"D"="A";
%left
"A"+(-14,-15.5)-"D"*{\afvjmcol{23}{Red}};
"A"+(-11,-17)*{\afvjumcol{4}{\boldsymbol\iota}{}{Red}};
%right
"A"+(11,-11.5)-"Dhalf"*{\afvjmcol{11}{Red}};
"A"+(14,-15.5)-"D"*{\afvjmcol{23}{Red}};
"A"+(11,-17)*{\afvjumcol{4}{}{\boldsymbol{\iota'}}{Red}};
% top boxes
(0,0)="A";
"A"+(-11,-14)*{\afcjrmcol{6}{20}{Red}};
"A"+(11,-14)*{\afcjlmcol{6}{20}{Red}};
"A"+( 0,  8)*{\afaidcol{}{}{}{}{}{}{Red}{Red}};
"A"+(-2, -8)*{\afawucol{}{}{}{}{}{Red}};
% join one
"A"+(2,-10)*{\afvjcol{12}{Red}};
% middle boxes
"A"+"D"="A";
"A"+(9.5,-10)*{\afcjlmcol{3}{12}{Red}};
"A"+( 2,-8)*{\afawucol{}{}{}{}{}{Red}};
%%%%% GREEN %%%%%
%% cocontractions
(0,0)="A";
%left
"A"+(-11,17)*{\afvjdmcol{4}{\boldsymbol\epsilon}{}{OliveGreen}};
"A"+"D"+(-14,15.5)*{\afvjmcol{23}{OliveGreen}};
"A"+"Dhalf"+(-11,11.5)*{\afvjmcol{11}{OliveGreen}};
%right
"A"+(11,17)*{\afvjdmcol{4}{}{\boldsymbol{\epsilon'}}{OliveGreen}};
"A"+"D"+(14,15.5)*{\afvjmcol{23}{OliveGreen}};
% middle boxes
"A"+"D"="A";
"A"+(-9.5,10)*{\afcjlmcol{3}{12}{OliveGreen}};
"A"+(-2, 8)*{\afawdcol{}{}{}{}{}{OliveGreen}};
% join two
"A"+(-2,-10)*{\afvjcol{12}{OliveGreen}};
% bottom boxes
"A"+"D"="A";
"A"+(-11,14)*{\afcjlmcol{6}{20}{OliveGreen}};
"A"+(11,14)*{\afcjrmcol{6}{20}{OliveGreen}};
"A"+( 0,-8)*{\afaiucol{}{}{}{}{}{}{OliveGreen}{OliveGreen}};
"A"+( 2, 8)*{\afawdcol{}{}{}{}{}{OliveGreen}};
%%%%% BLACK %%%%%
%% cocontractions
(0,0)="A";
%left
"A"+(-8,5.5)*{\afvjm3};
"A"+(-11,11)*{\affr88};
"A"+(-11,11)*{\copy\contrup};
%right
"A"+(8,5.5)*{\afvjm3};
"A"+"Dhalf"+(11,11.5)*{\afvjm{11}};
"A"+(11,11)*{\affr88};
"A"+(11,11)*{\copy\contrup};
%% contractions
"D"+"D"="A";
%left
"A"+(-11,-11.5)-"Dhalf"*{\afvjm{11}};
"A"+(-8,-5.5)*{\afvjm3};
"A"+(-11,-11)*{\affr88};
"A"+(-11,-11)*{\copy\contrdown};
%right
"A"+(8,-5.5)*{\afvjm3};
"A"+(11,-11)*{\affr88};
"A"+(11,-11)*{\copy\contrdown};
% top boxes
(0,0)="A";
"A"+(-5,  0)*{\affr{8}{8}};
"A"+(-6,  2)*{\aflabelright{f_1(\phi)}};
"A"+( 5,  0)*{\affr{8}{8}};
"A"+( 4,  2)*{\aflabelright{g_1(\psi)}};
% middle boxes
"A"+"D"="A";
"A"+(9.5,10)*{\afcjrm{3}{12}};
"A"+(-9.5,-10)*{\afcjrm{3}{12}};
"A"+(-5, 0)*{\affr{8}{8}};
"A"+(-6, 2)*{\aflabelright{f_2(\phi)}};
"A"+( 5, 0)*{\affr{8}{8}};
"A"+( 4, 2)*{\aflabelright{g_2(\psi)}};
% bottom boxes
"A"+"D"="A";
"A"+(-5, 0)*{\affr{8}{8}};
"A"+(-6, 2)*{\aflabelright{f_3(\phi)}};
"A"+( 5, 0)*{\affr{8}{8}};
"A"+( 4, 2)*{\aflabelright{g_3(\psi)}};
}\quad,
\]
where the evidenced interaction and cut vertices belong to the same connected component.
\end{definition}
%---------------

\TODO{Alessio: We cannot identify edges if they do not belong to the same connected component (they might be mapped to from occurrences of different atoms. I will remark somewhere...}

%----------------------------------------------
\begin{theorem}\label{theorem:PathBreakerSound}
Reduction $\to_\frpb$ is sound; moreover, if\/ $\Phi\to_\frpb\Psi$, then the size of $\Psi$ depends at most cubically on the size of $\Phi$.
\end{theorem}

\begin{proof}
Let $\Phi$ be a derivation with flow $\phi'$, such that $\phi'\to_\frpb\psi'$. We show that there exists a derivation $\Psi$ with flow $\psi'$ and with the same premiss and conclusion as $\Phi$. In the following, we refer to the figure in Definition~\vref{definition:PathBreaker}.

Since the evidenced interaction and cut vertices belong to the same connected component, we assume, by Remark~\vref{remark:AlternativeAiDecomposedForm}, that the following derivation is an $\ai$-decomposed form of $\Phi$:
\[
\vlder{\Phi'}{}
{
 \vlsbr[\beta\;.\;\vlinf{}{}{\fff}{\vls(a^\phi.\bar a^\psi)}]
}
{
 \vlsbr(\vlinf{}{}{\vls[a^\phi.\bar a^\psi]}{\ttt}\;.\;\alpha)
}\quad,
\]
for some atom $a$ and formulae $\alpha$ and $\beta$.

We combine three copies of $\Phi'$ to obtain the desired derivation $\Psi$ with flow $\psi'$ and the same premiss and conclusion as $\Phi$:

\TODO{Make an example of this use of switches:}

\newbox\DeltaTopK
\setbox\DeltaTopK=
\hbox{$
\vlder{\Phi'}{}
{
 \vlsbr[\beta\;.\;(\vlinf{}{}{\ttt}{a^{f_1(\phi)}}\;.\;\bar a^{g_1(\psi)})]
}
{
 \vlsbr(\vlinf{}{}{\vls[a^{f_1(\phi)}.\bar a^{g_1(\psi)}]}{\ttt}\;.\;\alpha)
}
$}
\newbox\DeltaK
\setbox\DeltaK=
\hbox{$
\vlder{\Phi'}{}
{
 \vlsbr[\beta\;.\;(a^{f_2(\phi)}\;.\;\vlinf{}{}{\ttt}{\bar a^{g_2(\psi)}})]
}
{
 \vlsbr([\vlinf{}{}{a^{f_2(\phi)}}{\fff}\;.\;\bar a^{g_2(\psi)}]\;.\;\alpha)
}
$}
\newbox\DeltaBotK
\setbox\DeltaBotK=
\hbox{$
\vlder{\Phi'}{}
{
 \vlsbr[\beta\;.\;\vlinf{}{}{\fff}{\vls(a^{f_3(\phi)}.\bar a^{g_3(\psi)})}]
}
{
 \vlsbr([a^{f_3(\phi)}\;.\;\vlinf{}{}{\bar a^{g_3(\psi)}}{\fff}]\;.\;\alpha)
}
$}
\[
\Psi\quad=\quad
\vlderivation
{
 \vlin{\cod}{}{\beta}
 {
  \vlin{\swi}{}
  {
   \vls
   [
    \vlinf{\cod}{}{\beta}{\vls[\beta.\beta]}
   \;\;\;\;.\;\;\;\;
    \box\DeltaBotK
   ]
  }
  {
   \vlin{\swi}{}
   {
    \vls
    (
%     \vlinf{\swi}{}
%     {
%      \vls
      [
       \beta
      \;\;\;\;.\;\;\;\;
       \box\DeltaK
      ]
%     }
%     {
%      \vls(\vlsmallbrackets[\beta.\bar a^\psi].\alpha)
%     }
    \;\;\;\;\;.\;\;\;\;\;
     \alpha
    )   
   }
   {
    \vlin{\cod}{}
    {
     \vls
     (
      \box\DeltaTopK
     \;\;\;\;.\;\;\;\;
      \vlinf{\cou}{}{\vls(\alpha.\alpha)}{\alpha}
     )
    }
    {
     \vlhy{\alpha}
    }
   }
  }
 } 
}\qquad.
\]
We know that the size of $\Phi'$ depends at most cubically on the size of $\Phi$ by Theorem~\vref{theorem:aiDecomposedForm}, and that the size of $\Psi$ depends at most quadratically on the size of $\alpha$ and $\beta$ by Lemma~\vref{lemma:GenericContraction}, so $\Psi$ depends at most cubically on the size of $\Phi$.
\end{proof}
%----------

\TODO{Alessio: Fixed like the previous Lemma and added reference to $g_1$, $g_2$ and $g_3$.}

%-------------------------------------
\begin{lemma}\label{lemma:PathBreaker}
Given two atomic flows $\phi$ and $\psi$, such that $\phi\to_\frpb\psi$; let $f_1$, $f_2$, $f_3$, $g_1$, $g_2$ and $g_3$ be the isomorphisms, let $\nu_\ai$ be the evidenced cut (resp., interaction) vertex described in the contractum, and let $\nu'_\ai$ be the evidenced cut (resp., interaction) vertex described in the redex of Definition~\ref{definition:PathBreaker}; then, given an interaction (resp., cut) vertex $\nu$ in $\psi$, there is an interaction (resp., cut) vertex $\nu'$ in $\phi$, such that
\begin{itemize}
\item for some $1\le i\le 3$, $\nu=f_i(\nu')$ or $\nu=g_i(\nu')$, or $\nu=\nu_\ai$ and $\nu'=\nu'_\ai$;
\item if there is a path from $\nu$ to $\bot$ (resp., $\top$) in $\psi$, then there is a path from $\nu'$ to $\bot$ (resp., $\top$) in $\phi$; and
\item if there is a cut (resp., interaction) vertex $\hat\nu$ in $\psi$, such that there is a path from $\nu$ to $\hat\nu$ in $\psi$, then there is a cut (resp., interaction) vertex $\hat\nu'$ in $\phi$, such that, for some $1\le i\le 3$, $\hat\nu=f_i(\nu')$ or $\hat\nu=g_i(\nu')$, or $\hat\nu=\nu_\ai$ and $\hat\nu'=\nu'_\ai$; and there is a path from $\nu'$ to $\hat\nu'$ in $\phi$.
\end{itemize}
\end{lemma}

\begin{proof}
In the following we refer to the figure in Definition~\ref{definition:PathBreaker}:
\begin{itemize}
  \item by definition;
  \item any path from $\nu$ to $\top$ (resp., $\bot$) in $\psi$ must contain an edge $\epsilon$, such that, for some upper (resp., lower) edge $\epsilon'$ of $\phi$ and some $1\le i\le 3$, $f_i(\epsilon')=\epsilon$ or $g_i(\epsilon')=\epsilon$. Hence, there is a path from $\nu'$ to $\top$ (resp., $\bot$) in $\phi$; and
 \item we have to consider two cases:
 \begin{itemize}
  \item for some $1\le i\le 3$, $\nu=f_i(\nu')$ and $\hat\nu=f_i(\hat\nu')$, or $\nu=g_i(\nu')$ and $\hat\nu=g_i(\hat\nu')$, then there is a path from $\nu'$ to $\hat\nu'$ in $\phi$; or
  \item $\nu=g_1(\nu')$ and $\hat\nu=g_2(\hat\nu')$, or $\nu=f_2(\nu')$ and $\hat\nu=f_3(\hat\nu')$ (resp., $\nu=g_2(\nu')$ and $\hat\nu=g_1(\hat\nu')$, or $\nu=f_3(\nu')$ and $\hat\nu=f_2(\hat\nu')$),then there is a path from $\nu'$ to $\nu_\ai$ in $\phi$.
 \end{itemize}
\end{itemize}
\end{proof}
%----------

%----------------------------
\newcommand{\PB}{\mathsf{PB}}
\begin{definition}\label{definition:DerPathBreaker}
The \emph{Path Breaker}, $\PB$, is an operator whose arguments are an atom $a$, and a derivation $\Phi$ that is not weakly streamlined with respect to $a$ or with respect to $\bar a$, with $\ai$-decomposed form 
\[
\vlder{\Phi'}{}
{
 \vlsbr
 [
  \beta
 \;.\;
  \vlinf{}{}
  {
   \fff
  }
  {
   \vls(a^\psi.\bar a)
  }
 \;.\;\cdots\;.\;
  \vlinf{}{}
  {
   \fff
  }
  {
   \vls(a^\psi.\bar a)
  }
 ]
}
{
 \vlsbr
 (
  \vlinf{}{}
  {
   \vls[a^\psi.\bar a]
  }
  {
   \ttt
  }
 \;.\;\cdots\;.\;
  \vlinf{}{}
  {
   \vls[a^\psi.\bar a]
  }
  {
   \ttt
  }
 \;.\;
  \alpha
 )
}\quad,
\]
and atomic flow
\[
\phi''\;=\;
\atomicflow
{
(-8, 7)*{\afvjdm{6}{\boldsymbol\epsilon}{}};
( 0, 8)*{\afaidm{}{}{}{}{}{}};
( 8, 7)*{\afvjdm{6}{}{\boldsymbol{\epsilon'}}};
(-5, 0)*{\affr{8}{8}};
(-4, 2)*{\aflabelright{\phi'}};
%---
( 5, 0)*{\affr{8}{8}};
( 6, 2)*{\aflabelright{\psi'}};
( 8,-7)*{\afvjum{6}{}{\boldsymbol{\iota'}}};
( 0,-8)*{\afaium{}{}{}{}{}{}};
(-8,-7)*{\afvjum{6}{\boldsymbol\iota}{}};
}
\quad,
\]
such that occurrences of $a$ do not appear in an interaction or cut instance in $\Phi'$. Consider the derivation
\[
\Psi\;=\;
\vlder{\Phi'}{}
{
 \vlsbr
 [
  \beta
 \;\;\;.\;\;\;
  \vlderivation
  {
   \vlin{}{}
   {
    \fff
   }
   {
    \vlde{}{\{\cod\}}
    {
     \vls(a.\bar a)
    }
    {
     \vlhy
     {
      \vls[(a.\bar a).\cdots.(a.\bar a)]
     }
    }
   }
  }
 ]
}
{
 \vlsbr
 (
  \vlderivation
  {
   \vlde{}{\{\cou\}}
   {
    \vls([a.\bar a].\cdots.[a.\bar a])
   }
   {
    \vlin{}{}
    {
     \vls[a.\bar a]
    }
    {
     \vlhy
     {
      \ttt
     }
    }
   }
  }
 \;\;\;.\;\;\;
  \alpha
 )
}\quad,
\]
with atomic flow
\[
\psi''\;=\;
\atomicflow{
(-11, 11.5)*{\afvjdm{15}{\boldsymbol\epsilon}{}};
( 13, 11.5)*{\afvjdm{15}{}{\boldsymbol{\epsilon'}}};
(  1, 18)*{\afaidex{}{}{}{}{}{}{12}{4}};
%-
( -5, 10)*{\affr68};
( -5, 10)*{\copy\contrup};
( -5,  5)*{\afvjm2};
( -8,  0)*{\affr{8}{8}};
( -4,  2)*{\aflabelleft{\phi'}};
( -5, -5)*{\afvjm2};
( -5,-10)*{\affr68};
( -5,-10)*{\copy\contrdown};
%-
(  7, 10)*{\affr68};
(  7, 10)*{\copy\contrup};
(  7,  5)*{\afvjm2};
( 10,  0)*{\affr{8}{8}};
( 14,  2)*{\aflabelleft{\psi'}};
(  7, -5)*{\afvjm2};
(  7,-10)*{\affr68};
(  7,-10)*{\copy\contrdown};
%-
(  1,-18)*{\afaiuex{}{}{}{}{}{}{12}{4}};
(-11,-11.5)*{\afvjum{15}{\boldsymbol\iota}{}};
( 13,-11.5)*{\afvjum{15}{}{\boldsymbol{\iota'}}};
%----
( -6,  0)*{\affr{14}{30}};
(  1, 13)*{\aflabelleft{\phi}};
(9.5,  0)*{\affr{13}{30}};
( 16, 13)*{\aflabelleft{\psi}};
}\quad.
\]

We then define $\PB(\Phi,a)$ to be such that $\Psi\to_\frpb\PB(\Phi,a)$, where $\phi$ and $\psi$ are the flows, by the same names, shown in Definition~\vref{definition:PathBreaker}.
\end{definition}
%---------------

%------------------------------------------------------------
\begin{proposition}\label{proposition:IsolatedSubflowRemover}
Given an atom $a$, and a derivation $\Phi$ that is not weakly streamlined with respect to both $a$ and $\bar a$,
\begin{enumerate}
\item $\PB(\Phi,a)$ is weakly streamlined with respect to both $a$ and $\bar a$;
\item for any atom $b$, if $\Phi$ is weakly streamlined with respect to $b$, then $\PB(\Phi,a)$ is weakly streamlined with respect to $b$; and
\item the size of\/ $\PB(\Phi,a)$ depends at most cubically on the size of\/ $\Phi$.
\end{enumerate}
\end{proposition}

\TODO{Prove:}

\begin{proof}
The statements follow by
\begin{enumerate}
\item Definition~\vref{definition:DerStreamlined} and Lemma~\vref{lemma:PathBreaker};
\item Definitions~\ref{definition:DerStreamlined} and Lemma~\ref{lemma:PathBreaker};
\item the statement follows by Theorem~\vref{theorem:SoundPathBreaker}.
\end{enumerate}
\end{proof}
%----------

%-----------------------------------------
\begin{example}\label{example:PathBreaker}
Given a derivation $\Phi$ where the atoms $a_1$ and $a_2$ occur, such that the atomic flow associated with $\Phi$ is
\[
\atomicflow
{
(-2,8)*{\afaid{}{}{}{}{}{}};
(4,6)*{\afvjm4};
(0,0)*{\affr{10}8};
(5,2)*{\aflabelleft{\phi_1}};
(-4,-6)*{\afvjm4};
(2,-8)*{\afaiu{}{}{}{}{}{}};
}\quad
\atomicflow
{
(-2,8)*{\afaid{}{}{}{}{}{}};
(4,6)*{\afvjm4};
(0,0)*{\affr{10}8};
(5,2)*{\aflabelleft{\phi_2}};
(-4,-6)*{\afvjm4};
(2,-8)*{\afaiu{}{}{}{}{}{}};
}\quad
\atomicflow
{
(0,6)*{\afvjm4};
(0,0)*{\affr88};
(4,2)*{\aflabelleft{\psi}};
(0,-6)*{\afvjm4};
}\quad,
\]

\TODO{Alessio said: `\emph{This is a long shot. Perhaps you can argue a bit more and more clearly about the red edges. Not clear what the latter and former subflows are.}'.}

where all the edges in $\phi_1$ are mapped to from $a_1$ and all the edges in $\phi_2$ are mapped to from $a_2$, and there are no edges in $\psi$ that are mapped to from $a_1$ or $a_2$, then the atomic flow associated with $\PB((\Phi,a_1),a_2)$ is

\TODO{adjust labels}

\TODO{state that we ignore isomorphisms}

\TODO{put flows on one page (how?)}

\[
\atomicflow
{
%cocontraction - top
(4,37.5)*{\afvjm{3}};
(4,32)*{\affr{50}8};
(4,32)*{\copy\contrup};
%contraction - bot
(-4,-32)*{\affr{50}8};
(-4,-32)*{\copy\contrdown};
(-4,-37.5)*{\afvjm{3}};
%---------------------
(4,-18)="D";
(0,-9)="Dhalf";
%----------------
%%first
(-20,0)="B";
% cocontractions
"B"-"D"-"D"-(-12,7)="A";
%left
"A"+"Dhalf"+"Dhalf"+(4,-4)*{\afvjm{42}};
"A"+"Dhalf"+(0,-4)*{\afvjm{24}};
"A"+(-4,-4)*{\afvjm6};
% contractions
"B"+"D"+"D"+(-12,7)="A";
%right
"A"+(4,4)*{\afvjm6};
"A"+(0,4)-"Dhalf"*{\afvjm{24}};
"A"+(-4,4)-"Dhalf"-"Dhalf"*{\afvjm{42}};
%---
% top boxes
"B"-"D"="A";
"A"+(-2, 8)*{\afaid{}{}{}{}{}{}};
"A"+( 0,-8)*{\afawu{}{}{}{}{}};
"A"+( 0, 0)*{\affr{10}{8}};
"A"+( 2, 2)*{\aflabelright{\phi_1}};
% join one
"A"+(4,-9)*{\afvjcol{10}{Red}};
% middle boxes
"B"="A";
"A"+(-4, 8)*{\afawd{}{}{}{}{}};
"A"+( 4,-8)*{\afawu{}{}{}{}{}};
"A"+( 0, 0)*{\affr{10}{8}};
"A"+( 2, 2)*{\aflabelright{\phi_1}};
% join two
"A"+(0,-9)*{\afvjcol{10}{Red}};
% bottom boxes
"B"+"D"="A";
"A"+(2,-8)*{\afaiu{}{}{}{}{}{}};
"A"+(0, 8)*{\afawd{}{}{}{}{}};
"A"+(0, 0)*{\affr{10}{8}};
"A"+( 2, 2)*{\aflabelright{\phi_1}};
%----------------
%%second
(0,0)="B";
% cocontractions
"B"-"D"-"D"-(-12,7)="A";
%left
"A"+"Dhalf"+"Dhalf"+(4,-4)*{\afvjm{42}};
"A"+"Dhalf"+(0,-4)*{\afvjm{24}};
"A"+(-4,-4)*{\afvjm6};
% contractions
"B"+"D"+"D"+(-12,7)="A";
%right
"A"+(4,4)*{\afvjm6};
"A"+(0,4)-"Dhalf"*{\afvjm{24}};
"A"+(-4,4)-"Dhalf"-"Dhalf"*{\afvjm{42}};
%---
% top boxes
"B"-"D"="A";
"A"+(-2, 8)*{\afaid{}{}{}{}{}{}};
"A"+( 0,-8)*{\afawu{}{}{}{}{}};
"A"+( 0, 0)*{\affr{10}{8}};
"A"+( 2, 2)*{\aflabelright{\phi_1}};
% join one
"A"+(4,-9)*{\afvjcol{10}{Red}};
% middle boxes
"B"="A";
"A"+(-4, 8)*{\afawd{}{}{}{}{}};
"A"+( 4,-8)*{\afawu{}{}{}{}{}};
"A"+( 0, 0)*{\affr{10}{8}};
"A"+( 2, 2)*{\aflabelright{\phi_1}};
% join two
"A"+(0,-9)*{\afvjcol{10}{Red}};
% bottom boxes
"B"+"D"="A";
"A"+(2,-8)*{\afaiu{}{}{}{}{}{}};
"A"+(0, 8)*{\afawd{}{}{}{}{}};
"A"+(0, 0)*{\affr{10}{8}};
"A"+( 2, 2)*{\aflabelright{\phi_1}};
%----------------
%%third
(20,0)="B";
% cocontractions
"B"-"D"-"D"-(-12,7)="A";
%left
"A"+"Dhalf"+"Dhalf"+(4,-4)*{\afvjm{42}};
"A"+"Dhalf"+(0,-4)*{\afvjm{24}};
"A"+(-4,-4)*{\afvjm6};
% contractions
"B"+"D"+"D"+(-12,7)="A";
%right
"A"+(4,4)*{\afvjm6};
"A"+(0,4)-"Dhalf"*{\afvjm{24}};
"A"+(-4,4)-"Dhalf"-"Dhalf"*{\afvjm{42}};
%---
% top boxes
"B"-"D"="A";
"A"+(-2, 8)*{\afaid{}{}{}{}{}{}};
"A"+( 0,-8)*{\afawu{}{}{}{}{}};
"A"+( 0, 0)*{\affr{10}{8}};
"A"+( 2, 2)*{\aflabelright{\phi_1}};
% join one
"A"+(4,-9)*{\afvjcol{10}{Red}};
% middle boxes
"B"="A";
"A"+(-4, 8)*{\afawd{}{}{}{}{}};
"A"+( 4,-8)*{\afawu{}{}{}{}{}};
"A"+( 0, 0)*{\affr{10}{8}};
"A"+( 2, 2)*{\aflabelright{\phi_1}};
% join two
"A"+(0,-9)*{\afvjcol{10}{Red}};
% bottom boxes
"B"+"D"="A";
"A"+(2,-8)*{\afaiu{}{}{}{}{}{}};
"A"+(0, 8)*{\afawd{}{}{}{}{}};
"A"+(0, 0)*{\affr{10}{8}};
"A"+( 2, 2)*{\aflabelright{\phi_1}};
}
\]
\[
\atomicflow
{
(12,-50)="D";
% cocontractions
(18,-11)="A";
%
"A"+(6,5.5)*{\afvjm3};
"A"+(6,0)*{\affr{34}8};
"A"+(6,0)*{\copy\contrup};
"A"+(2,-29)*{\afvjm{50}};
"A"+(6,-29)*{\afvjm{50}};
"A"+(10,-33)*{\afvjm{58}};
"A"+(14,-54)*{\afvjm{100}};
"A"+(18,-54)*{\afvjm{100}};
"A"+(22,-58)*{\afvjm{108}};
% === BOX ONE ===
(0,-27)="A";
%-
"A"+( -6,24)*{\afaidex{}{}{}{}{}{}31};
%-
"A"+(-12,16)*{\affr{10}8};
"A"+(-12,16)*{\copy\contrup};
"A"+(  0,16)*{\affr{10}8};
"A"+(  0,16)*{\copy\contrup};
%-
"A"+(-16,8)*{\afvj8};
"A"+(-8,8)*{\afcjr88};
"A"+(0,8)*{\afcjrm{16}8};
%
"A"+(-8,8)*{\afcjl88};
"A"+(0,8)*{\afvj8};
"A"+(8,8)*{\afcjrm88};
%
"A"+(0,8)*{\afcjl{16}8};
"A"+(8,8)*{\afcjl88};
"A"+(16,8)*{\afvjm8};
%-
"A"+(-12,0)*{\affr{10}8};
"A"+(0,0)*{\affr{10}8};
"A"+(12,0)*{\affr{10}8};
"A"+(-10,2)*{\aflabelright{\phi_2}};
"A"+(2,2)*{\aflabelright{\phi_2}};
"A"+(14,2)*{\aflabelright{\phi_2}};
%-
"A"+(-8,-8)*{\afcjl88};
"A"+(0,-8)*{\afcjlcol{16}8{Red}};
%-
"A"+(-8,-8)*{\afcjrm88};
"A"+(0,-8)*{\afvj8};
"A"+(8,-8)*{\afcjlcol88{Red}};
%
"A"+(0,-8)*{\afcjrm{16}8};
"A"+(8,-8)*{\afcjr88};
"A"+(16,-8)*{\afvjcol8{Red}};
%
"A"+(  0,-16)*{\affr{10}8};
"A"+(  0,-16)*{\copy\contrdown};
"A"+( 12,-16)*{\affr{10}8};
"A"+( 12,-16)*{\copy\contrdown};
%---
"A"+(0,-24)*{\afawu{}{}{}{}};
"A"+(12,-25)*{\afvjcol{10}{Red}};
% === BOX TWO ===
"A"+"D"="A";
%-
"A"+(-12,24)*{\afawd{}{}{}{}};
%-
"A"+(-12,16)*{\affr{10}8};
"A"+(-12,16)*{\copy\contrup};
"A"+(  0,16)*{\affr{10}8};
"A"+(  0,16)*{\copy\contrup};
%-
"A"+(-16,8)*{\afvj8};
"A"+(-8,8)*{\afcjrcol88{Red}};
"A"+(0,8)*{\afcjrm{16}8};
%
"A"+(-8,8)*{\afcjl88};
"A"+(0,8)*{\afvjcol8{Red}};
"A"+(8,8)*{\afcjrm88};
%
"A"+(0,8)*{\afcjl{16}8};
"A"+(8,8)*{\afcjlcol88{Red}};
%-
"A"+(-12,0)*{\affr{10}8};
"A"+(0,0)*{\affr{10}8};
"A"+(12,0)*{\affr{10}8};
"A"+(-10,2)*{\aflabelright{\phi_2}};
"A"+(2,2)*{\aflabelright{\phi_2}};
"A"+(14,2)*{\aflabelright{\phi_2}};
%-
"A"+(-8,-8)*{\afcjlcol88{Red}};
"A"+(0,-8)*{\afcjl{16}8};
%-
"A"+(-8,-8)*{\afcjrm88};
"A"+(0,-8)*{\afvjcol8{Red}};
"A"+(8,-8)*{\afcjl88};
%
"A"+(0,-8)*{\afcjrm{16}8};
"A"+(8,-8)*{\afcjrcol88{Red}};
"A"+(16,-8)*{\afvj8};
%
"A"+(  0,-16)*{\affr{10}8};
"A"+(  0,-16)*{\copy\contrdown};
"A"+( 12,-16)*{\affr{10}8};
"A"+( 12,-16)*{\copy\contrdown};
%---
"A"+(12,-24)*{\afawu{}{}{}{}};
"A"+(0,-25)*{\afvjcol{10}{Red}};
% === BOX THREE ===
"A"+"D"="A";
%-
"A"+(  0,24)*{\afawd{}{}{}{}};
%-
"A"+(-12,16)*{\affr{10}8};
"A"+(-12,16)*{\copy\contrup};
"A"+(  0,16)*{\affr{10}8};
"A"+(  0,16)*{\copy\contrup};
%-
"A"+(-16,8)*{\afvjcol8{Red}};
"A"+(-8,8)*{\afcjr88};
"A"+(0,8)*{\afcjrm{16}8};
%
"A"+(-8,8)*{\afcjlcol88{Red}};
"A"+(0,8)*{\afvj8};
"A"+(8,8)*{\afcjrm88};
%
"A"+(0,8)*{\afcjlcol{16}8{Red}};
"A"+(8,8)*{\afcjl88};
%-
"A"+(-12,0)*{\affr{10}8};
"A"+(0,0)*{\affr{10}8};
"A"+(12,0)*{\affr{10}8};
"A"+(-10,2)*{\aflabelright{\phi_2}};
"A"+(2,2)*{\aflabelright{\phi_2}};
"A"+(14,2)*{\aflabelright{\phi_2}};
%-
"A"+(-16,-8)*{\afvjm8};
"A"+(-8,-8)*{\afcjl88};
"A"+(0,-8)*{\afcjl{16}8};
%-
"A"+(-8,-8)*{\afcjrm88};
"A"+(8,-8)*{\afcjl88};
"A"+(0,-8)*{\afvj8};
%
"A"+(0,-8)*{\afcjrm{16}8};
"A"+(8,-8)*{\afcjr88};
"A"+(16,-8)*{\afvj8};
%
"A"+(  0,-16)*{\affr{10}8};
"A"+(  0,-16)*{\copy\contrdown};
"A"+( 12,-16)*{\affr{10}8};
"A"+( 12,-16)*{\copy\contrdown};
%-
"A"+(  6,-24)*{\afaiuex{}{}{}{}{}{}31};
%---
"A"+(-20,-16)="A";
"A"+(-20,58)*{\afvjm{108}};
"A"+(-16,54)*{\afvjm{100}};
"A"+(-12,54)*{\afvjm{100}};
"A"+(-8,33)*{\afvjm{58}};
"A"+(-4,29)*{\afvjm{50}};
"A"+( 0,29)*{\afvjm{50}};
"A"+(-4,0)*{\affr{34}8};
"A"+(-4,0)*{\copy\contrdown};
"A"+(-4,-5.5)*{\afvjm3};
}
\]
\[
\atomicflow{
(0,34.5)*{\afvjm3};
(0,29)*{\affr{82}8};
(0,29)*{\copy\contrup};
%
(0,-29)*{\affr{82}8};
(0,-29)*{\copy\contrdown};
(0,-34.5)*{\afvjm3};
%-------------------
(30,0)="B";
%---------------
(0,0)-"B"="A";
"A"+(-10,14.5)*{\afvjm{21}};
"A"+(0,14.5)*{\afvjm{21}};
"A"+(10,14.5)*{\afvjm{21}};
%---
"A"+(-10,0)*{\affr88};
"A"+( -9,2)*{\aflabelright\psi};
"A"+(  0,0)*{\affr88};
"A"+(  1,2)*{\aflabelright\psi};
"A"+( 10,0)*{\affr88};
"A"+( 11,2)*{\aflabelright\psi};
%---
"A"+(-10,-14.5)*{\afvjm{21}};
"A"+(0,-14.5)*{\afvjm{21}};
"A"+(10,-14.5)*{\afvjm{21}};
%---------------
(0,0)="A";
%---
"A"+(-10,14.5)*{\afvjm{21}};
"A"+(0,14.5)*{\afvjm{21}};
"A"+(10,14.5)*{\afvjm{21}};
%---
"A"+(-10,0)*{\affr88};
"A"+( -9,2)*{\aflabelright\psi};
"A"+(  0,0)*{\affr88};
"A"+(  1,2)*{\aflabelright\psi};
"A"+( 10,0)*{\affr88};
"A"+( 11,2)*{\aflabelright\psi};
%---
"A"+(-10,-14.5)*{\afvjm{21}};
"A"+(0,-14.5)*{\afvjm{21}};
"A"+(10,-14.5)*{\afvjm{21}};
%---------------
"A"+"B"="A";
%---
"A"+(-10,14.5)*{\afvjm{21}};
"A"+(0,14.5)*{\afvjm{21}};
"A"+(10,14.5)*{\afvjm{21}};
%---
"A"+(-10,0)*{\affr88};
"A"+( -9,2)*{\aflabelright\psi};
"A"+(  0,0)*{\affr88};
"A"+(  1,2)*{\aflabelright\psi};
"A"+( 10,0)*{\affr88};
"A"+( 11,2)*{\aflabelright\psi};
%---
"A"+(-10,-14.5)*{\afvjm{21}};
"A"+(0,-14.5)*{\afvjm{21}};
"A"+(10,-14.5)*{\afvjm{21}};
}\quad.
\]
By comparing the subflow containing $\phi_1$ with the subflow containing $\phi_2$ above, we can see why the procedure is non-confluent: We could modify our procedure in several ways to make the two subflows more similar. However, we can see no way of avoiding the fact that there are more paths passing through the edges marked in red in the latter than in the former subflow.
\end{example}
%------------

%======================================
\subsection{Multiple Isolated Subflows Removal}\label{subsection:MultipleIsolatedSubflowsRemoval}

\newcommand{\Gammasf}{\mathsf\Gamma}

%----------------------------------
\newcommand{\frmis}{{\mathsf{mis}}}
\begin{definition}\label{definition:MultipleIsolatedSubflowsRemoval}
For every $n>0$, given
\begin{itemize}
\item atoms $a_1$, $\dots$, $a_n$;
\item an $N>0$;
\item for $0\le k\le N$, formulae $\gamma_{k,1}$, $\dots$, $\gamma_{k,n}$, such that
\begin{itemize}
 \item $\gamma_{0,1}=\cdots=\gamma_{0,n}=\ttt$, and
 \item $\gamma_{N,1}=\cdots=\gamma_{N,n}=\fff$; and
\end{itemize}
\item for $1\le k\le N$, a derivation
\[
\Gammasf_k\quad=\quad
\vlder{}{\SKS\setminus\{\aid,\aiu\}}
{
 \vls([a_1.\gamma_{k,1}].\cdots.[a_n.\gamma_{k,n}])
}
{
 \vls[(a_1.\gamma_{k-1,1}).\cdots.(a_n.\gamma_{k-1,n})]
}\quad,
\]
\end{itemize}
let, for $1\le k\le N$, $\eta_k$ be the atomic flow of\/ $\Gammasf_k$, and let
\[
\mu_k\;=\;
\atomicflow
{
(  4,  0)*{\affr{8}{8}};
(  8,  2)*{\aflabelleft{f_{1,1}(\psi_1)}};
( 10,  0)*{\cdots};
( 16,  0)*{\affr{8}{8}};
( 20,  2)*{\aflabelleft{f_{1,l_1}(\psi_1)}};
( 22,  0)*{\cdots};
( 28,  0)*{\affr{8}{8}};
( 32,  2)*{\aflabelleft{f_{n,1}(\psi_n)}};
( 34,  0)*{\cdots};
( 40,  0)*{\affr{8}{8}};
( 44,  2)*{\aflabelleft{f_{n,l_n}(\psi_n)}};
}
\quad,
\]
where, for $1\le i\le n$, $l_i$ is the number of atom occurrences in $\gamma_{k,i}$, we define the reduction $\to_\frmis$ (where $\frmis$ stands for \emph{multiple isolated subflows}) as follows, for any atomic flow $\phi$ and any connected atomic flows $\psi_1$, $\dots$, $\psi_n$ that do not contain interaction or cut vertices:

\TODO{Decide on the direction of the dots}

\[
\atomicflow{
(-12,  8)*{\afvjdm8{}{\boldsymbol\epsilon}};
(  4, 10)*{\afaidex{}{}{}{}{}{}{20}{4}};
( -6,  5)*{\afvj2};
( -4,  6)*{\cdots};
( 14,  5)*{\afvj2};
(  1,  8)*{\afaidex{}{}{}{}{}{}{6}{4}};
( -7,  0)*{\affr{12}{8}};
( -4,  2)*{\aflabelright{\phi}};
(  4,  0)*{\affr{6}{8}};
(  4,  2)*{\aflabelright{\psi_1}};
(  9,  0)*{\cdots};
( 14,  0)*{\affr{6}{8}};
( 14,  2)*{\aflabelright{\psi_n}};
(  1, -8)*{\afaiuex{}{}{}{}{}{}{6}{4}};
( -6, -5)*{\afvj2};
( -4, -6)*{\cdots};
( 14, -5)*{\afvj2};
(  4,-10)*{\afaiuex{}{}{}{}{}{}{20}{4}};
(-12, -8)*{\afvjum8{}{\boldsymbol\iota}};
}
\quad\to_{\frmis_n}\quad
\atomicflow{
(-12, 56)*{\afvjm4};
(-12, 50)*{\copy\contrup};
(-12, 50)*{\affr{16}{8}};
( -5, 44)*{\afvjdm4{}{f_0(\boldsymbol\epsilon)}};
( -7, 34)*{\afcjlm{4}{24}};
(-12, 42)*{\cdots};
(-10,  2)*{\afcjlm{10}{32}};
(-15, 32)*{\afvjm{28}};
(-12,-13)*{\afcjlm{14}{42}};
(-19, 27)*{\afvjm{38}};
%--
( 5, 46)*{\afawdm{}{}{}{}};
( 0, 38)*{\affr{12}{8}};
( 1, 40)*{\aflabelright{f_0(\phi)}};
( 5, 33)*{\afvjm2};
( 9, 28)*{\affr{10}{8}};
(11, 30)*{\aflabelright{\eta_0}};
( 5, 23)*{\afvjm2};
(11, 23)*{\afvjm2};
( 0, 18)*{\affr{12}{8}};
( 1, 20)*{\aflabelright{f_1(\phi)}};
( 5, 13)*{\afvjm2};
(11, 13)*{\afvjm2};
(11, 18)*{\affr{6}{8}};
(11, 20)*{\aflabelright{\mu_1}};
( 9,  8)*{\affr{10}{8}};
(11, 10)*{\aflabelright{\eta_1}};
( 5,  3)*{\afvjm2};
(11,  3)*{\afvjm2};
%-
( 0,  1)*{\vdots};
%-
( 5, -3)*{\afvjm2};
(11, -3)*{\afvjm2};
( 9, -8)*{\affr{10}{8}};
( 9, -6)*{\aflabelright{\eta_{n-1}}};
( 5,-13)*{\afvjm2};
(11,-13)*{\afvjm2};
( 0,-18)*{\affr{12}{8}};
( 1,-16)*{\aflabelright{f_n(\phi)}};
(11,-18)*{\affr{6}{8}};
(11,-16)*{\aflabelright{\mu_n}};
( 5,-23)*{\afvjm2};
(11,-23)*{\afvjm2};
( 9,-28)*{\affr{10}{8}};
(11,-26)*{\aflabelright{\eta_n}};
( 5,-33)*{\afvjm2};
( 0,-38)*{\affr{12}{8}};
(-1,-36)*{\aflabelright{f_{n+1}(\phi)}};
( 5,-46)*{\afawum{}{}{}{}};
%--
(-12, 13)*{\afcjrm{14}{42}};
(-19,-27)*{\afvjm{38}};
(-10, -2)*{\afcjrm{10}{32}};
(-15,-32)*{\afvjm{28}};
(-12,-42)*{\cdots};
( -7,-34)*{\afcjrm{4}{24}};
( -5,-44)*{\afvjum4{}{f_{n+1}(\boldsymbol\iota)}};
(-12,-50)*{\affr{16}{8}};
(-12,-50)*{\copy\contrdown};
(-12,-56)*{\afvjm4};
}\quad.
\]
\end{definition}
%---------------

%------------------------------------------------------------------
\begin{theorem}\label{theorem:SoundMultipleIsolatedSubflowsRemoval}
For every $n>0$, reduction\/ $\to_{\frmis_n}$ is sound; moreover, if\/ $\Phi\to_{\frmis_n}\Psi$, then the size of $\Psi$ depends at most cubically on the size of $\Phi$ and at most quadratically on $\max\{\size{\Gammasf_1},\dots,\size{\Gammasf_N}\}$.
\end{theorem}

\begin{proof}
Let $\Phi$ be a derivation with flow $\phi'$, such that $\phi'\to_{\frmis_n}\psi'$. We show that there exists a derivation $\Psi$ with flow $\psi'$ and with the same premiss and conclusion as $\Phi$. In the following, we refer to the figures in Definition~\vref{definition:MultipleIsolatedSubflowsRemoval}.

Since each of $\psi_1$, $\dots$, $\psi_n$ is connected, we assume, by Remark~\vref{remark:AlternativeAiDecomposedForm}, that the following derivation is an $\ai$-decomposed form of $\Phi$:
\[
\vlder{\Phi'}{}
{
 \vls[\beta\;.\;\vlinf{}{}{\fff}{\vls(a_1.\bar a_1^{\psi_1})}\;.\;\vlinf{}{}{\fff}{\vls(a_n.\bar a_n^{\psi_n})}]
}
{
 \vls(\vlinf{}{}{\vls[a_1.\bar a_1^{\psi_1}]}{\ttt}\;.\;\vlinf{}{}{\vls[a_n.\bar a_n^{\psi_n}]}{\ttt}\;.\;\alpha)
}\quad,
\]
for some atoms $a_1$, $\dots$, $a_n$ (that, without loss of generality, we assume coincide with the atoms given in Definition~\ref{definition:MultipleIsolatedSubflowsRemoval}) and formulae $\alpha$ and $\beta$.

For every $0\le k\le N$, we obtain the derivation $\Phi_k$ from $\Phi'$ as follows:
\[
\Phi_k\quad=\quad
\vlder{\Phi'\{\bar a_1^{\psi_1}/\gamma_{k,1},\dots,\bar a_n^{\psi_n}/\gamma_{k,n}\}}{}
{
 \vlsbr[\beta.(a_1.\gamma_{k,1}).\cdots.(a_n.\gamma_{k,n})]
}
{
 \vls([a_1.\gamma_{k,1}].\cdots.[a_n.\gamma_{k,n}].\alpha)
}
\]
Since each of $\psi_1$, $\dots$, $\psi_n$ is connected and contains no interaction or cut vertices, the mapping from occurrences of $\bar a_i^{\psi_i}$ to edges of $\psi_i$ is surjective. Hence, we know that $\Phi_k$ has atomic flow
\[
\atomicflow{
(-5, 6)*{\afvjm4};
( 1, 6)*{\afvj4};
( 3, 6)*{\cdots};
( 5, 6)*{\afvj4};
(11, 6)*{\afvjm4};
( 0, 0)*{\affr{12}{8}};
( 3, 2)*{\aflabelright{\phi}};
(11, 0)*{\affr{6}{8}};
(11, 2)*{\aflabelright{\mu_k}};
(-5,-6)*{\afvjm4};
( 1,-6)*{\afvj4};
( 3,-6)*{\cdots};
( 5,-6)*{\afvj4};
(11,-6)*{\afvjm4};
}\quad.
\]
We combine $\Phi_0$, $\dots$, $\Phi_N$, $\Gammasf_1$, $\dots$, $\Gammasf_N$ to get the desired derivation $\Psi$ with atomic flow $\psi'$ and the same premiss and conclusion as $\Phi$:
\[
\vlderivation
{
 \vlin{\cod}{}
 {
  \beta
 }
 {
  \vlin{\swi}{}
  {
   \vls
   [
    \vlinf{\cod}{}{\beta}{\vls[\beta.\beta]}
   \;\;\;.\;\;\;
    \vlder{\Phi_N}{}
    {
     \vlsbr[\beta\;.\;(\vlinf{}{}{\ttt}{a_1}\;.\;\fff)\;.\;\cdots\;.\;(\vlinf{}{}{\ttt}{a_n}\;.\;\fff)]
    }
    {
     \vls([a_1.\gamma_{N,1}].\cdots.[a_n.\gamma_{N,n}].\alpha)
    }
   ]
  }
  {
   \vlin{\swi}{}
   {
    \vls
    (
     [
      \vlinf{\cod}{}{\beta}{\vls[\beta.\beta]}
     \;\;\;\;.\;\;\;\;
      \vlder{\Phi_{N-1}}{}
      {
       \vlsbr
       [
        \beta
       \;\;.\;\;
        \vlder{\Gammasf_N}{}
        {
         \vls([a_1.\gamma_{N,1}].\cdots.[a_n.\gamma_{N,n}])
        }
        {
         \vls[(a_1.\gamma_{N-1,1}).\cdots.(a_n.\gamma_{N-1,n})]
        }
       ]
      }
      {
       \vls([(a_1.\gamma_{N-1,1}).\cdots.(a_n.\gamma_{N-1,n})].\alpha)
      }
     ]
    \;\;\;\;.\;\;\;\;
     \alpha
    )
   }
   {
    \vlin{\swi}{}
    {
     \vdots
    }
    {
     \vlin{\swi}{}
     {
      \vls
      (
       [
        \beta
       \;\;\;\;.\;\;\;\;
        \vlder{\Phi_1}{}
        {
         \vlsbr
         [
          \beta
         \;\;.\;\;
          \vlder{\Gammasf_2}{}
          {
           \vls([a_1.\gamma_{2,1}].\cdots.[a_n.\gamma_{2,n}])
          }
          {
           \vls[(a_1.\gamma_{1,1}).\cdots.(a_n.\gamma_{1,n})]
          }
         ]
        }
        {
         \vls((a_1.\gamma_{0,1}).\cdots.(a_n.\gamma_{0,n}).\alpha)
        }
       ]
      \;\;\;\;.\;\;\;\;
       \vlinf{\cou}{}{\vls(\alpha.\alpha)}{\alpha}
      )
     }
     {
      \vlin{\cou}{}
      {
       \vls
       (
        \vlder{\Phi_0}{}
        {
         \vlsbr
         [
          \beta
         \;\;.\;\;
          \vlder{\Gammasf_1}{}
          {
           \vls([a_1.\gamma_{1,1}].\cdots.[a_n.\gamma_{1,n}])
          }
          {
           \vls[(a_1.\gamma_{0,1}).\cdots.(a_n.\gamma_{0,n})]
          }
         ]
        }
        {
         \vlsbr([\vlinf{}{}{a_1}{\fff}\;.\;\ttt]\;.\;\cdots\;.\;[\vlinf{}{}{a_n}{\fff}\;.\;\ttt]\;.\;\alpha)
        }
       \;\;\;\;.\;\;\;\;
        \vlinf{\cou}{}{\vls(\alpha.\alpha)}{\alpha}
       )
      }
      {
       \vlhy{\alpha}
      }
     }
    }
   }
  }
 }
}
\qquad.
\]
Since $\max\{\size{\gamma_{0,1}},\dots,\size{\gamma_{N,n}}\}$ is less than or equal to $\max\{\size{\Gammasf_1},\dots,\size{\Gammasf_N}\}$, we know that the size of $\Phi_0$, $\dots$, $\Phi_N$ depend at most cubically on the size of $\Phi$ and at most quadratically on the size of $\max\{\size{\Gammasf_1},\dots,\size{\Gammasf_N}\}$ by Theorem~\vref{theorem:aiDecomposedForm} and Proposition~\vref{proposition:DerivationSubstitution}, and that the size of $\Psi$ depends at most cubically on the size of $\alpha$ and $\beta$ by Lemma~\vref{lemma:GenericContraction}, so $\Psi$ depends at most cubically on the size of $\Phi$ and at most quadratically on the size of $\max\{\size{\Gammasf_1},\dots,\size{\Gammasf_N}\}$.
\end{proof}
%----------

%-------------------------------------
\begin{lemma}\label{lemma:MultipleIsolatedSubflowsRemovalPaths}
Given two atomic flows $\phi$ and $\psi$, such that $\phi\to_\frmis\psi$; let $f_0$, $\dots$, $f_n$ be the isomorphisms and let $\nu_{\ai,1}$, $\dots$, $\nu_{\ai,n}$ be the evidenced cut (resp., interaction) vertices described in Definition~\ref{definition:MultipleIsolatedSubflowsRemoval}; then, given an interaction (resp., cut) vertex $\nu$ in $\psi$, there is an interaction (resp., cut) vertex $\nu'$ in $\phi$, such that
\begin{itemize}
\item for some $0\le i\le n$, $\nu=f_i(\nu')$;
\item if there is a path from $\nu$ to $\bot$ (resp., $\top$) in $\psi$, then there is a path from $\nu'$ to $\bot$ (resp., $\top$) in $\phi$; and
\item if there is a cut (resp., interaction) vertex $\hat\nu$ in $\psi$, such that there is a path from $\nu$ to $\hat\nu$ in $\psi$, then there is a cut (resp., interaction) vertex $\hat\nu'$ in $\phi$, such that, for some $0\le i\le n$, $\hat\nu=f_i(\nu')$, or, for some $1\le i\le n$, $\hat\nu'=\nu_{\ai,i}$; and there is a path from $\nu'$ to $\hat\nu'$ in $\phi$.
\end{itemize}
\end{lemma}

\begin{proof}
In the following we refer to the figure in Definition~\ref{definition:MultipleIsolatedSubflowsRemoval}:
\begin{itemize}
 \item by definition;
 \item any path from $\nu$ to $\top$ (resp., $\bot$) in $\psi$ must contain an edge $\epsilon$, such that, for some upper (resp., lower) edge $\epsilon'$ of $\phi$ and some $0\le i\le n$, $f_i(\epsilon')=\epsilon$. Hence, there is a path from $\nu'$ to $\top$ (resp., $\bot$) in $\phi$; and
 \item we have to consider two cases:
 \begin{itemize}
  \item for some $0\le i\le n$, $\nu=f_i(\nu')$ and $\hat\nu=f_i(\hat\nu')$, then there is a path from $\nu'$ to $\hat\nu'$ in $\phi$; or
  \item for some $0\le i<j\le n$, $\nu=f_i(\nu')$ and $\hat\nu=f_j(\hat\nu')$ (resp., $\nu=f_j(\nu')$ and $\hat\nu=f_i(\hat\nu')$),then, for some $1\le i\le n$, there is a path from $\nu'$ to $\nu_{\ai,i}$ in $\phi$.
 \end{itemize}
\end{itemize}
\end{proof}
%----------

%--------------------------------
\newcommand{\MISR}{\mathsf{MISR}}
\begin{definition}\label{definition:MultipleIsolatedSubflowsRemover}
For every $n>0$, given the atoms, formulae and derivations described in Definition~\vref{definition:MultipleIsolatedSubflowsRemoval}, the \emph{Multiple Isolated Subflow Remover}, $\MISR_n$, is an operator whose arguments are atoms $a_1$, $\dots$, $a_n$ (that, without loss of generality, we assume coincide with the atoms given in Definition~\ref{definition:MultipleIsolatedSubflowsRemoval}), and a derivation $\Phi$ that is in simple form, but is not weakly streamlined, with respect to $a_1$, $\dots$, $a_n$, with $\ai$-decomposed form
\[
\vlder{\Phi'}{}
{
 \vlsbr
 [
  \beta
 \;.\;
  \vlinf{}{}
  {
   \fff
  }
  {
   \vls(a_n^{\psi_n}.\bar a_n)
  }
 \;.\;\cdots\;.\;
  \vlinf{}{}
  {
   \fff
  }
  {
   \vls(a_n^{\psi_n}.\bar a_n)
  }
 \;.\;\cdots\;.\;
  \vlinf{}{}
  {
   \fff
  }
  {
   \vls(a_1^{\psi_1}.\bar a_1)
  }
 \;.\;\cdots\;.\;
  \vlinf{}{}
  {
   \fff
  }
  {
   \vls(a_1^{\psi_1}.\bar a_1)
  }
 ]
}
{
 \vlsbr
 (
  \vlinf{}{}
  {
   \vls[a_1^{\psi_1}.\bar a_1]
  }
  {
   \ttt
  }
 \;.\;\cdots\;.\;
  \vlinf{}{}
  {
   \vls[a_1^{\psi_1}.\bar a_1]
  }
  {
   \ttt
  }
 \;.\;
  \vlinf{}{}
  {
   \vls[a_n^{\psi_n}.\bar a_n]
  }
  {
   \ttt
  }
 \;.\;\cdots\;.\;
  \vlinf{}{}
  {
   \vls[a_n^{\psi_n}.\bar a_n]
  }
  {
   \ttt
  }
 \;.\;
  \alpha
 )
}\quad,
\]
and atomic flow
\[
\phi''\;=\;
\atomicflow{
(-12,  8)*{\afvjdm8{}{\boldsymbol\epsilon}};
(  4, 10)*{\afaidmex{}{}{}{}{}{}{20}{4}};
( -6,  5)*{\afvjm2};
( -4,  6)*{\cdots};
( 14,  5)*{\afvjm2};
(  1,  8)*{\afaidmex{}{}{}{}{}{}{6}{4}};
( -7,  0)*{\affr{12}{8}};
( -4,  2)*{\aflabelright{\phi'}};
(  4,  0)*{\affr{6}{8}};
(  4,  2)*{\aflabelright{\psi'_1}};
(  9,  0)*{\cdots};
( 14,  0)*{\affr{6}{8}};
( 14,  2)*{\aflabelright{\psi'_n}};
(  1, -8)*{\afaiumex{}{}{}{}{}{}{6}{4}};
( -6, -5)*{\afvjm2};
( -4, -6)*{\cdots};
( 14, -5)*{\afvjm2};
(  4,-10)*{\afaiumex{}{}{}{}{}{}{20}{4}};
(-12, -8)*{\afvjum8{}{\boldsymbol\iota}};
}
\quad,
\]
where, for $1\le i\le n$, $\psi_i$ is the juxtaposition of all the isolated subflows mapped to from occurrences of $a_i$ in $\Phi$. Consider the derivation
\[
\Psi\;=\;
\vlder{\Phi'}{}
{
 \vlsbr
 [
  \beta
 \;\;\;.\;\;\;
  \vlderivation
  {
   \vlin{}{}
   {
    \fff
   }
   {
    \vlde{}{\{\cod\}}
    {
     \vls(a_n.\bar a_n)
    }
    {
     \vlhy
     {
      \vls[(a_n.\bar a_n).\cdots.(a_n.\bar a_n)]
     }
    }
   }
  }
 \;\;\;.\;\;\;\cdots\;\;\;.\;\;\;
  \vlderivation
  {
   \vlin{}{}
   {
    \fff
   }
   {
    \vlde{}{\{\cod\}}
    {
     \vls(a_1.\bar a_1)
    }
    {
     \vlhy
     {
      \vls[(a_1.\bar a_1).\cdots.(a_1.\bar a_1)]
     }
    }
   }
  }
 ]
}
{
 \vlsbr
 (
  \vlderivation
  {
   \vlde{}{\{\cou\}}
   {
    \vls([a_1.\bar a_1].\cdots.[a_1.\bar a_1])
   }
   {
    \vlin{}{}
    {
     \vls[a_1.\bar a_1]
    }
    {
     \vlhy
     {
      \ttt
     }
    }
   }
  }
 \;\;\;.\;\;\;\cdots\;\;\;.\;\;\;
  \vlderivation
  {
   \vlde{}{\{\cou\}}
   {
    \vls([a_n.\bar a_n].\cdots.[a_n.\bar a_n])
   }
   {
    \vlin{}{}
    {
     \vls[a_n.\bar a_n]
    }
    {
     \vlhy
     {
      \ttt
     }
    }
   }
  }
 \;\;\;.\;\;\;
  \alpha
 )
}\quad,
\]
with atomic flow
\[
\psi''\;=\;
\atomicflow{
(-16, 12)*{\afvjdm{16}{\boldsymbol\epsilon}{}};
(  7, 20)*{\afaidex{}{}{}{}{}{}{34}{4}};
(  4, 18)*{\afaidex{}{}{}{}{}{}{12}{4}};
%-
(-10, 15)*{\afvj2};
(-10, 10)*{\affr68};
(-10, 10)*{\copy\contrup};
(-10,  5)*{\afvjm2};
( -6,  5)*{\cdots};
( -2, 10)*{\affr68};
( -2, 10)*{\copy\contrup};
( -2,  5)*{\afvjm2};
( -9,  0)*{\affr{16}{8}};
( -1,  2)*{\aflabelleft{\phi'}};
(-10, -5)*{\afvjm2};
(-10,-10)*{\affr68};
(-10,-10)*{\copy\contrdown};
(-10,-15)*{\afvj2};
( -6, -5)*{\cdots};
( -2, -5)*{\afvjm2};
( -2,-10)*{\affr68};
( -2,-10)*{\copy\contrdown};
%-
(10, 10)*{\affr68};
(10, 10)*{\copy\contrup};
(10,  5)*{\afvjm2};
(24, 15)*{\afvj2};
(24, 10)*{\affr68};
(24, 10)*{\copy\contrup};
(24,  5)*{\afvjm2};
(10,  0)*{\affr{6}{8}};
(11,  2)*{\aflabelleft{\psi'_1}};
(18,  0)*{\cdots};
(24,  0)*{\affr{6}{8}};
(27,  2)*{\aflabelleft{\psi'_n}};
(10, -5)*{\afvjm2};
(10,-10)*{\affr68};
(10,-10)*{\copy\contrdown};
(24, -5)*{\afvjm2};
(24,-10)*{\affr68};
(24,-10)*{\copy\contrdown};
(24,-15)*{\afvj2};
%-
(-16,-12)*{\afvjum{16}{\boldsymbol\iota}{}};
(  4,-18)*{\afaiuex{}{}{}{}{}{}{12}{4}};
(  7,-20)*{\afaiuex{}{}{}{}{}{}{34}{4}};
%----
(-7,  0)*{\affr{22}{30}};
( 4, 13)*{\aflabelleft{\phi}};
(11,  0)*{\affr{10}{30}};
(16, 13)*{\aflabelleft{\psi_1}};
(25,  0)*{\affr{10}{30}};
(30, 13)*{\aflabelleft{\psi_n}};
}\quad.
\]
We then define $\MISR_n(\Phi,a_1,\dots,a_n)$ to be such that $\Psi\to_\fris\MISR(\Phi,a_1,\dots,a_n)$, where $\phi$, $\psi_1$, $\dots$, $\psi_n$ are the flows, by the same names, shown in Definition~\vref{definition:MultipleIsolatedSubflowsRemoval}.
\end{definition}
%---------------

%--------------------------------------------------------------------
\begin{proposition}\label{proposition:MultipleIsolatedSubflowRemover}
Given the atoms, formulae and derivations described in Definition~\vref{definition:MultipleIsolatedSubflowsRemoval}, and atoms $a_1$, $\dots$, $a_n$ and a derivation $\Phi$ that is in simple form, but is not weakly streamlined, with respect to $a_1$, $\dots$, $a_n$,
\begin{enumerate}
\item $\MISR_n(\Phi,a_1,\dots,a_n)$ is weakly streamlined with respect to $a_1$, $\dots$, $a_n$;
\item for any atom $b$,
\begin{itemize}
\item if $\Phi$ is weakly streamlined with respect to $b$, then $\MISR_n(\Phi,a_1,\dots,a_n)$ is weakly streamlined with respect to $b$, and
\item if $b$ is not the dual of any of $a_1$, $\dots$, $a_n$ and $\Phi$ is on simple form with respect to $b$, then $\MISR_n(\Phi,a_1,\dots,a_n)$ is on simple form with respect to $b$; and
\end{itemize}
\item the size of\/ $\MISR_n(\Phi,a_1,\dots,a_n)$ depends at most cubically on the size of\/ $\Phi$, and at most quadratically on $\max\{\size{\Gammasf_1},\dots,\size{\Gammasf_N}\}$.
\end{enumerate}
\end{proposition}

\TODO{Prove:}

\begin{proof}
The statements follow by
\begin{enumerate}
\item Definition~\vref{definition:DerStreamlined} and Lemma~\vref{lemma:MultipleIsolatedSubflowsRemovalPaths};
\item Definitions~\ref{definition:DerStreamlined} and \ref{definition:DerSimpleForm}, and Lemma~\ref{lemma:MultipleIsolatedSubflowsRemovalPaths};
\item the statement follows by Theorem~\vref{theorem:SoundMultipleIsolatedSubflowsRemoval}.
\end{enumerate}
\end{proof}
%----------

%--------------------------------------------------
\begin{remark}\label{remark:FromGammasToThresholds}
Given the atoms, formulae and derivations described in Definition~\ref{definition:MultipleIsolatedSubflowsRemoval}, it follows from an argument by induction on $k$, that, for every $1\le i\le n$ and every $0\le k\le N$, the formula $\gamma_{k,i}$ is
\begin{itemize}
 \item true if at least $k$ of the atoms $a_1$, $\dots$, $a_{i-1}$, $a_{i+1}$, $\dots$, $a_n$ are true; and
 \item false if at least $N-k$ of the atoms $a_1$, $\dots$, $a_{i-1}$, $a_{i+1}$, $\dots$, $a_n$ are false.
\end{itemize}
It follows by contradiction that $N\ge n$. Furthermore, if $N=n$, we know that $\gamma_{k,i}$ is true if and only if at least $k$ of the atoms $a_1$, $\dots$, $a_{i-1}$, $a_{i+1}$, $\dots$, $a_n$ are true. This would make $\gamma_{k,i}$ a \emph{threshold formula}, as we will se in the next section.
\end{remark}
%-----------

%=============================


\subsubsection{Threshold Formulae}\label{subsubsection:ThresholdFormulae}

\TODO{Change this introduction}

We present here the main construction of this paper, \emph{i.e.}, a class of derivations $\Gammasf$ that only depend on a given set of atoms and that allow us to normalise any proof containing those atoms. The complexity of the $\Gammasf$ derivations dominates the complexity of the normal proof, and is due to the complexity of certain `threshold formulae', on which the $\Gammasf$ derivations are based. The $\Gammasf$ derivations are constructed in Definition~\vref{definition:ThresholdDerivations}; this directly leads to Theorem~\vref{theorem:ThresholdDerivations}, which states a crucial property of the $\Gammasf$ derivations and which is the main result of this section.

\TODO{Define $\lfloor x\rfloor$.}

\TODO{Define $a^\phi$.}

%In the following, $\lfloor x\rfloor$ denotes the maximum integer $n$ such that $n\le x$.

There are several ways of encoding threshold functions into formulae, and the problem is to find, among them, an encoding that allows us to obtain Theorem~\vref{theorem:ThresholdDerivations}. Efficiently obtaining the property stated in Theorem~\vref{theorem:ThresholdDerivations} crucially depends also on the proof system we adopt.

\TODO{Lemma on substituting into isolated subflows}

Threshold formulae realise boolean threshold functions, which are defined as boolean functions that are true if and only if at least $k$ of $n$ inputs are true (see \cite{Wege:87:The-Comp:vn} for a thorough reference on threshold functions). 

The following class of threshold formulae, which we found to work for system $\SKS$, is a simplification of the one adopted in \cite{AtseGalePudl:02:Monotone:yu}.

\renewcommand{\th}[2]{\mathop{\thetaup_{#1}^{#2}}}
%-------------------------------------------------------------------------------
\begin{definition}\label{definition:ThresholdFormulae}
Consider $n>0$, distinct atoms $a_1$, \dots, $a_n$, and let $p=\lfloor n/2\rfloor$ and $q=n-p$; for $k\ge0$, we define the \emph{threshold formulae\/} $\th kn\avec1n$ as follows:
\begin{itemize}
%---------------------------------------
\item for any $n>0$ let $\th0n\avec1n\equiv\ttt$;
%---------------------------------------
\item for any $n>0$ and $k>n$ let $\th kn\avec1n\equiv\fff$;
%---------------------------------------
\item $\th11(a_1)\equiv a_1$;
%---------------------------------------
\item for any $n>1$ and $0<k\le n$ let
$\th kn\avec1n\equiv\bigvee_{\begin{subarray}{l}i+j=k      \\ 
                                                0\le i\le p\\ 
                                                0\le j\le q
                             \end{subarray}}
\vlsbr(\th ip\avec1p.\th jq\avec{p+1}n)$.
%---------------------------------------
\end{itemize}
\end{definition}

See, in Figure~\vref{figure:ThresholdFormulae}, some examples of threshold formulae.

\TODO{Check if this still holds, if it does, find a new explanation:}

The only reason why we require atoms to be distinct in threshold formulae is to avoid certain technical problems with substitutions in the definition of cut-free form, later on. However, there is no substantial difficulty in relaxing this definition to any set of atoms.

%-------------------------------------------------------------------------------
\begin{figure}
\vlsmallbrackets
\begin{eqnarray*}
%---------------------------------------
\th02(a,b)&\equiv&\ttt
\quad,\\
\noalign{\medskip}
%---------------------------------------
\th12(a,b)&\equiv&\vls[({\vlnos\th11(a)}.{\vlnos\th01(b)}).
                       ({\vlnos\th01(a)}.{\vlnos\th11(b)})]
           \equiv     [(a.\ttt).(\ttt.b)]\\
          &=     &\vls [a      .      b ]
\quad,\\
\noalign{\medskip}
%---------------------------------------
\th22(a,b)&\equiv&\vls({\vlnos\th11(a)}.{\vlnos\th11(b)})\\
          &\equiv&\vls(a.b)
\quad,\\
\noalign{\medskip}
%---------------------------------------
\th03(a,b,c)&\equiv&\ttt
\quad,\\
\noalign{\medskip}
%---------------------------------------
\th13(a,b,c)&\equiv&\vls[({\vlnos\th11(a)}.{\vlnos\th02(b,c)}).
                         ({\vlnos\th01(a)}.{\vlnos\th12(b,c)})]
             \equiv     [(a.\ttt).(\ttt.[(b.\ttt).(\ttt.c)])]\\
            &=     &\vls[a.b.c]
\quad,\\
\noalign{\medskip}
%---------------------------------------
\th23(a,b,c)&\equiv&\vls[({\vlnos\th11(a)}.{\vlnos\th12(b,c)}).
                    ({\vlnos\th01(a)}.{\vlnos\th22(b,c)})]\\
            &=     &\vls[(a.[b.c]).(b.c)]
\quad,\\
\noalign{\medskip}
%---------------------------------------
\th33(a,b,c)&\equiv&\vls({\vlnos\th11(a)}.{\vlnos\th22(b,c)})
             \equiv     [(a.(b.c))]\\
            &=     &\vls(a.b.c)
\quad,\\
\noalign{\medskip}
%---------------------------------------
\th05(a,b,c,d,e)&\equiv&\ttt
\quad,\\
\noalign{\medskip}
%---------------------------------------
\th15(a,b,c,d,e)&\equiv&\vls[({\vlnos\th12(a,b)}.{\vlnos\th03(c,d,e)}).
                             ({\vlnos\th02(a,b)}.{\vlnos\th13(c,d,e)})]\\
                &=     &\vls[a.b.c.d.e]
\quad,\\
\noalign{\medskip}
%---------------------------------------
\th25(a,b,c,d,e)&\equiv&\vls[({\vlnos\th22(a,b)}.{\vlnos\th03(c,d,e)}).
                             ({\vlnos\th12(a,b)}.{\vlnos\th13(c,d,e)}).
                             ({\vlnos\th02(a,b)}.{\vlnos\th23(c,d,e)})]\\
                &=     &\vls[(a.b                                    ).
                             ([a.b]             .[c.d.e]             ).
                                                 (c.[d.e]).(d.e)      ]
\quad,\\
\noalign{\medskip}
%---------------------------------------
\th35(a,b,c,d,e)&\equiv&\vls[({\vlnos\th22(a,b)}.{\vlnos\th13(c,d,e)}).
                             ({\vlnos\th12(a,b)}.{\vlnos\th23(c,d,e)}).
                             ({\vlnos\th02(a,b)}.{\vlnos\th33(c,d,e)})]\\
                &=     &\vls[(a.b               .[c.d.e]             ).
                             ([a.b]             .[(c.[d.e]).(d.e)]   ).
                                                 (c.d.e)              ]
\quad,\\
\noalign{\medskip}
%---------------------------------------
\th45(a,b,c,d,e)&\equiv&\vls[({\vlnos\th22(a,b)}.{\vlnos\th23(c,d,e)}).
                             ({\vlnos\th12(a,b)}.{\vlnos\th33(c,d,e)})]\\
                &=     &\vls[(a.b               .[(c.[d.e]).(d.e)]   ).
                             ([a.b]             .c.d.e               )]
\quad,\\
\noalign{\medskip}
%---------------------------------------
\th55(a,b,c,d,e)&\equiv&\vls({\vlnos\th22(a,b)}.{\vlnos\th33(c,d,e)})\\
                &=     &\vls(a.b.c.d.e)
\quad,\\
\noalign{\medskip}
%---------------------------------------
\th65(a,b,c,d,e)&\equiv&\fff
\quad.
\end{eqnarray*}
\caption{Examples of threshold formulae.}
\label{figure:ThresholdFormulae}
\end{figure}

The formulae for threshold functions adopted in \cite{AtseGalePudl:02:Monotone:yu} correspond, for each choice of $k$ and $n$, to $\bigvee_{i\ge k}\th in\avec1n$. We presume that \cite{AtseGalePudl:02:Monotone:yu} employs these more complicated formulae because the formalism adopted there, the sequent calculus, is less flexible than deep inference, requiring more information in threshold formulae in order to construct suitable derivations.

\TODO{Remove?}

%-------------------------------------------------------------------------------
\begin{remark}
For $n>0$, we have $\th1n\avec1n=\vls[a_1.\vldots.a_n]$ and $\th nn\avec1n=\vls(a_1.\vldots.a_n)$.
\end{remark}


\TODO{Read again:}
%-------------------------------------------------------------------------------
\begin{remark}\label{remark:ThersholdSubstitution}
Given a threshold formula $\th kn\avec1n$ and an atom $a_i$, both $\th kn\avec1n\{a_i/\fff\}$ and $\th kn\avec 1n\{a_i/\ttt\}$ are threshold formula as well. In particular, $\th kn\avec1n\{a_i/\fff\}$ (resp., $\th kn\avec 1n\{a_i/\ttt\}$) is true if and only if at least $k$ (resp., $k-1$) of the atoms $a_1$, $\dots$, $a_{i-1}$, $a_{i+1}$, $\dots$, $a_n$ are true.
\end{remark}

The size of the threshold formulae dominates the cost of the normalisation procedure, so, we evaluate their size. We leave as an exercise the proof of the following proposition.

%-------------------------------------------------------------------------------
\begin{proposition}\label{proposition:LargestThresholdFormula}
For any $n>0$ and $k\ge0$, $\size{\th kn\avec1n}\le\size{\th{\lfloor n/2\rfloor+1}n\avec1n}$.
\end{proposition}

%-------------------------------------------------------------------------------
\begin{lemma}\label{lemma:SizeThresholdMax}
The size of\/ $\th{\lfloor n/2\rfloor+1}n\avec1n$ is $n^{\Ord{\log n}}$.
\end{lemma}

%-------------------------------------------------------------------------------
\begin{proof}
Observe that $\size{\th kn\avec1n}\le\size{\th k{n+1}\avec1{n+1}}$. Let $p=\lfloor n/2\rfloor$ and $q=n-p$ and consider:
\begin{equation}\label{PropQuasIneq}
\begin{split}
\size{\th{p+1}n\avec1n}
&=\textstyle\sum_{\begin{subarray}{l}i+j=p+1    \\
                                     0\le i\le p\\
                                     0\le j\le q
                  \end{subarray}}
  \left(\size{\th ip\avec1p}+
        \size{\th jq\avec{p+1}n}\right)             \\
&\le\textstyle\sum_{\begin{subarray}{l}i+j=p+1\\
                                       0\le i,j\le q
                    \end{subarray}}
  \left(\size{\th iq\avec1q}+
        \size{\th jq\avec1q}\right)                 \\
&\le2(q+1)
  \size{\th{\lfloor q/2\rfloor+1}q\avec1q}\quad,
\end{split}
\end{equation}
where we use Proposition~\vref{proposition:LargestThresholdFormula}. We show that, for $h=2/(\log3-\log2)$ and for any $n>0$, we have $\size{\th{\lfloor n/2\rfloor+1}n\avec1n}\le n^{h\log n}$. We reason by induction on $n$; the case $n=1$ trivially holds. By the inequality~\eqref{PropQuasIneq}, and for $n>1$, we have
\begin{equation*}
\begin{split}
\size{\th{\lfloor n/2\rfloor+1}n\avec1n}
&\le2(n-\lfloor n/2\rfloor+1)
     (n-\lfloor n/2\rfloor)^{h\log(n-\lfloor n/2\rfloor)}       \\
&\le n^2n^{h\log(2n/3)}=n^{h\log n-h(\log3-\log2)+2}=n^{h\log n}
\quad.
\end{split}
\end{equation*}
\end{proof}

%-------------------------------------------------------------------------------
\begin{theorem}\label{theorem:SizeThreshold}
For any $k\ge0$ the size of\/ $\th kn\avec1n$ is $n^{\Ord{\log n}}$.
\end{theorem}

%-------------------------------------------------------------------------------
\begin{proof}
It immediately follows from Proposition~\vref{proposition:LargestThresholdFormula} and Lemma~\vref{lemma:SizeThresholdMax}.
\end{proof}

\subsubsection{Glue derivations}

\TODO{Rename or remove subsubsectoin}

\TODO{Define generic weakening.}

%-------------------------------------------------------------------------------
\begin{remark}\label{remark:UpsideDownCoweakening}
Given $n>1$, let $p=\lfloor n/2\rfloor$ and $q=n-p$. For $0\le k\le q$ and $1\le l\le p$, the following derivation is well defined:
\[
\vlinf{\gwu}
      {}
      {\fff}
      {\vls({\vlnos(\th pp\avec1p)}\{a_l/\fff\}.\th kq\avec{p+1}n)}
=
\vls(
\vlinf{\gwu}
      {}
      {\vls(\ttt)}
      {\vls(a_1.\cdots.a_{l-1}.a_{l+1}.\cdots.a_p.\th kq\avec{p+1}n)}
.\fff)
\quad.
\]
Analogously, for $0\le k\le p$ and $p+1\le l\le n$, we can define the following derivation:
\[
\vlinf{\gwu}
      {}
      {\fff}
      {\vls(\th kp\avec1p.{\vlnos(\th qq\avec{p+1}n)}\{a_l/\fff\})}
=
\vls(
\vlinf{\gwu}
      {}
      {\vls(\ttt)}
      {\vls(\th kp\avec1p.a_{p+1}.\cdots.a_{l-1}.a_{l+1}.\cdots.a_n)}
.\fff)
\quad.
\]
Both classes of derivations are used in Definition~\vref{definition:AuxillaryThresholdDerivation}.
\end{remark}

\newcommand{\Uth}[3]{\mathop{\mathsf\Upsilon_{#1,#2}^{#3}}}
\newcommand{\Dth}[3]{\mathop{\mathsf\Delta_{#1,#2}^{#3}}}
\newcommand{\Gth}[3]{\mathop{\Gammasf_{#1,#2}^{#3}}}
%-------------------------------------------------------------------------------
\begin{definition}\label{definition:AuxillaryThresholdDerivation}
Consider $n>0$, distinct atoms $a_1$, \dots, $a_n$, and let $p=\lfloor n/2\rfloor$ and $q=n-p$.
\begin{itemize}
%---------------------------------------
%---------------------------------------
\item
For $n>1$ and $1\le l\le n$, we define the derivations $\Uth kln\avec1n$ and $\Dth kln\avec1n$ as follows:
\[
\Uth kln\avec1n=\begin{cases}
\vlinf{\gwu}
      {}
      {\fff}
      {\vls({\vlnos(\th pp\avec1p)}\{a_l/\fff\}.\th{k-p}q\avec{p+1}n)}
             &\text{if $p\le k\le n$ and $l\le p$}\\
\noalign{\medskip}
\vlinf{\gwu}
      {}
      {\fff}
      {\vls(\th{k-q}p\avec1p.{\vlnos(\th qq\avec{p+1}n)}\{a_l/\fff\})}
             &\text{if $q\le k\le n$ and $p<l$}\\
\noalign{\medskip}
\fff         &\text{otherwise}
              \end{cases}
\]
and
\[
\Dth kln\avec1n=\begin{cases}
\vlinf{\gwd}
      {}
      {\th kq\avec{p+1}n}
      {\fff}
             &\text{if $0<k\le q$ and $l\le p$}\\
\noalign{\medskip}
\vlinf{\gwd}
      {}
      {\th kp\avec1p}
      {\fff}
             &\text{if $0<k\le p$ and $p<l$}\\
\noalign{\medskip}
\fff         &\text{otherwise}
              \end{cases}\quad.
\]
%---------------------------------------
%---------------------------------------
\item
For $k\ge0$ and $1\le l\le n$, we define the derivations $\vlsmash{\Gth kln\avec1n}$, recursively on $n$, as follows:
\begin{itemize}
%---------------------------------------
\item $\Gth 011(a_1)=\ttt$;
%---------------------------------------
\item for $k>0$, $\Gth k11(a_1)=\fff$;
%---------------------------------------
\item for $k>n$, $\Gth kln\avec1n=\fff$;
%---------------------------------------
\item for $n>1$ and $k\le n$, let
\[
\Gth kln\avec1n=\begin{cases}
%---------------------------------------
\vls[
\bigvee_{\begin{subarray}{l}i+j=k      \\ 
                            0\le i<p   \\ 
                            0\le j\le q
         \end{subarray}}(
\Gth ilp\avec1p.
\th jq\avec{p+1}n).
\Uth kln\avec1n.\Dth{k+1}ln\avec1n]
&\text{if $l\le p$}\\
\noalign{\medskip}
%---------------------------------------
\vls[
\bigvee_{\begin{subarray}{l}i+j=k      \\
                            0\le i\le p\\ 
                            0\le j<q
         \end{subarray}}(
\th ip\avec1p.
\Gth j{l-p}q\avec{p+1}n).
\Uth kln\avec1n.\Dth{k+1}ln\avec1n]
&\text{if $p<l$}
\end{cases}
\quad.
\]
%---------------------------------------
\end{itemize}
%---------------------------------------
%---------------------------------------
\end{itemize}
\end{definition}


%-------------------------------------------------------------------------------
\begin{example}\label{example:AuxillaryThresholdDerivations}
See, in Figure~\vref{figure:AuxillaryThresholdDerivations}, some examples of derivations $\vlsmash{\Gth kln\avec1n}$. Note that, for clarity, we removed all instances of the trivial derivations $\Uth112\avec12=\Uth122\avec12=\Uth113\avec13=\vldownsmash{\vlinf\gwu{}\fff\fff}$. We can do so because these derivation instances appear as disjuncts.
\end{example}

%-------------------------------------------------------------------------------
\begin{figure}
\begin{eqnarray*}
%---------------------------------------
\Gth 015\avecletter&=&
\vls [\ttt.\vlderivation{
\vlin{}{}{b}{
\vlhy{\vls \fff}
}}
.\vlderivation{
\vlin{}{}{\vls [c.d.e]}{
\vlhy{\vls \fff}
}}
]\quad,\\
\noalign{\smallskip}
%---------------------------------------
\Gth 115\avecletter&=&
\vls [b.([\ttt.\vlderivation{
\vlin{}{}{b}{
\vlhy{\vls \fff}
}}
].[c.d.e]).\vlderivation{
\vlin{}{}{\vls [(c.[d.e]).(d.e)]}{
\vlhy{\vls \fff}
}}
]\quad,\\
\noalign{\smallskip}
%---------------------------------------
\Gth 215\avecletter&=&
\vls [(b.[c.d.e]).([\ttt.\vlderivation{
\vlin{}{}{b}{
\vlhy{\vls \fff}
}}
].[(c.[d.e]).(d.e)]).\vlderivation{
\vlin{}{}{\vls \fff}{
\vlhy{\vls (\fff.b)}
}}
.\vlderivation{
\vlin{}{}{\vls (c.d.e)}{
\vlhy{\vls \fff}
}}
]\quad,\\
\noalign{\smallskip}
%---------------------------------------
\Gth 315\avecletter&=&
\vls [(b.[(c.[d.e]).(d.e)]).([\ttt.\vlderivation{
\vlin{}{}{b}{
\vlhy{\vls \fff}
}}
].c.d.e).\vlderivation{
\vlin{}{}{\vls \fff}{
\vlhy{\vls (\fff.b.[c.d.e])}
}}
]\quad,\\
\noalign{\smallskip}
%---------------------------------------
\Gth 415\avecletter&=&
\vls [(b.c.d.e).\vlderivation{
\vlin{}{}{\vls \fff}{
\vlhy{\vls (\fff.b.[(c.[d.e]).(d.e)])}
}}
]\quad,\\
\noalign{\smallskip}
%---------------------------------------
\Gth 515\avecletter&=&
\vlderivation{
\vlin{}{}{\vls \fff}{
\vlhy{\vls (\fff.b.c.d.e)}
}}
\quad,\\
\noalign{\smallskip}
%---------------------------------------
\Gth 035\avecletter&=&
\vls [\ttt.\vlderivation{
\vlin{}{}{\vls [d.e]}{
\vlhy{\vls \fff}
}}
.\vlderivation{
\vlin{}{}{\vls [a.b]}{
\vlhy{\vls \fff}
}}
]\quad,\\
\noalign{\smallskip}
%---------------------------------------
\Gth 135\avecletter&=&
\vls [([a.b].[\ttt.\vlderivation{
\vlin{}{}{\vls [d.e]}{
\vlhy{\vls \fff}
}}
]).d.e.\vlderivation{
\vlin{}{}{\vls (d.e)}{
\vlhy{\vls \fff}
}}
.\vlderivation{
\vlin{}{}{\vls (a.b)}{
\vlhy{\vls \fff}
}}
]\quad,\\
\noalign{\smallskip}
%---------------------------------------
\Gth 235\avecletter&=&
\vls [(a.b.[\ttt.\vlderivation{
\vlin{}{}{\vls [d.e]}{
\vlhy{\vls \fff}
}}
]).([a.b].[d.e.\vlderivation{
\vlin{}{}{\vls (d.e)}{
\vlhy{\vls \fff}
}}
]).(d.e).\vlderivation{
\vlin{}{}{\vls \fff}{
\vlhy{\vls (\fff.[d.e])}
}}
]\quad,\\
\noalign{\smallskip}
%---------------------------------------
\Gth 335\avecletter&=&
\vls [(a.b.[d.e.\vlderivation{
\vlin{}{}{\vls (d.e)}{
\vlhy{\vls \fff}
}}
]).([a.b].[(d.e).\vlderivation{
\vlin{}{}{\vls \fff}{
\vlhy{\vls (\fff.[d.e])}
}}
]).\vlderivation{
\vlin{}{}{\vls \fff}{
\vlhy{\vls (\fff.d.e)}
}}
]\quad,\\
\noalign{\smallskip}
%---------------------------------------
\Gth 435\avecletter&=&
\vls [(a.b.[(d.e).\vlderivation{
\vlin{}{}{\vls \fff}{
\vlhy{\vls (\fff.[d.e])}
}}
]).\vlderivation{
\vlin{}{}{\vls \fff}{
\vlhy{\vls ([a.b].\fff.d.e)}
}}
]\quad,\\
\noalign{\smallskip}
%---------------------------------------
\Gth 535\avecletter&=&
\vlderivation{
\vlin{}{}{\vls \fff}{
\vlhy{\vls (a.b.\fff.d.e)}
}}\quad,\\
\noalign{\smallskip}
%---------------------------------------
\Gth 055\avecletter&=&
\vls [\ttt.\vlderivation{
\vlin{}{}{d}{
\vlhy{\vls \fff}
}}
.\vlderivation{
\vlin{}{}{c}{
\vlhy{\vls \fff}
}}
.\vlderivation{
\vlin{}{}{\vls [a.b]}{
\vlhy{\vls \fff}
}}
]\quad,\\
\noalign{\smallskip}
%---------------------------------------
\Gth 155\avecletter&=&
\vls [([a.b].[\ttt.\vlderivation{
\vlin{}{}{d}{
\vlhy{\vls \fff}
}}
.\vlderivation{
\vlin{}{}{c}{
\vlhy{\vls \fff}
}}
]).(c.[\ttt.\vlderivation{
\vlin{}{}{d}{
\vlhy{\vls \fff}
}}
]).d.\vlderivation{
\vlin{}{}{\vls (a.b)}{
\vlhy{\vls \fff}
}}
]\quad,\\
\noalign{\smallskip}
%---------------------------------------
\Gth 255\avecletter&=&
\vls [(a.b.[\ttt.\vlderivation{
\vlin{}{}{d}{
\vlhy{\vls \fff}
}}
.\vlderivation{
\vlin{}{}{c}{
\vlhy{\vls \fff}
}}
]).([a.b].[(c.[\ttt.\vlderivation{
\vlin{}{}{d}{
\vlhy{\vls \fff}
}}
]).d]).(c.d).\vlderivation{
\vlin{}{}{\vls \fff}{
\vlhy{\vls (d.\fff)}
}}
]\quad,\\
\noalign{\smallskip}
%---------------------------------------
\Gth 355\avecletter&=&
\vls [(a.b.[(c.[\ttt.\vlderivation{
\vlin{}{}{d}{
\vlhy{\vls \fff}
}}
]).d]).([a.b].[(c.d).\vlderivation{
\vlin{}{}{\vls \fff}{
\vlhy{\vls (d.\fff)}
}}
]).\vlderivation{
\vlin{}{}{\vls \fff}{
\vlhy{\vls (c.d.\fff)}
}}
]\quad,\\
\noalign{\smallskip}
%---------------------------------------
\Gth 455\avecletter&=&
\vls [(a.b.[(c.d).\vlderivation{
\vlin{}{}{\vls \fff}{
\vlhy{\vls (d.\fff)}
}}
]).\vlderivation{
\vlin{}{}{\vls \fff}{
\vlhy{\vls ([a.b].c.d.\fff)}
}}
]\quad,\\
\noalign{\smallskip}
%---------------------------------------
\Gth 555\avecletter&=&
\vlderivation{
\vlin{}{}{\vls \fff}{
\vlhy{\vls (a.b.c.d.\fff)}
}}\quad.
\end{eqnarray*}
\caption{Examples of $\Gth kl5\avecletter$, where $\avecletter=(a,b,c,d,e)$.}
\label{figure:AuxillaryThresholdDerivations}
\end{figure}


%-------------------------------------------------------------------------------
\begin{theorem}\label{theorem:AuxillaryThresholdDerivations}
For any $n>0$, $k\ge0$ and\/ $1\le l\le n$, the derivation\/ $\vlsmash{\Gth kln\avec1n}$ has shape
\[
\vlder{}{\{\awd,\awu\}}{(\th{k+1}n\avec1n)\{a_l/\ttt\}}
                       {(\th kn\avec1n)\{a_l/\fff\}}
\quad,
\]
and\/ $\size{\Gth kln\avec1n}$ is $n^{\Ord{\log n}}$.
\end{theorem}

%-------------------------------------------------------------------------------
\begin{proof}
The shape of $\Gth kln\avec1n$ can be verified by inspecting Definition~\vref{definition:AuxillaryThresholdDerivation}. For example, this is the case when $n>1$ and $l\le p\le k<q$, where $p=\lfloor n/2\rfloor$ and $q=n-p$:
\vlstore{\noalign{\medskip}
\vls[
\textstyle\bigvee_{\begin{subarray}{l}i+j=k      \\
                                      0\le i<p   \\
                                      0\le j\le q
                   \end{subarray}}(
\vlder{\Gth ilp\avec1p}
      {}
      {(\th{i+1}p\avec1p)\{a_l/\ttt\}}
      {(\th ip\avec1p)\{a_l/\fff\}}
.
\th jq\avec{p+1}n)
.
\vlinf{\gwu}
      {}
      {\fff}
      {\vls({\vlnos(\th pp\avec1p)}\{a_l/\fff\}.\th{k-p}q\avec{p+1}n)}
.
\vlinf{\gwd}
      {}
      {\th{k+1}q\avec{p+1}n}
      {\fff}
]}
\begin{multline*}
\vlder{\Gth kln\avec1n}
      {}
      {(\th{k+1}n\avec1n)\{a_l/\ttt\}}
      {(\th kn\avec1n)\{a_l/\fff\}}
={}\\
\vlread
\quad.
\end{multline*}
(Remember that
\[
\th kn\avec1n\equiv\bigvee_{\begin{subarray}{l}
                            i+j=k\\ 
                            0\le i\le p\\ 
                            0\le j\le q
                            \end{subarray}}
                   \vlsbr(\th ip\avec1p.\th jq\avec{p+1}n)
\]
and $\th0p\avec1p\equiv\ttt$.) General (co)weak\-en\-ing rule instances can be replaced by atomic ones because of Proposition~\vref{TODO}. The size bound on $\Gth kln\avec1n$ follows from Proposition~\vref{TODO} and Theorem~\vref{theorem:SizeThreshold}.
\end{proof}

\TODO{Define operator for the following definition}

\begin{definition}\label{definition:ThresholdDerivations}
Consider $n>0$, distinct atoms $a_1$, \dots, $a_n$. For $k\ge0$, we define the derivations $\vlsmash{\Gth k{}n\avec1n}$ as follows:
\[
\Gth k{}n\avec1n\quad=\quad
\vlderivation
{
 \vlin{n\cdot\cou}{}
 {
  \vls
  (
   \vlder{}{}
   {
    \vls[a_1.(\th {k+1}n\avec1n)\{a_1/\fff\}]
   }
   {
    \th {k+1}n\avec1n
   }
  \;\;.\;\;
   \vlder{}{}
   {
    \vls[a_n.(\th {k+1}n\avec1n)\{a_n/\fff\}]
   }
   {
    \th {k+1}n\avec1n
   }
  )
 }
 {
  \vlin{n\cdot\cod}{}
  {
   \th {k+1}n\avec1n
  }
  {
   \vlhy
   {
    \vls
    [
     \vlder{}{}
     {
      \th {k+1}n\avec1n
     }
     {
      \vlsbr
      (
       a_1
      .
       \vlder{\Gth kin\avec1n}{}
       {
        (\th {k+1}n\avec1n)\{a_1/\ttt\}
       }
       {
        (\th kn\avec1n)\{a_1/\fff\}
       }
      )
     }
    \;.\;
     \vlder{}{}
     {
      \th {k+1}n\avec1n
     }
     {
      \vlsbr
      (
       a_n
      .
       \vlder{\Gth knn\avec1n}{}
       {
        (\th {k+1}n\avec1n)\{a_n/\ttt\}
       }
       {
        (\th kn\avec1n)\{a_n/\fff\}
       }
      )
     }
    ]
   }
  }
 }
}
\]
\end{definition}

%-------------------------------------------------------------------------------
\begin{theorem}\label{theorem:ThresholdDerivations}
For any $n>0$ and $k\ge0$, the derivation\/ $\vlsmash{\Gth k{}n\avec1n}$ has shape
\[
\vlder{}{\SKS\setminus\{\aid,\aiu\}}
{
 \vls([a_1.(\th{k+1}n\avec1n)\{a_1/\fff\}].\cdots.[a_n.(\th{k+1}n\avec1n)\{a_n/\fff\}])
}
{
 \vls[(a_1.(\th{k}n\avec1n)\{a_1/\fff\}).\cdots.(a_n.(\th{k}n\avec1n)\{a_n/\fff\})]
}
\quad,
\]
and\/ $\size{\Gth k{}n\avec1n}$ is $n^{\Ord{\log n}}$.
\end{theorem}


\newcommand{\frqmis}{{\mathsf{qmis}}}
%---------------------------------------
\begin{definition}\label{definition:QuasipolynomialMultipleSubflowRemoval}
We define the reduction $\to_\frqmis$ (where $\frqmis$ stands for \emph{quasipolynomial multiple isolated subflows}) as follows,
Let $\avec 1n$ be any atoms, then, for $1\le k\le n$, instantiate $\to_\frmis$ with
\begin{itemize}
\item the atomic flow of $\Gamma_k\avec 1n$ for $\gamma_k$, and
\item one copy of $\psi_i$ for every atom occurrence of $\th kn\avec 1n\{a_i/\fff\}$.
\end{itemize}
\end{definition}

\begin{theorem}\label{theorem:SoundQuasipolynomialMultipleSubflowRemoval}
$\to_\frqmis$ is sound.
\end{theorem}

\begin{proof}

\end{proof}

\TODO{Rephrase:}

\begin{theorem}\label{theorem:SizeQuasipolynomialMultipleSubflowRemoval}
$\to_\frqmis$ is quasipolynomial.
\end{theorem}

%======================================
\section{Local Transformations}

\begin{definition}\label{definition:FlowGraphicalExpressions}
In Figure~\vref{figure:ReductionRules}, we define graphical expressions of the kind $r\colon\phi'\to\psi'$, where $r$ is a name and $\phi'$ and $\psi'$ are flows.
\end{definition}

\TODO{In Figures~\vref{figure:ReductionRules}, \vref{figure:ReductionRulesWeakening} and \vref{figure:ReductionRulesContraction} extend the edges so the atomic flows on either side of the arrow has equal height.}

\newcommand{\rwdcd}{{{\mathsf w}{\downarrow}{\hbox{-}}{\mathsf c}{\downarrow}}}
\newcommand{\rwdiu}{{{\mathsf w}{\downarrow}{\hbox{-}}{\mathsf i}{\uparrow  }}}
\newcommand{\rwdwu}{{{\mathsf w}{\downarrow}{\hbox{-}}{\mathsf w}{\uparrow  }}}
\newcommand{\rwdcu}{{{\mathsf w}{\downarrow}{\hbox{-}}{\mathsf c}{\uparrow  }}}
\newcommand{\rcuwu}{{{\mathsf c}{\uparrow  }{\hbox{-}}{\mathsf w}{\uparrow  }}}
\newcommand{\rcdwu}{{{\mathsf c}{\downarrow}{\hbox{-}}{\mathsf w}{\uparrow  }}}
\newcommand{\rcdiu}{{{\mathsf c}{\downarrow}{\hbox{-}}{\mathsf i}{\uparrow  }}}
\newcommand{\rcdcu}{{{\mathsf c}{\downarrow}{\hbox{-}}{\mathsf c}{\uparrow  }}}
\newcommand{\ridwu}{{{\mathsf i}{\downarrow}{\hbox{-}}{\mathsf w}{\uparrow  }}}
\newcommand{\ridcu}{{{\mathsf i}{\downarrow}{\hbox{-}}{\mathsf c}{\uparrow  }}}
%---------------------------------------
\begin{figure}[tbp]
\[
\begin{array}{@{}c@{}c@{}}
%-------------------
\rwdcd\colon\quad\afraise{\atomicflow{
( 0  ,0)*{\afacd{}{}{}\one{}\two};
(-2  ,4)*{\afawdnw{}{}};
(-3.5,0)*{\invisiblemark};
( 3.5,0)*{\invisiblemark}}}
\quad\to\quad
\atomicflow{
( 0  ,3.3)*{\aflabelright{\one,\two}};
( 0  ,3  )*{\afvj6};
(-1.5,0  )*{\invisiblemark};
( 3  ,0  )*{\invisiblemark}}
&\qquad
%-------------------
\rcuwu\colon\quad\aflower{\atomicflowinv{
( 0  ,0)*{\afacu{}{}{}\two{}\one};
(-2  ,6)*{\afawunw{}{}};
( 3.5,0)*{\invisiblemark}}}
\quad\to\quad
\atomicflow{
(0,3  )*{\afvj6};
(0,3.3)*{\aflabelright{\one,\two}};
(3,0  )*{\invisiblemark}}
\\
%-------------------
\rwdiu\colon\quad\atomicflow{
( 0  ,0)*{\afaiu{}{}{}\one{}{}};
(-2  ,4)*{\afawdnw{}{}};
(-3.5,0)*{\invisiblemark};
( 3.5,0)*{\invisiblemark}}
\quad\to\quad
\aflower{\atomicflow{
( 0  ,0)*{\afawu{}{}{}\one};
(-1.5,0)*{\invisiblemark};
( 1.5,0)*{\invisiblemark}}}
&\qquad
%-------------------
\ridwu\colon\quad\atomicflowinv{
( 0  ,0)*{\afaid{}{}{}\one{}{}};
(-2  ,6)*{\afawunw{}{}};
( 3.5,0)*{\invisiblemark}}
\quad\to\quad
\afraise{\atomicflowinv{
(0  ,0)*{\afawd{}{}{}\one};
(1.5,0)*{\invisiblemark}}}
\\
%-------------------
\multispan2{\hfil$
\rwdwu\colon\quad\atomicflow{
( 0  ,6)*{\afawd{}{}{}{}};
( 0  ,0)*{\afawunw{}{}{}{}};
(-1.5,0)*{\invisiblemark};
( 1.5,0)*{\invisiblemark}}
\quad\to\quad
\atomicflow{}
$\hfil}\\
%-------------------
\rwdcu\colon\quad\afraise{\atomicflow{
( 0  ,-4)*{\afacu\one{}{}\two{}{}};
( 0  , 0)*{\afawdnw{}{}};
(-3.5, 0)*{\invisiblemark};
( 3.5, 0)*{\invisiblemark}}}
\quad\to\quad
\atomicflow{
(-2  ,-4)*{\afawd{}{}\one{}};
( 2  ,-4)*{\afawd{}{}{}\two};
(-3.5, 0)*{\invisiblemark};
( 3.5, 0)*{\invisiblemark}}
&\qquad
%-------------------
\rcdwu\colon\quad\aflower{\atomicflowinv{
( 0  ,-4)*{\afacd\one{}{}\two{}{}};
( 0  , 2)*{\afawunw{}{}};
(-3.5, 0)*{\invisiblemark};
( 3.5, 0)*{\invisiblemark}}}
\quad\to\quad
\atomicflowinv{
(-2  ,-4)*{\afawu{}{}\one{}};
( 2  ,-4)*{\afawu{}{}{}\two};
(-3.5, 0)*{\invisiblemark};
( 3.5, 0)*{\invisiblemark}}
\\
%-------------------
\rcdiu\colon\quad\aflower{\atomicflow{
(   6, 6)*{\afvjd4{}\three{}{}};
(   3, 0)*{\afaiuex{}{}{}{}{}{}32};
(   0, 4)*{\afacdnw\one{}{}\two};
(-3.5, 0)*{\invisiblemark};
( 7.5, 0)*{\invisiblemark}}}
\quad\to\quad
\aflower{\atomicflow{
(  10,8)*{\afacu{}{}{}{}{}\three};
(   0,8)*{\afvjd8\one{}};
(   4,8)*{\afvjd8{}\two};
(   6,2)*{\afaiunw{}{}};
(   6,0)*{\afaiuex{}{}{}{}{}{}31}}}
&\qquad
%-------------------
\ridcu\colon\quad\afraise{\atomicflowinv{
(   6,6)*{\afvju4{}\three{}{}};
(   3,0)*{\afaidex {}{}{}{}{}{}32};
(   0,6)*{\afacunw\one{}{}\two};
(-3.5,0)*{\invisiblemark};
( 7.5,0)*{\invisiblemark}}}
\quad\to\quad
\afraise{\atomicflowinv{
(  10,8)*{\afacd{}{}{}{}{}\three};
(   0,8)*{\afvju8\one{}};
(   4,8)*{\afvju8{}\two};
(   6,4)*{\afaidnw{}{}};
(   6,0)*{\afaidex{}{}{}{}{}{}31}}}
\\
%-------------------
\multispan2{\hfil$
\rcdcu\colon\quad\atomicflow{
( 0,6)*{\afacd\one{}{}\two{}{}};
( 0,0)*{\afacunw\three{}{}\four};
(-4,0)*{\invisiblemark};
( 4,0)*{\invisiblemark}}
\quad\to\quad
\atomicflow{
(0,12)*{\afacu{}{}{}{}\one{}};
(6,12)*{\afacu{}{}{}{}{}\two};
( 0,0)*{\afacd{}{}{}{}\three{}};
( 6,0)*{\afacd{}{}{}{}{}\four};
(-2,6)*{\afvj4};
( 8,6)*{\afvj4};
( 3,6)*{\afex24}}
$\hfil}\\
%%-------------------
%\rcd\colon\quad\aflower{\atomicflow{
%(   0, 4)*{\afacd\one{}{}\two{}\three};
%(-3.5, 0)*{\invisiblemark};
%( 3.5, 0)*{\invisiblemark}
%}}
%\quad\to\quad
%\aflower{\atomicflow{
%(13,12)*{\afaidex{}{}{}{}{}{}32};
%(16, 4)*{\afvju8{}\three};
%(10, 6)*{\afacunw{}{}{}{}};
%( 0, 8)*{\afvjd8\one{}};
%( 4, 8)*{\afvjd8{}\two};
%( 6, 2)*{\afaiunw{}{}};
%( 6, 0)*{\afaiuex{}{}{}{}{}{}31}}}
%&\qquad
%%-------------------
%\rcu\colon\quad\afraise{\atomicflowinv{
%(   0,-4)*{\afacu\one{}{}\two{}\three};
%(-3.5, 0)*{\invisiblemark};
%( 3.5, 0)*{\invisiblemark}
%}}
%\quad\to\quad
%\aflower{\atomicflow{
%(13,-12)*{\afaiuex{}{}{}{}{}{}32};
%(16, -4)*{\afvjd8{}\three};
%(10,-8.25)*{\afacdnw{}{}{}{}};
%( 0, -8)*{\afvju8\one{}};
%( 4, -8)*{\afvju8{}\two};
%( 6, -4)*{\afaidnw{}{}};
%( 6,  0)*{\afaidex{}{}{}{}{}{}31}}}
%\\
\end{array}
\]
\caption{Atomic-flow reduction rules.}
\label{figure:ReductionRules}
\end{figure}%

%---------------------------------------
\begin{example}\label{example:NoPolarityAssignment}
The `reduction' on the left, when used inside a larger atomic flow, might create a situation as on the right:
\nopagebreak[4]\medskip\afnegspace
\[
\atomicflow{
( 0  ,0)*{\afacu{}{}{}{}{}{}};
( 0  ,4)*{\afawdnw{}{}}}
\quad\to\quad
\atomicflow{
( 0  ,0)*{\afaid{}{}{}{}{}{}}}
\qquad\qquad
\atomicflow{
( 0  , 8)*{\afacu{}{}{}{}\ppl{}};
( 0  ,12)*{\afawdnw{}{}};
(-2  , 4)*{\aflabelleft\ppl};
( 2  , 4)*{\aflabelright\ppl};
( 0  , 0)*{\afacd{}{}{}{}\ppl{}};
(-3.5, 0)*{\invisiblemark};
( 3.5, 0)*{\invisiblemark}}
\quad\to\quad
\atomicflow{
( 0  ,4)*{\afaidnw{}{}};
( 0  ,0)*{\afacd\ppl{}{}{\scriptstyle?}\ppl{}};
(-3.5,0)*{\invisiblemark};
( 3.5,0)*{\invisiblemark}}
\quad,
\] 
where the graph at the right is not an atomic flow, for lack of a polarity assignment.
\end{example}

This prompts us to define reduction rules for atomic flows as follows.

%---------------------------------------
\begin{definition}\label{definition:FlowReductionRule}
An (\emph{atomic-flow}) \emph{reduction rule $r$ from flow $\phi'$ to flow $\psi'$}\index{reduction!rule} is a quadruple $(\phi',\psi',f,g)$ such that:
\begin{enumerate}
\item $f$ is a one-to-one map from the upper edges of $\phi'$ to the upper edges of $\psi'$,
\item $g$ is a one-to-one map from the lower edges of $\phi'$ to the lower edges of $\psi'$,
\item for every polarity assignment $\pi$ for $\phi'$, there is a polarity assignment $\pi'$ for $\psi'$ such that $\pi'(f(\epsilon))=\pi(\epsilon)$ and $\pi'(g(\epsilon'))=\pi(\epsilon')$, for any upper edge $\epsilon$ and any lower edge $\epsilon'$ of $\phi'$;
\end{enumerate}
we define reduction rules with graphical expressions $r\colon\phi'\to\psi'$, where $f$ and $g$ are indicated by labelling edges. For every reduction rule $r\colon\phi'\to\psi'$, the reduction ${\to_r}$ is defined, such that $\phi\to_r\psi$ if and only if $\phi'$ appears as a subgraph in $\phi$ and we obtain $\psi$ by replacing $\phi'$ with $\psi'$ in $\phi$, while respecting the correspondence of edges; we call this operation a \emph{reduction by $r$}\index{reduction!by rule}.
\end{definition}

\TODO{Consider making a remark about conserving polarity assignments.}

\begin{remark}\label{remark:FlowReductionRuleProperFlow}
The condition on polarity assignments for a reduction rule $r$ guarantees that the $\psi$ in $\phi\to_r\psi$ is a proper atomic flow, if $\phi$ is one.
\end{remark}

\begin{remark}\label{remark:FlowReductionRuleNoConnect}
Because of the condition on polarity assignments for reduction rules, two distinct connected components in a flow cannot be connected by a reduction. To see that this is impossible, consider the following `reduction rule', which violates the condition on polarity assignments:
\[
\aflower{\atomicflow{
(-2,0)*{\afawu{}{}{}{}};
( 2,0)*{\afawu{}{}{}{}}}}
\quad\to\quad
\aflower{\atomicflow{
( 0  , 2)*{\afaiu{}{}{}{}{}{}}}}
\quad.
\]
\afnegspace
For this `reduction rule' there exist both valid (left) and invalid (right) polarity assignments:
\[
\aflower{\atomicflow{
(-2  ,0)*{\afawu{}{}\ppl{}};
( 2  ,0)*{\afawu{}{}{}\pmi};
(-3.5,0)*{\invisiblemark};
( 3.5,0)*{\invisiblemark}}}
\quad\to\quad
\aflower{\atomicflow{
( 0  ,2)*{\afaiu\ppl{}{}\pmi{}{}};
(-3.5,0)*{\invisiblemark};
( 3.5,0)*{\invisiblemark}}}
\qquad\qquad
\aflower{\atomicflow{
(-2,0)*{\afawu{}{}\ppl{}};
( 2,0)*{\afawu{}{}{}\ppl};
(-3.5,0)*{\invisiblemark};
( 3.5,0)*{\invisiblemark}}}
\quad\to\quad
\aflower{\atomicflow{
( 0  , 2)*{\afaiu\ppl{}{}{\scriptstyle?}{}{}};
(-3.5, 0)*{\invisiblemark};
( 3.5, 0)*{\invisiblemark}}}
\quad.
\]
\afnegspace
\end{remark}

It is immediate to check:

\begin{proposition}\label{proposition:ValidReductionRules}
The graphical expressions in Figure~\vref{figure:ReductionRules} are atomic-flow reduction rules.
\end{proposition}

%---------------------------------------
\begin{definition}\label{definition:FlowRewritingSystem}
A finite set of reduction rules is a \emph{flow rewriting system}\index{flow rewriting system}. For every flow rewriting system $F=\{r_1,\dots,r_h\}$ we define ${\to_F}={\to_{r_1}\cup\cdots\cup{\to_{r_h}}}$. The reflexive transitive closure of $\to_F$ is denoted by $\to_F^\star$. Given a set of atomic flows $S$, we say that a flow rewriting system $F$ is \emph{terminating on $S$}\index{flow rewriting system!terminating} if there is no infinite chain $\phi_1\to_F\phi_2\to_F\cdots$, for every $\phi_1\in S$; if $F$ is terminating on the set of atomic flows, we say that it is \emph{terminating}. We say that atomic flow $\phi$ is \emph{normal}\index{flow rewriting system!normal} for flow rewriting system $F$ if there is no atomic flow $\psi$ such that $\phi\to_F\psi$.
\end{definition}

\newcommand{\frw}{{\mathsf w}}
%---------------------------------------
\begin{definition}\label{definition:FlowRewritingWeakening}
The following flow rewriting system is called $\frw$:
\[
\{\;\rwdcd\;,\;\rcuwu\;,\;\rwdiu\;,\;\ridwu\;,\;\rwdwu\;,\;\rwdcu\;,\;\rcdwu\;\}
\quad.
\]
\end{definition}

\newcommand{\frc}{{\mathsf c}}
%---------------------------------------
\begin{definition}\label{definition:FlowRewritingContraction}
The following flow rewriting system is called $\frc$:
\[
\{\;\rcdiu\;,\;\ridcu\;,\;\rcdcu\;\}\quad.
\]
\end{definition}

%=======================================
\subsection{Soundness}\label{subsection:soundness}

%---------------------------------------
\begin{definition}\label{definition:SoundRedcutionRule}
A reduction rule $r$ is \emph{sound}\index{reduction!rule!sound} if $\to_r$ is sound.
\end{definition}

The proof of the following theorem is essentially contained in Figures~\vref{figure:ReductionRulesWeakening} and \vref{figure:ReductionRulesContraction}.

%---------------------------------------
\begin{figure}[tbp]
\[
\begin{array}{@{}l@{}c@{}}
%---------------------------------------
\rwdcd\colon\hfil\afraise{\atomicflow{
( 0  ,0)*{\afacd\three{}{}\one{}\two};
(-2  ,4)*{\afawdnw{}{}};
(-3.5,0)*{\invisiblemark};
( 3.5,0)*{\invisiblemark}}}
\quad\to\quad
\atomicflow{
( 0  ,3.3)*{\aflabelright{\one,\two}};
( 0  ,3  )*{\afvj6};
(-1.5,0  )*{\invisiblemark};
( 3  ,0  )*{\invisiblemark}}
&\qquad
\vlder{\Phi}{}{\zeta\left\{\vlinf{}{}{a^\two}{\vls[a^\three.a^\one]}\right\}}
              {\xi\left\{\vlinf{}{}{a^\three}{\fff}\right\}}
\quad\to_\rwdcd\quad
\vlder{\Phi\{a^\three/\fff\}}{}{\zeta\left\{\vlinf{=}{}{a^{\one,\two}}{\vls[\fff.a^{\one,\two}]}\right\}}
              {\xi\{\fff\}}
\\
\noalign{\bigskip}
%---------------------------------------
\rwdiu\colon\hfil\atomicflow{
( 0  ,0)*{\afaiu\two {}{}\one{}{}};
(-2  ,4)*{\afawdnw{}{}};
(-3.5,0)*{\invisiblemark};
( 3.5,0)*{\invisiblemark}}
\quad\to\quad
\aflower{\atomicflow{
( 0  ,0)*{\afawu{}{}{}\one};
(-1.5,0)*{\invisiblemark};
( 1.5,0)*{\invisiblemark}}}
&\qquad
\vlder{\Phi}{}{\zeta\left\{\vlinf{}{}{\fff}{\vls(a^\two.\bar a^\one)}\right\}}
              {\xi\left\{\vlinf{}{}{a^\two}{\fff}\right\}}
\quad\to_\rwdiu\quad
\vlder{\Phi\{a^\two/\fff\}}{}{\zeta\left\{\vlinf{=}{}{\fff}{\vls(\fff\;.\;\vlinf{}{}{\ttt}{\bar a^\one})}\right\}}
              {\xi\{\fff\}}
\\
\noalign{\bigskip}
%---------------------------------------
\rwdwu\colon\hfil\atomicflow{
( 0  ,6)*{\afawd{}{}{}\one};
( 0  ,0)*{\afawunw{}{}{}{}};
(-1.5,0)*{\invisiblemark};
( 1.5,0)*{\invisiblemark}}
\quad\to\quad
\atomicflow{}
&\qquad
\vlder{\Phi}{}{\zeta\left\{\vlinf{}{}{\ttt}{a^\one}\right\}}
              {\xi\left\{\vlinf{}{}{a^\one}{\fff}\right\}}
\quad\to_\rwdwu\quad
\vlder{\Phi\{a^\one/\fff\}}{}{\zeta\left\{\vlderivation
                           {
                            \vlin{=}{}{\ttt}
                            {
                             \vlin{\swi}{}{\vls[(\fff.\fff).\ttt]}
                             {
                              \vlin{=}{}{\vls(\fff.[\fff.\ttt])}
                              {
                               \vlhy{\fff}
                              }
                             }
                            }
                           }\right\}}
              {\xi\{\fff\}}
\\
\noalign{\bigskip}
%---------------------------------------
\rwdcu\colon\hfil\afraise{\atomicflow{
( 0  ,-4)*{\afacu\one{}{}\two{}\three};
( 0  , 0)*{\afawdnw{}{}};
(-3.5, 0)*{\invisiblemark};
( 3.5, 0)*{\invisiblemark}}}
\quad\to\quad
\atomicflow{
(-2  ,-4)*{\afawd{}{}\one{}};
( 2  ,-4)*{\afawd{}{}{}\two};
(-3.5, 0)*{\invisiblemark};
( 3.5, 0)*{\invisiblemark}}
&\qquad
\vlder{\Phi}{}{\zeta\left\{\vlinf{}{}{\vls(a^\one.a^\two)}{a^\three}\right\}}
              {\xi\left\{\vlinf{}{}{a^\three}{\fff}\right\}}
\quad\to_\rwdcu\quad
\vlder{\Phi\{a^\three/\fff\}}{}{\zeta\left\{\vlinf{=}{}{\vls(\vlinf{}{}{a^\one}{\fff}\;.\;\vlinf{}{}{a^\two}{\fff})}{\fff}\right\}}
              {\xi\{\fff\}}
\\
%---------------------------------------
\end{array}
\]
\caption{`Downwards' reduction rules for weakening and their soundness.}
\label{figure:ReductionRulesWeakening}
\end{figure}%

\newbox\ContDownIntUp
\setbox\ContDownIntUp=
\hbox{$
\vlderivation
{
 \vlin{}{}{\fff}
 {
  \vlin{=}{}{\vlsbr(\vlinf{\swi}{}{\vlinf{=}{}{a^\one}{\vlsbr[\vlinf{}{}{\fff}{\vls(\bar a.a^\two)}\;.\;a^\one]}}{\vls(\bar a.[a^\two.a^\one])}\;\;\;.\;\;\;\bar a)}
  {
   \vlhy{\vlsbr([a^\one.a^\two]\;.\;\vlinf{}{}{\vls(\bar a.\bar a)}{\bar a^\three})}
  }
 }
}
$}
\newbox\ContDownContUp
\setbox\ContDownContUp=
\hbox{$
\vlinf{\med}{}{\vlsbr(\vlinf{}{}{a^\three}{\vls[a.a]}\;.\;\vlinf{}{}{a^\four}{\vls[a.a]})}
              {\vlsbr[\vlinf{}{}{\vls(a.a)}{a^\one}\;.\;\vlinf{}{}{\vls(a.a)}{a^\two}]}
$}
\newbox\ContDown
\setbox\ContDown=
\hbox{$
\vlinf\swi{}{
			 \vlsbr[\vlderivation
			        {
			         \vlin\swi{}
			         {
			          \vls[\vlinf{}{}
                           {\fff}
                           {\vls(\bar a.a^\one)}
                          .
                           \vlinf{}{}
                           {\fff}
                           {\vls(\bar a.a^\two)}
                          ]
                     }
                     {
                      \vlin\swi{}
                      {
                       \vls(\bar a.[(\bar a.a^\one).a^\two])
                      }
                      {
                       \vlhy
                       {\vls(\vlinf{}{}
                             {\vls(\bar a.\bar a)}
                             {\bar a}          
                            .
                             [a^\one.a^\two]
                            )
                       }
                      }
                     }
                    }
                   \;\;\;.
                    {a^\three}
                   ]
            }
            {
             \vls(\vlinf{}{}
                  {\vls[\bar a.a^\three]}
                  {\ttt}
                 .
                  [a^\one.a^\two]
                 )
            }
$}
%---------------------------------------
\begin{figure}[tbp]
\[
\begin{array}{@{}l@{}c@{}}
%---------------------------------------
\rcdiu\colon\aflower{\atomicflow{
(   6, 6)*{\afvjd4{}\three{}{}};
(   3, 0)*{\afaiuex{}\four{}{}{}{}32};
(   0, 4)*{\afacdnw\one{}{}\two};
(-3.5, 0)*{\invisiblemark};
( 7.5, 0)*{\invisiblemark}}}
\quad\to\quad
\aflower{\atomicflow{
(10,8)*{\afacu{}{}{}{}{}\three};
( 0,8)*{\afvjd8\one{}};
( 4,8)*{\afvjd8{}\two};
( 6,2)*{\afaiunw{}{}};
( 6,0)*{\afaiuex{}{}{}{}{}{}31}}}
&\qquad
\vlder{\Phi}{}{\zeta\left\{\vlinf{}{}{\fff}{\vls(a^\four.\bar a^\three)}\right\}}
              {\xi\left\{\vlinf{}{}{a^\four}{\vls[a^\one.a^\two]}\right\}}
\quad\to_\rcdiu\quad
\vlder{\Phi\{a^\four/\vlsmallbrackets\vlsbr[a^\one.a^\two]\}}{}{\zeta\left\{\box\ContDownIntUp\right\}}
                                     {\xi\left\{\vls[a^\one.a^\two]\right\}}
\\
\noalign{\bigskip}
%---------------------------------------
\rcdcu\colon\hfil\atomicflow{
( 0,6)*{\afacd\one{}{}\two{}\five};
( 0,0)*{\afacunw\three{}{}\four};
(-4,0)*{\invisiblemark};
( 4,0)*{\invisiblemark}}
\quad\to\quad
\atomicflow{
( 0,12)*{\afacu{}{}{}{}\one{}};
( 6,12)*{\afacu{}{}{}{}{}\two};
( 0, 0)*{\afacd{}{}{}{}\three{}};
( 6, 0)*{\afacd{}{}{}{}{}\four};
(-2, 6)*{\afvj4};
( 8, 6)*{\afvj4};
( 3, 6)*{\afex24}}
&\qquad
\vlder{\Phi}{}{\zeta\left\{\vlinf{}{}{\vls(a^\three.a^\four)}{a^\five}\right\}}
              {\xi\left\{\vlinf{}{}{a^\five}{\vls[a^\one.a^\two]}\right\}}
\quad\to_\rcdcu\quad
\vlder{\Phi\{a^\five/\vlsmallbrackets\vlsbr[a^\one.a^\two]\}}{}{\zeta\left\{\box\ContDownContUp\right\}}
                                     {\xi\left\{\vls[a^\one.a^\two]\right\}}
\\
%---------------------------------------
\end{array}
\]
\caption{`Downwards' reduction rules for contraction and their soundness.}
\label{figure:ReductionRulesContraction}
\end{figure}%

%---------------------------------------
\begin{theorem}\label{theorem:ReductionRulesSound}
The reduction rules\/ $\rwdcd$, $\rwdiu$, $\rwdwu $, $\rwdcu$, $\rcdiu$, $\rcdcu$, $\rcuwu$, $\ridwu$, $\rcdwu$ and $\ridcu$ are sound.
\end{theorem}

\begin{proof}
For $r\in\{\rwdcd,\rwdiu,\rwdwu,\rwdcu,\rcdiu,\rcdcu\}$ and $r\colon\phi'\to\psi'$ as in the left columns of Figures~\vref{figure:ReductionRulesWeakening} and \vref{figure:ReductionRulesContraction}, for every $\phi$ and $\psi$ such that $\phi\to_r\psi$ and for every $\Phi$ with flow $\phi$, the right columns of the tables provide reductions $\Phi\to_r\Psi$, where $\Psi$ has flow $\psi$, as follows. If $\Phi'\to_r\Psi'$ is the reduction provided by the table, then
\[
\Phi=
\vlderivation              {
\vlde{\Psi_2}{}{\beta  }  {
\vlde{\Phi' }{}{\beta' } {
\vlde{\Psi_1}{}{\alpha'}{
\vlhy          {\alpha }}}}}
\qquad\hbox{and}\qquad
\Psi=
\vlderivation              {
\vlde{\Psi_2}{}{\beta  }  {
\vlde{\Psi'}{}{\beta' } {
\vlde{\Psi_1}{}{\alpha'}{
\vlhy          {\alpha }}}}}
\quad.
\]
We can deal with the remaining rules by employing dual derivations to the ones shown.
\end{proof}

\TODO{Check why the inference rule $\swi'$ taken from AFI used atoms instead of formulae.}

%---------------------------------------
\begin{remark}\label{remark:ReductionRulesSoundIndependence}
The previous soundness theorem only depends on the switch and medial rules for the reductions in Figure~\vref{figure:ReductionRulesContraction}. Any system obtained from $\SKS$ by replacing $\swi$ and $\med$ with linear rules that can derive them would support a soundness theorem like the one above, for the same reduction rules. For example, we could think of replacing $\swi$ with the rule $\vlinf{\swi'}{}{\vls[(\alpha.\gamma).[\beta.\delta]]}{\vls([\alpha.\beta].[\gamma.\delta])}$, from which $\swi$ is derivable.
\end{remark}

%=======================================
\subsection{Termination and Confluence}\label{subsection:TerminationConfluence}

%---------------------------------------
\begin{remark}\label{RemCycle}
Flow rewriting system $\frc$ is not terminating:
\nopagebreak[4]\medskip\afnegspace
\[
\atomicflow{
( 4  ,8)*{\afaidnw{}{}};
( 6  ,6)*{\afvj4};
( 3  ,0)*{\afaiuex{\ppl}{}{}{\pmi}{}{}32};
( 0  ,4)*{\afacd{\ppl}{}{}{}{}{}};
(-3.5,0)*{\invisiblemark};
( 7.5,0)*{\invisiblemark}}
\quad\to_\frc\quad
\atomicflow{
( 9  ,12)*{\afaidex{\pmi}{}{}{\ppl}{}{}32};
( 6  , 8)*{\afacu{}{}{}{}{}{}};
(12  , 6)*{\afvj4};
( 0  , 6)*{\afvjd4\ppl{}};
( 2  , 2)*{\afaiunw{}{}};
(10  , 2)*{\afaiunw{}{}{}{}{}{}};
(-1.5, 0)*{\invisiblemark};
(13.5, 0)*{\invisiblemark}}
\quad\to_\frc\quad
\atomicflow{
( 6  ,8)*{\afaidnw{}{}};
(14  ,8)*{\afaidnw{}{}};
( 4  ,6)*{\afvj4};
(16  ,6)*{\afvj4};
( 2  ,0)*{\afaiu{\ppl}{}{}{}{}{}};
(13  ,0)*{\afaiuex{\ppl}{}{}{\pmi}{}{}32};
(10  ,4)*{\afacd{}{}{}{}{}{}};
(-1.5,0)*{\invisiblemark};
(17.5,0)*{\invisiblemark}}
\quad\to_\frc\quad\cdots\quad.
\]
\afnegspace
We see that if a contraction vertex belongs to an $\ai$-cycle, reductions by $\frc$ make it `bounce' in the $\ai$-cycle and create a trail; while bouncing, the vertex alternates between contraction and cocontraction; if we assign a polarity to the flow, the vertex alternates between being positive and negative.
\end{remark}


%---------------------------------------
\begin{theorem}\label{theorem:RewritingSystemContractionTerminating}
Flow rewriting system\/ $\frc$ is terminating on the set of cycle-free atomic flows.
\end{theorem}

%---------------------------------------
\begin{theorem}\label{theorem:RewritingSystemWeakeningTerminating}
Flow rewriting system\/ $\frw$ is terminating.
\end{theorem}

%---------------------------------------
\begin{proof}
At every reduction, either the number of vertices decreases, or it stays the same but the number of contraction and cocontraction vertices decreases.
\end{proof}

%---------------------------------------
\begin{theorem}\label{theorem:RewritingSystemContractionWeakeningConfluent}
Flow rewriting system\/ $\frw\cup\frc$ is confluent.
\end{theorem}

%=======================================
\subsection{Conservation of Paths}\label{subsection:FlowRewritePathConservation}

\TODO{Show that the number and length of `super paths' are conserved by all reduction rules.}

%========================================
\subsection{Complexity}\label{subsection:FlowRewriteComplexity}

%---------------------------------------
\begin{remark}\label{remark:RewriteContractionExponential}
Normalising by $\frc$ can blow the size of atomic flows exponentially, in particular in a situation like the following (noted by Lutz Stra{\ss}burger):
\[
\atomicflow{
(0,29  )*{\afacu{}{}{}{}{}{}};
(0,21  )*{\afacd{}{}{}{}{}{}};
(0,15.8)*{\vdots};
(0, 8  )*{\afacu{}{}{}{}{}{}};
(0, 0  )*{\afacd{}{}{}{}{}{}}}
\quad\to_\frc^\star\quad
\atomicflow{
( 0,29  )*{\afacuexsq{}{}{}{}{}{}21};
(-4,21  )*{\afacunw{}{}{}{}{}{}};
( 4,21  )*{\afacunw{}{}{}{}{}{}};
(-6,17.8)*{\vdots};
(-2,17.8)*{\vdots};
( 2,17.8)*{\vdots};
( 6,17.8)*{\vdots};
( 0, 4  )*{\afacdexsq{}{}{}{}{}{}21};
(-4,10  )*{\afacdnw{}{}{}{}{}{}};
( 4,10  )*{\afacdnw{}{}{}{}{}{}}}
\quad.
\]
In fact, if there are $n$ couples cocontraction/contraction like the two shown above on the left, then there are $2^n$ maximal $\ai$-paths, and their number (and length) is conserved by $\to_\frc^\star$ (see Remark~\vref{remark:TODO}). Exactly one $\ai$-path passes through each edge in the middle portion of the flow on the right.
\end{remark}

\TODO{Lemma about complexity of weakening reductions.}