\chapter{Introduction}

This thesis introduces \emph{atomic flows}, a language for studying normalisation of derivations in classical propositional logic.



Using atomic flows we both describe new and generalised normal forms of derivations as well as normalisation procedures with novel proprties.

We claim that atomic flows give us a more general view of normalisation because we get more, in terms of normal forms, of which the traditional normal forms are special cases, and in terms of more normalisation procedures and more freedom in designing them; and we pay less, in the sense that we use less of the information available in the derivations in order to normalise.

Atomic flows are designed by removing all non-essential information from derivations from the point of view of describing normal forms. It turns out that all we need in order to
\begin{itemize}
\item removes information, which is why we pay less; and
\item are symmetric which is why we get more.
\end{itemize}

Atomic flows give us a huge advantage since it gives us a language to work in.

\begin{itemize}
\item we define normal forms in terms of structural information; and
\item we perform normalisation using structural information.
\end{itemize}

In the past, in natural deduction and the sequent calculus, normalisation was considered a delicate property because the rules of a system had to be designed carefully to obtain the normalisation results.

We find that normalisation is not a delicate property, on the contrary, we can take great liberties in designing our rules.
\begin{itemize}
%----
\item
In traditional proof theory, a proof is on normal form if it does not contain the cut rule. Normalisation is therefore the same as cut elimination.
%----
\item
In deep inference we have a more liberal notion of what a proof is. In particular, we have proofs from which the cut rule can not be eliminated.
%----
\item
In this thesis we present a generalisation of normal forms of proofs. In particular we hold the view that:
%----
\item
A proof is a geometric object and a proof on normal form is a proof with a certain geometric property.
%----
\item
%----
\item
\end{itemize}

In the first part of the thesis I introduce derivations and summarise some standard results we will use later on. In the second part I define atomic flows and show their relationship with derivations (see Figure~\vref{figure:ExampleAtomicFlows}), and define normal forms of derivations in terms of atomic flows (see Definition~\vref{definition:FlowNormalForms}). In the third, and final, part of the thesis we define `atomic flow reductions', which 

In  we show some examples of extracting atomic flows from derivations. Once the relationship between derivations and their associated atomic flows are understood we use atomic flows in  to describe normal forms of derivations. Finally in Definition~\vref{definition:Simplifier}, Definition~\vref{definition:IsolatedSubflowRemoval}, Definition~\vref{definition:PathBreaker}, Definition~\vref{definition:MultipleIsolatedSubflowsRemoval}, and Figure~\vref{figure:ReductionRules} we show `atomic-flow reductions' which are the basic steps with which we can build normalisation procedures.