\chapter{Introduction}

We have a more general view of normalisation because we can do
\begin{itemize}
\item more; with
\item less
\end{itemize}
Atomic flows
\begin{itemize}
\item removes information, which is why we pay less; and
\item are symmetric which is why we get more.
\end{itemize}

Atomic flows give us a huge advantage since it gives us a language to work in.

\begin{itemize}
\item we define normal forms in terms of structural information; and
\item we perform normalisation using structural information.
\end{itemize}

In the past, in natural deduction and the sequent calculus, normalisation was considered a delicate property because the rules of a system had to be designed carefully to obtain the normalisation results.

We find that normalisation is not a delicate property, on the contrary, we can take great liberties in designing our rules.
\begin{itemize}
%----
\item
In traditional proof theory, a proof is on normal form if it does not contain the cut rule. Normalisation is therefore the same as cut elimination.
%----
\item
In deep inference we have a more liberal notion of what a proof is. In particular, we have proofs from which the cut rule can not be eliminated.
%----
\item
In this thesis we present a generalisation of normal forms of proofs. In particular we hold the view that:
%----
\item
A proof is a geometric object and a proof on normal form is a proof with a certain geometric property.
%----
\item
%----
\item
\end{itemize}

In Figure~\vref{figure:ExampleAtomicFlows} we show some examples of extracting atomic flows from derivations. Once the relationship between derivations and their associated atomic flows are understood we use atomic flows in Definition~\vref{definition:FlowNormalForms} to describe normal forms of derivations. Finally in Definition~\vref{definition:Simplifier}, Definition~\vref{definition:IsolatedSubflowRemoval}, Definition~\vref{definition:PathBreaker}, Definition~\vref{definition:MultipleIsolatedSubflowsRemoval}, and Figure~\vref{figure:ReductionRules} we show `atomic-flow reductions' which are the basic steps with which we can build normalisation procedures.