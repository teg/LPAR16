\chapter{Introduction}

The mathematical proofs of journal articles and text books are usually informal arguments expressed in a natural language. In \emph{proof theory} we consider proofs to be well-defined mathematical objects expressed in some formal language, which we call a \emph{formalism}.

Since proofs are mathematical objects, it is natural to ask the question:

\emph{When are two mathematical proofs the same?}

This question was considered for inclusion in Hilbert's famous 1900 lecture \cite{Thie:03:Hilberts:yu}, and remains open today.

Formal proofs, generally speaking, are dense and terse creations that completely obscure any ingenuity that might have existed in the mind of the mathematician. \emph{Semantics} is the discipline which studies the essence, or meaning of proofs. Finding a good semantics of proofs would solve Hilbert's problem.

%\emph{abstract}

%\emph{concrete}

This thesis is part of a program whose aim it is to find the semantics of proofs. The following slogan, which I will now explain, summarises our approach:
\[
\mbox{Locality}\qquad\rightarrow\qquad\mbox{Geometry}\qquad\rightarrow\qquad\mbox{Semantics}
\]

I will no argue why we believe locality will allow us to discover the geometric essence of proofs.

The presence of `inessential details' in a proof, we call \emph{bureaucracy}. We believe that bureaucracy in proofs is what obscures their meaning and eliminating bureaucracy is what will allow us to discover their geometric essence. As an example of the kind of bureaucracy we are interested in, consider a proof that contains two independent arguments, \emph{i.e.}, the order of the arguments does not matter. If the representation of the proof contains information of the order of the arguments, this is considered bureaucracy.

% linear sequent calculus = sequent calculus without weakening and contraction

We are influenced by the success of Girard's \emph{linear logic} and \emph{proof nets} \cite{Gira:87:Linear-L:wm}. Linear logic is roughly speaking a restriction of classical logic by only allowing \emph{linear} inference rules, \emph{i.e.}, rules that do not duplicate or destroy formulae. Proof nets are geometric representations of linear logic proofs, obtained by considering proofs modulo bureaucracy.

In the same way that linearity gave proof nets in the case of linear logic, we want to find a geometric representation of proofs independent of the logic. However, not all logics can be represented with only linear rules. This motivated a generalisation of linearity to \emph{locality}. An inference rule is said to be local if the amount of information needed to verify an instance of the rule is bounded by a constant.

We can show that linear rules are local. Furthermore, \emph{atomic} rules are a second class of local rules. A rule is atomic if it only applies to atoms. We can imagine rules which are not local: In order to verify an instance of a rule that duplicates a formula, the two copies of the formula must be compared. Since the size of a formula is unbounded, the rule is not local.

Intuitively, since local rules can only depend on a limited amount of information, the `interdependence' of instances of local rules is limited. In terms of the example of bureaucracy given above, local rules allow us to recognise more independent parts of a proof.

\emph{Deep inference} \cite{Gugl:06:A-System:kl} is a methodology which allows for more general formalisms than the traditional Gentzen style. In particular, deep-inference formalisms can express classical logic using only local inference rules \cite{BrunTiu:01:A-Local-:mz}, something which is impossible in the traditional formalisms \cite{Brun:03:Two-Rest:mn}.

The slogan above then reads: Locality, made possible by deep inference, is a tool for finding the geometric essence of proofs.




\TODO{proof system}


\TODO{logical system}


---

\emph{classical logic}

\emph{normalisation}

\emph{atomic flows} \cite{GuglGund:07:Normalis:lr}

