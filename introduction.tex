\chapter{Introduction}

\begin{itemize}

%----
\item
In traditional proof theory, a proof is on normal form if it does not contain the cut rule. Normalisation is therefore the same as cut elimination.
%----
\item
In deep inference we have a more liberal notion of what a proof is. In particular, we have proofs from which the cut rule can not be eliminated.
%----
\item
In this thesis we present a generalisation of normal forms of proofs. In particular we hold the view that:
%----
\item
A proof is a geometric object and a proof on normal form is a proof with a certain geometric property.
%----
\item
%----
\item
\end{itemize}