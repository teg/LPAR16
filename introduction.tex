\chapter{Introduction}

Mathematical proofs found in text books and journal articles are usually informal arguments, expressed in a natural language, possibly with the aid of figures or diagrams, whose aim it is to convince mathematicians that a certain statement is true. Due to their informal nature, it is not possible to formally express what constitutes a mathematical proof, let alone study its properties.

For this reason, \emph{proof theory} studies \emph{formal proofs}, which are precisely defined and automatically verifiable proofs. A formal proof is expressed in a formal language, called a \emph{formalism}, and the rules by which a proof is constructed are called \emph{inference rules}.   

Formal proofs are not very good at conveying the idea of an argument. In fact, they are usually extremely terse, almost unreadable. This begs the question: What is the `essence', or \emph{semantics} of a proof? An answer to this question would also solve a problem older than proof theory itself:

\emph{When are two mathematical proofs the same?}

This question was considered for inclusion in Hilbert's famous 1900 lecture \cite{Thie:03:Hilberts:yu}, and is for this reason known as \emph{Hilbert's 24th problem}.

It should not be expected that a semantics of proofs will be able to answer the above question for all intents and purposes. Depending on the reason and context for asking the question, proofs might be manipulated in different ways before they are compared. This process of manipulation is called \emph{normalisation}.

% 'semantics' used to be: 'bureaucracy-free essence'

This thesis is part of a program whose aim it is to answer Hilbert's question. The main ideas of this program are: 1) proofs are essentially geometric objects; 2) their shape is their semantics; 3) normalisation manipulates the shape of proofs under geometric invariants. The following slogan, which I will now explain, summarises our approach:
\[
\mbox{Locality}\qquad\rightarrow\qquad\mbox{Geometry}\qquad\rightarrow\qquad\mbox{Semantics}\quad.
\]
The presence of `inessential details' in a proof, we call \emph{bureaucracy}. We believe that bureaucracy in proofs is what obscures their meaning and eliminating bureaucracy is what will allow us to discover their geometric essence. As an example of the kind of bureaucracy we are interested in, consider a proof that contains two independent arguments, \emph{i.e.}, the order of the arguments does not matter. If, in order to represent the proof, one has to arbitrarily choose an order, this is considered bureaucracy.

% linear sequent calculus = sequent calculus without weakening and contraction

We are influenced by the success of Girard's \emph{linear logic} and \emph{proof nets} \cite{Gira:87:Linear-L:wm}. Linear logic is, roughly speaking, a restriction of classical logic by only allowing \emph{linear} inference rules, \emph{i.e.}, rules that do not duplicate or destroy formulae. Proof nets are geometric representations of linear logic proofs, which identify proofs modulo bureaucracy.

In the same way that linearity gave proof nets in the case of linear logic, we want to find a geometric representation of proofs independent of the logic. However, not all logics can be represented with only linear rules. This motivated a generalisation of linearity to \emph{locality}. An inference rule is said to be local if the amount of information needed to verify an instance of the rule is bounded by a constant.

We can show that linear rules are local. Furthermore, \emph{atomic} rules are a second class of local rules. A rule is atomic if it only applies to atoms. We can imagine rules which are not local: In order to verify an instance of a rule that duplicates a formula, the two copies of the formula must be compared. Since the size of a formula is unbounded, the rule is not local.

Intuitively, since local rules can only depend on a limited amount of information, the `interdependence' of instances of local rules is limited. In terms of the example of bureaucracy given above, local rules allow us to recognise more independent parts of a proof.

\emph{Deep inference} \cite{Gugl:06:A-System:kl} is a methodology which allows for more general formalisms than Gentzen's traditional \emph{natural deduction} and \emph{sequent calculus} \cite{Gent:69:Investig:xi}. In particular, deep-inference formalisms can express classical logic using only local inference rules \cite{BrunTiu:01:A-Local-:mz}, something which is impossible in the traditional formalisms \cite{Brun:03:Two-Rest:mn}.

The slogan above then reads: Locality, made possible by deep inference, is a tool for finding the geometric essence of proofs.

The contribution of this thesis is a general view of normalisation through a geometric language we call \emph{atomic flows} \cite{GuglGund:07:Normalis:lr}.

As for natural deduction and the sequent calculus, we intend normalisation as eliminating a certain kind of inference rule, called \emph{cut}, or, more in general, we eliminate `detours', from derivations. However, unlike in natural deduction or in the sequent calculus, derivations in deep-inference formalisms are symmetric in the premiss-conclusion axis, and `detours' turn out to be more than just the existence of cuts.

Atomic flows were motivated by wanting to generalise traditional normalisation, based on elimination of inference rules, to a new notion of normalisation based on elimination of dependencies between inference rule instances. In order to formalise the notion of `detours', atomic flows were defined by discarding from derivations all but the information about casual relations between creation and destruction of atoms.

We claim that atomic flows give us a more general view of normalisation because they gives us new normal forms, of which the traditional normal forms are special cases; at the same time they show that we need less of the information available to us in order to obtain normalisation.

Due to the top-down symmetry, and the generalisation of `detours', atomic flows allow us to describe a range of new normal forms. In the special cases where `cut elimination' makes sense, the new normal forms coincide with the traditional ones, as expected.

Conventional wisdom teaches us that in order to achieve normalisation a certain \emph{harmony} between the logical inference rules is needed. Atomic flows discard all information about logical relations and logical rules, so we found it surprising that the information contained in atomic flows is sufficient to design normalisation procedures. The locality provided by deep inference is what makes this simplification possible.

In fact, the principal advantage of atomic flows is that it allows us to use the graphical language of atomic flows to describe the gist of our results. Once the language is mastered, the technical details of most of our results can easily be reconstructed based on the illustrations alone.

The thesis is split in three parts. In the first part I introduce the deep-inference formalism we will be using and classical propositional logic. In the second part I define atomic flows, show how flows are related to derivations and define normal forms of derivations in terms of their associated atomic flows. Finally, in the last part I define several normalisation procedures based on rewriting atomic flows.