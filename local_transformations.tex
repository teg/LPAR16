\chapter{Local Transformations}

\begin{definition}
In Figure~\ref{FigRed}, we define graphical expressions of the kind $r\colon\phi'\to\psi'$, where $r$ is a name and $\phi'$ and $\psi'$ are flows.
\end{definition}

\newcommand{\rwdcd}{{{\mathsf w}{\downarrow}{\hbox{-}}{\mathsf c}{\downarrow}}}
\newcommand{\rwdiu}{{{\mathsf w}{\downarrow}{\hbox{-}}{\mathsf i}{\uparrow  }}}
\newcommand{\rwdwu}{{{\mathsf w}{\downarrow}{\hbox{-}}{\mathsf w}{\uparrow  }}}
\newcommand{\rwdcu}{{{\mathsf w}{\downarrow}{\hbox{-}}{\mathsf c}{\uparrow  }}}
\newcommand{\rcuwu}{{{\mathsf c}{\uparrow  }{\hbox{-}}{\mathsf w}{\uparrow  }}}
\newcommand{\rcdwu}{{{\mathsf c}{\downarrow}{\hbox{-}}{\mathsf w}{\uparrow  }}}
\newcommand{\rcdiu}{{{\mathsf c}{\downarrow}{\hbox{-}}{\mathsf i}{\uparrow  }}}
\newcommand{\rcdcu}{{{\mathsf c}{\downarrow}{\hbox{-}}{\mathsf c}{\uparrow  }}}
\newcommand{\ridwu}{{{\mathsf i}{\downarrow}{\hbox{-}}{\mathsf w}{\uparrow  }}}
\newcommand{\ridcu}{{{\mathsf i}{\downarrow}{\hbox{-}}{\mathsf c}{\uparrow  }}}
%---------------------------------------
\begin{figure}[tbp]
\[
\begin{array}{@{}c@{}c@{}}
%-------------------
\rwdcd\colon\quad\afraise{\atomicflow{
( 0  ,0)*{\afacd{}{}{}\one{}\two};
(-2  ,4)*{\afawdnw{}{}};
(-3.5,0)*{\invisiblemark};
( 3.5,0)*{\invisiblemark}}}
\quad\to\quad
\atomicflow{
( 0  ,3.3)*{\aflabelright{\one,\two}};
( 0  ,3  )*{\afvj6};
(-1.5,0  )*{\invisiblemark};
( 3  ,0  )*{\invisiblemark}}
&\qquad
%-------------------
\rcuwu\colon\quad\aflower{\atomicflowinv{
( 0  ,0)*{\afacu{}{}{}\two{}\one};
(-2  ,6)*{\afawunw{}{}};
( 3.5,0)*{\invisiblemark}}}
\quad\to\quad
\atomicflow{
(0,3  )*{\afvj6};
(0,3.3)*{\aflabelright{\one,\two}};
(3,0  )*{\invisiblemark}}
\\
%-------------------
\rwdiu\colon\quad\atomicflow{
( 0  ,0)*{\afaiu{}{}{}\one{}{}};
(-2  ,4)*{\afawdnw{}{}};
(-3.5,0)*{\invisiblemark};
( 3.5,0)*{\invisiblemark}}
\quad\to\quad
\aflower{\atomicflow{
( 0  ,0)*{\afawu{}{}{}\one};
(-1.5,0)*{\invisiblemark};
( 1.5,0)*{\invisiblemark}}}
&\qquad
%-------------------
\ridwu\colon\quad\atomicflowinv{
( 0  ,0)*{\afaid{}{}{}\one{}{}};
(-2  ,6)*{\afawunw{}{}};
( 3.5,0)*{\invisiblemark}}
\quad\to\quad
\afraise{\atomicflowinv{
(0  ,0)*{\afawd{}{}{}\one};
(1.5,0)*{\invisiblemark}}}
\\
%-------------------
\multispan2{\hfil$
\rwdwu\colon\quad\atomicflow{
( 0  ,6)*{\afawd{}{}{}{}};
( 0  ,0)*{\afawunw{}{}{}{}};
(-1.5,0)*{\invisiblemark};
( 1.5,0)*{\invisiblemark}}
\quad\to\quad
\atomicflow{}
$\hfil}\\
%-------------------
\rwdcu\colon\quad\afraise{\atomicflow{
( 0  ,-4)*{\afacu\one{}{}\two{}{}};
( 0  , 0)*{\afawdnw{}{}};
(-3.5, 0)*{\invisiblemark};
( 3.5, 0)*{\invisiblemark}}}
\quad\to\quad
\atomicflow{
(-2  ,-4)*{\afawd{}{}\one{}};
( 2  ,-4)*{\afawd{}{}{}\two};
(-3.5, 0)*{\invisiblemark};
( 3.5, 0)*{\invisiblemark}}
&\qquad
%-------------------
\rcdwu\colon\quad\aflower{\atomicflowinv{
( 0  ,-4)*{\afacd\one{}{}\two{}{}};
( 0  , 2)*{\afawunw{}{}};
(-3.5, 0)*{\invisiblemark};
( 3.5, 0)*{\invisiblemark}}}
\quad\to\quad
\atomicflowinv{
(-2  ,-4)*{\afawu{}{}\one{}};
( 2  ,-4)*{\afawu{}{}{}\two};
(-3.5, 0)*{\invisiblemark};
( 3.5, 0)*{\invisiblemark}}
\\
%-------------------
\rcdiu\colon\quad\aflower{\atomicflow{
(   6, 6)*{\afvjd4{}\three{}{}};
(   3, 0)*{\afaiuex{}{}{}{}{}{}32};
(   0, 4)*{\afacdnw\one{}{}\two};
(-3.5, 0)*{\invisiblemark};
( 7.5, 0)*{\invisiblemark}}}
\quad\to\quad
\aflower{\atomicflow{
(  10,8)*{\afacu{}{}{}{}{}\three};
(   0,8)*{\afvjd8\one{}};
(   4,8)*{\afvjd8{}\two};
(   6,2)*{\afaiunw{}{}};
(   6,0)*{\afaiuex{}{}{}{}{}{}31}}}
&\qquad
%-------------------
\ridcu\colon\quad\afraise{\atomicflowinv{
(   6,6)*{\afvju4{}\three{}{}};
(   3,0)*{\afaidex {}{}{}{}{}{}32};
(   0,6)*{\afacunw\one{}{}\two};
(-3.5,0)*{\invisiblemark};
( 7.5,0)*{\invisiblemark}}}
\quad\to\quad
\afraise{\atomicflowinv{
(  10,8)*{\afacd{}{}{}{}{}\three};
(   0,8)*{\afvju8\one{}};
(   4,8)*{\afvju8{}\two};
(   6,4)*{\afaidnw{}{}};
(   6,0)*{\afaidex{}{}{}{}{}{}31}}}
\\
%-------------------
\multispan2{\hfil$
\rcdcu\colon\quad\atomicflow{
( 0,6)*{\afacd\one{}{}\two{}{}};
( 0,0)*{\afacunw\three{}{}\four};
(-4,0)*{\invisiblemark};
( 4,0)*{\invisiblemark}}
\quad\to\quad
\atomicflow{
(0,12)*{\afacu{}{}{}{}\one{}};
(6,12)*{\afacu{}{}{}{}{}\two};
( 0,0)*{\afacd{}{}{}{}\three{}};
( 6,0)*{\afacd{}{}{}{}{}\four};
(-2,6)*{\afvj4};
( 8,6)*{\afvj4};
( 3,6)*{\afex24}}
$\hfil}\\
\end{array}
\]
\caption{Atomic-flow reduction rules.}
\label{FigRed}
\end{figure}%

%---------------------------------------
\begin{example}
The `reduction' on the left, when used inside a larger atomic flow, might create a situation as on the right:
\nopagebreak[4]\medskip\afnegspace
\[
\atomicflow{
( 0  ,0)*{\afacu{}{}{}{}{}{}};
( 0  ,4)*{\afawdnw{}{}}}
\quad\to\quad
\atomicflow{
( 0  ,0)*{\afaid{}{}{}{}{}{}}}
\qquad\qquad
\atomicflow{
( 0  , 8)*{\afacu{}{}{}{}\ppl{}};
( 0  ,12)*{\afawdnw{}{}};
(-2  , 4)*{\aflabelleft\ppl};
( 2  , 4)*{\aflabelright\ppl};
( 0  , 0)*{\afacd{}{}{}{}\ppl{}};
(-3.5, 0)*{\invisiblemark};
( 3.5, 0)*{\invisiblemark}}
\quad\to\quad
\atomicflow{
( 0  ,4)*{\afaidnw{}{}};
( 0  ,0)*{\afacd\ppl{}{}{\scriptstyle?}\ppl{}};
(-3.5,0)*{\invisiblemark};
( 3.5,0)*{\invisiblemark}}
\quad,
\] 
where the graph at the right is not an atomic flow, for lack of a polarity assignment.
\end{example}

This prompts us to define reduction rules and reductions for atomic flows as follows.

%---------------------------------------
\begin{definition}
An (\emph{atomic-flow}) \emph{reduction rule $r$ from flow $\phi'$ to flow $\psi'$} is a quadruple $(\phi',\psi',f,g)$ such that:
\begin{enumerate}
\item $f$ is a one-to-one map from the upper edges of $\phi'$ to the upper edges of $\psi'$,
\item $g$ is a one-to-one map from the lower edges of $\phi'$ to the lower edges of $\psi'$,
\item for every polarity assignment $\pi$ for $\phi'$, there is a polarity assignment $\pi'$ for $\psi'$ such that $\pi'(f(\epsilon))=\pi(\epsilon)$ and $\pi'(g(\epsilon'))=\pi(\epsilon')$, for any upper edge $\epsilon$ and any lower edge $\epsilon'$ of $\phi'$;
\end{enumerate}
we define reduction rules with graphical expressions $r\colon\phi'\to\psi'$, where $f$ and $g$ are indicated by labelling edges. A binary relation $R$ on the set of atomic flows is called an (\emph{atomic-flow}) \emph{reduction} if, whenever $\phi\mathrel{R}\psi$, there is a one-to-one map from the upper edges of $\phi$ to the upper edges of $\psi$ and a one-to-one map from the lower edges of $\phi$ to the lower edges of $\psi$. For every reduction rule $r\colon\phi'\to\psi'$, the reduction ${\to_r}$ is defined, such that $\phi\to_r\psi$ if and only if $\phi'$ appears as a subgraph in $\phi$ and we obtain $\psi$ by replacing $\phi'$ with $\psi'$ in $\phi$, while respecting the correspondence of edges; we call this operation a \emph{reduction by $r$}.
\end{definition}

\begin{remark}
The condition on polarity assignments for a reduction rule $r$ guarantees that the $\psi$ in $\phi\to_r\psi$ is a proper atomic flow, if $\phi$ is one.
\end{remark}

\begin{remark}
Because of the condition on polarity assignments for reduction rules, two distinct connected components in a flow cannot be connected by a reduction. To see that this is impossible, consider the following `reduction rule', which violates the condition on polarity assignments:
\[
\aflower{\atomicflow{
(-2,0)*{\afawu{}{}{}{}};
( 2,0)*{\afawu{}{}{}{}}}}
\quad\to\quad
\aflower{\atomicflow{
( 0  , 2)*{\afaiu{}{}{}{}{}{}}}}
\quad.
\]
\afnegspace
For this `reduction rule' there exist both valid (left) and invalid (right) polarity assignments:
\[
\aflower{\atomicflow{
(-2  ,0)*{\afawu{}{}\ppl{}};
( 2  ,0)*{\afawu{}{}{}\pmi};
(-3.5,0)*{\invisiblemark};
( 3.5,0)*{\invisiblemark}}}
\quad\to\quad
\aflower{\atomicflow{
( 0  ,2)*{\afaiu\ppl{}{}\pmi{}{}};
(-3.5,0)*{\invisiblemark};
( 3.5,0)*{\invisiblemark}}}
\qquad\qquad
\aflower{\atomicflow{
(-2,0)*{\afawu{}{}\ppl{}};
( 2,0)*{\afawu{}{}{}\ppl};
(-3.5,0)*{\invisiblemark};
( 3.5,0)*{\invisiblemark}}}
\quad\to\quad
\aflower{\atomicflow{
( 0  , 2)*{\afaiu\ppl{}{}{\scriptstyle?}{}{}};
(-3.5, 0)*{\invisiblemark};
( 3.5, 0)*{\invisiblemark}}}
\quad.
\]
\afnegspace
\end{remark}

It is immediate to check:

\begin{proposition}
The graphical expressions in Figure~\ref{FigRed} are atomic-flow reduction rules.
\end{proposition}

%---------------------------------------
\begin{definition}
A finite set of reduction rules is a \emph{flow rewriting system}. For every flow rewriting system $F=\{r_1,\dots,r_h\}$ we define ${\to_F}={\to_{r_1}\cup\cdots\cup{\to_{r_h}}}$. The reflexive transitive closure of $\to_F$ is denoted by $\to_F^\star$. Given a set of atomic flows $S$, we say that a flow rewriting system $F$ is \emph{terminating on $S$} if there is no infinite chain $\phi_1\to_F\phi_2\to_F\cdots$, for every $\phi_1\in S$; if $F$ is terminating on the set of atomic flows, we say that it is \emph{terminating}. We say that atomic flow $\phi$ is \emph{normal} for flow rewriting system $F$ if there is no atomic flow $\psi$ such that $\phi\to_F\psi$.
\end{definition}

\newcommand{\frw}{{\mathsf w}}
%---------------------------------------
\begin{definition}
The following flow rewriting system is called $\frw$:
\[
\{\;\rwdcd\;,\;\rcuwu\;,\;\rwdiu\;,\;\ridwu\;,\;\rwdwu\;,\;\rwdcu\;,\;\rcdwu\;\}
\quad.
\]
\end{definition}

\newcommand{\frc}{{\mathsf c}}
%---------------------------------------
\begin{definition}
The following flow rewriting system is called $\frc$:
\[
\{\;\rcdiu\;,\;\ridcu\;,\;\rcdcu\;\}\quad.
\]
\end{definition}

\section{Soundness}

%---------------------------------------
\begin{definition}
A reduction $R$ is \emph{sound} if, for every $\phi$ and $\psi$ such that $\phi\mathrel{R}\psi$ and for every derivation $\Phi$ with atomic flow $\phi$, there is a derivation $\Psi$ with atomic flow $\psi$ such that $\Phi$ and $\Psi$ have the same premiss and conclusion; in this case we write $\Phi\mathrel{R}\Psi$. A reduction rule $r$ is \emph{sound} if $\to_r$ is sound.
\end{definition}

The proof of the following theorem is essentially contained in Figures~\ref{FigRedW} and \ref{FigRedC}.

%TODO: define \ot (should use \shortleftarrow, but it gives errors)
\newcommand{\ot}{\mathbin\leftarrow}

%---------------------------------------
\begin{figure}[tbp]
\[
\begin{array}{@{}l@{}c@{}}
%---------------------------------------
\rwdcd\colon\hfil\afraise{\atomicflow{
( 0  ,0)*{\afacd\three{}{}\one{}\two};
(-2  ,4)*{\afawdnw{}{}};
(-3.5,0)*{\invisiblemark};
( 3.5,0)*{\invisiblemark}}}
\quad\to\quad
\atomicflow{
( 0  ,3.3)*{\aflabelright{\one,\two}};
( 0  ,3  )*{\afvj6};
(-1.5,0  )*{\invisiblemark};
( 3  ,0  )*{\invisiblemark}}
&\qquad
\vlder{\Phi}{}{\zeta\left\{\vlinf{}{}{a^\two}{\vls[a^\three.a^\one]}\right\}}
              {\xi\left\{\vlinf{}{}{a^\three}{\fff}\right\}}
\quad\to_\rwdcd\quad
\vlder{\Phi\{a^\three\ot\fff\}}{}{\zeta\left\{\vlinf{=}{}{a^{\one,\two}}{\vls[\fff.a^{\one,\two}]}\right\}}
              {\xi\{\fff\}}
\\
\noalign{\bigskip}
%---------------------------------------
\rwdiu\colon\hfil\atomicflow{
( 0  ,0)*{\afaiu\two {}{}\one{}{}};
(-2  ,4)*{\afawdnw{}{}};
(-3.5,0)*{\invisiblemark};
( 3.5,0)*{\invisiblemark}}
\quad\to\quad
\aflower{\atomicflow{
( 0  ,0)*{\afawu{}{}{}\one};
(-1.5,0)*{\invisiblemark};
( 1.5,0)*{\invisiblemark}}}
&\qquad
\vlder{\Phi}{}{\zeta\left\{\vlinf{}{}{\fff}{\vls(a^\two.\bar a^\one)}\right\}}
              {\xi\left\{\vlinf{}{}{a^\two}{\fff}\right\}}
\quad\to_\rwdiu\quad
\vlder{\Phi\{a^\two\ot\fff\}}{}{\zeta\left\{\vlinf{=}{}{\fff}{\vls(\fff\;.\;\vlinf{}{}{\ttt}{\bar a^\one})}\right\}}
              {\xi\{\fff\}}
\\
\noalign{\bigskip}
%---------------------------------------
\rwdwu\colon\hfil\atomicflow{
( 0  ,6)*{\afawd{}{}{}\one};
( 0  ,0)*{\afawunw{}{}{}{}};
(-1.5,0)*{\invisiblemark};
( 1.5,0)*{\invisiblemark}}
\quad\to\quad
\atomicflow{}
&\qquad
\vlder{\Phi}{}{\zeta\left\{\vlinf{}{}{\ttt}{a^\one}\right\}}
              {\xi\left\{\vlinf{}{}{a^\one}{\fff}\right\}}
\quad\to_\rwdwu\quad
\vlder{\Phi\{a^\one\ot\fff\}}{}{\zeta\left\{\vlderivation
                           {
                            \vlin{=}{}{\ttt}
                            {
                             \vlin{\swi}{}{\vls[(\fff.\fff).\ttt]}
                             {
                              \vlin{=}{}{\vls(\fff.[\fff.\ttt])}
                              {
                               \vlhy{\fff}
                              }
                             }
                            }
                           }\right\}}
              {\xi\{\fff\}}
\\
\noalign{\bigskip}
%---------------------------------------
\rwdcu\colon\hfil\afraise{\atomicflow{
( 0  ,-4)*{\afacu\one{}{}\two{}\three};
( 0  , 0)*{\afawdnw{}{}};
(-3.5, 0)*{\invisiblemark};
( 3.5, 0)*{\invisiblemark}}}
\quad\to\quad
\atomicflow{
(-2  ,-4)*{\afawd{}{}\one{}};
( 2  ,-4)*{\afawd{}{}{}\two};
(-3.5, 0)*{\invisiblemark};
( 3.5, 0)*{\invisiblemark}}
&\qquad
\vlder{\Phi}{}{\zeta\left\{\vlinf{}{}{\vls(a^\one.a^\two)}{a^\three}\right\}}
              {\xi\left\{\vlinf{}{}{a^\three}{\fff}\right\}}
\quad\to_\rwdcu\quad
\vlder{\Phi\{a^\three\ot\fff\}}{}{\zeta\left\{\vlinf{=}{}{\vls(\vlinf{}{}{a^\one}{\fff}\;.\;\vlinf{}{}{a^\two}{\fff})}{\fff}\right\}}
              {\xi\{\fff\}}
\\
%---------------------------------------
\end{array}
\]
\caption{`Downwards' reduction rules for weakening and their soundness.}
\label{FigRedW}
\end{figure}%

\newbox\ContDownIntUp
\setbox\ContDownIntUp=
\hbox{$
\vlderivation
{
 \vlin{}{}{\fff}
 {
  \vlin{=}{}{\vlsbr(\vlinf{\swi}{}{\vlinf{=}{}{a^\one}{\vlsbr[\vlinf{}{}{\fff}{\vls(\bar a.a^\two)}\;.\;a^\one]}}{\vls(\bar a.[a^\two.a^\one])}\;\;\;.\;\;\;\bar a)}
  {
   \vlhy{\vlsbr([a^\one.a^\two]\;.\;\vlinf{}{}{\vls(\bar a.\bar a)}{\bar a^\three})}
  }
 }
}
$}

\newbox\ContDownContUp
\setbox\ContDownContUp=
\hbox{$
\vlinf{\med}{}{\vlsbr(\vlinf{}{}{a^\three}{\vls[a.a]}\;.\;\vlinf{}{}{a^\four}{\vls[a.a]})}
              {\vlsbr[\vlinf{}{}{\vls(a.a)}{a^\one}\;.\;\vlinf{}{}{\vls(a.a)}{a^\two}]}
$}
%---------------------------------------
\begin{figure}[tbp]
\[
\begin{array}{@{}l@{}c@{}}
%---------------------------------------
\rcdiu\colon\aflower{\atomicflow{
(   6, 6)*{\afvjd4{}\three{}{}};
(   3, 0)*{\afaiuex{}\four{}{}{}{}32};
(   0, 4)*{\afacdnw\one{}{}\two};
(-3.5, 0)*{\invisiblemark};
( 7.5, 0)*{\invisiblemark}}}
\quad\to\quad
\aflower{\atomicflow{
(10,8)*{\afacu{}{}{}{}{}\three};
( 0,8)*{\afvjd8\one{}};
( 4,8)*{\afvjd8{}\two};
( 6,2)*{\afaiunw{}{}};
( 6,0)*{\afaiuex{}{}{}{}{}{}31}}}
&\qquad
\vlder{\Phi}{}{\zeta\left\{\vlinf{}{}{\fff}{\vls(a^\four.\bar a^\three)}\right\}}
              {\xi\left\{\vlinf{}{}{a^\four}{\vls[a^\one.a^\two]}\right\}}
\quad\to_\rcdiu\quad
\vlder{\Phi\{a^\four\ot\vls[a^\one.a^\two]\}}{}{\zeta\left\{\box\ContDownIntUp\right\}}
                                     {\xi\left\{\vls[a^\one.a^\two]\right\}}
\\
\noalign{\bigskip}
%---------------------------------------
\rcdcu\colon\hfil\atomicflow{
( 0,6)*{\afacd\one{}{}\two{}\five};
( 0,0)*{\afacunw\three{}{}\four};
(-4,0)*{\invisiblemark};
( 4,0)*{\invisiblemark}}
\quad\to\quad
\atomicflow{
( 0,12)*{\afacu{}{}{}{}\one{}};
( 6,12)*{\afacu{}{}{}{}{}\two};
( 0, 0)*{\afacd{}{}{}{}\three{}};
( 6, 0)*{\afacd{}{}{}{}{}\four};
(-2, 6)*{\afvj4};
( 8, 6)*{\afvj4};
( 3, 6)*{\afex24}}
&\qquad
\vlder{\Phi}{}{\zeta\left\{\vlinf{}{}{\vls(a^\three.a^\four)}{a^\five}\right\}}
              {\xi\left\{\vlinf{}{}{a^\five}{\vls[a^\one.a^\two]}\right\}}
\quad\to_\rcdcu\quad
\vlder{\Phi\{a^\five\ot\vls[a^\one.a^\two]\}}{}{\zeta\left\{\box\ContDownContUp\right\}}
                                     {\xi\left\{\vls[a^\one.a^\two]\right\}}
\\
%---------------------------------------
\end{array}
\]
\caption{`Downwards' reduction rules for contraction and their soundness.}
\label{FigRedC}
\end{figure}%

%---------------------------------------
\begin{theorem}\label{TheoSound}
The reduction rules\/ $\rwdcd$, $\rwdiu$, $\rwdwu $, $\rwdcu$, $\rcdiu$, $\rcdcu$, $\rcuwu$, $\ridwu$, $\rcdwu$ and\/ $\ridcu$ are sound.
\end{theorem}

\begin{proof}
For $r\in\{\rwdcd,\rwdiu,\rwdwu,\rwdcu,\rcdiu,\rcdcu\}$ and $r\colon\phi'\to\psi'$ as in the left columns of Figures~\ref{FigRedW} and \ref{FigRedC}, for every $\phi$ and $\psi$ such that $\phi\to_r\psi$ and for every $\Phi$ with flow $\phi$, the right columns of the tables provide reductions $\Phi\to_r\Psi$, where $\Psi$ has flow $\psi$, as follows. If $\Phi'\to_r\Psi'$ is the reduction provided by the table, then
\[
\Phi=
\vlderivation              {
\vlde{\Psi_2}{}{\beta  }  {
\vlde{\Phi' }{}{\beta' } {
\vlde{\Psi_1}{}{\alpha'}{
\vlhy          {\alpha }}}}}
\qquad\hbox{and}\qquad
\Psi=
\vlderivation              {
\vlde{\Psi_2}{}{\beta  }  {
\vlde{\Psi'}{}{\beta' } {
\vlde{\Psi_1}{}{\alpha'}{
\vlhy          {\alpha }}}}}
\quad.
\]
We can deal with the remaining rules by employing dual derivations to the ones shown.
\end{proof}

%---------------------------------------
\begin{remark}\label{RemIndep}
The previous soundness theorem only depends on the switch and medial rules for the reductions in Figure~\ref{FigRedC}. Any system obtained from $\SKS$ by replacing $\swi$ and $\med$ with linear rules that can derive them would support a soundness theorem like the one above, for the same reduction rules. For example, we could think of replacing $\swi$ with the rule $\vlinf{\swi'}{}{\vls[(a.c).[b.d]]}{\vls([a.b].[c.d])}$, from which $\swi$ is derivable.
\end{remark}


\section{Non-termination}

%---------------------------------------
\begin{remark}\label{RemCycle}
Flow rewriting system $\frc$ is not terminating:
\nopagebreak[4]\medskip\afnegspace
\[
\atomicflow{
( 4  ,8)*{\afaidnw{}{}};
( 6  ,6)*{\afvj4};
( 3  ,0)*{\afaiuex{\ppl}{}{}{\pmi}{}{}32};
( 0  ,4)*{\afacd{\ppl}{}{}{}{}{}};
(-3.5,0)*{\invisiblemark};
( 7.5,0)*{\invisiblemark}}
\quad\to_\frc\quad
\atomicflow{
( 9  ,12)*{\afaidex{\pmi}{}{}{\ppl}{}{}32};
( 6  , 8)*{\afacu{}{}{}{}{}{}};
(12  , 6)*{\afvj4};
( 0  , 6)*{\afvjd4\ppl{}};
( 2  , 2)*{\afaiunw{}{}};
(10  , 2)*{\afaiunw{}{}{}{}{}{}};
(-1.5, 0)*{\invisiblemark};
(13.5, 0)*{\invisiblemark}}
\quad\to_\frc\quad
\atomicflow{
( 6  ,8)*{\afaidnw{}{}};
(14  ,8)*{\afaidnw{}{}};
( 4  ,6)*{\afvj4};
(16  ,6)*{\afvj4};
( 2  ,0)*{\afaiu{\ppl}{}{}{}{}{}};
(13  ,0)*{\afaiuex{\ppl}{}{}{\pmi}{}{}32};
(10  ,4)*{\afacd{}{}{}{}{}{}};
(-1.5,0)*{\invisiblemark};
(17.5,0)*{\invisiblemark}}
\quad\to_\frc\quad\cdots\quad.
\]
\afnegspace
We see that if a contraction vertex belongs to an $\ai$-cycle, reductions by $\frc$ make it `bounce' in the $\ai$-cycle and create a trail; while bouncing, the vertex alternates between contraction and cocontraction; if we assign a polarity to the flow, the vertex alternates between being positive and negative.
\end{remark}

\section{Termination and Confluence}
%---------------------------------------
\begin{theorem}\label{TheoWTerm}
Flow rewriting system\/ $\frw$ is terminating.
\end{theorem}

%---------------------------------------
\begin{proof}
At every reduction, either the number of vertices decreases, or it stays the same but the number of contraction and cocontraction vertices decreases.
\end{proof}

%---------------------------------------
\begin{theorem}\label{TheoCTerm}
Flow rewriting system\/ $\frc$ is terminating on the set of cycle-free atomic flows.
\end{theorem}

%---------------------------------------
\begin{theorem}\label{TheoWCConf}
Flow rewriting system\/ $\frw\cup\frc$ is confluent.
\end{theorem}