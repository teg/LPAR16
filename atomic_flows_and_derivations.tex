\chapter{Atomic Flows and Derivations}\label{chapter:FlowsAndDerivations}

\section{Extracting Flows from Derivations}\label{section:ExtractingFlowsFromDerivations}

%---------------------------------------
\begin{proposition}\label{proposition:FlowUnique}
Given an\/ $\SKS$ derivation\/ $\Phi$, there is a unique atomic flow $\phi$ such that:
\begin{enumerate}
%-------------------
\item there is a surjective map between the set of atom occurrences of\/ $\Phi$ and the set of edges of $\phi$;
%-------------------
\item there is a bijective map between the set of structural inference rule instances of\/ $\Phi$ and the set of vertices of $\phi$, such that, for each inference rule $\rho$ that maps to a vertex $\nu$, the label of $\nu$, and the map between the atom occurrences in the premiss (resp., conclusion) of $\rho$ and the upper (resp., lower) edges of $\nu$ are given below, for each possible case of the inference rules:
\[
\begin{array}{@{}ccc@{}ccc@{}}
\vlinf{\aid}{}{\vls[a^\one.{\bar a^\two}]}{\ttt}&\mbox{to\/}&
\vcenter{\afaid\one{}{}\two{}{}}
\quad,&\qquad
\vlinf{\aiu}{}{\fff}{\vls(a^\one.{\bar a^\two})}&\mbox{to\/}&
\vcenter{\afaiu\one{}{}\two{}{}}
\quad,\\
\noalign{\medskip}
\vlinf{\awd}{}{a^\one}{\fff}                    &\mbox{to\/}&
\vcenter{\afawd{}{}{}\one{}} 
\quad,&\qquad
\vlinf{\awu}{}{\ttt}{a^\one}                    &\mbox{to\/}&
\vcenter{\afawu{}{}{}\one{}}
\quad,\\
\noalign{\medskip}
\vlinf{\acd}{}{a^\three}{\vls[a^\one.a^\two]}   &\mbox{to\/}&
\vcenter{\afacd\one{}{}\two{}\three}
\quad,&\qquad
\vlinf{\acu}{}{\vls(a^\two.a^\three)}{a^\one}   &\mbox{to\/}&
\vcenter{\afacu\two{}{}\three{}\one}
\quad,\\
\end{array}
\]
where the mapping is indicated by small numerals.
%-------------------
\item for each inference rule instance of\/ $\Phi$ of kind
\[\hss
\begin{array}{@{}r@{}l@{}}
\vlinf{\swi}{}{\vls[(\alpha.\beta).\gamma]}
              {\vls(\alpha.[\beta.\gamma])}           \quad,&\qquad
\vlinf{\med}{}{\vls([\alpha.\gamma].[\beta.\delta])}
              {\vls[(\alpha.\beta).(\gamma.\delta)]}  \quad,      \\
\noalign{\smallskip}
\vlinf{\coor}{}{\vls[\beta.\alpha]}{\vls[\alpha.\beta]}\quad,\qquad
\vlinf{\coand}{}{\vls(\beta.\alpha)}{\vls(\alpha.\beta)}\quad,&\qquad
\vlinf{\asor}{}{\vls[[\alpha.\beta].\gamma]}
         {\vls[\alpha.[\beta.\gamma]]}                \quad,\qquad
\vlinf{\asand}{}{\vls(\alpha.(\beta.\gamma))}
         {\vls((\alpha.\beta).\gamma)}                \quad,      \\
\noalign{\smallskip}
\vlinf{\fffd}{}{\vls[\alpha.\fff]}{\alpha}           \quad,\qquad
\vlinf{\tttd}{}{\vls(\alpha.\ttt)}{\alpha}           \quad,&\qquad
\vlinf{\fffu}{}{\alpha}{\vls(\ttt.\alpha)}        \qquad\hbox{and\/}\qquad
\vlinf{\tttu}{}{\alpha}{\vls[\fff.\alpha]}
\end{array}
\]
all the atom occurrences in $\alpha$, $\beta$, $\gamma$ and $\delta$ in the premiss are respectively mapped to the same edges of $\phi$ as the atom occurrences in $\alpha$, $\beta$, $\gamma$ and $\delta$ in the conclusion.
\end{enumerate}
\end{proposition}

\TODO{Do we ever label $\epsilon$ with $a$?}

%---------------------------------------
\begin{definition}\label{definition:AssociatedFlow}
Given a derivation $\Phi$, we say that the unique atomic flow $\phi$ defined in Proposition~\vref{proposition:FlowUnique} is the atomic flow \emph{associated with}\index{flow!associated with derivation} the derivation $\Phi$. Sometimes, when an atom occurrence $a$ in $\Phi$ maps to an edge $\epsilon$ in $\phi$, we decorate $a$ with the label $\epsilon$ or the label $\phi$.
\end{definition}
%---------------

\begin{theorem}\label{theorem:SurjectiveDerToFlow}
Every atomic flow is associated with some derivation.
\end{theorem}

%---------------------------------------
\begin{proof}
First, we show that, for any atom $a$ and formula contexts $\xi\vlhole$ and $\zeta\vlhole$, there exists a derivation
$
\vlder{}{\{\swi,\med\}}
{
 \vls[(\xi\{\{a\}.\zeta\{\fff\}).\ttt]
}
{
 \vls[(\xi\{\{\ttt\}.\zeta\{a\}).\ttt]
}
$
, in other words we can `move' the atom $a$ from the context $\xi\vlhole$ to the context $\zeta\vlhole$ by using a derivation whose atomic flow contains no vertices:
\[
\vlderivation
{
 \vlin{=}{}{\vls[(\xi\{a\}.\zeta\{\fff\}).\ttt]}
 {
  \vlin{=}{}
  {
   \vlsbr[\vlinf{\swi}{}{\vls[(\zeta\{\fff\}.\xi\{a\}).\ttt]}{\vls(\zeta\{\fff\}.[\xi\{a\}.\ttt])}\;.\;\ttt]
  }
  {
   \vlhy
   {
    \vlsbr[
    \vlderivation
    {
     \vlin{\swi}{}{\vls[([\xi\{a\}.\ttt].\zeta\{\fff\}).\ttt]}
     {
      \vlin{=}{}{\vls([\xi\{a\}.\ttt].[\zeta\{\fff\}.\ttt])}
      {
       \vlin{\med}{}{\vls([\xi\{a\}.\ttt].[\ttt.\zeta\{\fff\}])}
       {
        \vlin{=}{}{\vls[(\xi\{a\}.\ttt).(\ttt.\zeta\{\fff\})]}
        {
         \vlin{\supers}{}
         {
          \vls[\xi\{a\}.\zeta\{\fff\}]
         }
         {
          \vlhy
          {
           \vls(\xi\{\ttt\}.\zeta\{a\})
          }
         }
        }
       }
      }
     }
    }
    \;\;\;\;\;.\;\;\;\;\;\ttt]
   }
  }
 }
}
\quad.
\]
This construction can be used repeatedly to build the derivation $\Psi$, for $h\ge0$:
\[
\vlderivation                                                             {
\vlde{\Psi}{\{\swi,\med\}}
     {\vls[(\xi\{a_1 \}\cdots\{a_h \}.\zeta\{\fff\}\cdots\{\fff\}).\ttt]}{
\vlhy{\vls[(\xi\{\ttt\}\cdots\{\ttt\}.\zeta\{a_1 \}\cdots\{a_h \}).\ttt]}}}
\quad.
\]
We can now prove the theorem by induction on the number of vertices of a given atomic flow $\phi$. The cases where $\phi$ only has zero or one vertex are trivial. Let us then suppose that $\phi$ has more than one vertex; then $\phi$ can be considered as composed of two flows $\phi_1$ and $\phi_2$, each with fewer vertices than $\phi$, as follows:
\[
\phi\;=\;
\atomicflow
{
( 2, 12)*{\afvjm4};
( 2,  6)*{\affr{10}8};
( 4,  8)*{\aflabelright{\phi_1}};
( 6, -6)*{\afvjm{16}};
%---
(-2,  0)*{\afvjd4{\epsilon_1}{}};
( 0,  0)*{\cdots};
( 2,  0)*{\afvjd4{}{\epsilon_h}};
%---
(-6,  6)*{\afvjm{16}};
(-2, -6)*{\affr{10}8};
( 0, -4)*{\aflabelright{\phi_2}};
(-2,-12)*{\afvjm4};
}\quad,
\]
where $h\ge0$ (this can possibly be done in many different ways). By the inductive hypothesis, there exist derivations $\vlder{\Phi_1}{}{\zeta\{a_1^{\epsilon_1}\}\cdots\{a_h^{\epsilon_h}\}}{\gamma}$ and $\vlder{\Phi_2}{}{\delta}{\xi\{a_1^{\epsilon_1}\}\cdots\{a_h^{\epsilon_h}\}}$ whose flows are, respectively, $\phi_1$ and $\phi_2$. Using these, we can build
\[
\vlder{\Psi}{}
{
 \vlsbr[(
 \vlder{\Phi_2}{}{\delta}{\xi\{a_1^{\epsilon_1}\}\cdots\{a_h^{\epsilon_h}\}}
 \;\;.\;\;
 \zeta\{\fff\}\cdots\{\fff\})
 \;\;.\;\;
 \ttt]
}
{
 \vlsbr[(\xi\{\ttt\}\cdots\{\ttt\}
 \;\;.\;\;
 \vlder{\Phi_1}{}{\zeta\{a_1^{\epsilon_1}\}\cdots\{a_h^{\epsilon_h}\}}{\gamma}
 )
 \;\;.\;\;
 \ttt]
}
\quad,
\]
whose flow is $\phi$.
\end{proof}

\begin{definition}\label{definiton:FlowRestriction}
Given a derivation $\Phi$ with atomic flow $\phi$, and an atom $a$, the \emph{restriction of $\phi$ to $a$}\index{flow!restriction to atom} is the largest subflow $\phi_a$ of $\phi$, such that every edge of $\phi_a$ is mapped to from $a$ or $\bar a$.
\end{definition}

%=================================================================
\section{Normal Forms of Derivations}\label{section:DerNormalForm}

\begin{definition}\label{definition:aiDecomposedForm}
Given two derivations
\[
\vlder{\Phi}{}{\beta}{\alpha}
\quad\mbox{and}\quad
\Psi\;=\;
\vlder{}{\SKS\setminus\{\aid,\aiu\}}
{
 \vlsbr[\beta\;.\;\vlinf{}{}{\fff}{\vls(b_m.\bar b_m)}\;.\;\cdots\;.\;\vlinf{}{}{\fff}{\vls(b_1.\bar b_1)}]
}
{
 \vlsbr(\vlinf{}{}{\vls[a_1.\bar a_1]}{\ttt}\;.\;\cdots\;.\;\vlinf{}{}{\vls[a_n.\bar a_n]}{\ttt}\;.\;\alpha)
}\quad,
\]
for some atoms $a_1,\dots,a_n,b_1,\dots,b_m$, such that $\Phi$ and $\Psi$ have the same atomic flow, we say that $\Psi$ is an \emph{$\ai$-decomposed form of\/ $\Phi$}\index{$\ai$-decomposed form}.
\end{definition}

\begin{convention}\label{convention:AlternativeAiDecomposedForm}
Given a derivation $\Phi$ and an $\ai$-decomposed form of $\Phi$:
\[
\vlder{}{\SKS\setminus\{\aid,\aiu\}}
{
 \vlsbr[\beta\;.\;\vlinf{}{}{\fff}{\vls(d_l.\bar d_l)}\;.\;\cdots\;.\;\vlinf{}{}{\fff}{\vls(d_1.\bar d_1)}\;.\;\vlinf{}{}{\fff}{\vls(b_m.\bar b_m)}\;.\;\cdots\;.\;\vlinf{}{}{\fff}{\vls(b_1.\bar b_1)}]
}
{
 \vlsbr(\vlinf{}{}{\vls[a_1.\bar a_1]}{\ttt}\;.\;\cdots\;.\;\vlinf{}{}{\vls[a_n.\bar a_n]}{\ttt}\;.\;\vlinf{}{}{\vls[c_1.\bar c_1]}{\ttt}\;.\;\cdots\;.\;\vlinf{}{}{\vls[c_k.\bar c_k]}{\ttt}\;.\;\alpha)
}\quad,
\]
we sometimes want to single out only some of the interaction or cut instances. We therefore also call the following, partially sequentialised, derivation an $\ai$-decomposed form of $\Phi$:
\[
\vlderivation
{
 \vlin{=}{}
 {
  \vlsbr[\beta\;.\;\vlinf{}{}{\fff}{\vls(b_m.\bar b_m)}\;.\;\cdots\;.\;\vlinf{}{}{\fff}{\vls(b_1.\bar b_1)}]
 }
 {
  \vlde{}{\SKS\setminus\{\aid,\aiu\}}
  {
   \vlsbr[\beta\;.\;\vlinf{}{}{\fff}{\vls(d_l.\bar d_l)}\;.\;\cdots\;.\;\vlinf{}{}{\fff}{\vls(d_1.\bar d_1)}\;.\;(b_m.\bar b_m)\;.\;\cdots\;.\;(b_1.\bar b_1)]
  }
  {
   \vlin{=}{}
   {
    \vlsbr([a_1.\bar a_1]\;.\;\cdots\;.\;[a_n.\bar a_n]\;.\;\vlinf{}{}{\vls[c_1.\bar c_1]}{\ttt}\;.\;\cdots\;.\;\vlinf{}{}{\vls[c_k.\bar c_k]}{\ttt}\;.\;\alpha)
   }
   {
    \vlhy
    {
     \vlsbr(\vlinf{}{}{\vls[a_1.\bar a_1]}{\ttt}\;.\;\cdots\;.\;\vlinf{}{}{\vls[a_n.\bar a_n]}{\ttt}\;.\;\alpha)
    }
   }
  }
 }
}\quad.
\]
\end{convention}
%-----------

\TODO{Alessio: The bound on the size of the $\ai$-decomposed form is clearly not tight. Is it ok?}
%----------------------------------------------
\begin{theorem}\label{theorem:aiDecomposedForm}
Given a derivation $\Phi$, an $\ai$-decomposed form of\/ $\Phi$ whose size depends at most cubically on the size of $\Phi$ can be constructed.
\end{theorem}
\begin{proof}
Using Lemma~\vref{lemma:SuperSwitch} apply, from top-to-bottom and left-to-right, the following transformations to each of the identity and cut instances in $\Phi$:
\[
\vlderivation
{
 \vlde{\Psi'}{}{\beta}
 {
  \vlde{\Psi}{}{\xi\left\{\vlinf{}{}{\vls[a.{\bar a}]}{\ttt}\right\}}
  {
   \vlhy{\alpha}
  }
 }
}\quad\rightarrow\quad
\vlderivation
{
 \vlde{\Psi'}{}{\beta}
 {
  \vlin{\ssu}{}{\xi\vlsbr[a.{\bar a}]}
  {
   \vlhy{\vlsbr(\vlinf{}{}{\vls[a.{\bar a}]}{\ttt}\;\;.\;\;\vlder{\Psi}{}{\xi\{\ttt\}}{\alpha})}
  }
 }
}\qquad\mbox{and}\qquad
\vlderivation
{
 \vlde{\Psi'}{}{\beta}
 {
  \vlde{\Psi}{}{\xi\left\{\vlinf{}{}{\fff}{\vls(a.{\bar a})}\right\}}
  {
   \vlhy{\alpha}
  }
 }
}\quad\rightarrow\quad
\vlderivation
{
 \vlin{\ssd}{}{\vlsbr[\vlder{\Psi'}{}{\beta}{\xi\{\fff\}}\;\;.\;\;\vlinf{}{}{\fff}{\vls(a.{\bar a})}]}
 {
  \vlde{\Psi}{}{\xi\vlsbr(a.{\bar a})}
  {
   \vlhy{\alpha}
  }
 }
}\quad
\]
to obtain an $\ai$-decomposed form of $\Phi$. The size of the $\ai$-decomposed form obtained in this way depends at most cubically on the size of $\Phi$, since, by Lemma~\vref{lemma:SuperSwitch}, each of the transformations increase the size of the derivation at most quadratically and the number of transformations is bound by the size of $\Phi$.
\end{proof}

\begin{remark}\label{remakr:aiDecomposedFormUnique}
The only reason to insist on performing the transformations in the proof of Theorem~\vref{theorem:aiDecomposedForm} in a certain order is to ensure that the resulting derivation is unique. The uniqueness is useful in the following definition.
\end{remark}

\begin{definition}\label{definition:TheAiDecomposedForm}
Given a derivation $\Phi$, the $\ai$-decomposed form of $\Phi$ obtained as described in the proof of Theorem~\vref{theorem:aiDecomposedForm} is called \emph{the} (\emph{canonical}) \emph{$\ai$-decomposed form of\/ $\Phi$}\index{$\ai$-decomposed form!canonical}.
\end{definition}