\chapter{Normal Forms}

%---------------------------------------
\begin{definition}\label{definition:FlowWeaklyStreamlined}
An atomic flow is \emph{weakly streamlined} if it can be represented as
\[
\atomicflow{
(-10, 11.5)*{\afvjm4};
(  0, 13.5)*{\afawdm{}{}{}{}{}{}};
%---
( -8, 5.5)*{\copy\contrup};
( -5, 5.5)*{\affr{18}8};
( -2, 5.5)*{\copy\contrdown};
%---
(-10,-2.5)*{\afaium{}{}{}{}{}{}};
(  0,   0)*{\afvjm3};
( 10, 2.5)*{\afaidm{}{}{}{}{}{}};
%---
(  2,-5.5)*{\copy\contrup};
(  5,-5.5)*{\affr{18}8};
(  8,-5.5)*{\copy\contrdown};
%---
(  0,-13.5)*{\afawum{}{}{}{}{}{}};
( 10,-11.5)*{\afvjm4};
}\quad.
\]
\end{definition}

\begin{definition}\label{definition:FlowWeaklyCutFree}
A weakly streamlined atomic flow with no upper (resp., lower) edges is called \emph{weakly cut-free} (reps., \emph{weakly interaction-free}).
\end{definition}

%---------------------------------------
\begin{definition}\label{definition:FlowStreamlined}
An atomic flow is \emph{streamlined} if it can be represented as
\[
\atomicflow{
(-10,11)*{\afvjm3};
%---
(-15, 5.5)*{\copy\contrup};
(-10, 5.5)*{\affr{28}8};
( -5, 5.5)*{\copy\contrdown};
%---
(-20,-2.5)*{\afawum{}{}{}{}{}{}};
(-10,-2.5)*{\afaium{}{}{}{}{}{}};
(  0,   0)*{\afvjm3};
( 10, 2.5)*{\afaidm{}{}{}{}{}{}};
( 20, 2.5)*{\afawdm{}{}{}{}{}{}};
%---
(  5,-5.5)*{\copy\contrup};
( 10,-5.5)*{\affr{28}8};
( 15,-5.5)*{\copy\contrdown};
%---
(  10,-11)*{\afvjm3};
}\quad.
\]
\end{definition}

\TODO{A streamlined atomic flow with no pair of upper (resp., lower) edges with an $ai$-path between them is cut (resp., axiom) free.}

%---------------------------------------
\begin{proposition}\label{proposition:FlowStreamlinedNoUpper}
A streamlined atomic flow with no upper (resp., lower) edges can be represented as
\[
\atomicflow{
( 10,2.5)*{\afaidm{}{}{}{}{}{}};
( 20,2.5)*{\afawdm{}{}{}{}{}{}};
%---
( 12,-5.5)*{\copy\contrup};
( 15,-5.5)*{\affr{18}8};
( 18,-5.5)*{\copy\contrdown};
%---
(  15,-11)*{\afvjm3};
}\quad\mbox{and}\quad
\atomicflow{
(-15,11)*{\afvjm3};
%---
(-18, 5.5)*{\copy\contrup};
(-15, 5.5)*{\affr{18}8};
(-12, 5.5)*{\copy\contrdown};
%---
(-20,-2.5)*{\afawum{}{}{}{}{}{}};
(-10,-2.5)*{\afaium{}{}{}{}{}{}};
}\quad,
\]
respectively.
\end{proposition}

%---------------------------------------
\begin{definition}\label{definition:FlowSuperStreamlined}
An atomic flow is \emph{super streamlined} if it can be represented as
\[
\atomicflow{
(-18,8.5)*{\afawum{}{}{}{}};
( -5, 11)*{\afvjm3};
%---
( -8, 5.5)*{\copy\contrup};
( -5, 5.5)*{\affr{18}8};
( -2, 5.5)*{\copy\contrdown};
%---
(-10,-2.5)*{\afaium{}{}{}{}{}{}};
(  0,   0)*{\afvjm3};
( 10, 2.5)*{\afaidm{}{}{}{}{}{}};
%---
(  2,-5.5)*{\copy\contrup};
(  5,-5.5)*{\affr{18}8};
(  8,-5.5)*{\copy\contrdown};
%---
( 5, -11)*{\afvjm3};
(18,-8.5)*{\afawdm{}{}{}{}};
}\quad.
\]
\end{definition}

%---------------------------------------
\begin{proposition}\label{proposition:FlowSuperStreamlinedNoUpper}
A super streamlined atomic flow with no upper (resp., lower) edges can be represented as
\[
\atomicflow{
( 0,2.5)*{\afaidm{}{}{}{}{}{}};
%---
(-2,-5.5)*{\copy\contrup};
( 0,-5.5)*{\affr88};
( 2,-5.5)*{\copy\contrdown};
%---
( 0,-11)*{\afvjm3};
( 8,-8.5)*{\afawdm{}{}{}{}};
}
\quad\mbox{and}\quad
\atomicflow{
(-8,8.5)*{\afawum{}{}{}{}};
( 0,11)*{\afvjm3};
%---
(-2, 5.5)*{\copy\contrup};
( 0, 5.5)*{\affr88};
( 2, 5.5)*{\copy\contrdown};
%---
( 0,-2.5)*{\afaium{}{}{}{}{}{}};
}\quad,
\]
respectively.
\end{proposition}

%---------------------------------------
\begin{definition}\label{definition:FlowHyperStreamlined}
An atomic flow is \emph{hyper streamlined} if it can be represented as
\[
\atomicflow{
(-18,8.5)*{\afawum{}{}{}{}};
( -5, 11)*{\afvjm3};
%---
( -5, 5.5)*{\copy\contrup};
( -5, 5.5)*{\affr{18}8};
%---
(-10,-2.5)*{\afaium{}{}{}{}{}{}};
(  0,   0)*{\afvjm3};
( 10, 2.5)*{\afaidm{}{}{}{}{}{}};
%---
(  5,-5.5)*{\affr{18}8};
(  5,-5.5)*{\copy\contrdown};
%---
( 5, -11)*{\afvjm3};
(18,-8.5)*{\afawdm{}{}{}{}};
}\quad.
\]
\end{definition}

%---------------------------------------
\begin{proposition}\label{proposition:FlowHyperStreamlinedNoUpper}
A hyper streamlined atomic flow with no upper (resp., lower) edges can be represented as
\[
\atomicflow{
( 0,2.5)*{\afaidm{}{}{}{}{}{}};
%---
( 0,-5.5)*{\affr88};
( 0,-5.5)*{\copy\contrdown};
%---
( 0,-11)*{\afvjm3};
( 8,-8.5)*{\afawdm{}{}{}{}};
}
\quad\mbox{and}\quad
\atomicflow{
(-8,8.5)*{\afawum{}{}{}{}};
( 0,11)*{\afvjm3};
%---
( 0, 5.5)*{\copy\contrup};
( 0, 5.5)*{\affr88};
%---
( 0,-2.5)*{\afaium{}{}{}{}{}{}};
}\quad,
\]
respectively.
\end{proposition}

\TODO{Maybe think of a different name than `simple form' (it doesn't look very simple...).}

%---------------------------------------
\begin{definition}\label{definition:FlowSimpleForm}
An atomic flow is \emph{on simple form with respect to the polarity assignment $\pi$} if it can be represented as
\[
\atomicflow{
(-21, 14)*{\afvjm4};
(-21,  8)*{\affr88};
(-22, 10)*{\aflabelleft{\ppl}};
%-
(-24,  0)*{\afvjm8};
%-
(-21, -8)*{\affr88};
(-22, -6)*{\aflabelleft{\ppl}};
(-21,-14)*{\afvjm4};
%-
( -6, 16)*{\afaidm{}{}{}{}{}{}};
( -4,  8)*{\afvjm8};
(-11,  8)*{\affr88};
(-12, 10)*{\aflabelleft{\ppl}};
(-16,  0)*{\afcjrm48};
%-
(-16,  0)*{\afcjlm48};
(-11, -8)*{\affr88};
(-12, -6)*{\aflabelleft{\ppl}};
( -4, -8)*{\afvjm8};
( -6,-16)*{\afaium{}{}{}{}{}{}};
%--
(  6,  8)*{\afaidm{}{}{}{}{}{}};
(  0, 10)*{\afvjm{12}};
(  0,  0)*{\affr{10}8};
( -2,  2)*{\aflabelleft{\pmi}};
(  0,-10)*{\afvjm{12}};
( 11,  0)*{\affr88};
( 10,  2)*{\aflabelleft{\ppl}};
(  6, -8)*{\afaium{}{}{}{}{}{}};
}\qquad
\atomicflow{
(-10, 11.5)*{\afvjm4};
(  0, 13.5)*{\afawdm{}{}{}{}{}{}};
%---
( -8, 5.5)*{\copy\contrup};
( -5, 5.5)*{\affr{18}8};
( -2, 5.5)*{\copy\contrdown};
%---
(-10,-2.5)*{\afaium{}{}{}{}{}{}};
(  0,   0)*{\afvjm3};
( 10, 2.5)*{\afaidm{}{}{}{}{}{}};
%---
(  2,-5.5)*{\copy\contrup};
(  5,-5.5)*{\affr{18}8};
(  8,-5.5)*{\copy\contrdown};
%---
(  0,-13.5)*{\afawum{}{}{}{}{}{}};
( 10,-11.5)*{\afvjm4};
}\quad.
\]
\end{definition}

%---------------------------------------
\begin{proposition}\label{proposition:FlowSimpleNoUpper}
Given a polarity assignment $\pi$, an atomic flow on simple form with respect to $\pi$ with no upper (resp., lower) edges can be represented as
\[
\atomicflow{
( -5,  8)*{\afaidm{}{}{}{}{}{}};
(-10,  0)*{\affr88};
(-11,  2)*{\aflabelleft{\ppl}};
(-10, -6)*{\afvjm4};
%--
(  5,  8)*{\afaidm{}{}{}{}{}{}};
(  0,  0)*{\affr88};
( -1,  2)*{\aflabelleft{\pmi}};
( -3, -6)*{\afvjm4};
( 10,  0)*{\affr88};
(  9,  2)*{\aflabelleft{\ppl}};
(  5, -8)*{\afaium{}{}{}{}{}{}};
}\qquad
\atomicflow{
( -5, 13.5)*{\afawdm{}{}{}{}{}{}};
%---
( -8, 5.5)*{\copy\contrup};
( -5, 5.5)*{\affr{18}8};
( -2, 5.5)*{\copy\contrdown};
%---
(-10,-2.5)*{\afaium{}{}{}{}{}{}};
(  0,   0)*{\afvjm3};
( 10, 2.5)*{\afaidm{}{}{}{}{}{}};
%---
(  2,-5.5)*{\copy\contrup};
(  5,-5.5)*{\affr{18}8};
(  8,-5.5)*{\copy\contrdown};
%---
(  0,-13.5)*{\afawum{}{}{}{}{}{}};
( 10,-11.5)*{\afvjm4};
}\quad\mbox{and}\quad
\]
\[
\atomicflow{
( -5, -8)*{\afaium{}{}{}{}{}{}};
(-10,  0)*{\affr88};
(-11,  2)*{\aflabelleft{\ppl}};
(-10,  6)*{\afvjm4};
%--
(  5, -8)*{\afaium{}{}{}{}{}{}};
(  0,  0)*{\affr88};
( -1,  2)*{\aflabelleft{\pmi}};
( -3,  6)*{\afvjm4};
( 10,  0)*{\affr88};
(  9,  2)*{\aflabelleft{\ppl}};
(  5,  8)*{\afaidm{}{}{}{}{}{}};
}\qquad
\atomicflow{
(-10, 11.5)*{\afvjm4};
(  0, 13.5)*{\afawdm{}{}{}{}{}{}};
%---
( -8, 5.5)*{\copy\contrup};
( -5, 5.5)*{\affr{18}8};
( -2, 5.5)*{\copy\contrdown};
%---
(-10,-2.5)*{\afaium{}{}{}{}{}{}};
(  0,   0)*{\afvjm3};
( 10, 2.5)*{\afaidm{}{}{}{}{}{}};
%---
(  2,-5.5)*{\copy\contrup};
(  5,-5.5)*{\affr{18}8};
(  8,-5.5)*{\copy\contrdown};
%---
(  5,-13.5)*{\afawum{}{}{}{}{}{}};
}\quad,
\]
respectively.
\end{proposition}

%---------------------------------------
\begin{definition}\label{definition:FlowWeaklyStreamlinedPolarity}
An atomic flow is \emph{weakly streamlined with respect to the polarity assignment $\pi$} if it can be represented as
\[
\atomicflow{
(-21, 14)*{\afvjm4};
(-21,  8)*{\affr88};
(-22, 10)*{\aflabelleft{\ppl}};
%-
(-24,  0)*{\afvjm8};
%-
(-21, -8)*{\affr88};
(-22, -6)*{\aflabelleft{\ppl}};
(-21,-14)*{\afvjm4};
%-
( -6, 16)*{\afaidm{}{}{}{}{}{}};
( -4,  8)*{\afvjm8};
(-11,  8)*{\affr88};
(-12, 10)*{\aflabelleft{\ppl}};
(-16,  0)*{\afcjrm48};
%-
(-16,  0)*{\afcjlm48};
(-11, -8)*{\affr88};
(-12, -6)*{\aflabelleft{\ppl}};
( -4, -8)*{\afvjm8};
( -6,-16)*{\afaium{}{}{}{}{}{}};
%--
(  2, 10)*{\afvjm{12}};
( -1,  0)*{\affr88};
( -2,  2)*{\aflabelleft{\pmi}};
(  2,-10)*{\afvjm{12}};
}\qquad
\atomicflow{
(-10, 11.5)*{\afvjm4};
(  0, 13.5)*{\afawdm{}{}{}{}{}{}};
%---
( -8, 5.5)*{\copy\contrup};
( -5, 5.5)*{\affr{18}8};
( -2, 5.5)*{\copy\contrdown};
%---
(-10,-2.5)*{\afaium{}{}{}{}{}{}};
(  0,   0)*{\afvjm3};
( 10, 2.5)*{\afaidm{}{}{}{}{}{}};
%---
(  2,-5.5)*{\copy\contrup};
(  5,-5.5)*{\affr{18}8};
(  8,-5.5)*{\copy\contrdown};
%---
(  0,-13.5)*{\afawum{}{}{}{}{}{}};
( 10,-11.5)*{\afvjm4};
}\quad.
\]
\end{definition}

\TODO{Define the inverse of a polarity assignment.}

\begin{proposition}\label{proposition:FlowWeaklyStreamlinedPolarity}
Given a polarity assignment $\pi$, an atomic flow that is weakly streamlined with respect to both $\pi$ and $\bar\pi$ is weakly streamlined. 
\end{proposition}

\begin{proposition}\label{proposition:FlowWeaklyCutFreePolarity}
Given a polarity assignment $\pi$, a weakly streamlined atomic flow with respect to $\pi$ that has no upper (resp., lower) edges is weakly cut-free (resp., weakly identity-free).
\end{proposition}

\TODO{Consider defining `ordered choice of components' somewhere else.}

\TODO{This might be more suitable somewhere else.}

\newcommand{\Core}{\mathsf{Core}}
\begin{definition}\label{definition:FlowCore}
Given an atomic flow
\[
\phi\;\;=\;\;\atomicflow
{
(-13,0)*{\affr{22}{20}};
(-5,8)*{\aflabelright{\phi_1}};
%
(-20,10)*{\afvjdm{12}{\boldsymbol\epsilon'_1}{}};
(-13, 8)*{\afaidm{}{}{}{}{}{}};
( -6,10)*{\afvjdm{12}{}{\boldsymbol\epsilon''_1}};
(-18, 0)*{\affr{8}{8}};
(-17, 2)*{\aflabelright{\phi'_1}};
( -8, 0)*{\affr{8}{8}};
( -7, 2)*{\aflabelright{\phi''_1}};
( -6,-10)*{\afvjum{12}{}{\boldsymbol\iota''_1}};
(-13,-8)*{\afaium{}{}{}{}{}{}};
(-20,-10)*{\afvjum{12}{\boldsymbol\iota'_1}{}};
%------------
(0,0)*{\cdots};
%------------
(13,0)*{\affr{22}{20}};
(21,8)*{\aflabelright{\phi_n}};
%
(20, 11)*{\afvjdm{14}{}{\boldsymbol\epsilon''_n}};
(13,  8)*{\afaidm{}{}{}{}{}{}};
( 6, 11)*{\afvjdm{14}{\boldsymbol\epsilon'_n}{}};
( 8,  0)*{\affr{8}{8}};
( 9,  2)*{\aflabelright{\phi'_n}};
(18,  0)*{\affr{8}{8}};
(19,  2)*{\aflabelright{\phi''_n}};
( 6,-11)*{\afvjum{14}{\boldsymbol\iota'_n}{}};
(13,-8)*{\afaium{}{}{}{}{}{}};
(20,-11)*{\afvjum{14}{}{\boldsymbol\iota''_n}};
}\qquad
\atomicflow{
(-10, 13.5)*{\afvjm8};
(  0, 13.5)*{\afawdm{}{}{}{}{}{}};
%---
( -8, 5.5)*{\copy\contrup};
( -5, 5.5)*{\affr{18}8};
( -2, 5.5)*{\copy\contrdown};
%---
(-10,-2.5)*{\afaium{}{}{}{}{}{}};
(  0,   0)*{\afvjm3};
( 10, 2.5)*{\afaidm{}{}{}{}{}{}};
%---
(  2,-5.5)*{\copy\contrup};
(  5,-5.5)*{\affr{18}8};
(  8,-5.5)*{\copy\contrdown};
%---
(  0,-13.5)*{\afawum{}{}{}{}{}{}};
( 10,-13.5)*{\afvjm8};
%---------
( 0, 0)*{\affr{30}{32}};
(12,14)*{\aflabelright{\psi}};
}\quad,
\]
where, for $1\le i\le n$, the subflows $\phi'_i$ and $\phi''_i$ contain no identity and no cut vertices and $\phi_i$ is a connected, non-weakly-streamlined subflow, we define the \emph{core of $\phi$}, denoted $\Core(\phi)$, to be
\[
\atomicflow
{
(-21,11)*{\afvjdm{14}{\boldsymbol\epsilon'_1}{}};
(-17,16)*{\afvjd4{}{\lambda'_1}};
(-17,10)*{\affr{6}{8}};
(-17,10)*{\copy\contrup};
(-17, 5)*{\afvjm{2}};
(-18, 0)*{\affr{8}{8}};
(-17, 2)*{\aflabelright{\phi'_1}};
(-21,-11)*{\afvjum{14}{\boldsymbol\iota'_1}{}};
(-17,-16)*{\afvju4{}{\mu'_1}};
(-17,-10)*{\affr{6}{8}};
(-17,-10)*{\copy\contrdown};
(-17, -5)*{\afvjm{2}};
%
( -9,16)*{\afvjd4{}{\lambda''_1}};
( -9,10)*{\affr{6}{8}};
( -9,10)*{\copy\contrup};
( -9, 5)*{\afvjm{2}};
( -5,11)*{\afvjdm{14}{}{\boldsymbol\epsilon''_1}};
( -8, 0)*{\affr{8}{8}};
( -7, 2)*{\aflabelright{\phi''_1}};
( -9,-16)*{\afvju4{}{\mu''_1}};
( -9,-10)*{\affr{6}{8}};
( -9,-10)*{\copy\contrdown};
( -9, -5)*{\afvjm{2}};
( -5,-11)*{\afvjum{14}{}{\boldsymbol\iota''_1}};
( -8, 0)*{\affr{8}{8}};
%------------
(0,0)*{\cdots};
%------------
( 9,16)*{\afvjd4{}{\lambda'_n}};
( 9,10)*{\affr{6}{8}};
( 9,10)*{\copy\contrup};
( 9, 5)*{\afvjm{2}};
( 5,11)*{\afvjdm{14}{\boldsymbol\epsilon'_n}{}};
( 8, 0)*{\affr{8}{8}};
( 9, 2)*{\aflabelright{\phi'_n}};
( 9,-16)*{\afvju4{}{\mu'_n}};
( 9,-10)*{\affr{6}{8}};
( 9,-10)*{\copy\contrdown};
( 9, -5)*{\afvjm{2}};
( 5,-11)*{\afvjum{14}{\boldsymbol\iota'_n}{}};
( 8, 0)*{\affr{8}{8}};
%
(21,11)*{\afvjdm{14}{}{\boldsymbol\epsilon''_n}};
(17,16)*{\afvjd4{}{\lambda''_n}};
(17,10)*{\affr{6}{8}};
(17,10)*{\copy\contrup};
(17, 5)*{\afvjm{2}};
(18, 0)*{\affr{8}{8}};
(19, 2)*{\aflabelright{\phi''_n}};
(21,-11)*{\afvjum{14}{}{\boldsymbol\iota''_n}};
(17,-16)*{\afvju4{}{\mu''_n}};
(17,-10)*{\affr{6}{8}};
(17,-10)*{\copy\contrdown};
(17, -5)*{\afvjm{2}};
}\qquad
\atomicflow{
(-10, 13.5)*{\afvjm8};
(  0, 13.5)*{\afawdm{}{}{}{}{}{}};
%---
( -8, 5.5)*{\copy\contrup};
( -5, 5.5)*{\affr{18}8};
( -2, 5.5)*{\copy\contrdown};
%---
(-10,-2.5)*{\afaium{}{}{}{}{}{}};
(  0,   0)*{\afvjm3};
( 10, 2.5)*{\afaidm{}{}{}{}{}{}};
%---
(  2,-5.5)*{\copy\contrup};
(  5,-5.5)*{\affr{18}8};
(  8,-5.5)*{\copy\contrdown};
%---
(  0,-13.5)*{\afawum{}{}{}{}{}{}};
( 10,-13.5)*{\afvjm8};
%---------
( 0, 0)*{\affr{30}{32}};
(12,14)*{\aflabelright{\psi}};
}\quad.
\]
For every $1\le i\le n$, the \emph{new upper edge of} $\phi'_i$ (resp., $\phi''_i$) \emph{in $\Core(\phi)$} is $\lambda'_i$ (resp., $\lambda''_i$), the \emph{new lower edge of} $\phi'_i$ (resp., $\phi''_i$) \emph{in $\Core{\phi}$} is $\mu'_i$ (resp., $\mu''_i$) and the \emph{new upper} (resp., \emph{lower}) \emph{edges of $\phi_i$ in $\Core{\phi}$} are $\lambda'_i$ and $\lambda''_i$ (resp., $\mu'_i$ and $\mu''_i$).
We say that $\phi_1$, $\dots$, $\phi_n$ is an \emph{ordered choice of components from $\phi$}.
\end{definition}

\newcommand{\Exp}{\mathsf{Exp}}

\begin{definition}\label{definition:FlowExperiment}
Given a polarity assignment $\pi$ and an atomic flow
\[
\phi\;\;=\;\;\atomicflow
{
(0,17.5)*{\invisiblemark};
%---
(-13,0)*{\affr{22}{20}};
(-5,8)*{\aflabelright{\phi_1}};
%
(-13, 8)*{\afaidm{}{}{}{}{}{}};
(-18, 0)*{\affr{8}{8}};
(-19, 2)*{\aflabelleft\ppl};
(-17, 2)*{\aflabelright{\phi'_1}};
( -8, 0)*{\affr{8}{8}};
( -9, 2)*{\aflabelleft\pmi};
( -7, 2)*{\aflabelright{\phi''_1}};
( -6,-11)*{\afvjum{14}{}{\boldsymbol\iota''_1}};
(-13,-8)*{\afaium{}{}{}{}{}{}};
(-20,-11)*{\afvjum{14}{\boldsymbol\iota'_1}{}};
%------------
(0,0)*{\cdots};
%------------
(13,0)*{\affr{22}{20}};
(21,8)*{\aflabelright{\phi_n}};
%
(13, 8)*{\afaidm{}{}{}{}{}{}};
( 8, 0)*{\affr{8}{8}};
( 7, 2)*{\aflabelleft\ppl};
( 9, 2)*{\aflabelright{\phi'_n}};
(18, 0)*{\affr{8}{8}};
(17, 2)*{\aflabelleft\pmi};
(19, 2)*{\aflabelright{\phi''_n}};
( 6,-11)*{\afvjum{14}{\boldsymbol\iota'_n}{}};
(13,-8)*{\afaium{}{}{}{}{}{}};
(20,-11)*{\afvjum{14}{}{\boldsymbol\iota''_n}};
}\qquad
\atomicflow{
( -5, 13.5)*{\afawdm{}{}{}{}{}{}};
%---
( -8, 5.5)*{\copy\contrup};
( -5, 5.5)*{\affr{18}8};
( -2, 5.5)*{\copy\contrdown};
%---
(-10,-2.5)*{\afaium{}{}{}{}{}{}};
(  0,   0)*{\afvjm3};
( 10, 2.5)*{\afaidm{}{}{}{}{}{}};
%---
(  2,-5.5)*{\copy\contrup};
(  5,-5.5)*{\affr{18}8};
(  8,-5.5)*{\copy\contrdown};
%---
(  0,-13.5)*{\afawum{}{}{}{}{}{}};
( 10,-13.5)*{\afvjm8};
%---------
( 0, 0)*{\affr{30}{32}};
(12,14)*{\aflabelright{\psi}};
}\quad,
\]
where, for $1\le i\le n$, $\phi_i$ is a connected, non-weakly-cut-free subflow, we define the \emph{experiment on $\phi$ with respect to $\pi$}, denoted $\Exp(\phi,\pi)$, to be
\[
\atomicflow
{
(0,17.5)*{\invisiblemark};
%-----
(-18, 6)*{\afvjdm4{}{\boldsymbol{\lambda'_1}}};
( -8, 8)*{\afawdm{}{}{}{}{}{}};
(-18, 0)*{\affr{8}{8}};
(-19, 2)*{\aflabelleft\ppl};
(-17, 2)*{\aflabelright{\phi'_1}};
( -8, 0)*{\affr{8}{8}};
( -9, 2)*{\aflabelleft\pmi};
( -7, 2)*{\aflabelright{\phi''_1}};
( -6,-11)*{\afvjum{14}{}{\boldsymbol\iota''_1}};
(-13,-8)*{\afaium{}{}{}{}{}{}};
(-20,-11)*{\afvjum{14}{\boldsymbol\iota'_1}{}};
%------------
(0,0)*{\cdots};
%------------
( 8, 6)*{\afvjdm4{}{\boldsymbol{\lambda'_n}}};
(18, 8)*{\afawdm{}{}{}{}{}{}};
( 8, 0)*{\affr{8}{8}};
( 7, 2)*{\aflabelleft\ppl};
( 9, 2)*{\aflabelright{\phi'_n}};
(18, 0)*{\affr{8}{8}};
(17, 2)*{\aflabelleft\pmi};
(19, 2)*{\aflabelright{\phi''_n}};
( 6,-11)*{\afvjum{14}{\boldsymbol\iota'_n}{}};
(13,-8)*{\afaium{}{}{}{}{}{}};
(20,-11)*{\afvjum{14}{}{\boldsymbol\iota''_n}};
}\qquad
\atomicflow{
( -5, 13.5)*{\afawdm{}{}{}{}{}{}};
%---
( -8, 5.5)*{\copy\contrup};
( -5, 5.5)*{\affr{18}8};
( -2, 5.5)*{\copy\contrdown};
%---
(-10,-2.5)*{\afaium{}{}{}{}{}{}};
(  0,   0)*{\afvjm3};
( 10, 2.5)*{\afaidm{}{}{}{}{}{}};
%---
(  2,-5.5)*{\copy\contrup};
(  5,-5.5)*{\affr{18}8};
(  8,-5.5)*{\copy\contrdown};
%---
(  0,-13.5)*{\afawum{}{}{}{}{}{}};
( 10,-13.5)*{\afvjm8};
%---------
( 0, 0)*{\affr{30}{32}};
(12,14)*{\aflabelright{\psi}};
}\quad.
\]
For every $1\le i\le n$, the \emph{new upper edges of} $\phi_i$ are $\boldsymbol{\lambda'_i}$.
\end{definition}


\begin{definition}\label{definition:FlowCore}
Given an atomic flow $\phi$ and a polarity assignment $\pi$, such that
\[
\phi\;=\;
\atomicflow{
(-21, 14)*{\afvjm4};
(-21,  8)*{\affr88};
(-22, 10)*{\aflabelleft{\ppl}};
%-
(-24,  0)*{\afvjm8};
%-
(-21, -8)*{\affr88};
(-22, -6)*{\aflabelleft{\ppl}};
(-21,-14)*{\afvjm4};
%-
( -6, 16)*{\afaidm{}{}{}{}{}{}};
( -4,  8)*{\afvjm8};
(-11,  8)*{\affr88};
(-12, 10)*{\aflabelleft{\ppl}};
(-16,  0)*{\afcjrm48};
%-
(-16,  0)*{\afcjlm48};
(-11, -8)*{\affr88};
(-12, -6)*{\aflabelleft{\ppl}};
( -4, -8)*{\afvjm8};
( -6,-16)*{\afaium{}{}{}{}{}{}};
%--
( 12, 10)*{\afaidmex{}{}{}{}{}{}{20}{4}};
(  2,  5)*{\afvjm2};
(  4,  6)*{\cdots};
( 22,  5)*{\afvjm2};
(  9,  8)*{\afaidmex{}{}{}{}{}{}{6}{4}};
(  1,  0)*{\affr{12}{8}};
( -2,  2)*{\aflabelleft{\pmi}};
( 12,  0)*{\affr{6}{8}};
( 12,  2)*{\aflabelright{\phi_1}};
( 12,  2)*{\aflabelleft{\ppl}};
( 17,  0)*{\cdots};
( 22,  0)*{\affr{6}{8}};
( 22,  2)*{\aflabelright{\phi_n}};
( 22,  2)*{\aflabelleft{\ppl}};
(  9, -8)*{\afaiumex{}{}{}{}{}{}{6}{4}};
(  2, -5)*{\afvjm2};
(  4, -6)*{\cdots};
( 22, -5)*{\afvjm2};
( 12,-10)*{\afaiumex{}{}{}{}{}{}{20}{4}};
}\qquad
\atomicflow{
(-10, 13.5)*{\afvjm8};
(  0, 13.5)*{\afawdm{}{}{}{}{}{}};
%---
( -8, 5.5)*{\copy\contrup};
( -5, 5.5)*{\affr{18}8};
( -2, 5.5)*{\copy\contrdown};
%---
(-10,-2.5)*{\afaium{}{}{}{}{}{}};
(  0,   0)*{\afvjm3};
( 10, 2.5)*{\afaidm{}{}{}{}{}{}};
%---
(  2,-5.5)*{\copy\contrup};
(  5,-5.5)*{\affr{18}8};
(  8,-5.5)*{\copy\contrdown};
%---
(  0,-13.5)*{\afawum{}{}{}{}{}{}};
( 10,-13.5)*{\afvjm8};
%---------
( 0, 0)*{\affr{30}{32}};
(12,14)*{\aflabelright{\psi}};
}\quad,
\]
where, for $1\le i\le n$, the subflow $\phi_i$ is a connected subflow, we define the \emph{core of $\phi$ with respect to $\pi$}, denoted $\Core(\phi,\pi)$, to be
\[
\atomicflow{
(-21, 14)*{\afvjm4};
(-21,  8)*{\affr88};
(-22, 10)*{\aflabelleft{\ppl}};
%-
(-24,  0)*{\afvjm8};
%-
(-21, -8)*{\affr88};
(-22, -6)*{\aflabelleft{\ppl}};
(-21,-14)*{\afvjm4};
%-
( -6, 16)*{\afaidm{}{}{}{}{}{}};
( -4,  8)*{\afvjm8};
(-11,  8)*{\affr88};
(-12, 10)*{\aflabelleft{\ppl}};
(-16,  0)*{\afcjrm48};
%-
(-16,  0)*{\afcjlm48};
(-11, -8)*{\affr88};
(-12, -6)*{\aflabelleft{\ppl}};
( -4, -8)*{\afvjm8};
( -6,-16)*{\afaium{}{}{}{}{}{}};
%--
( 10, 10)*{\affr{6}{8}};
( 10, 10)*{\copy\contrup};
( 10,-10)*{\affr{6}{8}};
( 10,-10)*{\copy\contrdown};
(  2, 10)*{\affr{6}{8}};
(  2, 10)*{\copy\contrup};
(  2,-10)*{\affr{6}{8}};
(  2,-10)*{\copy\contrdown};
(  2, 16)*{\afvj4};
(  2,  5)*{\afvjm2};
(  6,5.5)*{\cdots};
( 10, 16)*{\afvj4};
( 10,  5)*{\afvjm2};
( 18, 16)*{\afvjd4{}{\lambda_1}};
( 28, 16)*{\afvjd4{}{\lambda_n}};
( 18,  5)*{\afvjm2};
( 28,  5)*{\afvjm2};
(  4,  0)*{\affr{18}{8}};
( -2,  2)*{\aflabelleft{\pmi}};
( 18, 10)*{\affr{6}{8}};
( 18, 10)*{\copy\contrup};
( 18,  0)*{\affr{6}{8}};
( 18,-10)*{\affr{6}{8}};
( 18,-10)*{\copy\contrdown};
( 18,  2)*{\aflabelright{\phi_1}};
( 18,  2)*{\aflabelleft{\ppl}};
( 23,  0)*{\cdots};
( 28, 10)*{\affr{6}{8}};
( 28, 10)*{\copy\contrup};
( 28,  0)*{\affr{6}{8}};
( 28,-10)*{\affr{6}{8}};
( 28,-10)*{\copy\contrdown};
( 28,  2)*{\aflabelright{\phi_n}};
( 28,  2)*{\aflabelleft{\ppl}};
(  2, -5)*{\afvjm2};
(  2,-16)*{\afvj4};
(  6, -5)*{\cdots};
( 10,-16)*{\afvj4};
( 10, -5)*{\afvjm2};
( 18, -5)*{\afvjm2};
( 28, -5)*{\afvjm2};
( 18,-16)*{\afvju4{}{\mu_1}};
( 28,-16)*{\afvju4{}{\mu_n}};
}\qquad
\atomicflow{
(-10, 13.5)*{\afvjm8};
(  0, 13.5)*{\afawdm{}{}{}{}{}{}};
%---
( -8, 5.5)*{\copy\contrup};
( -5, 5.5)*{\affr{18}8};
( -2, 5.5)*{\copy\contrdown};
%---
(-10,-2.5)*{\afaium{}{}{}{}{}{}};
(  0,   0)*{\afvjm3};
( 10, 2.5)*{\afaidm{}{}{}{}{}{}};
%---
(  2,-5.5)*{\copy\contrup};
(  5,-5.5)*{\affr{18}8};
(  8,-5.5)*{\copy\contrdown};
%---
(  0,-13.5)*{\afawum{}{}{}{}{}{}};
( 10,-13.5)*{\afvjm8};
%---------
( 0, 0)*{\affr{30}{32}};
(12,14)*{\aflabelright{\psi}};
}\quad.
\]
For every $1\le i\le n$, the \emph{new upper} (resp., \emph{lower}) \emph{edge of $\phi_i$ in $\Core{\phi,\pi}$} is $\lambda_i$ (resp., $\mu_i$). We say that $\phi_1$, $\dots$, $\phi_n$ is an \emph{ordered choice of components from $\phi$ with respect to $\pi$}.
\end{definition}

