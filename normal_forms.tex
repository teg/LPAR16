\chapter{Normal Forms}\label{chapter:NormalForms}

Intro:
\begin{itemize}
\item Sequent Calculus: normal form means `cut free'. Not topological. Syntactic.
\item To get syntax independence of proofs we need syntax independence of normalisation.
\item in deep inference we have locality and top-down symmetry.
\begin{itemize}
\item locality gives topology; and
\item symmetry gives generalisation of cut-elimination.
\end{itemize}
The minimal ammount of information needed to describe these things: Atomic Flows.
\end{itemize}

Not a big surprise that we can describe this with atomic flows (by design). Later we get the big surprise that we can obtain this using (only) atomic flows.

The central idea in this chapter is the notion of \emph{streamlining}, which we express in terms of atomic flows. If we consider identities and weakenings to be the `creators' of atoms, and cuts and coweakening as the `destroyers' of atoms, then an atomic flow is weakly streamlined if no atom is first created and then destroyed. Alternatively, an atomic flow is streamlined if every edge appears in a path to $\top$ or to $\bot$. The shape of a streamlined atomic flow is given in case (\ref{definition:FlowNormalForms:item:Streamlined}) of Definition~\ref{definition:FlowNormalForms}.

The most challenging aspect of streamlining is the elimination of paths from interaction to cut vertices. For this reason, we define the notion of \emph{weakly streamlined} atomic flows, in case (\ref{definition:FlowNormalForms:item:WeaklyStreamlined}) of Definition~\ref{definition:FlowNormalForms}. An atomic flow is weakly streamlined if it contains no paths from interaction to cut vertices. This is the topic of Chapter~\vref{chapter:GlobalReductions}.

A path can be eliminated by removing the edges that make up the path. However, we might imagine a situation where an edge belongs to two paths, one we want to eliminate and one we want to keep. An atomic flow is in \emph{simple form}, if this situation does not occur. One approach to eliminating paths from a flow is to transform it into simple form and then eliminating the edges connecting interaction and cut vertices.

Sometimes, the elimination of edges mapped to by an atom $a$ might interfere with the elimination of edges mapped to from the atom $\bar a$. For this reason, we find it convenient to define special cases of simple form and weakly streamlined, where for every pair of dual atoms the edges mapped to from one of them are ignored. These are cases (\ref{definition:FlowNormalForms:item:SimpleForm}) and (\ref{definition:FlowNormalForms:item:WeaklyStreamlinedPolarity}) of Definition~\ref{definition:FlowNormalForms}.

The intuition behind the other normal forms are:
\begin{itemize}
\item a weakly streamlined flow is streamlined if it contains no paths from interaction (resp., cut) to coweakening (resp., weakeinng) vertices, or from weakening to coweakening vertices;
\item a streamlined flow is \emph{super streamlined} if it contains no paths from (co)weakening to (co)contraction vertices; and
\item a super streamlined flow is streamlined if it contains no path whose first edge is an upper edge of a cocontraction vertex and last edge is the lower edge of a contraction vertex.
\end{itemize}



\TODO{Alessio said: `\emph{I think it's a mistake to consider a derivation weakly streamlined with respect to a polarity. It would be better to say that it is weakly streamlined with the respect to the positive subflow, given a polarity, or something like this.}'}

%---------------------------------------
\begin{definition}\label{definition:FlowNormalForms}
An atomic flow is
\begin{enumerate}
%---
\item\label{definition:FlowNormalForms:item:SimpleForm}
\emph{in simple form with respect to the polarity assignment $\pi$}\index{simple form!flow} if it can be represented as
\[
\atomicflow{
(-20, 14)*{\afvjm4};
(-20,  8)*{\affr68};
(-20, 10)*{\aflabelleft{\ppl}};
%-
(-22,  0)*{\afvjm8};
%-
(-20, -8)*{\affr68};
(-20, -6)*{\aflabelleft{\ppl}};
(-20,-14)*{\afvjm4};
%-
( -3, 16)*{\afaidmex{}{}{}{}{}{}92};
(-12,  8)*{\affr68};
(-12, 10)*{\aflabelleft{\ppl}};
(-15,  0)*{\afcjrm68};
%-
(-15,  0)*{\afcjlm68};
(-12, -8)*{\affr68};
(-12, -6)*{\aflabelleft{\ppl}};
( -3,-16)*{\afaiumex{}{}{}{}{}{}92};
%--
( -1,  8)*{\afaidmex{}{}{}{}{}{}32};
( -4,  0)*{\affr68};
( -4,  2)*{\aflabelleft{\ppl}};
(  6,  8)*{\afvjm8};
( 10, 10)*{\afvjm{12}};
(  6,  0)*{\affr{10}8};
(  4,  2)*{\aflabelleft{\pmi}};
(  6, -8)*{\afvjm8};
( 10,-10)*{\afvjm{12}};
( -1, -8)*{\afaiumex{}{}{}{}{}{}32};
}\qquad
\atomicflow{
( -8, 12.75)*{\afvjm{6.5}};
(  0, 13.5)*{\afawdm{}{}{}{}{}{}};
%---
( -6, 5.5)*{\copy\contrup};
( -4, 5.5)*{\affr{10}8};
( -2, 5.5)*{\copy\contrdown};
%---
( -5,-2.5)*{\afaium{}{}{}{}{}{}};
(  0,   0)*{\afvjm3};
(  5, 2.5)*{\afaidm{}{}{}{}{}{}};
%---
(  2,-5.5)*{\copy\contrup};
(  4,-5.5)*{\affr{10}8};
(  6,-5.5)*{\copy\contrdown};
%---
(  0,-13.5)*{\afawum{}{}{}{}{}{}};
(  8,-12.75)*{\afvjm{6.5}};
}\quad;
\]
%---
\item\label{definition:FlowNormalForms:item:WeaklyStreamlinedPolarity}
\emph{weakly streamlined with respect to the polarity assignment $\pi$}\index{streamlined!weakly!with respect to polarity assignment} if it can be represented as
\[
\atomicflow{
(-20, 14)*{\afvjm4};
(-20,  8)*{\affr68};
(-20, 10)*{\aflabelleft{\ppl}};
%-
(-22,  0)*{\afvjm8};
%-
(-20, -8)*{\affr68};
(-20, -6)*{\aflabelleft{\ppl}};
(-20,-14)*{\afvjm4};
%-
( -9, 16)*{\afaidmex{}{}{}{}{}{}32};
(-12,  8)*{\affr68};
(-12, 10)*{\aflabelleft{\ppl}};
(-15,  0)*{\afcjrm68};
%-
(-15,  0)*{\afcjlm68};
(-12, -8)*{\affr68};
(-12, -6)*{\aflabelleft{\ppl}};
( -9,-16)*{\afaiumex{}{}{}{}{}{}32};
%--
( -6,  8)*{\afvjm8};
(  0, 10)*{\afvjm{12}};
( -3,  0)*{\affr88};
( -4,  2)*{\aflabelleft{\pmi}};
( -6, -8)*{\afvjm8};
(  0,-10)*{\afvjm{12}};
}\qquad
\atomicflow{
( -8, 12.75)*{\afvjm{6.5}};
(  0, 13.5)*{\afawdm{}{}{}{}{}{}};
%---
( -6, 5.5)*{\copy\contrup};
( -4, 5.5)*{\affr{10}8};
( -2, 5.5)*{\copy\contrdown};
%---
( -5,-2.5)*{\afaium{}{}{}{}{}{}};
(  0,   0)*{\afvjm3};
(  5, 2.5)*{\afaidm{}{}{}{}{}{}};
%---
(  2,-5.5)*{\copy\contrup};
(  4,-5.5)*{\affr{10}8};
(  6,-5.5)*{\copy\contrdown};
%---
(  0,-13.5)*{\afawum{}{}{}{}{}{}};
(  8,-12.75)*{\afvjm{6.5}};
}\quad;
\]
%---
\item\label{definition:FlowNormalForms:item:WeaklyStreamlined}
\emph{weakly streamlined}\index{streamlined!weakly} if it can be represented as
\[
\atomicflow{
( -8, 11.5)*{\afvjm4};
(  0, 13.5)*{\afawdm{}{}{}{}{}{}};
%---
( -6, 5.5)*{\copy\contrup};
( -4, 5.5)*{\affr{10}8};
( -2, 5.5)*{\copy\contrdown};
%---
( -5,-2.5)*{\afaium{}{}{}{}{}{}};
(  0,   0)*{\afvjm3};
(  5, 2.5)*{\afaidm{}{}{}{}{}{}};
%---
(  2,-5.5)*{\copy\contrup};
(  4,-5.5)*{\affr{10}8};
(  6,-5.5)*{\copy\contrdown};
%---
(  0,-13.5)*{\afawum{}{}{}{}{}{}};
(  8,-11.5)*{\afvjm4};
}\quad;
\]
%---
\item\label{definition:FlowNormalForms:item:Streamlined}
\emph{streamlined}\index{streamlined} if it can be represented as
\[
\atomicflow{
( -5, 11.5)*{\afvjm4};
%---
(-7.5, 5.5)*{\copy\contrup};
(  -5, 5.5)*{\affr{12}8};
(-2.5, 5.5)*{\copy\contrdown};
%---
(-10,-2.5)*{\afawum{}{}{}{}{}{}};
( -5,-2.5)*{\afaium{}{}{}{}{}{}};
(  0,   0)*{\afvjm3};
(  5, 2.5)*{\afaidm{}{}{}{}{}{}};
( 10, 2.5)*{\afawdm{}{}{}{}{}{}};
%---
(2.5,-5.5)*{\copy\contrup};
(  5,-5.5)*{\affr{12}8};
(7.5,-5.5)*{\copy\contrdown};
%---
(  5,-11.5)*{\afvjm4};
}\quad;
\]
%---
\item\label{definition:FlowNormalForms:item:SuperStreamlined}
\emph{super streamlined}\index{streamlined!super} if it can be represented as
\[
\atomicflow{
(-12,  9.5)*{\afawum{}{}{}{}{}{}};
%---
( -4, 11.5)*{\afvjm4};
%---
( -6, 5.5)*{\copy\contrup};
( -4, 5.5)*{\affr{10}8};
( -2, 5.5)*{\copy\contrdown};
%---
( -5,-2.5)*{\afaium{}{}{}{}{}{}};
(  0,   0)*{\afvjm3};
(  5, 2.5)*{\afaidm{}{}{}{}{}{}};
%---
(  2,-5.5)*{\copy\contrup};
(  4,-5.5)*{\affr{10}8};
(  6,-5.5)*{\copy\contrdown};
%---
(  4,-11.5)*{\afvjm4};
%---
( 12, -9.5)*{\afawdm{}{}{}{}{}{}};
}\quad;\quad\mbox{and}
\]
%---
\item\label{definition:FlowNormalForms:item:HyperStreamlined}
\emph{hyper streamlined}\index{streamlined!hyper} if it can be represented as
\[
\atomicflow{
(-12,  9.5)*{\afawum{}{}{}{}{}{}};
%---
( -4, 11.5)*{\afvjm4};
%---
( -4, 5.5)*{\copy\contrup};
( -4, 5.5)*{\affr{10}8};
%---
( -5,-2.5)*{\afaium{}{}{}{}{}{}};
(  0,   0)*{\afvjm3};
(  5, 2.5)*{\afaidm{}{}{}{}{}{}};
%---
(  4,-5.5)*{\affr{10}8};
(  4,-5.5)*{\copy\contrdown};
%---
(  4,-11.5)*{\afvjm4};
%---
( 12, -9.5)*{\afawdm{}{}{}{}{}{}};
}\quad.
\]
\end{enumerate}
\end{definition}

\TODO{Ensure that all the normal forms of atomic flows have a corresponding definition for derivations.}

\begin{definition}\label{definition:DerSimpleForm}
A derivation with associated atomic flow $\phi$ is \emph{in simple form with respect to} (\emph{the atom}) $a$\index{simple form!derivation!with respect to atom}, if $\pi$ is a polarity assignment for $\phi$, such that the edges in $\phi$ mapped to from occurrences of $a$ have a positive polarity, and the restriction of $\phi$ to $a$ is in simple form with respect to $\pi$.
\end{definition}

\begin{definition}\label{definition:DerStreamlined}
A derivation with associated atomic flow $\phi$ is \emph{weakly streamlined} (resp., \emph{streamlined}\index{streamlined!derivation}, \emph{super streamlined} and \emph{hyper streamlined}) if $\phi$ is \emph{weakly streamlined} (resp., \emph{streamlined}, \emph{super streamlined} and \emph{hyper streamlined}). The derivation is \emph{weakly streamlined with respect to} (\emph{the atom}) $a$\index{streamlined!weakly!with respect to atom}, if $\pi$ is a polarity assignment for $\phi$, such that the edges in $\phi$ mapped to from occurrences of $a$ have a positive polarity, and the restriction of $\phi$ to $a$ is weakly streamlined with respect to $\pi$.
\end{definition}

%---------------------------------------
\begin{example}\label{example:Streamlined}
The first flow is weakly streamlined, the other two are hyper streamlined:
\[
\aflower{\atomicflow{
(0,6)*{\afacd{}{}{}{}{}{}};
(4,6)*{\afawd{}{}{}{}};
(2,0)*{\afaiunw{}{}}}}
\quad,\qquad
\atomicflow{
( 2,0)*{\afacd{}{}{}{}{}{}};
( 0,4)*{\afawdnw{}{}};
( 6,8)*{\afacu{}{}{}{}{}{}};
(10,2)*{\afaiunw{}{}{}{}};
(12,8)*{\afvj8}}
\qquad\hbox{and}\qquad
\atomicflow{
(10,8)*{\afacu{}{}{}{}{}{}};
(16,8)*{\afvj8};
( 2,4)*{\afaidnw{}{}};
( 0,0)*{\afvj8};
(14,2)*{\afaiunw{}{}};
( 6,0)*{\afacd{}{}{}{}{}{}}}
\quad.
\]
\end{example}

\begin{proposition}\label{proposition:FlowWeaklyStreamlinedPolarity}
Given a polarity assignment $\pi$, an atomic flow that is weakly streamlined with respect to both $\pi$ and $\bar\pi$ is weakly streamlined. 
\end{proposition}

\begin{proposition}\label{proposition:FlowCutFree}
A streamlined atomic flow with no pair of upper (resp., lower) edges such that there is an $\ai$-path between them, contains no cut (resp., axiom) vertices.
\end{proposition}

%---------------------------------------
\begin{proposition}\label{proposition:FlowNormalFormsNoUpper}
Given an atomic flow with no upper (resp., lower) edges, it can be represented as
\begin{enumerate}
\item\label{proposition:FlowNormalFormsNoUpper:item:Streamlined}
\[
\atomicflow{
(  5, 2.5)*{\afaidm{}{}{}{}{}{}};
( 10, 2.5)*{\afawdm{}{}{}{}{}{}};
%---
(4,-5.5)*{\copy\contrup};
(6,-5.5)*{\affr{10}8};
(8,-5.5)*{\copy\contrdown};
%---
(6,-11.5)*{\afvjm4};
(0,-18)*{\invisiblemark};
}
\qquad
\left(\mbox{resp.,}\quad
\atomicflow{
( 0,18)*{\invisiblemark};
(-6,11.5)*{\afvjm4};
%---
(-8, 5.5)*{\copy\contrup};
(-6, 5.5)*{\affr{10}8};
(-4, 5.5)*{\copy\contrdown};
%---
(-10,-2.5)*{\afawum{}{}{}{}{}{}};
( -5,-2.5)*{\afaium{}{}{}{}{}{}};
}
\quad\right)
\quad,
\]
if it is streamlined;
%---
\item\label{proposition:FlowNormalFormsNoUpper:item:SuperStreamlined}
\[
\atomicflow{
(  4, 2.5)*{\afaidm{}{}{}{}{}{}};
%---
(  2,-5.5)*{\copy\contrup};
(  4,-5.5)*{\affr88};
(  6,-5.5)*{\copy\contrdown};
%---
(  4,-11.5)*{\afvjm4};
%---
( 11, -9.5)*{\afawdm{}{}{}{}{}{}};
(0,-18)*{\invisiblemark};
}
\qquad
\left(\mbox{resp.,}\quad
\atomicflow{
( 0,18)*{\invisiblemark};
(-11,  9.5)*{\afawum{}{}{}{}{}{}};
%---
( -4, 11.5)*{\afvjm4};
%---
( -6, 5.5)*{\copy\contrup};
( -4, 5.5)*{\affr88};
( -2, 5.5)*{\copy\contrdown};
%---
( -4,-2.5)*{\afaium{}{}{}{}{}{}};
}
\quad\right)
\quad,
\]
if it is super streamlined; and
%---
\item\label{proposition:FlowNormalFormsNoUpper:item:HyperStreamlined}
\[
\atomicflow{
(  4, 2.5)*{\afaidm{}{}{}{}{}{}};
%---
(  4,-5.5)*{\affr88};
(  4,-5.5)*{\copy\contrdown};
%---
(  4,-11.5)*{\afvjm4};
%---
( 11, -9.5)*{\afawdm{}{}{}{}{}{}};
(0,-18)*{\invisiblemark};
}
\qquad
\left(\mbox{resp.,}\quad
\atomicflow{
( 0,18)*{\invisiblemark};
(-11,  9.5)*{\afawum{}{}{}{}{}{}};
%---
( -4, 11.5)*{\afvjm4};
%---
( -4, 5.5)*{\copy\contrup};
( -4, 5.5)*{\affr88};
%---
( -4,-2.5)*{\afaium{}{}{}{}{}{}};
}
\quad\right)
\quad,
\]
if it is hyper streamlined.
\end{enumerate}
\end{proposition}

\begin{remark}\label{remark:DerCutFree}
A streamlined proof is cut-free, a super streamlined proof is analytic and a hyper streamlined proof is in system $\KS$.
\end{remark}
