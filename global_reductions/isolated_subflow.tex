\newcommand{\ISR}{\mathsf{ISR}}

Given a derivation $\Phi$ in simple form with respect to an atom $a$, the operator, $\ISR$, defined in this section produces a derivation with the same premiss and conclusion as $\Phi$, which is weakly streamlined with respect to $a$.

We will see later how a derivation containing $n$ atoms can be weakly streamlined by two applications of $\Simpl$ and $n$ applications of $\ISR$. This is the most basic procedure for obtaining a weakly streamlined derivation, in particular it only deals with one atom at a time. In the following sections we will see how we can deal with several atoms in parallell.

The operator is defined in terms of the following flow reduction.

%--------------------------------
\newcommand{\fris}{{\mathsf{is}}}
\begin{definition}\label{definition:IsolatedSubflowRemoval}
We define the reduction $\to_\fris$ (where $\fris$ stands for \emph{isolated subflow}\index{Isolated Subflow Remover!reduction}) as follows, for any atomic flow $\phi$ and any connected atomic flow $\psi$ that does not contain identity or cut vertices:
\[
\vcenter{\hbox{\includegraphics{Figures/redISRRedex}}}
\quad\to_\fris\quad
\vcenter{\hbox{\includegraphics{Figures/redISRContractum}}}
\quad.
\]
\end{definition}
%---------------

%-------------------------------------------------------------
\begin{remark}\label{remark:IsolatedSubflowRemovalRestriction}
The condition on the atomic flow $\psi$ in Definition~\ref{definition:IsolatedSubflowRemoval} ensures that all the edges in $\psi$ are mapped to from occurrences of the same atom. However, the reduction would still be sound if, at the expense of a slightly more verbose soundness proof, we relaxed the condition to say that the upper and lower edge of $\psi$ belong to the same connected component.
\end{remark}

%---------------------------------------------------------
\begin{theorem}\label{theorem:SoundIsolatedSubflowRemoval}
Reduction\/ $\to_\fris$ is sound; moreover, if\/ $\Phi\to_\fris\Psi$, then the size of\/ $\Psi$ depends  polynomially on the size of\/ $\Phi$.
\end{theorem}

\begin{proof}
Let $\Phi$ be a derivation with flow $\phi'$, such that $\phi'\to_\fris\psi'$. We show that there exists a derivation $\Psi$ with flow $\psi'$ and with the same premiss and conclusion as $\Phi$. In the following, we refer to the figure in Definition~\ref{definition:IsolatedSubflowRemoval}.

Since $\psi$ is connected, we assume, by Convention~\vref{convention:AlternativeAiDecomposedForm}, that the following derivation is an $\ai$-decomposed form of $\Phi$:
\[
\vlder{\Phi'}{}
{
 \vlsbr[\beta\;.\;\vlinf{}{}{\fff}{\vls(a^\psi.\bar a)}]
}
{
 \vlsbr(\vlinf{}{}{\vls[a^\psi.\bar a]}{\ttt}\;.\;\alpha)
}\quad,
\]
for some atom $a$ and formulae $\alpha$ and $\beta$.

We obtain the two derivations $\Phi_\ttt$ and $\Phi_\fff$ from $\Phi'$ as follows:
\[
\Phi_\ttt\;=\;
\vlder{\Phi'\{a^\psi/\ttt\}}{}
{
 \vls[\beta.\bar a]
}
{
 \vls([\ttt.\bar a].\alpha)
}
\qquad\mbox{and}\qquad
\Phi_\fff\;=\;
\vlder{\Phi'\{a^\psi/\fff\}}{}
{
 \vls[\beta.(\fff.\bar a)]
}
{
 \vls(\bar a.\alpha)
}\quad.
\]

Since $\psi$ is connected and contains no identity or cut vertices, the mapping from all the occurrences $a^\psi$ to edges of $\psi$ is surjective. Hence, we know that both derivation $\Phi_\ttt$ and $\Phi_\fff$ have a flow isomorphic to $\phi$. We combine $\Phi_\ttt$ and $\Phi_\fff$ to get the desired derivation $\Psi$ with flow $\psi'$ and the same premiss and conclusion as $\Phi$:
\[
\Psi\;=\;
\vlderivation
{
 \vlin{\cod}{}
 {
  \beta
 }
 {
  \vlin{\swi}{}
  {
   \vls
   [
    \beta
   \;\;\;.\;\;\;
    \vlder{\Phi_\fff}{}
    {
     \vlsbr[\beta\;.\;(\fff\;.\;\vlinf{}{}{\ttt}{\bar a})]
    }
    {
     \vls(\bar a.\alpha)
    }
   ]
  }
  {
   \vlin{\cou}{}
   {
    \vls
    (
     \vlder{\Phi_\ttt}{}
     {
      \vls[\beta.\bar a]
     }
     {
      \vlsbr([\ttt\;.\;\vlinf{}{}{\bar a}{\fff}]\;.\;\alpha)
     }
    \;\;\;.\;\;\;
     \alpha
    )
   }
   {
    \vlhy{\alpha}
   }
  }
 }
}\quad.
\]
We know that the size of $\Phi_\ttt$ and the size of $\Phi_\fff$ depend polynomially on the size of $\Phi$ by Theorem~\vref{theorem:aiDecomposedForm} and Proposition~\vref{proposition:DerivationSubstitution}, and that the size of $\Psi$ depends at most quadratically on the size of $\alpha$ and $\beta$ by Lemma~\vref{lemma:GenericContraction}, so the size of $\Psi$ depends polynomially on the size of $\Phi$.
\end{proof}
%----------

\Tom{Exchanged $\top$ and $\bot$ in item 2, and clarified statement.}

%-------------------------------------
\begin{lemma}\label{lemma:IsolatedSubflowRemovalPaths}
Given two atomic flows $\phi$ and $\psi$, such that $\phi\to_\fris\psi$; in the following we refer to Definition~\ref{definition:IsolatedSubflowRemoval}:
\begin{itemize}
\item let $f_1$ and $f_2$ be the evidenced isomorphisms; and
\item let $\nu_\ai$ be the evidenced cut (resp., interaction) vertex,
\end{itemize}
then, given an interaction (resp., cut) vertex $\nu$ in $\psi$, there is an interaction (resp., cut) vertex $\nu'$ in $\phi$, such that
\begin{itemize}
\item $\nu=f_1(\nu')$ or $\nu=f_2(\nu')$;
\item if there is a path from $\nu$ to $\bot$ (resp., $\top$) in $\psi$, then there is a path from $\nu'$ to $\bot$ (resp., $\top$) in $\phi$; and
\item if there is a cut (resp., interaction) vertex $\hat\nu$ in $\psi$, such that there is a path from $\nu$ to $\hat\nu$ in $\psi$, then there is a cut (resp., interaction) vertex $\hat\nu'$ in $\phi$, such that $\hat\nu=f_1(\nu')$ or $\hat\nu=f_2(\nu')$, or $\hat\nu'=\nu_\ai$; and there is a path from $\nu'$ to $\hat\nu'$ in $\phi$.
\end{itemize}
\end{lemma}

\begin{proof}
We consider each case separately:
\begin{itemize}
 \item by definition;
 \item any path from $\nu$ to $\bot$ (resp., $\top$) in $\psi$ must contain an edge $\epsilon$, such that, for some lower (resp., upper) edge $\epsilon'$ of $\phi$, $f_1(\epsilon')=\epsilon$ or $f_2(\epsilon')=\epsilon$. Hence, there is a path from $\nu'$ to $\bot$ (resp., $\top$) in $\phi$; and
 \item we have to consider two cases:
 \begin{itemize}
  \item $\nu=f_1(\nu')$ and $\hat\nu=f_1(\hat\nu')$, or $\nu=f_2(\nu')$ and $\hat\nu=f_2(\hat\nu')$, then there is a path from $\nu'$ to $\hat\nu'$ in $\phi$; or
  \item $\nu=f_1(\nu')$ and $\hat\nu=f_2(\hat\nu')$ (resp., $\nu=f_2(\nu')$ and $\hat\nu=f_1(\hat\nu')$),then there is a path from $\nu'$ to $\nu_\ai$ in $\phi$.
 \end{itemize}
\end{itemize}
\end{proof}
%----------

%------------------------------
\begin{definition}\label{definition:IsolatedSubflowRemover}
The \emph{Isolated Subflow Remover}\index{Isolated Subflow Remover!operator}, $\ISR$, is an operator whose arguments are an atom $a$ and a derivation $\Phi$ that is in simple form with respect to $a$. If $\Phi$ is weakly streamlined with respect to $a$, then $\ISR(\Phi,a)=\Phi$; otherwise, consider the following  $\ai$-decomposed form of $\Phi$:
\[
\vlder{\Phi'}{}
{
 \vlsbr
 [
  \beta
 \;.\;
  \vlinf{}{}
  {
   \fff
  }
  {
   \vls(a^{\psi'}.\bar a)
  }
 \;.\;\cdots\;.\;
  \vlinf{}{}
  {
   \fff
  }
  {
   \vls(a^{\psi'}.\bar a)
  }
 ]
}
{
 \vlsbr
 (
  \vlinf{}{}
  {
   \vls[a^{\psi'}.\bar a]
  }
  {
   \ttt
  }
 \;.\;\cdots\;.\;
  \vlinf{}{}
  {
   \vls[a^{\psi'}.\bar a]
  }
  {
   \ttt
  }
 \;.\;
  \alpha
 )
}\quad,
\]
with atomic flow
\[
\vcenter{\hbox{\includegraphics{Figures/defISR1}}}
\quad,
\]
where $\psi'$ is the juxtaposition of all the isolated subflows mapped to from occurrences of $a$ in $\Phi$. Consider the derivation
\[
\Psi\;=\;
\vlder{\Phi'}{}
{
 \vlsbr
 [
  \beta
 \;\;\;.\;\;\;
  \vlderivation
  {
   \vlin{}{}
   {
    \fff
   }
   {
    \vlde{}{\{\cod\}}
    {
     \vls(a.\bar a)
    }
    {
     \vlhy
     {
      \vls[(a.\bar a).\cdots.(a.\bar a)]
     }
    }
   }
  }
 ]
}
{
 \vlsbr
 (
  \vlderivation
  {
   \vlde{}{\{\cou\}}
   {
    \vls([a.\bar a].\cdots.[a.\bar a])
   }
   {
    \vlin{}{}
    {
     \vls[a.\bar a]
    }
    {
     \vlhy
     {
      \ttt
     }
    }
   }
  }
 \;\;\;.\;\;\;
  \alpha
 )
}\quad,
\]
with atomic flow
\[
\psi''\;=\;
\vcenter{\hbox{\includegraphics{Figures/defISR2}}}
\quad.
\]
We then define $\ISR(\Phi,a)$ to be such that $\Psi\to_\fris\ISR(\Phi,a)$, where $\phi$ and $\psi$ are the flows, by the same names, shown in Definition~\vref{definition:IsolatedSubflowRemoval}.
\end{definition}
%---------------

%------------------------------------------------------------
\begin{proposition}\label{proposition:IsolatedSubflowRemover}
Given an atom $a$ and a derivation $\Phi$ that is in simple form with respect to $a$,
\begin{enumerate}
\item $\ISR(\Phi,a)$ is weakly streamlined with respect to $a$;
\item for any atom $b$,
\begin{itemize}
\item if $\Phi$ is weakly streamlined with respect to $b$, then $\ISR(\Phi,a)$ is weakly streamlined with respect to $b$, and
\item if $b$ is not the dual of $a$ and $\Phi$ is in simple form with respect to $b$, then $\ISR(\Phi,a)$ is in simple form with respect to $b$; and
\end{itemize}
\item the size of\/ $\ISR(\Phi,a)$ depends polynomially on the size of\/ $\Phi$.
\end{enumerate}
\end{proposition}

\TODO{Alessio said: `\emph{This should be OK despite a possible mistake in 6.1.11. In fact, I think it would be better to justify point 2 by something else than (whatever becomes of) Lemma 6.1.11. The reason is that independent copies are made of the flows of b's, and then they are joined by (co)contractions. (Right?)}'.}

\begin{proof}
If $\Phi$ is weakly streamlined with respect to $a$, the result is trivial. Assume $\Phi$ is not weakly streamlined with respect to $a$, and let $\phi$, $\psi$, $\phi'$, $\psi'$ and $\psi''$ be the atomic flows given in Definition~\ref{definition:IsolatedSubflowRemover}, then
\begin{enumerate}
\item by definition there is no path in $\phi$ from an interaction to a cut vertex whose edges are mapped to from instances of $a$. By Lemma~\vref{lemma:IsolatedSubflowRemovalPaths}, we know that if there is a path from an interaction to a cut vertex in the atomic flow of $\ISR(\Phi,a)$ whose edges are mapped to from instances of $a$, then there must be a path from an interaction to a cut vertex in $\phi$ whose edges are mapped to from instances $a$. Hence, the statement follows by contradiction;
\item
\begin{itemize}
 \item if there is a path from an interaction (resp., cut) vertex in the atomic flow of $\ISR(\Phi,a)$ whose edges are mapped to from instances of $b$, then, by \newline Lemma~\vref{lemma:IsolatedSubflowRemovalPaths}, there is a path from an interaction (resp., cut) vertex in $\phi$, so also in $\phi'$, whose edges are mapped to from instances of $b$. Hence, the statement follows by contradiction; and
 \item if there is an interaction (resp., cut) vertex $\nu$ and a cut (resp., interaction) vertex $\hat\nu$ in the atomic flow of $\ISR(\Phi,a)$ such that there is a path from $\nu$ to $\hat\nu$ and a path from $\nu$ to $\bot$ (resp., $\top$), both of whose edges are mapped to from instances of $b$, then, by Lemma~\vref{lemma:IsolatedSubflowRemovalPaths}, there is an interaction (resp., cut) vertex $\nu'$ and a cut (resp., interaction) vertex $\hat\nu'$ in $\phi$ such that there is a path from $\nu$ to $\hat\nu$ and a path from $\nu$ to $\bot$ (resp., $\top$), both of whose edges are mapped to from instances of $b$. Furthermore, since we can assume that $b$ is not $a$ or $\bar a$, $\phi$ restricted to $b$ equals $\phi'$ restricted to $b$. Hence, the statement follows by contradiction.
\end{itemize}
\item the statement follows by Theorem~\vref{theorem:SoundIsolatedSubflowRemoval}.
\end{enumerate}
\end{proof}
%----------

\TODO{Copy example from AF1.}
