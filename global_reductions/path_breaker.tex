\newcommand{\PB}{\mathsf{PB}}

Given a derivation $\Phi$ and an atom $a$, the operator, $\PB$, defined in this section produces a derivation with the same premiss and conclusion as $\Phi$, which is weakly streamlined with respect to both $a$ and $\bar a$. This operator is a strict improvement over $\ISR$, since it does not require the input derivation to be in simple form, and it deals with the dual atoms in parallell. We will see later how a derivation containing $n$ atoms can be weakly streamlined by $n/2$ applications of $\PB$.

The operator is defined in terms of the following flow reduction.

%--------------------------------
\newcommand{\frpb}{{\mathsf{pb}}}
\begin{definition}\label{definition:PathBreaker}
We define the reduction $\to_\frpb$ (where $\frpb$ stands for \emph{path breaker}\index{Path Breaker!reduction}) as follows, for any two flows $\phi$ and $\psi$:
\[
\vcenter{\hbox{\includegraphics{Figures/redPBRedex}}}
\quad\to_\frpb\quad
\vcenter{\hbox{\includegraphics{Figures/redPBContractum}}}
\quad,
\]
where we call the evidenced interaction (resp., cut) vertex in the redex $\nu'_\aid$ (resp., $\nu'_\aiu$) and the evidenced interaction (resp., cut) vertex in the contractum $\nu_\aid$ (resp., $\nu_\aiu$); and where there is a path from $\nu'_\aid$ to $\nu'_\aiu$.
\end{definition}
%---------------

%----------------------------------------------
\begin{theorem}\label{theorem:SoundPathBreaker}
Reduction $\to_\frpb$ is sound; moreover, if\/ $\Phi\to_\frpb\Psi$, then the size of $\Psi$ depends polynomially on the size of $\Phi$.
\end{theorem}

\begin{proof}
Let $\Phi$ be a derivation with flow $\phi'$, such that $\phi'\to_\frpb\psi'$. We show that there exists a derivation $\Psi$ with flow $\psi'$ and with the same premiss and conclusion as $\Phi$. In the following, we refer to the figure in Definition~\vref{definition:PathBreaker}.

Since the evidenced interaction and cut vertices belong to the same connected component, we assume, by Convention~\vref{convention:AlternativeAiDecomposedForm}, that the following derivation is an $\ai$-decomposed form of $\Phi$:
\[
\vlder{\Phi'}{}
{
 \vlsbr[\beta\;.\;\vlinf{}{}{\fff}{\vls(a^\phi.\bar a^\psi)}]
}
{
 \vlsbr(\vlinf{}{}{\vls[a^\phi.\bar a^\psi]}{\ttt}\;.\;\alpha)
}\quad,
\]
for some atom $a$ and formulae $\alpha$ and $\beta$.

We combine three copies of $\Phi'$ to obtain the desired derivation $\Psi$ with flow $\psi'$ and the same premiss and conclusion as $\Phi$:

\newbox\DeltaTopK
\setbox\DeltaTopK=
\hbox{$
\vlder{\Phi'}{}
{
 \vlsbr[\beta\;.\;(\vlinf{}{}{\ttt}{a^{f_1(\phi)}}\;.\;\bar a^{g_1(\psi)})]
}
{
 \vlsbr(\vlinf{}{}{\vls[a^{f_1(\phi)}.\bar a^{g_1(\psi)}]}{\ttt}\;.\;\alpha)
}
$}
\newbox\DeltaK
\setbox\DeltaK=
\hbox{$
\vlder{\Phi'}{}
{
 \vlsbr[\beta\;.\;(a^{f_2(\phi)}\;.\;\vlinf{}{}{\ttt}{\bar a^{g_2(\psi)}})]
}
{
 \vlsbr([\vlinf{}{}{a^{f_2(\phi)}}{\fff}\;.\;\bar a^{g_2(\psi)}]\;.\;\alpha)
}
$}
\newbox\DeltaBotK
\setbox\DeltaBotK=
\hbox{$
\vlder{\Phi'}{}
{
 \vlsbr[\beta\;.\;\vlinf{}{}{\fff}{\vls(a^{f_3(\phi)}.\bar a^{g_3(\psi)})}]
}
{
 \vlsbr([a^{f_3(\phi)}\;.\;\vlinf{}{}{\bar a^{g_3(\psi)}}{\fff}]\;.\;\alpha)
}
$}
\[
\Psi\quad=\quad
\vlderivation
{
 \vlin{\cod}{}{\beta}
 {
  \vlin{\swi}{}
  {
   \vls
   [
    \vlinf{\cod}{}{\beta}{\vls[\beta.\beta]}
   \;\;\;\;.\;\;\;\;
    \box\DeltaBotK
   ]
  }
  {
   \vlin{\swi}{}
   {
    \vls
    (
%     \vlinf{\swi}{}
%     {
%      \vls
      [
       \beta
      \;\;\;\;.\;\;\;\;
       \box\DeltaK
      ]
%     }
%     {
%      \vls(\vlsmallbrackets[\beta.\bar a^\psi].\alpha)
%     }
    \;\;\;\;\;.\;\;\;\;\;
     \alpha
    )
   }
   {
    \vlin{\cod}{}
    {
     \vls
     (
      \box\DeltaTopK
     \;\;\;\;.\;\;\;\;
      \vlinf{\cou}{}{\vls(\alpha.\alpha)}{\alpha}
     )
    }
    {
     \vlhy{\alpha}
    }
   }
  }
 }
}\qquad.
\]
We know that the size of $\Phi'$ depends at most cubically on the size of $\Phi$ by \newline Theorem~\vref{theorem:aiDecomposedForm}, and that the size of $\Psi$ depends at most quadratically on the size of $\alpha$ and $\beta$ by Lemma~\vref{lemma:GenericContraction}, so $\Psi$ depends polynomially on the size of $\Phi$.
\end{proof}
%----------

\Tom{Added explanation:}

We now show the basic properties of $\to_\frpb$. Namely, that the reduction does not create any `new' interaction or cut vertices, that it does not create any `new' paths between interaction or $\top$ and cut or $\bot$ vertices, and that it breaks all the paths between the evidenced interaction and cut vertices.

\Tom{Changed from itemize to enumerate.}

%-------------------------------------
\begin{lemma}\label{lemma:PathBreaker}
In the following we refer to the names given in Definition~\vref{definition:PathBreaker}. Given two flows $\phi$ and $\psi$, such that $\phi\to_\frpb\psi$, then, given an interaction (resp., cut) vertex $\nu$ in $\psi$, there is an interaction (resp., cut) vertex $\nu'$ in $\phi$, such that
\begin{enumerate}
\item for some $1\le i\le 3$, $\nu=f_i(\nu')$ or $\nu=g_i(\nu')$, or $\nu=\nu_\aid$ and $\nu'=\nu'_\aid$ (resp., $\nu=\nu_\aiu$ and $\nu'=\nu'_\aiu$);
\item if there is a path from $\nu$ to $\bot$ (resp., $\top$) in $\psi$, then there is a path from $\nu'$ to $\bot$ (resp., $\top$) in $\phi$; and
\item if there is a cut (resp., interaction) vertex $\hat\nu$ in $\psi$, such that there is a path from $\nu$ to $\hat\nu$ in $\psi$, then there is a cut (resp., interaction) vertex $\hat\nu'$ in $\phi$, such that, for some $1\le i\le 3$, $\hat\nu=f_i(\nu')$ or $\hat\nu=g_i(\nu')$, or $\hat\nu=\nu_\aiu$ and $\hat\nu'=\nu'_\aiu$ (resp., $\hat\nu=\nu_\aid$ and $\hat\nu'=\nu'_\aid$); and there is a path from $\nu'$ to $\hat\nu'$ in $\phi$.
\item there is no path from $\nu_\aid$ to $\nu_\aiu$.
\end{enumerate}
\end{lemma}

\TODO{Check commens from Alessio}
\Tom{Added one case and moved proof that used to be in proposition.}

\begin{proof}
We consider each case separately:
\begin{enumerate}
  \item by definition;
  \item any path from $\nu$ to $\bot$ (resp., $\top$) in $\psi$ must contain an edge $\epsilon$, such that, for some lower (resp., upper) edge $\epsilon'$ of $\phi$ and some $1\le i\le 3$, $f_i(\epsilon')=\epsilon$ or $g_i(\epsilon')=\epsilon$. Hence, there is a path from $\nu'$ to $\bot$ (resp., $\top$) in $\phi$; and
  \item we have to consider two cases:
  \begin{enumerate}
    \item for some $1\le i\le 3$, $\nu=f_i(\nu')$ and $\hat\nu=f_i(\hat\nu')$, or $\nu=g_i(\nu')$ and $\hat\nu=g_i(\hat\nu')$, then there is a path from $\nu'$ to $\hat\nu'$ in $\phi$; or
    \item $\nu=g_1(\nu')$ and $\hat\nu=g_2(\hat\nu')$, or $\nu=f_2(\nu')$ and $\hat\nu=f_3(\hat\nu')$ (resp., $\nu=g_2(\nu')$ and $\hat\nu=g_1(\hat\nu')$, or $\nu=f_3(\nu')$ and $\hat\nu=f_2(\hat\nu')$), then there is a path from $\nu'$ to $\nu'_\aiu$ (resp., $\nu'_\aid$) in $\phi$.
  \end{enumerate}
  \item in Definition~\ref{definition:PathBreaker} we have colored the edges that might occur in paths from $\nu_\aid$ in red and paths that might occur in path to $\nu_\aiu$ in green. Since the red and the green edges never coincide, there can be no paths from $\nu_\aid$ to $\nu_\aiu$
\end{enumerate}
\end{proof}
%----------

%----------------------------
\begin{definition}\label{definition:DerPathBreaker}
The \emph{Path Breaker}\index{Path Breaker!operator}, $\PB$, is an operator whose arguments are an atom $a$ and a derivation $\Phi$. If $\Phi$ is weakly streamlined with respect to both $a$ and $\bar a$, then $\PB(\Phi,a)=\Phi$; otherwise, consider the following $\ai$-decomposed form of $\Phi$:
\[
\vlder{\Phi'}{}
{
 \vlsbr
 [
  \beta
 \;.\;
  \vlinf{}{}
  {
   \fff
  }
  {
   \vls(a^\psi.\bar a)
  }
 \;.\;\cdots\;.\;
  \vlinf{}{}
  {
   \fff
  }
  {
   \vls(a^\psi.\bar a)
  }
 ]
}
{
 \vlsbr
 (
  \vlinf{}{}
  {
   \vls[a^\psi.\bar a]
  }
  {
   \ttt
  }
 \;.\;\cdots\;.\;
  \vlinf{}{}
  {
   \vls[a^\psi.\bar a]
  }
  {
   \ttt
  }
 \;.\;
  \alpha
 )
}\quad,
\]
with flow
\[
\phi''\;=\;
\vcenter{\hbox{\includegraphics{Figures/defPB1}}}
\quad,
\]
such that occurrences of $a$ do not appear in an interaction or cut instance in $\Phi'$. Consider the derivation
\[
\Psi\;=\;
\vlder{\Phi'}{}
{
 \vlsbr
 [
  \beta
 \;\;\;.\;\;\;
  \vlderivation
  {
   \vlin{}{}
   {
    \fff
   }
   {
    \vlde{}{\{\cod\}}
    {
     \vls(a.\bar a)
    }
    {
     \vlhy
     {
      \vls[(a.\bar a).\cdots.(a.\bar a)]
     }
    }
   }
  }
 ]
}
{
 \vlsbr
 (
  \vlderivation
  {
   \vlde{}{\{\cou\}}
   {
    \vls([a.\bar a].\cdots.[a.\bar a])
   }
   {
    \vlin{}{}
    {
     \vls[a.\bar a]
    }
    {
     \vlhy
     {
      \ttt
     }
    }
   }
  }
 \;\;\;.\;\;\;
  \alpha
 )
}\quad,
\]
with flow
\[
\psi''\;=\;
\vcenter{\hbox{\includegraphics{Figures/defPB2}}}
\quad.
\]
We then define $\PB(\Phi,a)$ to be such that $\Psi\to_\frpb\PB(\Phi,a)$, where $\phi$ and $\psi$ are the flows, by the same names, shown in Definition~\vref{definition:PathBreaker}.
\end{definition}
%---------------

\Tom{Moved some details to the previous lemma.}

%------------------------------------------------------------
\begin{proposition}\label{proposition:PathBreaker}
Given an atom $a$ and a derivation $\Phi$,
\begin{enumerate}
\item $\PB(\Phi,a)$ is weakly streamlined with respect to both $a$ and $\bar a$;
\item for any atom $b$, if $\Phi$ is weakly streamlined with respect to $b$, then $\PB(\Phi,a)$ is weakly streamlined with respect to $b$; and
\item the size of\/ $\PB(\Phi,a)$ depends polynomially on the size of\/ $\Phi$.
\end{enumerate}
\end{proposition}

\Tom{Fixed typo in second case:}

\begin{proof}
If $\Phi$ is weakly streamlined with respect to both $a$ and $\bar a$, the result is trivial. Assume $\Phi$ is not weakly streamlined with respect to both $a$ and $\bar a$, and let $\phi$, $\psi$, $\phi'$, $\psi'$, $\phi''$ and $\psi''$ be the flows given in Definition~\vref{definition:DerPathBreaker} and let $\nu_\aid$ (resp., $\nu_\aiu$) be the evidenced interaction (resp., cut) vertex in $\psi''$, then
\begin{enumerate}
\item by Definition~\ref{definition:DerPathBreaker} all the paths from an interaction (resp., cut) vertex whose edges are mapped to from instances of $a$ or $\bar a$ must start from $\nu_\aid$ (resp., $\nu_\aiu$). The statement then follows by Lemma~\vref{lemma:PathBreaker};
\item if the flow of $\PB(\Phi,a)$ contains a path from an interaction vertex to a cut vertex  whose edges are mapped to from instances of $b$, then, by Lemma~\ref{lemma:PathBreaker}, there is a path from an interaction to a cut vertex in $\phi$ or $\psi$, so also in $\phi'$ or $\psi'$, whose edges are mapped to from instances of $b$. Hence, the statement follows by contradiction; and
\item the statement follows by Theorem~\vref{theorem:SoundPathBreaker}.
\end{enumerate}
\end{proof}
%----------

\Tom{Added informal text:}

We now give an example of an application of $\PB$. In particular we want to show its inherent non-confluency.

%-----------------------------------------
\begin{example}\label{example:PathBreaker}
Given a derivation $\Phi$ where the atoms $a_1$ and $a_2$ occur, such that the flow associated with $\Phi$ is
\[
\vcenter{\hbox{\includegraphics{Figures/exPB1}}}
\quad,
\]
and where all the edges in $\phi_1$ are mapped to from $a_1$ and all the edges in $\phi_2$ are mapped to from $a_2$, and there are no edges in $\psi$ that are mapped to from $a_1$ or $a_2$, then the flow associated with $\PB((\Phi,a_1),a_2)$ is the juxtaposition of the following three flows (where indications of the different isomorphisms are left out):

\[
\vcenter{\hbox{\includegraphics{Figures/exPB2}}}
\quad.
\]
We marked some edges in red to point out the fundamental difference between the subflows containing $\phi_1$ and the subflows containing $\phi_2$.
\end{example}
%------------
