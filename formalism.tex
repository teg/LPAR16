\chapter{The Functorial Calculus}\label{chapter:TheFunctorialCalculus}

\section{Definitions}\label{section:FunctorialCalculus}

\TODO{Give credit to Francois}

\begin{definition}\label{definition:InferenceRuleInstance}
An \emph{inference rule instance}, $\vlinf{\rho}{}{\beta}{\alpha}$, consists of two formulae, $\alpha$ and $\beta$, called \emph{premiss} and \emph{conclusion} respectively.
\end{definition}

\TODO{Make the label on the derivation in item 2 smaller.}

\begin{definition}\label{definition:Derivation}
Fix a set of formulae, a set of inference rule instances and a set of logical connectives, then a \emph{(functorial calculus) derivation $\vlder{\Psi}{}{\beta}{\alpha}$ from (the formula) $\alpha$ to (the formula) $\beta$}, is defined to be
\begin{enumerate}
 \item a formula, $\Psi=\alpha=\beta$,

 \item a vertical composition
 \[
 \Psi\;=\;
 \vlinf{\rho}{}
 {
  \vlder{\Phi_2}{}
  {
   \beta_2
  }
  {
   \alpha_2
  }
 }
 {
  \vlder{\Phi_1}{}
  {
   \beta_1
  }
  {
   \alpha_1
  }
 }
 \quad,
 \]
 where $\vlinf{\rho}{}{\alpha_2}{\beta_1}$ is an inference rule instance, and $\vlder{\Phi_1}{}{\beta_1}{\alpha_1}$ and $\vlder{\Phi_2}{}{\beta_2}{\alpha_1}$ are derivations, and
 \item a horizontal composition
 \[
 \Psi=\vlder{\circ(\Phi_1,\dots,\Phi_n)}{}{\circ(\beta_1,\dots,\beta_n)}{\circ(\alpha_1,\dots,\alpha_n)}\quad,
 \]
 where $\circ$ is an $n$-ary logical connective, $\alpha=\circ(\alpha_1,\dots,\alpha_n)$, $\beta=\circ(\beta_1,\dots,\beta_n)$ and $\vlder{\Phi_1}{}{\beta_1}{\alpha_1}$, $\dots$, $\vlder{\Phi_n}{}{\beta_n}{\alpha_n}$ are derivations.
\end{enumerate}
\end{definition}

\begin{remark}\label{remark:DerAssociativeComposition}
Given derivations $\vlder{\Phi_1}{}{\beta_1}{\alpha_1}$, $\vlder{\Phi_2}{}{\beta_2}{\alpha_2}$ and $\vlder{\Phi_3}{}{\beta_3}{\alpha_3}$, and inference rule instances $\vlinf{\rho_1}{}{\alpha_2}{\beta_1}$ and $\vlinf{\rho_2}{}{\alpha_3}{\beta_2}$ we consider
\[
\vlinf{\rho_2}{}
{
 \vlder{\Phi_3}{}
 {
  \beta_3
 }
 {
  \alpha_3
 }
}
{
 \left(
 \vlinf{\rho_1}{}
 {
  \vlder{\Phi_2}{}
  {
   \beta_2
  }
  {
   \alpha_2
  }
 }
 {
  \vlder{\Phi_1}{}
  {
   \beta_1
  }
  {
   \alpha_1
  }
 }
 \right)
}
\quad\mbox{and}\quad
\vlinf{\rho_1}{}
{
 \left(
 \vlinf{\rho_2}{}
 {
  \vlder{\Phi_3}{}
  {
   \beta_3
  }
  {
   \alpha_3
  }
 }
 {
  \vlder{\Phi_2}{}
  {
   \beta_2
  }
  {
   \alpha_2
  }
 }
 \right)
}
{
 \vlder{\Phi_1}{}
 {
  \beta_1
 }
 {
  \alpha_1
 }
}
\quad,
\]
to be equal, and we denote them both by:
\[
\vlderivation
{
 \vlde{\Phi_3}{}
 {
  \beta_3
 }
 {
  \vlin{\rho_2}{}
  {
   \alpha_3
  }
  {
   \vlde{\Phi_2}{}
   {
    \beta_2
   }
   {
    \vlin{\rho_1}{}
    {
     \alpha_2
    }
    {
     \vlde{\Phi_1}{}
     {
      \beta_1
     }
     {
      \vlhy
      {
       \alpha_1
      }
     }
    }
   }
  }
 }
}
\quad.
\]
\end{remark}

\begin{lemma}\label{lemma:DerComposition}
Given two derivations $\vlder{\Phi_1}{}{\beta}{\alpha}$ and $\vlder{\Phi_2}{}{\gamma}{\beta}$, a derivation $\vlder{\Phi_1;\Phi_2}{}{\gamma}{\alpha}$ can be constructed.
\end{lemma}

\begin{proof}
We argue by structural induction on $\Phi_1$ and $\Phi_2$
\begin{enumerate}

 \item if $\Phi_1=\beta$ then $\Psi=\Phi_2$;

 \item if $\Phi_2=\beta$ then $\Psi=\Phi_1$;

 \item if $\Phi_1\;=\;\vlder{\vlinf{\rho}{}{\Phi_1''}{\Phi_1'}}{}{\beta}{\alpha}$, then $\Phi_1'';\Phi_2$ can be constructed by the inductive hypothesis and we can build $\Psi\;=\;\vlder{\vlinf{\rho}{}{\Phi_1'';\Phi_2}{\Phi_1'}}{}{\gamma}{\alpha}$;

 \item if $\Phi_2\;=\;\vlder{\vlinf{\rho}{}{\Phi_2''}{\Phi_2'}}{}{\gamma}{\beta}$, then $\Phi_1;\Phi_2'$ can be constructed by the inductive hypothesis and we can build $\Psi\;=\;\vlder{\vlinf{\rho}{}{\Phi_2''}{\Phi_1;\Phi_2'}}{}{\gamma}{\alpha}$;

 \item if $\Phi_1=\circ(\Phi_{1,1},\dots,\Phi_{1,n})$ and $\Phi_2=\circ(\Phi_{2,1},\dots,\Phi_{2,n})$ then $\Phi_{1,1};\Phi_{2,1},\dots,\Phi_{1,n};\Phi_{2,n}$ can be constructed by the inductive hypothesis and we can build $\Phi_1;\Phi_2=\circ(\Phi_{1,1};\Phi_{2,1},\dots,\Phi_{1,n};\Phi_{2,n})$.

\end{enumerate}
\end{proof}

\begin{definition}\label{definition:DerComposition}
Given $\Phi_1$ and $\Phi_2$ as in the premiss of Lemma~\vref{lemma:DerComposition}, the derivation $\Phi_1;\Phi_2$ constructed in the proof of the lemma is denoted
\[
\vlderivation
{
 \vlde{\Phi_2}{}{\gamma}
 {
  \vlde{\Phi_1}{}{\beta}
  {
   \vlhy{\alpha}
  }
 }
}\quad.
\]
\end{definition}

\section{Comparison with the Calculus of Structures}\label{section:CalculusOfStructures}

\TODO{Double check the following definition, that it is in line with Paola and and Alessio's complexity paper.}

\newcommand{\size}[1]{{\left\vert #1\right\vert}}\vlupdate\size
\begin{definition}\label{definition:DerSize}
The \emph{size} $\size\alpha$ of a formula $\alpha$, and the \emph{size} $\size\Phi$ of a derivation $\Phi$, is the number of unit and atom occurrences appearing in it.
\end{definition}

\begin{theorem}
The functorial calculus p-simulates the calculus of structures and the calculus of structures p-simulates the functorial calculus.
\end{theorem}

\begin{theorem}
A functorial calculus derivation corresponds to an equivalence class of calculus of structures derivations modulo bureaucracy of type A.
\end{theorem}

\TODO{Compare with the paper by Stephane and Kai.}