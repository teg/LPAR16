%-------------------------------------------

\hyphenation{co-weak-en-ing}
\hyphenation{La-mar-che}
\hyphenation{quasi-poly-no-mial}

\renewcommand{\le}{\leqslant}
\renewcommand{\ge}{\geqslant}

%-------------------------------------------

\newtheorem{theorem}{Theorem}[section]
\newtheorem{lemma}[theorem]{Lemma}
\newtheorem{proposition}[theorem]{Proposition}
\newtheorem{corollary}[theorem]{Corollary}

\theoremstyle{definition}
\newtheorem{definition}[theorem]{Definition}
\newtheorem{example}[theorem]{Example}

\theoremstyle{remark}
\newtheorem{remark}[theorem]{Remark}

%-------------------------------------------

\newif\ifnonotes\nonotesfalse
\newcommand{\Ale }[1]{\ifnonotes\else{\color{blue}     \noindent{\bf A: }#1}\fi}
\newcommand{\TODO}[1]{\ifnonotes\else{\color{red}    \noindent{\bf TODO }#1}\fi}
%\nonotestrue % If this is commented, notes appear in the paper (if any)

%-------------------------------------------

%TODO: put this somewhere clever:

\newcommand{\avecletter}{{\boldsymbol a}}
\newcommand{\avec}[2]{\avecletter_{#1}^{#2}}
\newcommand{\bvecletter}{{\boldsymbol b}}
\newcommand{\bvec}[2]{\bvecletter_{#1}^{#2}}
\newcommand{\Ord}[1]{{\mathsf O}(#1)}

%TODO: define this
\newcommand{\weaken}{\mathsf{w}}
\newcommand{\gwu}{{\weaken{\uparrow}}}
\newcommand{\gwd}{{\weaken{\uparrow}}}
\newcommand{\KS}{\mathsf{KS}}

%TODO: this is moved here to make the preview work
\renewcommand{\th}[2]{\mathop{\thetaup_{#1}^{#2}}}
