\chapter{Propositional Classical Logic}

\newcommand{\SKS}{\mathsf{SKS}}

\section{System $\SKS$}

\newcommand{\fff}{\mathsf f}
\newcommand{\ttt}{\mathsf t}
%---------------------------------------
\begin{definition}
\emph{Formulae}, $\alpha$, $\beta$, $\gamma$, $\delta$ are freely built from: \emph{units}, $\fff$ (false), $\ttt$ (true); \emph{atoms}, $a$, $b$, $c$, $d$, $e$; \emph{disjunction} and \emph{conjunction}, ${\vlsbr[\alpha.\beta]}$ and $\vlsbr(\alpha.\beta)$. On the set of atoms a (non-identical) involution $\bar\cdot$ is defined, and dual atom occurrences, as $a$ and $\bar a$, can appear in formulae. We denote \emph{contexts}, \emph{i.e.}, formulae with a hole, by $\xi\vlhole$ and $\zeta\vlhole$.
\end{definition}

%---------------------------------------
\begin{remark}
Negation is only defined for atoms, which is not a limitation thanks to De Morgan laws.
\end{remark}


\newcommand{\ai   }{\mathsf{ai}}
\newcommand{\aw   }{\mathsf{aw}}
\newcommand{\ac   }{\mathsf{ac}}
\newcommand{\aid  }{\ai{\downarrow}}
\newcommand{\awd  }{\aw{\downarrow}}
\newcommand{\acd  }{\ac{\downarrow}}
\newcommand{\aiu  }{\ai{\uparrow}}
\newcommand{\awu  }{\aw{\uparrow}}
\newcommand{\acu  }{\ac{\uparrow}}
\newcommand{\swi  }{\mathsf{s}}
\newcommand{\med  }{\mathsf{m}}
\newcommand{\asor }{=_{\vee\mathsf{a}}}
\newcommand{\asand}{=_{\wedge\mathsf{a}}}
\newcommand{\coor }{=_{\vee\mathsf{c}}}
\newcommand{\coand}{=_{\wedge\mathsf{c}}}
\newcommand{\unor }{=_{\vee\mathsf{u}}}
\newcommand{\unand}{=_{\wedge\mathsf{u}}}
\newcommand{\idor }{=_{\vee\mathsf{i}}}
\newcommand{\idand}{=_{\wedge\mathsf{i}}}

\begin{definition}
System $\SKS$ is defined by the following \emph{structural} rules:
\[
\vlinf{\aid}{}{\vls[a.\bar a]}{\ttt}\qquad
\vlinf{\awd}{}{a}{\fff}\qquad
\vlinf{\acd}{}{a}{\vls[a.a]}
\]
\[
\vlinf{\aiu}{}{\fff}{\vls[a.\bar a]}\qquad
\vlinf{\awu}{}{\ttt}{a}\qquad
\vlinf{\acu}{}{\vls(a.a)}{a}\quad,
\]
the two \emph{logical} rules
\[
\vlinf{\swi}{}{\vls[(\alpha.\beta).\gamma]}{\vls(\alpha.[\beta.\gamma])}\qquad
\vlinf{\med}{}{\vls([\alpha.\delta].[\beta.\gamma])}{\vls[(\alpha.\beta).(\gamma.\delta)]}
\]
and, for $\rho\in\{\coor,\coand,\asor,\asand,\unor,\unand,\idor,\idand\}$, the \emph{invertible} rule $\vlinf{\rho}{}{\delta}{\gamma}$, such that $\gamma$ and $\delta$ are opposite sides of
\vlstore{
\vls[\alpha.\beta]         &\coor\vls[\beta.\alpha]         \quad,&
\vls[\alpha.\fff]          &\unor\vls[\alpha]               \quad,\\
\vls(\alpha.\beta)         &\coand\vls(\beta.\alpha)         \quad,&
\vls(\alpha.\ttt)          &\unand\vls(\alpha)               \quad,\\
\vls[[\alpha.\beta].\gamma]&\asor\vls[\alpha.[\beta.\gamma]]\quad,&
\vls[\ttt.\ttt]            &\idor\vls[\ttt]                 \quad,\\
\vls((\alpha.\beta).\gamma)&\asand\vls(\alpha.(\beta.\gamma))\quad,&
\vls(\fff.\fff)            &\idand\vls(\fff)                 \quad,}
\begin{align*}
\vlread
\end{align*}
respectively.
\end{definition}

%TODO: check and prove (Complexity paper by Paola and Alessio)

\begin{lemma}
If there exists a derivation $\vlder{}{\{\coor,\coand,\asor,\asand,\unor,\unand,\idor,\idand\}}{\beta}{\alpha}$, then there exists a derivation $\vlder{\Phi}{\{\coor,\coand,\asor,\asand,\unor,\unand,\idor,\idand\}}{\beta}{\alpha}$, such that $|\Phi|\le|\alpha+\beta|^2$.
\end{lemma}

%TODO: reread

\begin{remark}
When $\vlder{}{\{\coor,\coand,\asor,\asand,\unor,\unand,\idor,\idand\}}{\beta}{\alpha}$, we say that $\alpha=\beta$ and we often write $\vlinf{=}{}{\beta}{\alpha}$ instead of the derivation. Using this shorthand only compresses derivations by a polynomial, so it does not affect any complexity results.
\end{remark}

\begin{lemma}\label{LemSuperSwitch}
Given a context $\xi\vlhole$ and a formula $\alpha$ there exist derivations $\vlder{}{\{\swi\}}{\xi\{\alpha\}}{\vls(\alpha.\xi\{\ttt\})}$ and $\vlder{}{\{\swi\}}{\vls[\xi\{\fff\}.\alpha]}{\xi\{\alpha\}}$.
\end{lemma}

\begin{proof}
We show how to construct the first derivation, the second one can be done symmetrically. We argue by induction on the number of atoms in $\xi\vlhole$. The base case, $\xi\vlhole=\vlhole$, is trivial and the inductive cases are:

\[
\vlderivation
{
 \vlin{=}{}{\xi\{\alpha\}}
 {
  \vlin{\swi}{}{\vlsbr[\vlder{\Psi}{\{\swi\}}{\xi'\{\alpha\}}{\vls(\alpha.\xi'\{\ttt\})}\;\;.\;\;\beta]}
  {
   \vlin{=}{}{\vls(\alpha.[\xi'\{\ttt\}.\beta])}
   {
    \vlhy{\vls(\alpha.\xi\{\ttt\})}
   }
  }
 }
}\qquad\mbox{and}\qquad
\vlderivation
{
 \vlin{=}{}{\xi\{\alpha\}}
 {
  \vlin{=}{}{\vlsbr(\vlder{\Psi'}{\{\swi\}}{\xi'\{\alpha\}}{\vls(\alpha.\xi'\{\ttt\})}\;\;.\;\;\beta)}
  {
   \vlhy{\vls(\alpha.\xi\{\ttt\})}
  }
 }
}\quad,
\]
for some $\xi'\vlhole$ and $\beta$ where $\beta$ is not a unit and $\Psi$ and $\Psi'$ exist by the inductive hypothesis.
\end{proof}

%TODO: introduce notation for super switch

\begin{lemma}\label{LemGenericContraction}
Given a formula $\alpha$ and a positive integer $n$, there exist derivations $\vlupsmash{\vlder{}{\{\acd,\med\}}{\alpha}{\bigvee_{i=1}^{n}\alpha}}$ and $\vlupsmash{\vlder{}{\{\acu,\med\}}{\bigwedge_{i=1}^{n}\alpha}{\alpha}}$.
\end{lemma}

\begin{proof}
We show how to construct the first derivation, the second one can be done symmetrically. We argue by induction on $n$. The first base case, $n=1$, is trivial and for the second base case, $n=2$, we argue by induction on the number of atom occurrences in $\alpha$. We have to consider the following base case and two inductive cases:
\[
\vlinf{\acd}{}{a}{\vls[a.a]}
\qquad\hbox{,}\qquad
\vlderivation
{
 \vlin{=}{}{\vls(\alpha.\beta)}
 {
  \vlin{\med}{}{\vlsbr(\vlder{}{\{\acd,\med\}}{\alpha}{\vls[\alpha.\alpha]}\;\;.\;\;\vlder{}{\{\acd,\med\}}{\beta}{\vls[\beta.\beta]})}
  {
   \vlhy{\vls[(\alpha.\beta).(\alpha.\beta)]}
  }
 }
}\qquad\hbox{and}\qquad
\vlderivation
{
 \vlin{=}{}{\vls[\alpha.\beta]}
 {
  \vlin{=}{}{\vlsbr[\vlder{}{\{\acd,\med\}}{\alpha}{\vls[\alpha.\alpha]}\;\;.\;\;\vlder{}{\{\acd,\med\}}{\beta}{\vls[\beta.\beta]}]}
  {
   \vlhy{\vls[[\alpha.\beta].[\alpha.\beta]]}
  }
 }
}\quad.
\]
Finally, we show the inductive case, $n>2$:
\[
\vlderivation
{
 \vlde{}{\{\acd,\med\}}{\alpha}
 {
  \vlin{=}{}{\vlsbr[\vlder{}{\{\acd,\med\}}{\alpha}{\bigvee_{i=1}^{n-\lfloor n/2\rfloor}\alpha}\;\;.\;\;\vlder{}{\{\acd,\med\}}{\alpha}{\bigvee_{i=1}^{\lfloor n/2\rfloor}\alpha}]}
  {
   \vlhy{\bigvee_{i=1}^{n}\alpha}
  }
 }
}\quad.
\]
\end{proof}

\newcommand{\contr}{\mathsf{c}}
\newcommand{\cod}{{\contr{\downarrow}}}
\newcommand{\cou}{{\contr{\uparrow}}}

\begin{remark}\label{RemGenericContraction}
In the non-atomic version of system $\SKS$ the derivations shown in Lemma~\ref{LemGenericContraction} correspond to repeated applications of (co)contractions. For this reason we sometimes write the inference rules $\vlinf{n\cdot\cod}{}{\alpha}{\bigvee_{i=1}^{n}\alpha}$ and $\vlinf{n\cdot\cou}{}{\bigwedge_{i=1}^{n}\alpha}{\alpha}$ instead of the derivations $\vlder{}{\{\acd,\med\}}{\alpha}{\bigvee_{i=1}^{n}\alpha}$ and $\vlupsmash{\vlder{}{\{\acu,\med\}}{\bigwedge_{i=1}^{n}\alpha}{\alpha}}$, respectively.
\end{remark}

\section{A System with Implication}

\newcommand{\inte}{\mathsf{i}}
\newcommand{\wea  }{\mathsf{w}}
\newcommand{\con  }{\mathsf{c}}
\newcommand{\wead }{{\wea{\downarrow}}}
\newcommand{\cond }{{\con{\downarrow}}}
\newcommand{\weau }{{\wea{\uparrow}}}
\newcommand{\conu }{{\con{\uparrow}}}
\newcommand{\swio }{\mathsf{s_1}}
\newcommand{\swit }{\mathsf{s_2}}
\newcommand{\asso }{\mathsf{ass}}

\begin{definition}

\[
\vlinf{\swi }{}{\vls[(\alpha.\beta).\gamma]}{\vls(\alpha.[\beta.\gamma])}\qquad
\vlinf{\swio}{}{\vls(\beta\vlim(\alpha.\gamma))}{\vls(\alpha.(\beta\vlim\gamma))}\qquad
\vlinf{\swit}{}{\vls((\alpha\vlim\beta)\vlim\gamma)}{\vls(\alpha.(\beta\vlim\gamma))}\qquad
\vlinf{\asso}{}{\vls[(\alpha\vlim\beta).\gamma]}{\vls(\alpha\vlim[\beta.\gamma])}
\]

\[
\vlinf{\cond}{}{\alpha}{\vls[\alpha.\alpha]}\qquad
\vlinf{\conu}{}{\vls(\alpha.\alpha)}{\alpha}\qquad
\vlinf{\wead}{}{\alpha}{\fff}\qquad
\vlinf{\weau}{}{\ttt}{\alpha}\qquad
\vlinf{\inte}{}{\vls(\alpha\vlim\alpha)}{\ttt}
\]
\end{definition}

\begin{theorem}
The above system is implicationally complete for propositional classical logic.
\end{theorem}