\newcommand{\SKS}{\mathsf{SKS}}

\section{System $\SKS$}

\newcommand{\ai   }{\mathsf{ai}}
\newcommand{\aw   }{\mathsf{aw}}
\newcommand{\ac   }{\mathsf{ac}}
\newcommand{\aid  }{\ai{\downarrow}}
\newcommand{\awd  }{\aw{\downarrow}}
\newcommand{\acd  }{\ac{\downarrow}}
\newcommand{\aiu  }{\ai{\uparrow}}
\newcommand{\awu  }{\aw{\uparrow}}
\newcommand{\acu  }{\ac{\uparrow}}
\newcommand{\swi  }{\mathsf{s}}
\newcommand{\med  }{\mathsf{m}}
\newcommand{\asor }{=_{\vee\mathsf{a}}}
\newcommand{\asand}{=_{\wedge\mathsf{a}}}
\newcommand{\coor }{=_{\vee\mathsf{c}}}
\newcommand{\coand}{=_{\wedge\mathsf{c}}}
\newcommand{\fffd }{{=_{\fff}}{\downarrow}}
\newcommand{\fffu }{{=_{\fff}}{\uparrow}}
\newcommand{\tttd }{{=_{\ttt}}{\downarrow}}
\newcommand{\tttu }{{=_{\ttt}}{\uparrow}}
\newcommand{\tttord }{{=_{\ttt\vee}}{\downarrow}}
\newcommand{\fffandd }{{=_{\fff\wedge}}{\downarrow}}
\newcommand{\tttoru }{{=_{\ttt\vee}}{\uparrow}}
\newcommand{\fffandu }{{=_{\fff\wedge}}{\uparrow}}


\begin{definition}\label{definition:SKS}
System $\SKS$ is defined by the following \emph{structural} inference rules:
\[
\vlinf{\aid}{}{\vls[a.\bar a]}{\ttt}\qquad
\vlinf{\awd}{}{a}{\fff}\qquad
\vlinf{\acd}{}{a}{\vls[a.a]}
\]
\[
\vlinf{\aiu}{}{\fff}{\vls(a.\bar a)}\qquad
\vlinf{\awu}{}{\ttt}{a}\qquad
\vlinf{\acu}{}{\vls(a.a)}{a}\quad,
\]
the \emph{logical} inference rules:
\[
\vlinf{\swi}{}{\vls[(A.B).C]}{\vls(A.[B.C])}\qquad
\vlinf{\med}{}{\vls([A.C].[B.D])}{\vls[(A.B).(C.D)]}
\quad,
\]
and the \emph{invertible} rules:
\[
\vlinf{\coor}{}{\vls[B.A]}{\vls[A.B]}\qquad
\vlinf{\coand}{}{\vls(B.A)}{\vls(A.B)}\qquad
\vlinf{\asor}{}{\vls[[A.B].C]}{\vls[A.[B.C]]}\qquad
\vlinf{\asand}{}{\vls((A.B).C)}{\vls(A.(B.C))}
\]
\[
\vlinf{\fffd}{}{\vls[A.\fff]}{A}\qquad
\vlinf{\tttd}{}{\vls(A.\ttt)}{A}\qquad
\vlinf{\fffu}{}{A}{\vls(A.\ttt)}\qquad
\vlinf{\tttu}{}{A}{\vls[A.\fff]}
\]
\[
\vlinf{\fffandd}{}{\vls(\fff.\fff)}{\fff}\qquad
\vlinf{\tttord}{}{\vls[\ttt.\ttt]}{\ttt}\qquad
\vlinf{\fffandu}{}{\ttt}{\vls[\ttt.\ttt]}\qquad
\vlinf{\tttoru}{}{\fff}{\vls(\fff.\fff)}\quad.
\]
\end{definition}

\TODO{Reference Complexity paper by Alessio and Paola:}

\begin{definition}\label{definition:EquationsShorthand}
If there is a derivation $\vlder{\Phi}{\{\coor,\coand,\asor,\asand,\fffd,\tttd,\fffu,\tttu,\fffandd,\tttord,\fffandu,\tttoru\}}{\beta}{\alpha}$, we say that $\alpha=\beta$ and we often write $\vlinf{=}{}{\beta}{\alpha}$ instead of\/ $\Phi$.
\end{definition}

\begin{remark}\label{remark:ContextClosure}
By Definition~\vref{definition:EquationsShorthand} and Lemma~\vref{lemma:DerInContext}, for any formulae $\alpha$ and $\beta$ and context $\xi\vlhole$ we have that $\alpha=\beta$ implies $\xi\{\alpha\}=\xi\{\beta\}$.
\end{remark}

\begin{remark}\label{remark:ImplicitEquations}
When we work in (subsystems of) $\SKS$, we often omit mentioning the invertible rules $\coor$, $\coand$, $\asor$, $\asand$, $\fffd$, $\tttd$, $\fffu$, $\tttu$, $\fffandd$, $\tttord$, $\fffandu$ and $\tttoru$. \emph{E.g.}, when we refer to the system $\{\swi\}$ we mean the system $\{\swi,\coor,\coand,\asor,\asand,\fffd,\tttd,\fffu,\tttu,\fffandd,\tttord,\fffandu,\tttoru\}$, and we write derivations like
\[
\vlinf{\swi}{}{\vls[\alpha.(\beta.\gamma)]}{\vls([\alpha.\beta].\gamma)}\quad,
\]
instead of
\[
\vlderivation
{
 \vlin{=}{}
 {
  \vls[\alpha.(\beta.\gamma)]
 }
 {
  \vlin{\swi}{}
  {
   \vls[(\gamma.\beta).\alpha]
  }
  {
   \vlin{=}{}
   {
    \vls(\gamma.[\beta.\alpha])
   }
   {
    \vlhy
    {
     \vls([\alpha.\beta].\gamma)
    }
   }
  }
 }
}\quad.
\]
See the proofs of Theorems~\vrefrange{theorem:PathBreakerSound}{theorem:MultipleIsolatedSubflowsRemovalMaybeSound} for more examples of implicit equations.
\end{remark}

\begin{lemma}\label{lemma:SuperSwitch}
Given a context $\xi\vlhole$ and a formula $\alpha$ there exist derivations $\vlder{}{\{\swi\}}{\xi\{\alpha\}}{\vls(\alpha.\xi\{\ttt\})}$ and $\vlder{}{\{\swi\}}{\vls[\xi\{\fff\}.\alpha]}{\xi\{\alpha\}}$.
\end{lemma}

\begin{proof}
We show how to construct the first derivation, the second one can be done symmetrically. We argue by induction on the number of atoms in $\xi\vlhole$. The base case, $\xi\vlhole=\vlhole$, is trivial and the inductive cases are:

\[
\vlderivation
{
 \vlin{=}{}{\xi\{\alpha\}}
 {
  \vlin{\swi}{}{\vlsbr[\vlder{\Psi}{\{\swi\}}{\xi'\{\alpha\}}{\vls(\alpha.\xi'\{\ttt\})}\;\;.\;\;\beta]}
  {
   \vlin{=}{}{\vls(\alpha.[\xi'\{\ttt\}.\beta])}
   {
    \vlhy{\vls(\alpha.\xi\{\ttt\})}
   }
  }
 }
}\qquad\mbox{and}\qquad
\vlderivation
{
 \vlin{=}{}{\xi\{\alpha\}}
 {
  \vlin{=}{}{\vlsbr(\vlder{\Psi'}{\{\swi\}}{\xi'\{\alpha\}}{\vls(\alpha.\xi'\{\ttt\})}\;\;.\;\;\beta)}
  {
   \vlhy{\vls(\alpha.\xi\{\ttt\})}
  }
 }
}\quad,
\]
for some $\xi'\vlhole$ and $\beta$ where $\beta$ is not a unit and $\Psi$ and $\Psi'$ exist by the inductive hypothesis.
\end{proof}

\newcommand{\supers}{\mathsf{ss}}
\newcommand{\ssu}{\supers\uparrow}
\newcommand{\ssd}{\supers\downarrow}

\begin{remark}\label{remark:SuperSwitch}
We often write $\vlinf{\ssu}{}{\xi\{\alpha\}}{\vls(\alpha.\xi\{\ttt\})}$ and $\vlinf{\ssd}{}{\vls[\xi\{\ttt\}.\alpha]}{\xi\{\alpha\}}$, instead of, respectively, the derivations $\vlder{}{\{\swi\}}{\xi\{\alpha\}}{\vls(\alpha.\xi\{\ttt\})}$ and $\vlder{}{\{\swi\}}{\vls[\xi\{\fff\}.\alpha]}{\xi\{\alpha\}}$, as defined in the proof of Lemma~\vref{lemma:SuperSwitch}. Instead of the derivation
\[
\vlinf{\swi}{}
{
 \vls[\zeta\{\fff\}\;.\;\vlinf{\ssu}{}{\xi\{\alpha\}}{\vls(\alpha.\xi\{\ttt\})}]
}
{
 \vls(\vlinf{\ssd}{}{\vls[\zeta\{\fff\}.\alpha]}{\zeta\{\alpha\}}\;.\;\xi\{\ttt\})
}
\]
we write $\vlinf{\supers}{}{\vls[\zeta\{\fff\}.\xi\{\alpha\}]}{\vls(\zeta\{\alpha\}.\xi\{\ttt\})}$.
\end{remark}

\begin{lemma}\label{lemma:GenericWeakening}
Given a formula $\alpha$, there exist derivations $\vlupsmash{\vlder{}{\{\awd,\swi\}}{\alpha}{\fff}}$ and $\vlupsmash{\vlder{}{\{\awu,\swi\}}{\ttt}{\alpha}}$.
\end{lemma}

\begin{proof}
We show how to construct the first derivation, the second one can be done symmetrically.
Let $a_1$, $\dots$, $a_n$ be the atoms appearing in $\alpha$, then there exists a derivation
\[
\vlder{}{\{\awd\}}{\alpha}{\alpha\{a_1/\fff,\dots,a_n/\fff\}}
\quad.
\]
Since $\alpha\{a_1/\fff,\dots,a_n/\fff\}$ contains no atoms, there exists a derivation
\[
\vlder{}{\{\fffd,\tttd,\fffandd,\tttord\}}{\alpha\{a_1/\fff,\dots,a_n/\fff\}}{\fff}
\qquad\mbox{or}\qquad
\vlderivation
{
 \vlde{}{\{\fffd,\tttd,\fffandd,\tttord\}}
 {
  \alpha\{a_1/\fff,\dots,a_n/\fff\}
 }
 {
  \vlin{=}{}
  {
   \ttt
  }
  {
   \vlin{\swi}{}
   {
    \vls[(\fff.\ttt).\ttt]
   }
   {
    \vlin{=}{}
    {
     \vls(\fff.[\ttt.\fff])
    }
    {
     \vlhy
     {
      \fff
     }
    }
   }
  }
 }
}
\quad.
\]
\end{proof}

\begin{lemma}\label{lemma:GenericContraction}
Given a formula $\alpha$, there exist derivations $\vlupsmash{\vlder{}{\{\acd,\med\}}{\alpha}{\vls[\alpha.\alpha]}}$ and $\vlupsmash{\vlder{}{\{\acu,\med\}}{\vls(\alpha.\alpha)}{\alpha}}$.
\end{lemma}

\begin{proof}
We show how to construct the first derivation, the second one can be done symmetrically. We argue by induction on the number of atoms appearing in $\alpha$. We have to consider the following base case and two inductive cases:
\[
\vlinf{}{}{a}{\vls[a.a]}
\qquad\hbox{,}\qquad
\vlinf{\med}{}
{
 \vls
 (
  \vlder{}{\{\acd,\med\}}
  {
   \alpha
  }
  {
   \vls[\alpha.\alpha]
  }
 \;\;.\;\;
  \vlder{}{\{\acd,\med\}}
  {
   \beta
  }
  {
   \vls[\beta.\beta]
  }
 )
}
{
 \vls[(\alpha.\beta).(\alpha.\beta)]
}
\qquad\hbox{and}\qquad
\vlinf{=}{}
{
 \vls
 [
  \vlder{}{\{\acd,\med\}}
  {
   \alpha
  }
  {
   \vls[\alpha.\alpha]
  }
 \;\;.\;\;
  \vlder{}{\{\acd,\med\}}
  {
   \beta
  }
  {
   \vls[\beta.\beta]
  }
 ]
}
{
 \vls[[\alpha.\beta].[\alpha.\beta]]
}
\quad.
\]
\end{proof}

%\begin{lemma}\label{lemma:RepeatedGenericContraction}
%Given a formula $\alpha$ and a positive integer $n$, there exist derivations $\vlupsmash{\vlder{}{\{\acd,\med\}}{\alpha}{\bigvee_{i=1}^{n}\alpha}}$ and $\vlupsmash{\vlder{}{\{\acu,\med\}}{\bigwedge_{i=1}^{n}\alpha}{\alpha}}$.
%\end{lemma}

%\begin{proof}
%We show how to construct the first derivation, the second one can be done symmetrically. We argue by induction on $n$. The first base case, $n=1$, is trivial and for the second base case, $n=2$, we argue by induction on the number of atom occurrences in $\alpha$. We have to consider the following base case and two inductive cases:
%\[
%\vlinf{\acd}{}{a}{\vls[a.a]}
%\qquad\hbox{,}\qquad
%\vlderivation
%{
% \vlin{=}{}{\vls(\alpha.\beta)}
% {
%  \vlin{\med}{}{\vlsbr(\vlder{}{\{\acd,\med\}}{\alpha}{\vls[\alpha.\alpha]}\;\;.\;\;\vlder{}{\{\acd,\med\}}{\beta}{\vls[\beta.\beta]})}
%  {
%   \vlhy{\vls[(\alpha.\beta).(\alpha.\beta)]}
%  }
% }
%}\qquad\hbox{and}\qquad
%\vlderivation
%{
% \vlin{=}{}{\vls[\alpha.\beta]}
% {
%  \vlin{=}{}{\vlsbr[\vlder{}{\{\acd,\med\}}{\alpha}{\vls[\alpha.\alpha]}\;\;.\;\;\vlder{}{\{\acd,\med\}}{\beta}{\vls[\beta.\beta]}]}
%  {
%   \vlhy{\vls[[\alpha.\beta].[\alpha.\beta]]}
%  }
% }
%}\quad.
%\]
%Finally, we show the inductive case, $n>2$:
%\[
%\vlderivation
%{
% \vlde{}{\{\acd,\med\}}{\alpha}
% {
%  \vlin{=}{}{\vlsbr[\vlder{}{\{\acd,\med\}}{\alpha}{\bigvee_{i=1}^{n-\lfloor n/2\rfloor}\alpha}\;\;.\;\;\vlder{}{\{\acd,\med\}}{\alpha}{\bigvee_{i=1}^{\lfloor n/2\rfloor}\alpha}]}
%  {
%   \vlhy{\bigvee_{i=1}^{n}\alpha}
%  }
% }
%}\quad.
%\]
%\end{proof}

\newcommand{\contr}{\mathsf{c}}
\newcommand{\cod}{{\contr{\downarrow}}}
\newcommand{\cou}{{\contr{\uparrow}}}
\newcommand{\weakn}{\mathsf{w}}
\newcommand{\wed}{{\weakn{\downarrow}}}
\newcommand{\weu}{{\weakn{\uparrow}}}

\begin{remark}\label{remark:GenericContraction}
In the non-atomic version of system $\SKS$ the derivations shown in the proofs of Lemma~\ref{lemma:GenericWeakening} and Lemma~\vref{lemma:GenericContraction} correspond to (co)weakening and (co)contractions, respectively. For this reason we sometimes write the inference rules $\vlinf{\wed}{}{\alpha}{\fff}$, $\vlinf{\weu}{}{\ttt}{\alpha}$, $\vlinf{\cod}{}{\alpha}{\vls[\alpha.\alpha]}$ and $\vlinf{\cou}{}{\vls(\alpha.\alpha)}{\alpha}$ instead of the derivations $\vlder{}{\{\awd,\swi\}}{\alpha}{\fff}$, $\vlder{}{\{\awu,\swi\}}{\ttt}{\alpha}$, $\vlder{}{\{\acd,\med\}}{\alpha}{\vls[\alpha.\alpha]}$ and $\vlupsmash{\vlder{}{\{\acu,\med\}}{\vls(\alpha.\alpha)}{\alpha}}$, respectively.
\end{remark}

\begin{theorem}\label{theorem:SKSComplete}
System $\SKS$ is complete for propositional classical logic.
\end{theorem}

\begin{proof}
Consider a tautology $\alpha$. We show by induction on the number of atoms in $\alpha$ that there exists a proof of $\alpha$ in $\SKS$. For the base case, let $\alpha$ consist only of units. Then, since $\alpha$ is a tautology, we can build
\[
\vlder{}{\{\fffd,\tttd,\fffandd,\tttord\}}{\alpha}{\ttt}\quad.
\]

For the inductive case, let $\alpha$ be a tautology containing instances of the atom $a$. We consider two cases:
\begin{itemize}
\item if $\alpha$ does not contain an instance of $\bar a$, then $\alpha\{a/\fff\}$ is a tautology, so by the inductive hypothesis we can build
\[
\vlderivation
{
 \vlde{}{\{\awd\}}
 {
  \alpha
 }
 {
  \vlde{}{}
  {
   \alpha\{a/\fff\}
  }
  {
   \vlhy{\ttt}
  }
 }
}\quad;
\]
\item otherwise, both $\alpha\{a/\ttt,\bar a/\fff\}$ and $\alpha\{a/\fff,\bar a/\ttt\}$ are tautologies, so by the inductive hypothesis we can build
\[
\vlderivation
{
 \vlin{\cod}{}
 {
  \alpha
 }
 {
  \vlin{\swi}{}
  {
   \vls
   [
    \vlder{}{\{\ssu\}}
    {
     \alpha
    }
    {
     \vlsbr
     (
      \vlder{}{\{\acu\}}
      {
       \vls(a.\cdots.a)
      }
      {
       a
      }
     \;\;.\;\;
      \vlder{}{\{\awd\}}
      {
       \alpha\{a/\ttt\}
      }
      {
       \alpha\{a/\ttt,\bar a/\fff\}
      }
     )
    }
   \;\;\;\;.\;\;\;\;
    \vlder{}{\{\ssu\}}
    {
     \alpha
    }
    {
     \vlsbr
     (
      \vlder{}{\{\acu\}}
      {
       \vls(\bar a.\cdots.\bar a)
      }
      {
       \bar a
      }
     \;\;.\;\;
      \vlder{}{\{\awd\}}
      {
       \alpha\{\bar a/\ttt\}
      }
      {
       \alpha\{a/\fff,\bar a/\ttt\}
      }
     )
    }
   ]
  }
  {
   \vlde{}{}
   {
    \vlsbr
    (
     \vlinf{\swi}{}
     {
      \vls[(a.\alpha\{a/\ttt,\bar a/\fff\}).\bar a]
     }
     {
      \vls
      (
       \vlinf{}{}{\vls[a.\bar a]}{\ttt}
      \;.\;
       \alpha\{a/\ttt,\bar a/\fff\}
      )
     }
    \;\;.\;\;
     \alpha\{a/\fff,\bar a/\ttt\}
    )
   }
   {
    \vlhy
    {
     \ttt
    }
   }
  }
 }
}\quad.
\]
\end{itemize}
\end{proof}
