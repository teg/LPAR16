\chapter{Propositional Classical Logic}

\newcommand{\fff}{\mathsf f}
\newcommand{\ttt}{\mathsf t}
%---------------------------------------
\begin{definition}
\emph{Formulae}, $\alpha$, $\beta$, $\gamma$, $\delta$ are freely built from: \emph{units}, $\fff$ (false), $\ttt$ (true); \emph{atoms}, $a$, $b$, $c$, $d$, $e$; \emph{disjunction} and \emph{conjunction}, ${\vlsbr[\alpha.\beta]}$ and $\vlsbr(\alpha.\beta)$. On the set of atoms a (non-identical) involution $\bar\cdot$ is defined, and dual atom occurrences, as $a$ and $\bar a$, can appear in formulae. We denote \emph{contexts}, \emph{i.e.}, formulae with a hole, by $\xi\vlhole$ and $\zeta\vlhole$.
\end{definition}

%---------------------------------------
\begin{remark}
Negation is only defined for atoms, which is not a limitation thanks to De Morgan laws.
\end{remark}

\TODO{Define how the inference rules we are about to define are used to generate the inference rule instances.}

\newcommand{\ai   }{\mathsf{ai}}
\newcommand{\aw   }{\mathsf{aw}}
\newcommand{\ac   }{\mathsf{ac}}
\newcommand{\aid  }{\ai{\downarrow}}
\newcommand{\awd  }{\aw{\downarrow}}
\newcommand{\acd  }{\ac{\downarrow}}
\newcommand{\aiu  }{\ai{\uparrow}}
\newcommand{\awu  }{\aw{\uparrow}}
\newcommand{\acu  }{\ac{\uparrow}}
\newcommand{\swi  }{\mathsf{s}}
\newcommand{\med  }{\mathsf{m}}
\newcommand{\asor }{=_{\vee\mathsf{a}}}
\newcommand{\asand}{=_{\wedge\mathsf{a}}}
\newcommand{\coor }{=_{\vee\mathsf{c}}}
\newcommand{\coand}{=_{\wedge\mathsf{c}}}
\newcommand{\unor }{=_{\vee\mathsf{u}}}
\newcommand{\unand}{=_{\wedge\mathsf{u}}}
\newcommand{\idor }{=_{\vee\mathsf{i}}}
\newcommand{\idand}{=_{\wedge\mathsf{i}}}

\newcommand{\SKS}{\mathsf{SKS}}

\begin{definition}
System $\SKS$ is defined by the following \emph{structural} rules:
\[
\vlinf{\aid}{}{\vls[a.\bar a]}{\ttt}\qquad
\vlinf{\awd}{}{a}{\fff}\qquad
\vlinf{\acd}{}{a}{\vls[a.a]}
\]
\[
\vlinf{\aiu}{}{\fff}{\vls[a.\bar a]}\qquad
\vlinf{\awu}{}{\ttt}{a}\qquad
\vlinf{\acu}{}{\vls(a.a)}{a}\quad,
\]
the two \emph{logical} rules
\[
\vlinf{\swi}{}{\vls[(\alpha.\beta).\gamma]}{\vls(\alpha.[\beta.\gamma])}\qquad
\vlinf{\med}{}{\vls([\alpha.\delta].[\beta.\gamma])}{\vls[(\alpha.\beta).(\gamma.\delta)]}
\]
and, for each $\rho\in\{\coor,\coand,\asor,\asand,\unor,\unand,\idor,\idand\}$, the \emph{invertible} rule $\vlinf{\rho}{}{\delta}{\gamma}$, such that $\gamma$ and $\delta$ are opposite sides of
\vlstore{
\vls[\alpha.\beta]         &\coor\vls[\beta.\alpha]         \quad,&
\vls[\alpha.\fff]          &\unor\vls[\alpha]               \quad,\\
\vls(\alpha.\beta)         &\coand\vls(\beta.\alpha)         \quad,&
\vls(\alpha.\ttt)          &\unand\vls(\alpha)               \quad,\\
\vls[[\alpha.\beta].\gamma]&\asor\vls[\alpha.[\beta.\gamma]]\quad,&
\vls[\ttt.\ttt]            &\idor\vls[\ttt]                 \quad,\\
\vls((\alpha.\beta).\gamma)&\asand\vls(\alpha.(\beta.\gamma))\quad,&
\vls(\fff.\fff)            &\idand\vls(\fff)                 \quad,}
\begin{align*}
\vlread
\end{align*}
respectively.
\end{definition}

\begin{theorem}
A deep inference system $\mathcal{S}$ is complete for propositional classical logic if
\begin{enumerate}
 \item Given a formula $\alpha$ and an atom $a$, there exists a derivation 
  \[
   \vlder{}{\mathcal{S}}
   {
    \alpha
   }
   {
    \vls(\alpha\{a/\ttt,\bar a/\fff\}.\alpha\{a/\fff,\bar a/\ttt\})
   }\quad
  \]
 and
 \item Given a formula $\alpha$ that is logically equivalent to $\ttt$ and that does not contain any atoms, there exists a derivation
  \[
   \vlder{}{\mathcal{S}}
   {
    \alpha
   }
   {
    \ttt
   }\quad.
  \]
\end{enumerate}
\end{theorem}

\begin{proof}[Sketch of Proof]
Let $\alpha$ be a tautology. Apply (1) repeatedly to obtain a conjunction of formulae not containing atoms. Each of the conjuncts is a truth value assignment to $\alpha$, so the conjunction is logically equivalent to $\ttt$. Apply (2) to obtain the proof.
\end{proof}


\TODO{Check and prove (Complexity paper by Alessio and Paola).}

\begin{lemma}
If there exists a derivation $\vlder{}{\{\coor,\coand,\asor,\asand,\unor,\unand,\idor,\idand\}}{\beta}{\alpha}$, then there exists a derivation $\vlder{\Phi}{\{\coor,\coand,\asor,\asand,\unor,\unand,\idor,\idand\}}{\beta}{\alpha}$, such that $|\Phi|\le(|\alpha|+|\beta|)^2$.
\end{lemma}

%TODO: reread

\begin{remark}
When $\vlder{}{\{\coor,\coand,\asor,\asand,\unor,\unand,\idor,\idand\}}{\beta}{\alpha}$, we say that $\alpha=\beta$ and we often write $\vlinf{=}{}{\beta}{\alpha}$ instead of the derivation. Using this shorthand only compresses derivations by a polynomial, so it does not affect any complexity results.
\end{remark}

\begin{lemma}\label{LemSuperSwitch}
Given a context $\xi\vlhole$ and a formula $\alpha$ there exist derivations $\vlder{}{\{\swi\}}{\xi\{\alpha\}}{\vls(\alpha.\xi\{\ttt\})}$ and $\vlder{}{\{\swi\}}{\vls[\xi\{\fff\}.\alpha]}{\xi\{\alpha\}}$.
\end{lemma}

\begin{proof}
We show how to construct the first derivation, the second one can be done symmetrically. We argue by induction on the number of atoms in $\xi\vlhole$. The base case, $\xi\vlhole=\vlhole$, is trivial and the inductive cases are:

\[
\vlderivation
{
 \vlin{=}{}{\xi\{\alpha\}}
 {
  \vlin{\swi}{}{\vlsbr[\vlder{\Psi}{\{\swi\}}{\xi'\{\alpha\}}{\vls(\alpha.\xi'\{\ttt\})}\;\;.\;\;\beta]}
  {
   \vlin{=}{}{\vls(\alpha.[\xi'\{\ttt\}.\beta])}
   {
    \vlhy{\vls(\alpha.\xi\{\ttt\})}
   }
  }
 }
}\qquad\mbox{and}\qquad
\vlderivation
{
 \vlin{=}{}{\xi\{\alpha\}}
 {
  \vlin{=}{}{\vlsbr(\vlder{\Psi'}{\{\swi\}}{\xi'\{\alpha\}}{\vls(\alpha.\xi'\{\ttt\})}\;\;.\;\;\beta)}
  {
   \vlhy{\vls(\alpha.\xi\{\ttt\})}
  }
 }
}\quad,
\]
for some $\xi'\vlhole$ and $\beta$ where $\beta$ is not a unit and $\Psi$ and $\Psi'$ exist by the inductive hypothesis.
\end{proof}

\newcommand{\supers}{\mathsf{ss}}
\newcommand{\ssu}{\supers\uparrow}
\newcommand{\ssd}{\supers\downarrow}

\TODO{Point out that `only containing switches' or $\{\swi\}$ allows the use of equations.}
\TODO{Define `macro rules'}

\begin{remark}\label{RemSuperSwitch}
To ease legibility we introduce three macro rules for some derivations built up of switches. Instead of the derivations $\vlder{}{\{\swi\}}{\xi\{\alpha\}}{\vls(\alpha.\xi\{\ttt\})}$ and $\vlder{}{\{\swi\}}{\vls[\xi\{\fff\}.\alpha]}{\xi\{\alpha\}}$, as defined in the proof of Lemma~\ref{LemSuperSwitch}, we write $\vlinf{\ssu}{}{\xi\{\alpha\}}{\vls(\alpha.\xi\{\ttt\})}$ and $\vlinf{\ssd}{}{\vls[\xi\{\ttt\}.\alpha]}{\xi\{\alpha\}}$, respectively. Instead of the derivation
\[
\vlinf{\swi}{}
{
 \vls[\zeta\{\fff\}\;.\;\vlinf{\ssu}{}{\xi\{\alpha\}}{\vls(\alpha.\xi\{\ttt\})}]
}
{
 \vls(\vlinf{\ssd}{}{\vls[\zeta\{\fff\}.\alpha]}{\zeta\{\alpha\}}\;.\;\xi\{\ttt\})
}
\]
we write $\vlinf{\supers}{}{\vls[\zeta\{\fff\}.\xi\{\alpha\}]}{\vls(\zeta\{\alpha\}.\xi\{\ttt\})}$.
\end{remark}

\TODO{Use notation for super switch.}

\begin{lemma}\label{LemGenericContraction}
Given a formula $\alpha$ and a positive integer $n$, there exist derivations $\vlupsmash{\vlder{}{\{\acd,\med\}}{\alpha}{\bigvee_{i=1}^{n}\alpha}}$ and $\vlupsmash{\vlder{}{\{\acu,\med\}}{\bigwedge_{i=1}^{n}\alpha}{\alpha}}$.
\end{lemma}

\begin{proof}
We show how to construct the first derivation, the second one can be done symmetrically. We argue by induction on $n$. The first base case, $n=1$, is trivial and for the second base case, $n=2$, we argue by induction on the number of atom occurrences in $\alpha$. We have to consider the following base case and two inductive cases:
\[
\vlinf{\acd}{}{a}{\vls[a.a]}
\qquad\hbox{,}\qquad
\vlderivation
{
 \vlin{=}{}{\vls(\alpha.\beta)}
 {
  \vlin{\med}{}{\vlsbr(\vlder{}{\{\acd,\med\}}{\alpha}{\vls[\alpha.\alpha]}\;\;.\;\;\vlder{}{\{\acd,\med\}}{\beta}{\vls[\beta.\beta]})}
  {
   \vlhy{\vls[(\alpha.\beta).(\alpha.\beta)]}
  }
 }
}\qquad\hbox{and}\qquad
\vlderivation
{
 \vlin{=}{}{\vls[\alpha.\beta]}
 {
  \vlin{=}{}{\vlsbr[\vlder{}{\{\acd,\med\}}{\alpha}{\vls[\alpha.\alpha]}\;\;.\;\;\vlder{}{\{\acd,\med\}}{\beta}{\vls[\beta.\beta]}]}
  {
   \vlhy{\vls[[\alpha.\beta].[\alpha.\beta]]}
  }
 }
}\quad.
\]
Finally, we show the inductive case, $n>2$:
\[
\vlderivation
{
 \vlde{}{\{\acd,\med\}}{\alpha}
 {
  \vlin{=}{}{\vlsbr[\vlder{}{\{\acd,\med\}}{\alpha}{\bigvee_{i=1}^{n-\lfloor n/2\rfloor}\alpha}\;\;.\;\;\vlder{}{\{\acd,\med\}}{\alpha}{\bigvee_{i=1}^{\lfloor n/2\rfloor}\alpha}]}
  {
   \vlhy{\bigvee_{i=1}^{n}\alpha}
  }
 }
}\quad.
\]
\end{proof}

\newcommand{\contr}{\mathsf{c}}
\newcommand{\cod}{{\contr{\downarrow}}}
\newcommand{\cou}{{\contr{\uparrow}}}

\begin{remark}\label{RemGenericContraction}
In the non-atomic version of system $\SKS$ the derivations shown in Lemma~\ref{LemGenericContraction} correspond to repeated applications of (co)contractions. For this reason we sometimes write the inference rules $\vlinf{n\cdot\cod}{}{\alpha}{\bigvee_{i=1}^{n}\alpha}$ and $\vlinf{n\cdot\cou}{}{\bigwedge_{i=1}^{n}\alpha}{\alpha}$ instead of the derivations $\vlder{}{\{\acd,\med\}}{\alpha}{\bigvee_{i=1}^{n}\alpha}$ and $\vlupsmash{\vlder{}{\{\acu,\med\}}{\bigwedge_{i=1}^{n}\alpha}{\alpha}}$, respectively.
\end{remark}

%===============================================================================
\section{Threshold Formulae}\label{SectThresh}

\TODO{Make some comments about how this subsection will both serve as an introduction to many deep inference techniques as well as prepare definitions for later use.}

\TODO{Integrate this section better.}

\TODO{Check that is the most up-to-date relative to the submitted paper.}

%REMOVE THIS
\newcommand{\Gammasf}{\mathsf\Gamma}

\begin{remark}
\[
\th{k}{n-1}(\avec{1}{n}\setminus a_i)=\th{k}{n}\avec{1}{n}\{a_i/\fff\}
\]
\end{remark}

Given a threshold formula $\th kn\avec1n$, we can consider, for each $a_l$ such that $1\le l\le n$, the formulae $(\th kn\avec1n)\{a_l/\fff\}$ and $(\th{k+1}n\avec1n)\{a_l/\ttt\}$: we call both of them, informally, `pseudocomplements' of $a_l$. The reason for this name is that we can manage to replace, in a given proof, all occurrences of those $\bar a_l$ that appear in cut instances with the pseudocomplements of $a_l$. The cut instances and their corresponding identity instances are then removed, leaving us with derivations whose premiss and conclusion contain each a threshold formula. Moreover, the $k$-level of the threshold formula in the premiss is one less than the $k$-level of the threshold formula in the conclusion. This way, we obtain several derivations, corresponding to increasing values of $k$, that we are able to stitch together until we get a normalised proof.

All this, of course, needs clarification, but we think that it is helpful to provide a summary here of the main constructions that allow for this stitching operation. Let us read derivations top-down; the following are the steps that we need to perform, for $0\le k\le n$.
\begin{enumerate}
%---------------------------------------
\item\label{ItemOne} Build
\[
\vlder{}{}{\vlsmallbrackets\vls[a_l.(\th kn\avec1n)\{a_l/\fff\}]}
          {\th kn\avec1n}
\quad,
\]
\emph{i.e.}, create, from a $k$-level threshold formula, a disjunction between $a_l$ and its pseudocomplement $(\th kn\avec1n)\{a_l/\fff\}$ (Proposition~\ref{PropAuxNorm}); then replace the pseudocomplement into $\bar a_l$, for each identity instance.
%---------------------------------------
\item\label{ItemTwo} Increase the $k$-level by using the derivations
\[
\vlder{}{}{(\th{k+1}n\avec1n)\{a_l/\ttt\}}
          {(\th kn\avec1n)\{a_l/\fff\}}
\]
(Theorem~\ref{TheoThrDer}); these are the $\Gammasf$ derivations mentioned in the introduction to this section.
%---------------------------------------
\item\label{ItemThree} For each cut instance, collect the conjunction between $a_l$ and its pseudocomplement $(\th{k+1}n\avec1n)\{a_l/\ttt\}$; then build
\[
\vlder{}{}{\th{k+1}n\avec1n}
          {\vlsmallbrackets\vls(a_l.(\th{k+1}n\avec1n)\{a_l/\ttt\})}
\quad,
\]
\emph{i.e.}, create a $(k+1)$-level threshold formula (Proposition~\ref{PropAuxNorm}).
%---------------------------------------
\end{enumerate}
The derivations mentioned above do not require any use of identity and cut, and allow us to move, in $n+1$ steps, from $\th 0n\avec1n\equiv\ttt$ to $\th{n+1}n\avec1n\equiv\fff$, which is the secret to success. The constructions in~\ref{ItemOne} and \ref{ItemThree} are deep-inference routine and introduce low complexity. We deal now with the crucial step~\ref{ItemTwo}, by designing Definition~\ref{DefThrDer}, and then checking it carefully, so as to get the property stated in Theorem~\ref{TheoThrDer}.

Definition~\ref{DefThrDer} is technical, but its philosophy is simple; all one has to do to build the derivations is to
\begin{itemize}
 \item identify the atom occurrences in the premiss that do not occur in the conclusion and remove them using coweakenings and
 \item identify the atom occurrences in the conclusion that do not occur in the premiss and add them using weakenings.
\end{itemize}
Adding and removing atom occurrences deep inside a formula in this way is greatly simplified by the fact that we are using deep inference.

In order to be certain of its correctness, we have implemented it as a program \cite{Gugl:09:th.pl:rz}. The presentation could be slightly simplified, but we prefer it this way because this exactly corresponds to the implementation. It can be useful to read the definition together with the examples in Figures~\ref{FigPThEx} and \ref{FigThrEx}, which have been generated by the program, and by keeping in mind that the goal is to obtain the property stated in Theorem~\ref{TheoThrDer}.

\newcommand{\Gth}[3]{\mathop{\Gammasf_{#1,#2}^{#3}}}
%-------------------------------------------------------------------------------
\begin{figure}
\begin{eqnarray*}
%---------------------------------------
\Gth 015\avecletter&=&
\vls [\ttt.\vlderivation{
\vlin{}{}{b}{
\vlhy{\vls \fff}
}}
.\vlderivation{
\vlin{}{}{\vls [c.d.e]}{
\vlhy{\vls \fff}
}}
]\quad,\\
\noalign{\smallskip}
%---------------------------------------
\Gth 115\avecletter&=&
\vls [b.([\ttt.\vlderivation{
\vlin{}{}{b}{
\vlhy{\vls \fff}
}}
].[c.d.e]).\vlderivation{
\vlin{}{}{\vls [(c.[d.e]).(d.e)]}{
\vlhy{\vls \fff}
}}
]\quad,\\
\noalign{\smallskip}
%---------------------------------------
\Gth 215\avecletter&=&
\vls [(b.[c.d.e]).([\ttt.\vlderivation{
\vlin{}{}{b}{
\vlhy{\vls \fff}
}}
].[(c.[d.e]).(d.e)]).\vlderivation{
\vlin{}{}{\vls \fff}{
\vlhy{\vls (\fff.b)}
}}
.\vlderivation{
\vlin{}{}{\vls (c.d.e)}{
\vlhy{\vls \fff}
}}
]\quad,\\
\noalign{\smallskip}
%---------------------------------------
\Gth 315\avecletter&=&
\vls [(b.[(c.[d.e]).(d.e)]).([\ttt.\vlderivation{
\vlin{}{}{b}{
\vlhy{\vls \fff}
}}
].c.d.e).\vlderivation{
\vlin{}{}{\vls \fff}{
\vlhy{\vls (\fff.b.[c.d.e])}
}}
]\quad,\\
\noalign{\smallskip}
%---------------------------------------
\Gth 415\avecletter&=&
\vls [(b.c.d.e).\vlderivation{
\vlin{}{}{\vls \fff}{
\vlhy{\vls (\fff.b.[(c.[d.e]).(d.e)])}
}}
]\quad,\\
\noalign{\smallskip}
%---------------------------------------
\Gth 515\avecletter&=&
\vlderivation{
\vlin{}{}{\vls \fff}{
\vlhy{\vls (\fff.b.c.d.e)}
}}
\quad,\\
\noalign{\smallskip}
%---------------------------------------
\Gth 035\avecletter&=&
\vls [\ttt.\vlderivation{
\vlin{}{}{\vls [d.e]}{
\vlhy{\vls \fff}
}}
.\vlderivation{
\vlin{}{}{\vls [a.b]}{
\vlhy{\vls \fff}
}}
]\quad,\\
\noalign{\smallskip}
%---------------------------------------
\Gth 135\avecletter&=&
\vls [([a.b].[\ttt.\vlderivation{
\vlin{}{}{\vls [d.e]}{
\vlhy{\vls \fff}
}}
]).d.e.\vlderivation{
\vlin{}{}{\vls (d.e)}{
\vlhy{\vls \fff}
}}
.\vlderivation{
\vlin{}{}{\vls (a.b)}{
\vlhy{\vls \fff}
}}
]\quad,\\
\noalign{\smallskip}
%---------------------------------------
\Gth 235\avecletter&=&
\vls [(a.b.[\ttt.\vlderivation{
\vlin{}{}{\vls [d.e]}{
\vlhy{\vls \fff}
}}
]).([a.b].[d.e.\vlderivation{
\vlin{}{}{\vls (d.e)}{
\vlhy{\vls \fff}
}}
]).(d.e).\vlderivation{
\vlin{}{}{\vls \fff}{
\vlhy{\vls (\fff.[d.e])}
}}
]\quad,\\
\noalign{\smallskip}
%---------------------------------------
\Gth 335\avecletter&=&
\vls [(a.b.[d.e.\vlderivation{
\vlin{}{}{\vls (d.e)}{
\vlhy{\vls \fff}
}}
]).([a.b].[(d.e).\vlderivation{
\vlin{}{}{\vls \fff}{
\vlhy{\vls (\fff.[d.e])}
}}
]).\vlderivation{
\vlin{}{}{\vls \fff}{
\vlhy{\vls (\fff.d.e)}
}}
]\quad,\\
\noalign{\smallskip}
%---------------------------------------
\Gth 435\avecletter&=&
\vls [(a.b.[(d.e).\vlderivation{
\vlin{}{}{\vls \fff}{
\vlhy{\vls (\fff.[d.e])}
}}
]).\vlderivation{
\vlin{}{}{\vls \fff}{
\vlhy{\vls ([a.b].\fff.d.e)}
}}
]\quad,\\
\noalign{\smallskip}
%---------------------------------------
\Gth 535\avecletter&=&
\vlderivation{
\vlin{}{}{\vls \fff}{
\vlhy{\vls (a.b.\fff.d.e)}
}}\quad,\\
\noalign{\smallskip}
%---------------------------------------
\Gth 055\avecletter&=&
\vls [\ttt.\vlderivation{
\vlin{}{}{d}{
\vlhy{\vls \fff}
}}
.\vlderivation{
\vlin{}{}{c}{
\vlhy{\vls \fff}
}}
.\vlderivation{
\vlin{}{}{\vls [a.b]}{
\vlhy{\vls \fff}
}}
]\quad,\\
\noalign{\smallskip}
%---------------------------------------
\Gth 155\avecletter&=&
\vls [([a.b].[\ttt.\vlderivation{
\vlin{}{}{d}{
\vlhy{\vls \fff}
}}
.\vlderivation{
\vlin{}{}{c}{
\vlhy{\vls \fff}
}}
]).(c.[\ttt.\vlderivation{
\vlin{}{}{d}{
\vlhy{\vls \fff}
}}
]).d.\vlderivation{
\vlin{}{}{\vls (a.b)}{
\vlhy{\vls \fff}
}}
]\quad,\\
\noalign{\smallskip}
%---------------------------------------
\Gth 255\avecletter&=&
\vls [(a.b.[\ttt.\vlderivation{
\vlin{}{}{d}{
\vlhy{\vls \fff}
}}
.\vlderivation{
\vlin{}{}{c}{
\vlhy{\vls \fff}
}}
]).([a.b].[(c.[\ttt.\vlderivation{
\vlin{}{}{d}{
\vlhy{\vls \fff}
}}
]).d]).(c.d).\vlderivation{
\vlin{}{}{\vls \fff}{
\vlhy{\vls (d.\fff)}
}}
]\quad,\\
\noalign{\smallskip}
%---------------------------------------
\Gth 355\avecletter&=&
\vls [(a.b.[(c.[\ttt.\vlderivation{
\vlin{}{}{d}{
\vlhy{\vls \fff}
}}
]).d]).([a.b].[(c.d).\vlderivation{
\vlin{}{}{\vls \fff}{
\vlhy{\vls (d.\fff)}
}}
]).\vlderivation{
\vlin{}{}{\vls \fff}{
\vlhy{\vls (c.d.\fff)}
}}
]\quad,\\
\noalign{\smallskip}
%---------------------------------------
\Gth 455\avecletter&=&
\vls [(a.b.[(c.d).\vlderivation{
\vlin{}{}{\vls \fff}{
\vlhy{\vls (d.\fff)}
}}
]).\vlderivation{
\vlin{}{}{\vls \fff}{
\vlhy{\vls ([a.b].c.d.\fff)}
}}
]\quad,\\
\noalign{\smallskip}
%---------------------------------------
\Gth 555\avecletter&=&
\vlderivation{
\vlin{}{}{\vls \fff}{
\vlhy{\vls (a.b.c.d.\fff)}
}}\quad.
\end{eqnarray*}
\caption{Examples of $\Gth kl5\avecletter$, where $\avecletter=(a,b,c,d,e)$.}
\label{FigPThEx}
\end{figure}

\TODO{Define generic weakening.}

%-------------------------------------------------------------------------------
\begin{remark}
Given $n>1$, let $p=\lfloor n/2\rfloor$ and $q=n-p$. For $0\le k\le q$ and $1\le l\le p$, the following derivation is well defined:
\[
\vlinf{\gwu}
      {}
      {\fff}
      {\vls({\vlnos(\th pp\avec1p)}\{a_l/\fff\}.\th kq\avec{p+1}n)}
=
\vls(
\vlinf{\gwu}
      {}
      {\vls(\ttt)}
      {\vls(a_1.\cdots.a_{l-1}.a_{l+1}.\cdots.a_p.\th kq\avec{p+1}n)}
.\fff)
\quad.
\]
Analogously, for $0\le k\le p$ and $p+1\le l\le n$, we can define the following derivation:
\[
\vlinf{\gwu}
      {}
      {\fff}
      {\vls(\th kp\avec1p.{\vlnos(\th qq\avec{p+1}n)}\{a_l/\fff\})}
=
\vls(
\vlinf{\gwu}
      {}
      {\vls(\ttt)}
      {\vls(\th kp\avec1p.a_{p+1}.\cdots.a_{l-1}.a_{l+1}.\cdots.a_n)}
.\fff)
\quad.
\]
Both classes of derivations are used in Definition~\ref{DefThrDer}.
\end{remark}

