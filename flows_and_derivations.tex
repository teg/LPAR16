\chapter{Flows and Derivations}

\section{Extracting Flows from Derivations}

%TODO: define atomic flows to always be considered modulo isomorphism

%---------------------------------------
\begin{proposition}\label{PropUnFl}
Given an\/ $\SKS$ derivation\/ $\Phi$, there is a unique atomic flow $\phi$ (modulo isomorphisms) such that:
\begin{enumerate}
%-------------------
\item there is a surjective map between the set of atom occurrences of\/ $\Phi$ and the set of edges of $\phi$;
%-------------------
\item for each vertical composition $\vlsmash{\vlinf{\rho}{}{\Phi_2}{\Phi_1}}$ of\/ $\Phi$, where $\rho\in\{\aid,\aiu,\awd,\awu,\acd,\acu\}$ and $\vlinf{\rho}{}{\beta}{\alpha}$ is a rule instance, the atom occurrences in $\vlinf{\rho}{}{\beta}{\alpha}$ are mapped to edges of $\phi$ such that the edges are related with vertices as indicated below, for each possible case of the inference rules:
\[
\begin{array}{@{}ccc@{}ccc@{}}
\vlinf{\aid}{}{\vls[a^\one.{\bar a^\two}]}{\ttt}&\mbox{to\/}&
\vcenter{\afaid\one{}{}\two{}{}}
\quad,&\qquad
\vlinf{\aiu}{}{\fff}{\vls(a^\one.{\bar a^\two})}&\mbox{to\/}&
\vcenter{\afaiu\one{}{}\two{}{}}
\quad,\\
\noalign{\medskip}
\vlinf{\awd}{}{a^\one}{\fff}                    &\mbox{to\/}&
\vcenter{\afawd{}{}{}\one{}} 
\quad,&\qquad
\vlinf{\awu}{}{\ttt}{a^\one}                    &\mbox{to\/}&
\vcenter{\afawu{}{}{}\one{}}
\quad,\\
\noalign{\medskip}
\vlinf{\acd}{}{a^\three}{\vls[a^\one.a^\two]}   &\mbox{to\/}&
\vcenter{\afacd\one{}{}\two{}\three}
\quad,&\qquad
\vlinf{\acu}{}{\vls(a^\two.a^\three)}{a^\one}   &\mbox{to\/}&
\vcenter{\afacu\two{}{}\three{}\one}
\quad,\\
\end{array}
\]
where the mapping is indicated by small numerals.
%-------------------
\item for each vertical composition of\/ $\Phi$ of kind
\[\hss
\begin{array}{@{}r@{}l@{}}
\vlinf{\swi}{}{\vls[(\alpha.\beta).\gamma]}
              {\vls(\alpha.[\beta.\gamma])}           \quad,&\qquad
\vlinf{\med}{}{\vls([\alpha.\gamma].[\beta.\delta])}
              {\vls[(\alpha.\beta).(\gamma.\delta)]}  \quad,      \\
\noalign{\smallskip}
\vlinf={}{\vls[\beta.\alpha]}{\vls[\alpha.\beta]}\quad,&\qquad
\vlinf={}{\vls(\beta.\alpha)}{\vls(\alpha.\beta)}\quad,      \\
\noalign{\smallskip}
\vlinf={}{\vls[\alpha.[\beta.\gamma]]}
         {\vls[[\alpha.\beta].\gamma]}                \quad,&\qquad
\vlinf={}{\vls[[\alpha.\beta].\gamma]}
         {\vls[\alpha.[\beta.\gamma]]}                \quad,      \\
\noalign{\smallskip}
\vlinf={}{\vls(\alpha.(\beta.\gamma))}
         {\vls((\alpha.\beta).\gamma)}                \quad,&\qquad
\vlinf={}{\vls((\alpha.\beta).\gamma)}
         {\vls(\alpha.(\beta.\gamma))}                \quad,      \\
\noalign{\smallskip}
\vlinf={}{\{\alpha\}}{\vls[\alpha.\fff]}           \quad,\qquad
\vlinf={}{\vls[\alpha.\fff]}{\{\alpha\}}           \quad,&\qquad
\vlinf={}{\{\alpha\}}{\vls(\alpha.\ttt)}        \qquad\hbox{and\/}\qquad
\vlinf={}{\vls(\alpha.\ttt)}{\{\alpha\}}
\end{array}
\]
all the atom occurrences in $\alpha$, $\beta$, $\gamma$ and $\delta$ in the premiss are respectively mapped to the same edges of $\phi$ as the atom occurrences in $\alpha$, $\beta$, $\gamma$ and $\delta$ in the conclusion.
\end{enumerate}
\end{proposition}

%---------------------------------------
\begin{definition}
Given a derivation $\Phi$, we say that the unique atomic flow $\phi$ defined in Proposition~\ref{PropUnFl} is the atomic flow \emph{associated with} the derivation $\Phi$. Sometimes, when an atom occurrence $a$ in $\Phi$ maps to an edge $\epsilon$ in $\phi$, we decorate $\epsilon$ with the label $a$.
\end{definition}

\begin{theorem}
Every atomic flow is associated with some derivation.
\end{theorem}

\section{Size}

\begin{theorem}
Given a derivation $\Phi$ with associated flow $\phi$, $|\Phi|\leq|\phi|^3$.
\end{theorem}